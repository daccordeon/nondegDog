\chapter*{Abstract}
\chaptermark{Abstract}
% self-contained and less than 1 pg!
% no references

% GW, motivation
Gravitational waves are ripples in spacetime that carry information about the massive astrophysical sources they come from. Within the last decade, interferometric detectors like the Advanced~Laser~Interferometric~Gravitational-Wave~Observatory have used detections of gravitational waves with frequency around 100~Hz from the late inspiral of binary mergers of black holes and neutron stars to learn about such systems.
Gravitational waves around 1~kHz are yet to be detected but could contain other valuable information, e.g.\ 1--4~kHz gravitational waves from the post-merger remnant of binary neutron star systems could be used to further constrain the neutron star equation-of-state and better understand the exotic states of matter within.
% However, gravitational waves are predicted to be emitted from other sources but are yet to be detected because they are outside the sensitive frequency range, and therefore there is valuable astrophysical information that is currently going uncollected. For example, 1--4~kHz gravitational waves from the merger and post-merger remnant of binary neutron star systems are predicted to contain information that could be used to further constrain the neutron star equation-of-state and better understand the exotic states of matter within.
% The ability to detect kilohertz (around 1~kHz) gravitational waves could also lead to better understanding of the origin of low-mass black holes, the post-bounce dynamics of core-collapse supernovae, and the discovery of exotic or primordial sources in the stochastic gravitational-wave background.
% why they can't be seen
% existing proposals for kHz sensitivity
Current interferometric gravitational-wave detectors are not sensitive at kilohertz because of a classical limit on their quantum noise--limited sensitivity and their use of optical cavities, where the quantum noise in the measurement comes from the fundamental quantum uncertainty in the state of the laser light. Existing proposals to improve kilohertz sensitivity use quantum squeezing inside the detector to improve the quantum noise, e.g.\ degenerate internal squeezing, or counter the cavity resonance to improve the response to the gravitational-wave signal, e.g.\ stable optomechanical filtering. However, these two proposals require low optical and mechanical loss, respectively.

In this thesis, I investigate nondegenerate internal squeezing that combines these two existing proposals and has not been fully examined before.
% conclusions
I use an analytic, Hamiltonian model to characterise this configuration's quantum noise--limited sensitivity, stability, squeezing threshold, tolerance to optical loss, and different readout schemes. For a realistic future gravitational-wave detector, I find that nondegenerate internal squeezing is more loss-resistant than degenerate internal squeezing and is a viable, all-optical alternative to stable optomechanical filtering. I also find that it could feasibly improve kilohertz gravitational-wave detection and that using a different readout scheme makes it instead promising for broadband 100--4000~Hz detection. 


