\chapter*{Abstract}
\chaptermark{Abstract}
\addcontentsline{toc}{chapter}{Abstract} 

% self-contained and less than 1 pg!
% no references

% GW, motivation
Gravitational waves are ``ripples'' in spacetime emitted by massive astrophysical events. % that they carry away information about.
Over the past decade, interferometric detectors like the Advanced~Laser~Interferometric~Gravitational-Wave~Observatory have been used to measure gravitational waves with frequency around 100~Hz from the binary mergers of black holes and neutron stars to learn about such systems.
Other frequencies of gravitational waves are thought to exist and contain valuable information but are yet to be detected, e.g.\ detecting kilohertz (1--4~kHz) gravitational waves from binary neutron star mergers could be used to further constrain the neutron star equation-of-state and better understand the exotic states of matter within.
% However, gravitational waves are predicted to be emitted from other sources but are yet to be detected because they are outside the sensitive frequency range, and therefore there is valuable astrophysical information that is currently going uncollected. For example, 1--4~kHz gravitational waves from the merger and post-merger remnant of binary neutron star systems are predicted to contain information that could be used to further constrain the neutron star equation-of-state and better understand the exotic states of matter within.
% The ability to detect kilohertz (around 1~kHz) gravitational waves could also lead to better understanding of the origin of low-mass black holes, the post-bounce dynamics of core-collapse supernovae, and the discovery of exotic or primordial sources in the stochastic gravitational-wave background.
% why they can't be seen
% existing proposals for kHz sensitivity
Current interferometric gravitational-wave detectors are not as sensitive at kilohertz compared to 100~Hz because of their use of optical cavities  and a limit on their quantum noise--limited sensitivity, where the quantum noise in the measurement comes from the fundamental quantum uncertainties in the state of the laser light.
% Two existing proposals to improve kilohertz sensitivity are (1) degenerate internal squeezing that uses quantum squeezing inside the detector to improve the quantum noise or (2) stable optomechanical filtering that counters the cavity resonance to improve the response to the gravitational-wave signal. However, these two proposals require low optical loss and mechanical loss, respectively.
%a configuration that combines these two existing proposals and has not been thoroughly examined to date.

In this thesis, I consider how to improve the kilohertz sensitivity of future gravitational-wave detectors. I investigate nondegenerate internal squeezing, a configuration that uses a nonlinear crystal to manipulate the state of the light inside the detector. This technique, called ``squeezing'', avoids the above limit on the quantum noise--limited sensitivity. Nondegenerate internal squeezing improves the quantum noise--limited sensitivity by amplifying the response to the gravitational-wave signal more than it increases the quantum noise. However, it has not been thoroughly examined to date.

% conclusions
I derive an analytic, Hamiltonian model of nondegenerate internal squeezing and calculate its sensitivity and stability.
Then, I analyse its tolerance to the realistic optical losses in a future gravitational-wave detector and show that it is robust to detection loss in the output chain. I also compare its loss tolerance to other proposed configurations to improve kilohertz sensitivity.
% For a realistic future gravitational-wave detector, I find that nondegenerate internal squeezing is more detection loss-resistant than degenerate internal squeezing and is a viable, all-optical alternative to stable optomechanical filtering.
Finally, I find that it could feasibly improve 1--4~kHz sensitivity, and I explore an alternative readout scheme which is promising for broadband 0.1--4~kHz sensitivity.


