\section{Appendix A: Nondegenerate internal squeezing, shot noise only}

\jam{(For reference only, delete this from the actual thesis)}


The full Hamiltonian of the system is given by: $$\hat{H}=\hat{H}_0+\hat{H}_\mathrm{I}+\hat{H}_{\text{GW}}|_{\text{RP}=0}+\hat{H}_{\gamma_R}+\hat{H}_{\gamma_a}+\hat{H}_{\gamma_b}+\hat{H}_{\gamma_c}.$$ Where each term describes:
\begin{itemize}
\item $\hat{H}_0=\hbar\omega_0(\hat{a}^\dagger\hat{a}+\frac{1}{2})+\hbar\omega_0(\hat{b}^\dagger\hat{b}+\frac{1}{2})+\hbar\omega_\mathrm{idler}(\hat{c}^\dagger\hat{c}+\frac{1}{2})$, the uncoupled behaviour of the optical modes, in the Interaction Picture, the model can ignore this behaviour (moving it from the operators onto the states) \jam{(is RWA also used?)}. $\phi$ is the pump phase. % and uses a rotating frame at the carrier frequency in order to ignore the evolution of the optical modes from $\hat{H}_0$ \jam{(be careful, are we? -- also, is RWA used?)}
\item $\hat{H}_\mathrm{I}=i\hbar\omega_s(\hat{a}\hat{b}^\dag-\hat{a}^\dag\hat{b})+\hbar\chi(e^{i\phi}\hat{b}^\dag\hat{c}^\dag+e^{-i\phi}\hat{b}\hat{c})$, the interaction of the three optical modes $\hat{a}, \hat{b}, \hat{c}$, where a semi-classical approximation has been taken to the pump field \jam{(spell out what this means for $\chi$)} and a single-mode approximation to each of the cavity modes which are assumed to be on resonance \jam{(this needs more discussion, do more in deg. int. sqz. section)}
\item $\hat{H}_\mathrm{GW}|_{\text{RP}=0}=\alpha L_\mathrm{arm}h (\frac{\hat{a}+\hat{a}^\dag}{\sqrt{2}})$, \jam{($\alpha$ is $\alpha_\mathrm{GW}$, not the alpha in Li)} the coupling to the gravitational-wave signal $h(t)$
\item $\hat{H}_{\gamma_R}$ \jam{(formula, state but just use Langevin terms)}, the readout of the $\hat{b}$ mode
\item $\hat{H}_{\gamma_i}$ for $i=a,b,c$, the intra-cavity loss ports, these and $\hat{H}_{\gamma_R}$ give the standard \jam{(word choice)} Langevin terms in the Heisenberg-Langevin equations of motion \jam{(cite Gardiner and Collete?)}.
\end{itemize}

From this Hamiltonian $\hat{H}$, I find the Heisenberg-Langevin equations of motion in the Interaction Picture (separating away the evolution with respect to $\hat{H}_0$):
$$\begin{cases}
\dot{\hat{a}}=-\omega_s\hat{b} - \gamma_a \hat{a} + \sqrt{2\gamma_a}\hat{n}^L_a-\frac{i}{\hbar}\alpha L_\mathrm{arm}h\frac{1}{\sqrt{2}}\\
\dot{\hat{b}}=\omega_s\hat{a} -ie^{i\phi} \chi\hat{c}^\dagger - \gamma^b_\mathrm{tot} \hat{b} + \sqrt{2\gamma_R}\hat{B}_\mathrm{in} + \sqrt{2\gamma_b}\hat{n}^L_b\\
\dot{\hat{c}}=-ie^{i\phi}\chi\hat{b}^\dagger - \gamma_c \hat{c} + \sqrt{2\gamma_c}\hat{n}^L_c
\end{cases}$$
I separate out the fluctuating part of each mode $\hat{a}(t)=\langle\hat{a}\rangle+\delta\hat{a}(t)$ from its time average (or large classical motion) $\langle\hat{a}\rangle$. This does not change the form of the equations. \jam{(Why? This is not clear, the LHS does not change but the time-averages remain)} 
% solving the equations
In Fourier space, these equations become (with the simplified notation $\tilde{\delta\hat{Q}}(\Omega)\mapsto\hat{Q}(\Omega)$ and sign convention $\partial_t\mapsto-i\Omega$):
\begin{equation}
\begin{cases}
-i\Omega\hat{a}(\Omega)=-\omega_s\hat{b}(\Omega) - \gamma_a \hat{a}(\Omega) + \sqrt{2\gamma_a}\hat{n}^L_a(\Omega)-i\beta\tilde{h}(\Omega)\\
-i\Omega\hat{b}(\Omega)=\omega_s\hat{a}(\Omega) -ie^{i\phi} \chi\hat{c}^\dagger(-\Omega) - \gamma^b_\mathrm{tot} \hat{b}(\Omega) + \sqrt{2\gamma_R}\hat{B}_\mathrm{in}(\Omega) + \sqrt{2\gamma_b}\hat{n}^L_b(\Omega)\\
-i\Omega\hat{c}(\Omega)=-ie^{i\phi}\chi\hat{b}^\dagger(-\Omega) - \gamma_c \hat{c}(\Omega) + \sqrt{2\gamma_c}\hat{n}^L_c(\Omega).
\end{cases}
\end{equation}

Where $\beta=\frac{\alpha L_\mathrm{arm}}{\sqrt{2}\hbar}$.

Solving these Equations for simultaneous solutions of the $\hat{Q}(\Omega)$ and $\hat{Q}^\dag(-\Omega)$ fields is easier when expressed in matrix form. Let $\vec{\hat{Q}}(\Omega)=[\hat{Q}(\Omega),\hat{Q}^\dag(-\Omega)]^T$, then the above equations can be re-written as:
$$\begin{cases}
(\gamma_a-i\Omega)\vec{\hat{a}}(\Omega)=-\omega_s\vec{\hat{b}}(\Omega) + \sqrt{2\gamma_a}\vec{\hat{n}}^L_a(\Omega)-i\beta\begin{bsmallmatrix}
1 & 0 \\ 
0 & -1
\end{bsmallmatrix}\vec{\tilde{h}}(\Omega)\\
(\gamma^b_\mathrm{tot}-i\Omega)\vec{\hat{b}}(\Omega)=\omega_s\vec{\hat{a}}(\Omega) - \chi\begin{bsmallmatrix}
0 & ie^{i\phi} \\ 
-ie^{-i\phi} & 0
\end{bsmallmatrix}\vec{\hat{c}}(\Omega) + \sqrt{2\gamma_R}\vec{\hat{B}}_\mathrm{in}(\Omega) + \sqrt{2\gamma_b}\vec{\hat{n}}^L_b(\Omega)\\
(\gamma_c-i\Omega)\vec{\hat{c}}(\Omega)=- \chi\begin{bsmallmatrix}
0 & ie^{i\phi} \\ 
-ie^{-i\phi} & 0
\end{bsmallmatrix}\vec{\hat{b}}(\Omega) + \sqrt{2\gamma_c}\vec{\hat{n}}^L_c(\Omega).
\end{cases}$$

Where $\mathrm{I}$ is the $2\times2$ identity matrix. Solving for $\vec{\hat{b}}(\Omega)$, these give
\begin{align}
\vec{\hat{b}}(\Omega)=\mathrm{M}_b^{-1}( &\omega_s\mathrm{M}_a^{-1}(\sqrt{2\gamma_a}\vec{\hat{n}}^L_a(\Omega)-i\beta\begin{bsmallmatrix}
1 & 0 \\ 
0 & -1
\end{bsmallmatrix}\vec{\tilde{h}}(\Omega)) - \chi\begin{bsmallmatrix}
0 & ie^{i\phi} \\ 
-ie^{-i\phi} & 0
\end{bsmallmatrix}\mathrm{M}_c^{-1}\sqrt{2\gamma_c}\vec{\hat{n}}^L_c(\Omega)\\&+ \sqrt{2\gamma_R}\vec{\hat{B}}_\mathrm{in}(\Omega) + \sqrt{2\gamma_b}\vec{\hat{n}}^L_b(\Omega)).
\end{align}
Where the matrices $\mathrm{M}_i,\; i=a,b,c$ are given by:
\begin{equation}
\begin{cases}
\mathrm{M}_a = (\gamma_a-i\Omega)\mathrm{I}\\
\mathrm{M}_b = (\gamma^b_\mathrm{tot}-i\Omega)\mathrm{I} + \omega_s^2 \mathrm{M}_a^{-1} - \chi^2 \begin{bsmallmatrix}
0 & ie^{i\phi} \\ 
-ie^{-i\phi} & 0
\end{bsmallmatrix} \mathrm{M}_c^{-1} \begin{bsmallmatrix}
0 & ie^{i\phi} \\ 
-ie^{-i\phi} & 0
\end{bsmallmatrix}\\
\mathrm{M}_c = (\gamma_c-i\Omega)\mathrm{I}.
\end{cases}
\end{equation}

Having found the light inside the cavities, I want to find the light at the photodetector. The light right outside and travelling away from the signal-recycling mirror is given by the Input/Output (I/O) relation: $\hat{B}_\mathrm{out}=\hat{B}_\mathrm{in}-\sqrt{2\gamma_R}\hat{b}$. Using the beamsplitter model of detection loss at the photodiode, for a beamsplitter of reflectivity $R_\mathrm{PD}$, the quadratures of light at the photodetector are given by: $\vec{\hat{X}}_\mathrm{PD}=\sqrt{R_\mathrm{PD}} \vec{\hat{X}}_\mathrm{PD}^L + \sqrt{1-R_\mathrm{PD}} \vec{\hat{X}}_{B_\mathrm{out}}$. Where $\vec{\hat{X}}_Q=\Gamma \vec{\hat{Q}}=[\hat{X}_{1,Q},\hat{X}_{2,Q}]^T$ for $\hat{X}_{i,Q}$ the ith quadrature of $Q$ and $\Gamma=\frac{1}{\sqrt{2}}\begin{bsmallmatrix}
1 & 1 \\ 
-i & i\end{bsmallmatrix}$ the quadrature matrix.
Putting everything together, I find the quadratures of the light at the photodetector to be given by:
\begin{align}
\vec{\hat{X}}_\mathrm{PD}(\Omega)&=
% \sqrt{R_\mathrm{PD}} \vec{\hat{X}}_\mathrm{PD}^L(\Omega) + \sqrt{1-R_\mathrm{PD}} \Gamma(\hat{B}_\mathrm{in}(\Omega)-\sqrt{2\gamma_R}\hat{b}(\Omega))\\
% &=  \sqrt{R_\mathrm{PD}} \vec{\hat{X}}_\mathrm{PD}^L(\Omega) + \sqrt{1-R_\mathrm{PD}} \Gamma(\mathrm{I}-2\gamma_R\mathrm{M}_b^{-1})\Gamma^{-1}\vec{\hat{X}}_{B_\mathrm{in}}(\Omega)\\
% &-\sqrt{1-R_\mathrm{PD}} \Gamma\sqrt{2\gamma_R}\mathrm{M}_b^{-1}
% (\omega_s\mathrm{M}_a^{-1}\sqrt{2\gamma_a}\vec{\hat{n}}^L_a(\Omega)
% -\omega_s\mathrm{M}_a^{-1}i\beta\begin{bsmallmatrix}
% 1 & 0 \\ 
% 0 & -1
% \end{bsmallmatrix}\vec{\tilde{h}}(\Omega) \\& \hspace{5cm}+ \chi\begin{bsmallmatrix}
% 0 & 1 \\ 
% 1 & 0
% \end{bsmallmatrix}\mathrm{M}_c^{-1}\sqrt{2\gamma_c}\vec{\hat{n}}^L_c(\Omega) + \sqrt{2\gamma_b}\vec{\hat{n}}^L_b(\Omega))\\&=
\mathrm{R_{in}}\vec{\hat{X}}_{B_\mathrm{in}}(\Omega)
+ \mathrm{R}^L_a\vec{\hat{X}}^L_a(\Omega)
+ \mathrm{R}^L_b\vec{\hat{X}}^L_b(\Omega)
+ \mathrm{R}^L_c\vec{\hat{X}}^L_c(\Omega)
+ \mathrm{R}^L_\mathrm{PD}\vec{\hat{X}}_\mathrm{PD}^L(\Omega)
+ \mathrm{T}\vec{\tilde{h}}(\Omega).
\end{align}

Where $\vec{\hat{X}}_{B_\mathrm{in}}(\Omega)$ is the quadrature vector of the vacuum from the main vacuum port behind the signal-recycling mirror, $\vec{\hat{X}}^L_a(\Omega)$ is the vacuum from the $\hat{a}$ intra-cavity loss, etc.. And where the transfer function matrices $\mathrm{R}_i, \mathrm{T}$ for the noises and signal, respectively, are given by the following (where the frequency dependence is inside each $\mathrm{M}_i^{-1}$):
\begin{equation}
\begin{cases}
\mathrm{R_{in}}=\sqrt{1-R_\mathrm{PD}} \Gamma(\mathrm{I}-2\gamma_R\mathrm{M}_b^{-1})\Gamma^{-1}\\
\mathrm{R}^L_a=-\sqrt{1-R_\mathrm{PD}} 2\sqrt{\gamma_R\gamma_c}\omega_s\Gamma\mathrm{M}_b^{-1}\mathrm{M}_a^{-1}\Gamma^{-1}\\
\mathrm{R}^L_b=-\sqrt{1-R_\mathrm{PD}} 2\sqrt{\gamma_R\gamma_b}\Gamma\mathrm{M}_b^{-1}\Gamma^{-1}\\
\mathrm{R}^L_c=-\sqrt{1-R_\mathrm{PD}} 2\sqrt{\gamma_R\gamma_c}(-\chi)\Gamma\mathrm{M}_b^{-1}\begin{bsmallmatrix}
0 & ie^{i\phi} \\ 
-ie^{-i\phi} & 0
\end{bsmallmatrix}\mathrm{M}_c^{-1}\Gamma^{-1}\\
\mathrm{R}^L_\mathrm{PD}=\sqrt{R_\mathrm{PD}}\\
\mathrm{T}=\sqrt{1-R_\mathrm{PD}} \sqrt{2\gamma_R}i\omega_s\beta \Gamma\mathrm{M}_b^{-1}\mathrm{M}_a^{-1}\begin{bsmallmatrix}
1 & 0 \\ 
0 & -1
\end{bsmallmatrix}.
\end{cases}
\end{equation} 


\jam{(set up the combined noise, spectral densities matrix Sx)}

\begin{equation}
\mathrm{S}_X(\Omega)2\pi\delta(\Omega-\Omega')=\ev{\vec{\hat{X}}_\mathrm{PD}(\Omega)\cdot\vec{\hat{X}}_\mathrm{PD}(\Omega')^\dag}
\end{equation}

\jam{(explain uncorrelated vacuum assumption -- i and j are not indices of vectors)}

\begin{equation}
\ev{\vec{\hat{X}}_{(i)}(\Omega)\cdot\vec{\hat{X}}_{(j)}(\Omega')^\dag}=\delta_{i,j}\mathrm{I}\,2\pi\delta(\Omega-\Omega'),\quad \mathrm{S_{vac}}=\mathrm{I}=\begin{bsmallmatrix}
1 & 0 \\ 
0 & 1
\end{bsmallmatrix}
\end{equation}

\jam{(versus single-sided power spectral density of single function, not a vector: $A\circ B=\dfrac{1}{2}(A\cdot B+B\cdot A)$, check factors of two again --> single sided, vac=1, quadratures normalised as 1/rt2, quadrature is Hermitian and therefore commutes with its dagger)}


\begin{align}
\mathcal{S}_X(\Omega)2\pi\delta(\Omega-\Omega')&=\ev{\vec{\hat{X}}_{\mathrm{PD},2}(\Omega)\circ\vec{\hat{X}}_{\mathrm{PD},2}(\Omega')^\dag},\quad \mathcal{S}_\mathrm{vac}=1\\
\mathcal{S}_X(\Omega)&=1 +8 (1-R_\text{PD}) \gamma_c \gamma_R \chi ^2 \Omega ^4 \left(\gamma_a^2+\Omega ^2\right)\\
&/(\Omega ^4 (\left(\gamma_c^2+\Omega ^2\right) \omega_s^4-2 \left(\gamma_a \gamma_c \chi ^2-\gamma_a \gamma^b_\text{tot} \left(\gamma_c^2+\Omega ^2\right)+\Omega ^2 \left(\gamma_c^2+\chi ^2+\Omega ^2\right)\right) \omega_s^2\\
&+\left(\gamma_a^2+\Omega ^2\right) \left(\Omega ^4+\left({\gamma^b_\text{tot}}^2+\gamma_c^2+2 \chi ^2\right) \Omega ^2+\left(\chi ^2-\gamma^b_\text{tot} \gamma_c\right)^2\right)))
\end{align}

\begin{align}
\abs{\mathcal{T}(\Omega)}^2&=\abs{(\mathrm{T}.\begin{bsmallmatrix}
1\\ 
1
\end{bsmallmatrix})_2}^2\\
&=4 (1-R_\text{PD})\beta^2 \gamma_R \omega_s^2\left(\gamma_c^2+\Omega ^2\right)\\
&/(\Omega ^4 (\left(\gamma_c^2+\Omega ^2\right) \omega_s^4-2 \left(\gamma_a \gamma_c \chi ^2-\gamma_a \gamma^b_\text{tot} \left(\gamma_c^2+\Omega ^2\right)+\Omega ^2 \left(\gamma_c^2+\chi ^2+\Omega ^2\right)\right) \omega_s^2\\
&+\left(\gamma_a^2+\Omega ^2\right) \left(\Omega ^4+\left({\gamma^b_\text{tot}}^2+\gamma_c^2+2 \chi ^2\right) \Omega ^2+\left(\chi ^2-\gamma^b_\text{tot} \gamma_c\right)^2\right)))
\end{align}

I define the sensitivity of the detector to be the noise-to-signal ratio, i.e. the ratio of the noise and signal transfer functions (or, the signal-normalised noise).
Assuming that we measure the second quadrature, e.g. via balanced homodyne readout \jam{(address optimisation of quadrature angle, might not be 2 just because signal is only there. Also, need to describe homodyne readout somewhere)}, this gives:
$$S_h(\Omega)=\frac{(\mathrm{S}_X(\Omega))_{2,2}}{\abs{\mathcal{T}(\Omega)}^2}.$$





