\chapter{Alternative readout schemes for nondegenerate internal squeezing}
\label{chp:idler_readout}
% only a short chapter! --> too long right now!

% motivate by: signal readout requires the idler readout port to be closed...
% chronological does not matter, make this flow from the previous chapter --> emergence of a new possibility separate from initial interest in kilohertz sensitivity
% don't say preliminary, idler readout is characterised (threshold, stability, limits, tolerance to losses, and performance against readout rates) and give potential results for combined readout, be explicit about what steps should be taken next
In this chapter, I explore using the idler mode for the readout of nondegenerate internal squeezing. In the previous chapter, I showed that idler loss and the idler readout rate limit the sensitivity of signal readout. Here, I show how the idler readout rate can instead be used beneficially to measure the gravitational-wave signal.
Since a readout scheme using the mechanical idler mode of stable optomechanical filtering has been proposed~\cite{liEnhancingInterferometerSensitivity2021}, a comprehensive understanding of nondegenerate internal squeezing also requires idler readout to be understood. This work emerged during the present research and is somewhat separate from my initial motivation to improve kilohertz sensitivity. %is motivated by the use of idler and combined readouts for the nondegenerate OPO, see Section~\ref{sec:nOPO_combined_readout}, and
Firstly, in Section~\ref{sec:nIS_idlerRO_model}, I define idler readout and explain how it can be combined, incoherently and coherently, with signal readout. % I will also discuss how it relate to the optomechanical analogue.
% explain that I leave combined readout to future work
Secondly, in Section~\ref{sec:idlerRO_initial_results}, I characterise these alternative readout schemes, including their stability, threshold, and high loss limit. I also compare the general behaviour of idler readout to signal readout. % Although I leave studying coherently combined readout to future work, I detail what steps should be taken next.
Thirdly, in Section~\ref{sec:idlerRO_losses}, I find the tolerance of idler readout to realistic optical losses compared to signal readout. %I will show that idler readout is dominated by signal loss\jam{(check)}, but that its tolerance to it is different to signal readout's tolerance to idler loss.
I also find the effect of changing the pump phase.  
%, where the idler readout's tolerance to signal loss is particularly different.
Finally, in Section~\ref{sec:idlerRO_feasibility}, I consider the feasibility of using an alternative readout scheme for broadband gravitational-wave detection. %, and argue that it is feasible to use for a future detector and warrants further consideration.


\section{Conceptual understanding and model}
\label{sec:nIS_idlerRO_model}
% short section, is a separate section necessary?
% review idler and combined signal-idler readout from nOPO chapter

\begin{figure}[!ht]
    \centering
    \includegraphics[width=0.65\textwidth]{idler_RO_config.pdf}
    \caption{Nondegenerate internal squeezing possible readout schemes that were represented by the photodetector in Fig.~\ref{fig:nIS_mode_diagram}. The light can either be (i) split at a dichroic optic to measure the signal and idler separately or (ii) coherently combined at the same photodetector, where the vertical ellipses represent where the readout schemes join the main configuration. When measuring the modes separately, the quadratures of the mode can be still be coherently combined, e.g.\ the idler quadratures can be combined at angle $\psi_1$.}
    \label{fig:idler_RO_config}
\end{figure}

The idler mode $\hat c$ at $\omega_0+\Delta$ can be used for measurement (called \emph{``idler readout''}) because the squeezer couples the gravitational-wave signal from the signal mode $\hat b$ at $\omega_0$ into the idler mode. Therefore, unlike signal readout, the idler readout's performance cannot be compared to when the squeezer is off because then there is no signal. This change in mode structure also means that idler readout is not PT-symmetric in the lossless limit and, therefore, I expect different behaviour than the signal readout~\footnote{Although the internal mode structure of the configuration is the same (see Appendix~\ref{app:mode_structure}), the idler readout scheme ``sees'' the configuration differently.}.
The idler readout rate~\footnote{Which can be made different to the signal readout rate by using a dichroic signal-recycling mirror.}, previously a source of loss for the signal readout, can be used to measure the idler mode separately or coherently combined with the signal mode as shown in Fig.~\ref{fig:idler_RO_config}. That measuring the idler mode simply requires detecting the light leaking out through the signal-recycling mirror is an advantage of this all-optical configuration in comparison to measuring the mechanical idler mode of stable optomechanical filtering~\cite{liEnhancingInterferometerSensitivity2021}.

Idler and coherently combined readout \emph{use the same model as signal readout} from Section~\ref{sec:nIS_model} and combine the quadratures at the photodetector as described for the nondegenerate OPO in Section~\ref{sec:nOPO_combined_readout}. Specifically, let the signal and idler quadratures of $\vec{\hat X}_\text{PD}(\Omega)$ from Eq.~\ref{eq:nIS_Xpd} be coherently combined to measure~\footnote{Here, the combination angles $\psi_0,\psi_1,\psi_2\in[0,2\pi)$ can be chosen by the phase of the local oscillator in a homodyne readout scheme for example~\cite{danilishinQuantumMeasurementTheory2012}.}
\begin{equation}\label{eq:Xcom_three_angles}
\hat X_\text{com}(\Omega)=\left(\cos(\psi_2)\cos(\psi_0) , \cos(\psi_2)\sin(\psi_0) , \sin(\psi_2)\cos(\psi_1) , \sin(\psi_2)\sin(\psi_1)\right)\cdot\vec{\hat X}_\text{PD}(\Omega).
\end{equation}
By Eq.~\ref{eq:nIS_sigRO_signal_response}, there is gravitational-wave signal in every quadrature except the first signal quadrature $\hat{X}_{b,1}$ and the idler combination $-\sin(\phi)\hat{X}_{c,1}+\cos(\phi)\hat{X}_{c,2}$ for the pump phase $\phi$. Therefore, $\hat{X}_\text{com}$ contains the gravitational-wave signal unless $\psi_0=\psi_2=0$ or $\psi_2=\pi/2,\;\psi_1=\phi+\pi/2$, respectively, as long as the squeezer is on. 
Let the coherently combined readout be defined to measure $\hat{X}_\text{com}$ for some arbitrary combination of signal and idler that contains the gravitational-wave signal, where the readout angles $(\psi_0,\psi_1,\psi_2)$ can change with frequency.
Let the idler readout be defined to maximise the gravitational-wave signal response by measuring $\cos(\phi)\hat{X}_{c,1}+\sin(\phi)\hat{X}_{c,2}$, i.e.\ by matching $\psi_1$ to the pump phase $\phi$~\footnote{This is the idler quadrature connected to the second signal quadrature, which maximises the signal, by the squeezer.}. Like the signal readout measuring $\hat{X}_{b,2}$ in Eq.~\ref{eq:nIS_sigRO_sens}, this combination might not maximise the sensitivity because it does not consider the noise. %I will examine the effect of not measuring exactly this combination later
% I defined the signal readout to maximise the gravitational-wave signal by measuring $\hat{X}_{b,2}$, see Eq.~\ref{eq:nIS_sigRO_sens}, similarly, I define the idler readout to maximise the signal by measuring $\cos(\phi)\hat{X}_{c,1}+\sin(\phi)\hat{X}_{c,2}$. I will discuss the tolerance to the combination angle $\psi_1$ later. As before, this choice might not maximise the sensitivity because it does not consider the noise.
% I model these alternative readout schemes by using a linear combination of the signal and idler quadratures at the photodetector, as described for the nondegenerate OPO in Section~\ref{sec:nOPO_combined_readout}.
% mode structure changes, cannot compare to two-cavity detector
% When the squeezer is off, no gravitational-wave signal enters the idler mode and therefore nondegenerate internal squeezing with idler readout cannot be directly compared to the two-cavity interferometer, unlike the signal readout. Therefore, I expect the idler readout's noise and signal responses to be different to the signal readout's.
% coherent combination of mechanical and optical mode is more difficult than optical-optical where you just expose a PD to both at once?
% coherent vs incoherent combinations
Finally, \emph{incoherently combined readout} uses the separate signal and idler readouts in Fig.~\ref{fig:idler_RO_config} at different frequencies. This readout scheme does not measure the correlations between the signal and idler unlike the coherently combined readout as discussed in Section~\ref{sec:nOPO_combined_readout}. Therefore, the incoherently combined readout can only achieve the envelope of the separate readouts' sensitivity curves but the coherently combined readout might achieve better, e.g.\ because the correlations might reduce the noise.
% As discussed in Section~\ref{sec:nOPO_combined_readout}, the coherently combined readout depends on correlations between the signal and idler. 

% The signal and idler readouts can be combined coherently or incoherently. Firstly, measuring $\hat{X}_\text{com}$ at the photodetector\jam{(explain how? state that the configuration diagram with one photodetector is a simplification)} coherently combines the signal and idler because the signal-idler correlations appear in the measurement, see Eq.~\ref{eq:Scom_nOPO_eg}. Secondly, the measurements from separate signal and idler readouts can be incoherently combined such that the correlations do not appear~\cite{}, e.g.\ as $c_0 S_{1,1}+ c_1 S_{3,3}$~\footnote{Where the spectrum at each readout is calculated and then combined with the other, rather than the measurements being combined and then the spectrum taken.}, and the combination of signal and idler can be different at different frequencies. Therefore, the sensitivity of coherently combined readout can behave differently to the separate signal and idler readouts because of the new terms from the correlations but the sensitivity of incoherently combined readout is some linear combination of the signal and idler readout curves at each frequency, e.g.\ the envelope of the two curves.


\section{Results}
\label{sec:idlerRO_initial_results}
% mirror chapters 4,5

% I now present the results for the idler and combined readouts. %, (1) their stability and threshold, (2) the general behaviour of idler readout, and (3) their reduction to OPOs in the high arm loss limit.  %This is similar to Section~\ref{sec:nIS_sigRO_results} for the signal readout.
I analyse these alternative readouts \emph{using the same methodology as the signal readout} in the previous two chapters. 
% 2-3 sentences of immediate results
To start with, some of the behaviour is the same as the signal readout. 
The idler and combined readouts have the same singularity threshold and stability as signal readout in Section~\ref{sec:stability_and_threshold}. This is because all of the (co)variances of the signal and idler modes at the photodetector from Eq.~\ref{eq:nIS_Sx} have a common denominator that physically relates to them both being inside the signal-recycling cavity\jam{(is this true?)}. %It also makes physical sense as these properties depend on the intra-cavity modes and not on which mode or quadrature is measured\jam{(but what about unstable vs stable optomechanical filtering?)}.  
%the stability of the intra-cavity fields driven by the signal and noise\jam{(am I conflating the stability of the output field and the intra-cavity field?)} and the gain-loss balancing to determine threshold only depend on the coupling rates out of the interferometer, not on which quadrature or field is then measured.
The high arm loss limit is also the same as signal readout in Section~\ref{sec:nOPO_reduction} as the shot noise of idler readout reduces to that of a nondegenerate OPO between the signal-recycling mirror and a fully-reflective input test mass. The coherently, equally combined readout, i.e.\ with $\psi_2=\pi/4$, reduces to a degenerate OPO\jam{(check)} as in Section~\ref{sec:nOPO_combined_readout}~\footnote{Although I thought that this readout might also make nondegenerate internal squeezing recover degenerate internal squeezing, this is not true as the noise response has squeezing and anti-squeezing at different frequencies in the same quadrature, unlike the degenerate case. I suspect that this is because the idler is not resonant in the arms and therefore the signal and idler modes are not symmetric. This could be verified by making the idler frequency resonant in the arms, which I leave to future work.}.
 %, as shown in Fig.~\ref{fig:nIS_idler_nOPO_limit}.
% This is the same behaviour as the signal readout in Section~\ref{sec:nOPO_reduction}.
% Therefore, I expect that in the high arm loss limit the combined readout reduces to the combined readout from the same nondegenerate OPO. This is indeed true, as $\hat X_\text{com}$ with balanced $\psi_2=\pi/4$\jam{(check)} recovers the squeezed variance of a degenerate OPO when the signal and idler losses are symmetric\jam{(check)} as in Section~\ref{sec:nOPO_combined_readout}.
% interesting that combined nIS does not reduce to dIS, perhaps this is because the idler is not resonant in the arms
% Because combined readout of a nondegenerate OPO recovers a degenerate OPO, I thought that it might make nondegenerate internal squeezing recover degenerate internal squeezing, but this is false as the combined readout displays squeezing at some frequencies and anti-squeezing at other frequencies. I suspect that this is because the idler is not resonant in the arms and therefore the signal and idler modes are not symmetric which is required in the OPO case\jam{(can I justify this more?)}. 
Due to limited research time, I have not explored coherently combined readout further yet. I will leave it to future work and focus on the idler readout here. 

% \subsection{General behaviour of idler readout}
\label{sec:nIS_general_behaviour_idler}

% plot: separate idler and signal readout, 
% tradeoff of bandwidth and DC peak sensitivity + interaction with RP
\begin{figure}[t]
    \centering
    \includegraphics[width=\textwidth]{nIS_signal_vs_idler_ROs.pdf}
    % showcase same lines without RP?
    \caption{\jam{(Check what happens if signal readout rate is changed instead.)} Nondegenerate internal squeezing idler readout compared to signal readout for the same squeezing ($95\%$ threshold), showing the noise (upper-left panel), signal (bottom-left panel), and sensitivity (right panel). I use the parameter set from Tab.~\ref{tab:idler_RO_parameters} but with 500~Hz signal readout rate and 5~Hz idler readout rate by appropriately changing the signal-recycling length and respective transmissivities to adjust for the added loss with both ports open. The astrophysical kilohertz sensitivity target is shown for later comparison. Idler readout trades sensitivity above and at the peak (e.g.\ 0.5--10~kHz) for sensitivity below the peak (e.g. 50--500~Hz) down to where the improvement in the signal DC response cancels with the amplified radiation-pressure noise (e.g.\ 30~Hz). If the idler readout rate is instead greater than the signal readout rate (not shown) the two sensitivities are approximately equal except above 10~kHz. %It might be optimal for broadband improvement to use an incoherently combined readout scheme that achieves the envelope of the two curves with a smaller idler readout rate, e.g.\ 5~Hz.
    % When the idler readout rate is larger than the signal readout rate\jam{(check that it is the comparison that matters)}, the difference between the readouts is smaller. And for idler smaller than signal readout rates, e.g.\ 5 and 500~Hz, respectively, an incoherent, frequency-dependent readout scheme is useful, e.g.\ use idler around 100~Hz and signal around 1000~Hz.
    }
    \label{fig:nIS_signal_vs_idler_ROs}
\end{figure}

\begin{table}
\centering
% \begin{tabular}{@{}ll|ll@{}}
% \toprule
% carrier wavelength, $\lambda_0$ & 2 $\mu\text{m}$ & \cellcolor[HTML]{EFEFEF}signal mode transmissivity, $T_{\text{SRM},b}$ & \cellcolor[HTML]{EFEFEF} 0 \\
% arm cavity length, $L_\text{arm}$ & 4 km & \cellcolor[HTML]{EFEFEF}signal readout rate, $\gamma^b_R$ & \cellcolor[HTML]{EFEFEF} 0 \\
% signal-recycling cavity length, $L_\text{SRC}$ & 1.124 km & \cellcolor[HTML]{EFEFEF}idler mode transmissivity, $T_{\text{SRM},c}$ & \cellcolor[HTML]{EFEFEF} 0.046 \\
% circulating arm power, $P_c$ & 3 MW & \cellcolor[HTML]{EFEFEF}idler readout rate, $\gamma^c_R$ & \cellcolor[HTML]{EFEFEF} 500 Hz \\
% test mass mass, $M$ & 200 kg & arm intra-cavity loss, $T_{l,a}$ & 100 ppm \\
% radiation pressure, $\rho$ & $\neq0$ & signal mode intra-cavity loss, $T_{l,b}$ & 1000 ppm \\
% input test mass transmissivity, $T_\text{ITM}$ & 0.197 & idler mode intra-cavity loss, $T_{l,c}$ & 1000 ppm \\
% sloshing frequency, $\omega_s$ & 5 kHz & detection loss, $R_\text{PD}$ & $10\%$ \\ \bottomrule
% \end{tabular}
\begin{tabular}{@{}ll|ll@{}}
\toprule
signal mode transmissivity, $T_{\text{SRM},b}$ & 0 & idler mode transmissivity, $T_{\text{SRM},c}$ & 0.046 \\ 
signal readout rate, $\gamma^b_R$ & 0 & idler readout rate, $\gamma^c_R$ & 500 Hz \\ \bottomrule
\end{tabular}
\caption{Idler readout parameter set is the same as Tab.~\ref{tab:signal_RO_parameters} for the signal readout except with the signal and idler readout rates exchanged as shown. The idler readout results use these parameters unless stated otherwise.}
\label{tab:idler_RO_parameters}
\end{table}

The \emph{general behaviour} of idler readout is different to signal readout which I compare to in Fig.~\ref{fig:nIS_signal_vs_idler_ROs}~\footnote{Here, I change the readout rates from Tab.~\ref{tab:idler_RO_parameters} by fixing the signal transmissivity and changing the signal-recycling cavity length and idler transmissivity.}. The noise response of the idler readout has a peak with less anti-squeezing but at the same frequency\jam{(check, why?)} as the signal readout because the peak frequency is determined by the common singularity threshold frequency. The idler also has worse radiation-pressure noise than the signal\jam{(quantify)}\jam{(why?)}.
The signal response of the idler decreases the response at and above the peak but improves it below the peak down to DC compared to the signal readout. 
% The result is that the signal transfer function improvement with idler readout persists at low frequencies but radiation pressure means that there's no sensitivity improvement. 
The resulting sensitivity of the idler compared to the signal readout is the same below 30~Hz, improves from $\sim$30--500~Hz, and worsens above 500~Hz as shown in Fig.~\ref{fig:nIS_signal_vs_idler_ROs}. The squeezer parameter must be non-zero for idler readout, and increasing the squeezer parameter improves the idler sensitivity from around 50~Hz up to the peak frequency but also increases the radiation-pressure noise. 
The result that the idler readout improves the sensitivity around 100~Hz agrees with Ref.~\cite{liEnhancingInterferometerSensitivity2021} that considers idler readout of the mechanical mode for the optomechanical analogue.

For \emph{different readout rates}, the general behaviour is the same: idler readout is better at ``low--middle'' frequencies and worse at ``high'' frequencies than signal readout. If the idler readout rate is increased, then the DC signal response improvement diminishes at the same rate that the radiation-pressure noise increases such that the ``low'' frequency sensitivity remains the same\jam{(why?)}. The idler readout is most useful when the readout rates are comparable or the idler is smaller than the signal readout rate, e.g.\ the case shown in Fig.~\ref{fig:nIS_signal_vs_idler_ROs}.\jam{(why? quantify by change in integrated sensitivity?)}
% section 3-B in Li 2021 has noticed this already! ``During this work, I became aware\jam{(do I need to couch this? can I just say that this agress with an existing result)} of a related result in Ref.~\cite{liEnhancingInterferometerSensitivity2021} for the optomechanical analogue of this configuration.''
% both readout ports are open, therefore strong loss in each mode

 % and will show that signal loss affects the idler readout\jam{(but might be beneficial?)}. 
% I will address the coherently combined readout below.

% \subsection{Reduction to OPOs}
% \label{sec:nIS_idlerRO_reduction_to_OPOs}
% reduction to nOPO and dOPO (recovers squeezing)
% reduction to OPOs, combined reduction to dOPO

% \begin{figure}
% 	\centering
% 	\includegraphics[width=\textwidth]{nIS_idler_nOPO_limit.pdf}
% 	\caption{\jam{(Purpose: show limit)}\jam{(Cut or move to appendix but talk about result in text. Add predicted nOPO variance to compare to. Just show noise response.)} Nondegenerate internal squeezing idler readout shot noise response recovers nondegenerate OPO with fully reflective input test mass in the high arm loss limit, like signal readout in Fig.~\ref{fig:nIS_signal_nOPO_limit}.}
% 	\label{fig:nIS_idler_nOPO_limit}
% \end{figure}
% \begin{figure}
% 	\centering
% 	\includegraphics[width=\textwidth]{nIS_combinedRO_dOPO_limit.pdf}
% 	\caption{\jam{(Purpose: show that limit work for combined readout)}\jam{(Cut or move to appendix. Show only noise response. Show expected limit. Plot side-by-side with Fig.~\ref{fig:nIS_idler_nOPO_limit} limit to compress plots.)} Nondegenerate internal squeezing coherently combined readout with equal mixing, $\psi_2=\pi/4$, showing that the shot noise response in the limit of large arm loss recovers that of a degenerate OPO with fully reflective input test mass. This agrees with the limits for the separate readouts and that the nondegenerate OPO with combined readout recovers a degenerate OPO, as shown in Fig.~\ref{fig:nOPO_combined_readout}. The signal and idler readout and loss rates are symmetric.}
% 	\label{fig:nIS_combinedRO_dOPO_limit}
% \end{figure}


\section{Idler readout tolerance to optical loss}
\label{sec:idlerRO_losses}

% idler readout tolerance to losses
\begin{figure}
    \centering
    \includegraphics[width=0.6\textwidth]{nIS_idlerRO_tolerance_Rpd.pdf} 
    \caption{Nondegenerate internal squeezing idler readout tolerance to detection loss ($R_\text{PD}$). Idler readout's tolerance to detection loss is similar to signal readout's, shown in Fig.~\ref{fig:nIS_sigRO_tolerance_Rpd}, with uniform loss of sensitivity except around the peak frequency and where the radiation-pressure noise dominates because there the loss in signal and noise are roughly equal. I do not compare the idler readout to the performance without squeezing because no gravitational-wave signal reaches the idler mode with the squeezer off. I use the parameter set in Tab.~\ref{tab:idler_RO_parameters} and $95\%$ threshold.}
    \label{fig:nIS_idlerRO_tolerance_Rpd}
\end{figure}
\begin{figure}
	\centering
	\includegraphics[width=0.6\textwidth]{nIS_idlerRO_tolerance_Tlb.pdf}
	\caption{\jam{(Side-by-side with idler loss?)} Nondegenerate internal squeezing idler readout tolerance to signal mode intra-cavity loss ($T_{l,b}$). Signal loss decreases the peak frequency and worsens the radiation pressure noise of idler readout but broadens the sensitivity (e.g.\ from 10--1000~Hz) independently of the idler readout rate\jam{(quantify by integrated sensitivity)}. Opening the signal readout port as in Fig.~\ref{fig:nIS_signal_vs_idler_ROs} would introduce signal loss on the order of 10000~ppm. I use the parameter set in Tab.~\ref{tab:idler_RO_parameters} and $95\%$ threshold.}
	\label{fig:nIS_idlerRO_tolerance_Tlb}
\end{figure}
\begin{figure}[t]
	\centering
	\includegraphics[width=0.6\textwidth]{nIS_idlerRO_tolerance_Tlc.pdf}
	\caption{\jam{(Explain RPN change in left column.)} Nondegenerate internal squeezing idler readout tolerance to idler intra-cavity loss ($T_{l,c}$). Idler readout is tolerant to idler loss but less resistant than signal readout is to signal loss, shown in Fig.~\ref{fig:nIS_sigRO_tolerance_Tlb}. In either case, the readout rate dominates the realistic intra-cavity loss rate. The idler readout is least tolerant at 10--1000~Hz, but increasing the idler loss ten-fold from the realistic level decreases sensitivity by less than a factor of two. %Although they initially look similar, this tolerance is different to detection loss in Fig.~\ref{fig:nIS_idlerRO_tolerance_Rpd} as idler loss decreases the radiation-pressure noise and does not affect the sensitivity above the peak.
    I use the parameter set in Tab.~\ref{tab:idler_RO_parameters} and $95\%$ threshold.}
	\label{fig:nIS_idlerRO_tolerance_Tlc}
\end{figure}
% \begin{figure}
%     \centering
%     \includegraphics[width=\textwidth]{nIS_idlerRO_tolerance_Tla.pdf} 
%     \caption{\jam{(Purpose: show loss tolerance to compare to signal readout 4/4)} Nondegenerate internal squeezing idler readout tolerance to loss (4/4): arm intra-cavity losses. Increasing the arm loss decreases the peak sensitivity and the radiation-pressure noise--limited sensitivity but increases the sensitivity above the peak and below the peak down to $\sim10$~Hz\jam{(quantify change in integrated sensitivity)}. Increasing the idler readout rate decreases the effect of arm loss.\jam{(is this because the peak is smaller?)} }
%     \label{fig:nIS_idlerRO_tolerance_Tla}
% \end{figure}
% \begin{figure}
%     \centering
%     \includegraphics[width=\textwidth]{nIS_idlerRO_noise_budget.pdf} 
%     \caption{\jam{(Purpose: show all losses)} Nondegenerate internal squeezing idler readout noise response to each noise source, including the readout port, each loss port, and the total noise. The idler readout rate is 500~Hz and the signal readout port is closed. The noise from all of the realistic losses is small compared to, i.e.\ around 10~dB lower than, the vacuum through the readout port, except around the anti-squeezed peak where the signal loss dominates. The loss-based noise is dominated by idler loss below 10~Hz and detection loss above 10~Hz, but the sensitivity is more affected by signal loss because of its effect on the signal response compared to the other realistic losses.\jam{(quantify and clarify in the text)} Compare to signal readout in Fig.~\ref{fig:nIS_sigRO_noise_budget}.}
%     \label{fig:nIS_idlerRO_noise_budget}
% \end{figure}

% tolerance to losses? see problem with position of signal readout tolerance above
% decrease in peak sensitivity, large increase in bandwidth

I consider the tolerance of idler readout to optical loss using the parameter set in Tab.~\ref{tab:idler_RO_parameters}. I use a higher idler readout rate here than in Fig.~\ref{fig:nIS_signal_vs_idler_ROs} (500 versus 5~Hz respectively) because closing the signal readout port reduces the loss and narrows the peak, therefore, meaning that a lower idler readout rate is not required to narrow the peak~\footnote{Narrowing the peak increases its height, trading bandwidth for peak sensitivity.}. %Increasing the idler readout rate with the signal readout port closed increases the width of the peak more slowly than if the signal readout rate is increased for signal readout with the idler readout port closed\jam{(quantify)}. I suspect that it is related to the decreased noise and signal peak with the idler readout in Fig.~\ref{fig:nIS_signal_vs_idler_ROs}.}\jam{(do I need to plot this?)}.
% Similarly to Section~\ref{sec:nIS_tolerance_to_losses}, I close the signal readout port and vary the idler readout rate by changing the length of the signal-recycling cavity which also scales each of the intra-cavity losses\jam{(try changing Tsrm instead?)}.
Using the same methodology as the signal readout in Section~\ref{sec:nIS_tolerance_to_losses}, I consider the detection, signal, idler, and arm losses in turn.

Firstly, as shown in Fig.~\ref{fig:nIS_idlerRO_tolerance_Rpd}, the realistic \emph{detection loss} affects the idler readout similarly to the signal readout as it uniformly worsens sensitivity by $10\%$ except where the noise is far from vacuum, e.g.\ around the peak and below 3~Hz. %, i.e.\ when it is strongly anti-squeezed or dominated by radiation pressure, since the decrease in noise and signal response are then approximately equal.
This tolerance is independent of the idler readout rate. %Increasing the idler readout rate with the signal readout port closed increases the bandwidth of the amplification peak but far more slowly than the corresponding change in the signal readout, e.g.\ compare the 50~kHz readout rate plots in Figs.~\ref{fig:nIS_idlerRO_tolerance_Rpd}~\ref{fig:nIS_sigRO_tolerance_Rpd}\jam{(check, quantify, explain why?)}. %Like for signal readout, detection loss can be directly addressed by Caves's amplification which means that the realistic $\sim10\%$ loss of sensitivity away from the amplification peak\jam{(check)} can be partially mitigated\jam{(quantify)} unlike the other losses.

% for idler and arm loss, watch out for Tlc=Tla and the change to nOPO-like behaviour, but should use the realistic losses and call this out if it can be seen
Secondly, as shown in Fig.~\ref{fig:nIS_idlerRO_tolerance_Tlb}, the \emph{signal mode intra-cavity loss} affects the idler readout differently than the way that any of the losses affected the signal readout, which reflects the change in mode structure. The signal loss increases the radiation-pressure noise\jam{(why?)}, broadens the shot noise peak\jam{(check the effect on the shot noise peak, does it also decrease?)}, and decreases the signal peak, but uniformly, strongly (e.g.\ by at least a factor of two) amplifies the signal response away from the peak\jam{(why?)}. The effect on the signal response is similar to how loss damps a harmonic oscillator resonance, lowering the peak but broadening the bandwidth, except that the broadening extends to all frequencies away from the peak\jam{(check and cite comparison)}. The net result of signal loss is that the sensitivity worsens at the peak and at low frequencies below $\sim10$~Hz where the radiation-pressure noise dominates, but improves at all other frequencies\jam{(quantify integrated sensitivity change)}. I find that idler readout is strongly affected by realistic signal loss and that changing the idler readout rate does not improve the tolerance. When the signal readout port is open, the signal loss effectively increases by 46000~ppm and the idler readout sensitivity resembles Fig.~\ref{fig:nIS_signal_vs_idler_ROs} with diminished a peak but broad sensitivity from 10--1000~Hz. I will later consider whether purposefully opening the signal readout port could be used to improve broadband sensitivity.  % using idler readout. %Although, strangely, it might be useful for broadband detection, I find that idler readout is strongly affected by realistic signal loss.

Thirdly, as shown in Fig.~\ref{fig:nIS_idlerRO_tolerance_Tlc}, the \emph{idler mode intra-cavity loss} decreases the idler readout's sensitivity away from the peak. This is unlike the signal readout's tolerance to either signal loss, which decreased the sensitivity at the peak in Fig.~\ref{fig:nIS_sigRO_tolerance_Tlb}, or to idler loss, which decreased the sensitivity everywhere but improved the radiation-pressure noise in Fig.~\ref{fig:nIS_sigRO_tolerance_Tlc}. %Idler loss in idler readout decreases the anti-squeezing at the peak similarly for noise and signal\jam{(check)}, improves the radiation-pressure noise the same as signal readout\jam{(why?)}, decreases the DC signal response, and has a negligible effect on frequencies higher than the peak\jam{(quantify)}. When the idler readout rate is increased, the radiation-pressure noise and DC signal response decrease less\jam{(quantify, why?)}.
However, at realistic 1000~ppm idler loss, the effect on idler readout is negligible because it is dominated by the noise through the readout port, in same way that realistic signal loss is negligible for signal readout.

% but high arm loss will reduce circulating power, and the sensitivity will worsen overall (I do not include the affect of these losses on the circulating power, a low loss approximation)
Finally, similarly to signal readout, realistic \emph{arm intra-cavity loss} has a negligible effect on idler readout if the circulating power is fixed, e.g.\ increasing the arm loss a hundredfold affects the peak sensitivity by less than a factor of two. % If the arm loss is made unrealistically high, then the effect on the sensitivity is similar to the signal mode loss except that it diminishes with increased idler readout rate\jam{(why?)}, but I do not consider this behaviour.
% Finally, as shown in Fig.~\ref{fig:nIS_idlerRO_tolerance_Tla}, the arm intra-cavity loss affects idler readout somewhat similarly to the signal loss\jam{(why?)}. The radiation-pressure noise is amplified, the shot noise peak is broadened\jam{(but decreased?)}, and the signal response peak is decreased but its bandwidth is broadened, i.e.\ the signal response is improved for frequencies away from the peak. The result is that the sensitivity is decreased at the peak and at low frequencies below $\sim10$~Hz where the radiation-pressure noise dominates but is improved everywhere else\jam{(quantify change in integrated sensitivity?)}. However, unlike signal loss, the effect of arm loss decreases with increased idler readout rate\jam{(why?)} which is also true for signal readout with increased signal readout rate in Fig.~\ref{fig:nIS_sigRO_tolerance_Tla}. The signal peak frequency also changes with readout rate due to the change in singularity threshold frequency\jam{(check this, shouldn't the shot noise singularity now be at DC?)}. But otherwise, the idler readout's tolerance to arm loss is different to signal readout's, the radiation-pressure noise and signal responses are different and the change around $T_{l,a}=T_{l,c}$ is not as dramatic\jam{(why when threshold is the same? does this mean that that explanation for signal readout was wrong?)}. However, in either case, the effect of realistic arm loss is negligible\jam{(quantify)} and, in practice, any broadband benefit to idler readout with high arm loss is negated by the loss in circulating power. %\jam{(prove this or note that I do not include this effect)}.

\jam{(Have I shown more interpretation than just describing these plots?)}

% although the tolerance to the different losses is different to signal readout, which is allowed because of the differet mode structure, at the realistic level of loss the arm and corresponding SRC losses are negligible and the readout is dominated by the opposide mode in the SRC and the detection loss
In summary, idler readout is affected differently to signal readout by some of the losses, which is due to the different mode structure, i.e.\ which losses the noise and signal encounter on their way to each readout.
For realistic losses, arm and signal losses affect the noise negligibly compared to idler and detection loss. However, \emph{the signal loss has the dominant effect on the sensitivity} out of the losses because of its effect on the signal response as seen by comparing Fig.~\ref{fig:nIS_idlerRO_tolerance_Tlb} to Figs.~\ref{fig:nIS_idlerRO_tolerance_Rpd}~\ref{fig:nIS_idlerRO_tolerance_Tlc}. Like the signal readout, the dominant noise above 100~Hz remains the shot noise from the readout port rather than any of the losses.
\jam{(what does the optomechanical analogue with mechanical idler readout predict?)}
% Signal loss decreases the peak and radiation-pressure noise--limited sensitivity while broadening the bandwidth of idler readout, unlike idler loss which improves the radiation-pressure noise but worsens the rest of the sensitivity of signal readout.
% I will consider later whether increased signal loss could be beneficial for broadband detection using the idler readout.



\subsection{Variational idler readout and tolerance to pump phase}
\label{sec:idlerRO_pump_phase}
% stability of idler readout against pump power and phase: how sensitive is the scheme to ratio of threshold and $\phi$ (when $\psi$ combination angles fixed), give value in $\%$ and radians.

% \begin{figure}
% 	\centering
% 	\includegraphics[width=\textwidth]{nIS_idlerRO_tolerance_chi.pdf}
% 	\caption{\jam{(Purpose: compare squeezer parameter tolerance to signal readout)}\jam{(Is this necessary?)} Nondegenerate internal squeezing idler readout sensitivity tolerance to pump power (squeezer parameter $\chi$), varying the idler readout rate with the signal readout port closed. Decreasing the pump power improves the radiation-pressure noise slightly\jam{(quantify)}, decreases the sensitivity uniformly from around 50~Hz up to the peak frequency, and minimally affects the sensitivity above the peak frequency. This is because the squeezer increases the radiation-pressure noise, improves the sensitivity from around 50~Hz up to the peak frequency, and does not affect the signal or noise beyond the peak. The tolerance to pump power is independent of the idler readout rate with the signal readout port closed\jam{(why?)}, but with the signal readout port open the sensitivity change for the same change of ratio to threshold is larger but is localised to the peak frequency and decreases with increased idler readout rate, e.g.\ with 500~Hz signal readout a change of $0.04$ to threshold produces a greater sensitivity change than $0.15$ shown here around the peak\jam{(check, clarify, why?)}.
%     The tolerance with the signal readout port open is similar to signal readout, shown in Fig.~\ref{fig:nIS_sens_target}, where increased readout increases the tolerance\jam{(quantify and explain why?)}.}
% 	\label{fig:nIS_idlerRO_tolerance_chi}
% \end{figure}
\begin{figure}
    \centering
    \includegraphics[width=0.7\textwidth]{nIS_idlerRO_tolerance_phi.pdf}
    \caption{\jam{(Compare to signal readout's tolerance?)} Nondegenerate internal squeezing idler readout tolerance to the relative phase between the pump phase $\phi$ and idler combination angle $\psi_1$. Changing the relative phase affects the radiation-pressure noise, signal peak, and DC signal response. The signal response decreases because there is no signal response for a relative phase of $\pm\pi/2$ as discussed in Section~\ref{sec:nIS_idlerRO_model}. The tolerance is high to realistic variations in the relative phase (which are far smaller than $\pi/4$~\cite{Yap:19}). If the relative phase is fixed within $(-\pi/2,0)$, then the noise is squeezed between 1--20~Hz and the sensitivity is improved. If the idler readout rate is changed, then the position of the squeezing peak changes\jam{(why?)}. I use the parameter set in Tab.~\ref{tab:idler_RO_parameters}.}
    \label{fig:nIS_idlerRO_tolerance_phi}
\end{figure}

In Section~\ref{sec:nIS_idlerRO_model}, I defined the idler readout to optimise the signal response by matching the readout combination angle to the pump phase, e.g.\ $\psi_1-\phi=0$. Now, I consider the tolerance to changes in the relative phase $\psi_1-\phi$. %, where if the relative phase reaches $\pm\pi/2$ then the signal response vanishes as discussed in Section~\ref{sec:nIS_idlerRO_model}\jam{(verify this)}.
% I consider how tolerant the idler readout is to changes in the pump phase.
% Let the combination angles be fixed as $\psi_1=\psi_2=\pi/2$ and the pump phase varied away from $\pi/2$.
As shown in Fig.~\ref{fig:nIS_idlerRO_tolerance_phi}, a change of $\pi/4$ in the relative phase changes the sensitivity by at most a factor of two, and since $\pi/4$ is far larger than the realistic variation in the controlled relative phase~\cite{Yap:19}, I assume that the relative phase is fixed. %This tolerance is independent of the readout rates\jam{(check)}\jam{(why does the DC signal decrease with readout rate?)}.
The sign of the relative phase does not affect the sensitivity except at ``low'' frequencies where the noise for negative relative phases is squeezed\jam{(why does the sign matter?)}, e.g.\ $\psi_1-\phi=-\pi/4$ around 5~Hz in Fig.~\ref{fig:nIS_idlerRO_tolerance_phi}. This is because of correlations between the idler quadratures created by the radiation pressure interaction since the relevant off-diagonal terms in Eq.~\ref{eq:nIS_Sx} and the squeezing vanishing if the radiation pressure is turned off~\footnote{This is an example of ``ponderomotive'' squeezing where the optomechanical interaction at the test mass squeezes the reflected light and introduces correlations between the signal quadratures that appear in the idler via the squeezer~\cite{kimble_2001}. The optomechanical interaction produces squeezing because it couples the amplitude of the light, which affects the radiation pressure, to the propagation phase of the light acquired by the displacement of the mass.}. Similar squeezing occurs when the signal quadratures are combined\jam{(is there more to say?)}.
% The squeezing peak frequency changes with the idler readout rate\jam{(why?)}.
This squeezing could be used in a variational readout scheme~\cite{zhangBroadbandSignalRecycling2021} where the relative phase is made frequency-dependent to squeeze the radiation-pressure noise around 1--10~Hz using $\psi_1-\phi<0$ but use the optimal signal response $\psi_1-\phi=0$ above 10~Hz. Although at these frequencies below 100~Hz, quantum noise is not the dominant noise source for current gravitational-wave detectors~\cite{buikemaSensitivityPerformanceAdvanced2020}, it might limit future detectors as it is the fundamental noise floor and therefore this scheme is worth further examination.
To determine whether this variational readout scheme could be feasible for future detectors, the relative size of the other 1--100~Hz frequency noise sources would have to be considered, e.g.\ thermal, seismic, Newtonian, and control-system noise~\cite{buikemaSensitivityPerformanceAdvanced2020}.
This could form part of a broader study of the coherently combined readout scheme which I leave to future work.
% The first steps to understand this behaviour would be to (1) verify the result by plotting the idler-idler covariance $(\text{S}_X)_{3,4}$ for various pump phases to see when it is non-zero and to plot the idler readout's noise in the other combined idler quadrature with $\phi\mapsto\phi+\pi/2$\jam{(check)}, (2) plot for several more phase differences $\psi_1-\phi$ to see when the squeezing is present and maximal, and (3) experiment with the readout rates and other parameters to see what controls where the squeezing appears.

% mention what happens when idler readout rate is kept high and signal rate is varied, see nIS_ROs_tolerance_chi_large_idler_ROrate.pdf
% For changes in the pump power, i.e.\ the squeezer parameter $\chi$, the sensitivity is only affected significantly when the idler readout rate is small compared to the signal readout rate\jam{(check)}, e.g.\ a change of $4\%$ in the pump power\jam{(is this large?)} affects the sensitivity\jam{by ... when the readout rates are equal and by ... when the signal readout rate is zero (quantify using integrated sensitivity?)}.\jam{(check the effect of $\chi$)} %If the pump power varies 
% If the idler readout rate is fixed at 500~Hz and instead the signal readout rate is varied\jam{(over what range?)}, the effect is negligible\jam{(quantify, do I need to show another plot?)}.
% I do not know why low idler readout rate makes the system more sensitive to the squeezer parameter but it might be because less vacuum from the readout port enters the system and therefore more anti-squeezing can be produced\jam{(compare to signal readout, is this because of the lossless limit?)}.
% the squeezer increases the radiation-pressure noise, improves the sensitivity from around 50~Hz up to the peak frequency, and does not affect the signal or noise beyond the peak. The tolerance to pump power is independent of the idler readout rate with the signal readout port closed\jam{(why?)}, but with the signal readout port open the sensitivity change for the same change of ratio to threshold is larger but is localised to the peak frequency and decreases with increased idler readout rate, e.g.\ with 500~Hz signal readout a change of $0.04$ to threshold produces a greater sensitivity change than $0.15$ shown here around the peak\jam{(check, clarify, why?)}.
% idler readout transfer functions: noise indep of pump phase but noise dep on relative phase of pump to linear combination of idler quadratures
% check squeezing? MJ: variational readout can recover squeezing (different homodyne angle at different freq), RPN creates i-i covariances by itself as part of ponderomotive squeezing (from the arms?)?
% this is a potential/preliminary result
% optimise the readout angle? could do for signal as well --> future work

% Here's evidence that the squeezing seen in the combined idler quadratures is ponderomotive:
%     (Plot 1) Combined idler quadratures. If the radiation pressure is turned off, then the squeezing vanishes.
%     (Plot 2) Combined signal quadratures. If the squeezer is turned off, then squeezing is seen which disappears if the RP is turned off.
%     Checking the covariance matrix confirms that the s-s and i-i covariances vanish with rho=0.

% The latest idler readout plots:
% (Plot 1) Dashed lines are without radiation pressure. The result is that the signal transfer function improvement with idler readout persists at low frequencies but radiation pressures means that there's no sensitivity improvement. 
% (Plots 2-3) Zoom-in on peak frequency, which changes with the readout rate. This is because the singularity threshold frequency depends on the losses, although I normalise the squeezer parameter to the singularity threshold value I don't normalise the frequency axis. The singularity frequencies here are: {853, 1624, 3534, 4763} Hz (truncated at the decimal point). Plot 2 is for 95% of the squeezer parameter threshold; Plot 3 is on squeezer parameter threshold (where limited resolution means that the peak is finite). When the signal readout is turned off, the effect is smaller but non-zero.
% (Plots 4-5) Tolerance to pump power. Showing the effect of changing the ratio to threshold by plus/minus 4%. The effect appears larger at smaller idler readout rates with fixed signal readout rate (Plot 4) and negligible when idler readout is large and signal readout is varied (Plot 5). I suspect that this is due to the likeness to the lossless case with no idler readout (Li et al, 2020) where 100% threshold is dramatic compared to 90% threshold -- compared to at large readout rates where 90 and 100% threshold are largely the same. Why low signal readout does not produce the same effect is because it is further from that lossless configuration, but I am less confident about this.
% The regime that is most promising is for readout rates roughly equal, so this effect for low idler readout's is less of an issue, and it appears similar to the signal readout anyway. 
% (Plot 6) Tolerance to pump phase. Fixing the combination angle psi1=pi/2 and changing the pump phase phi from pi/2 by plus/minus pi/4. Plot 6 is labelled as "combined" but the angles are selecting idler readout only. The effect is that the sensitivity seems tolerant to pump phase, independently of the readout rates (which surprised me, recall that psi1=phi is optimal for signal but the change is small). 


\section{Idler readout for gravitational-wave detection}
\label{sec:idlerRO_feasibility}
% showcase the reasons for further interest in this configuration --> comparison to signal is sufficient?

% idler readout also looks quite promising but demands a lot more investigation before drawing strong conclusions, does it?, need to be critical of my methodology and approach

% \begin{figure}
% 	\centering
% 	\includegraphics[width=\textwidth]{nIS_idlerRO_ideal_losses.pdf}
% 	\caption{\jam{(Purpose: show what idler readout might achieve)}\jam{(Update description for new $\alpha$ scaling.)}\jam{(Shorten this caption and move information into text. Compare to signal readout plot. Why doesn't the peak move?)} Nondegenerate internal squeezing idler readout compared to kilohertz sensitivity target for ideal and realistic losses, different squeezer parameters, 500~Hz idler readout rate, 2~kHz sloshing frequency, and no signal readout rate. The target is not achieved at the peak frequency for realistic or ideal losses and $0.95$ threshold\jam{(check)}. However, it can achieved with 10~dB external squeezing\jam{(check def)} for realistic losses, with the signal readout port open, and 5~kHz sloshing frequency but only narrowly at the peak frequency. The sloshing frequency was changed to move the peak into 1--4~kHz\jam{(why is the peak at the sloshing frequency?)}. The idler readout rate could be lowered further but then the target would be achieved over an even smaller band\jam{(quantify)}. Introducing more signal loss decreases the peak and therefore does not help to achieve the target\jam{(check)}. The changes in squeezer parameter and losses do not affect the sensitivity as much as the signal readout in Fig.~\ref{fig:nIS_sens_target}\jam{(why? quantify)}.}
% 	\label{fig:nIS_idlerRO_ideal_losses}
% \end{figure}

I now consider the feasibility of using the idler readout for gravitational-wave detection. By Fig.~\ref{fig:nIS_signal_vs_idler_ROs}, the idler readout performs worse at kilohertz than the signal readout\jam{(why?)} and therefore I do not consider using the idler readout for improving kilohertz sensitivity. %~\footnote{To quantify the difference, even with the signal readout port closed, 10~dB of injected external squeezing, and realistic losses, idler readout can only achieve the target at a far narrower peak than signal readout, e.g.\ less than 10~Hz compared to around 500~Hz\jam{(check)}. Such Dirac delta--like sensitivity is not viable because the target astrophysical frequencies are not exact and the gravitational waves are predicted to change in frequency over time.}.
However, in my exploration of nondegenerate internal squeezing, the possibility of improving broadband sensitivity has emerged despite my initial motivation to improve kilohertz sensitivity. Signal readout is already promising for 0.1--4~kHz broadband detection as discussed in Section~\ref{sec:nIS_sigRO_feasibility}, but idler readout performs better from 10--1000~Hz than signal readout by Fig.~\ref{fig:nIS_signal_vs_idler_ROs}~\footnote{Another benefit of idler readout is that the frequency difference $\Delta$ can be chosen to match the highest quantum efficiency photodiodes available since $\Delta$ does not otherwise affect the sensitivity in Eq.~\ref{eq:nIS_Xpd}. This is promising because $2 \mu\text{m}$ signal readout currently has low quantum efficiency~\cite{singh_2019}.\jam{(quantify improvement)}}. %, where the signal loss from opening the signal readout port decreases the idler readout's peak but broadens its bandwidth as shown in Fig.~\ref{fig:nIS_idlerRO_tolerance_Tlb}.
An incoherently combined readout scheme could achieve the envelope of the signal and idler sensitivities such that the idler could be used at ``low'' frequencies (0.01--1~kHz) and the signal at ``high'' frequencies (1--4~kHz). %This beats the Mizuno limit because that limit applies separately to each measurement but not to the incoherent combination\jam{(check)}~\cite{}. %QCRB is about information contained regardless of readout?
In Fig.~\ref{fig:nIS_signal_vs_idler_ROs}, the incoherently combined readout has at least $2\times10^{-24}\text{Hz}^{-1/2}$ sensitivity from around $\sim$~80--2500~Hz\jam{(check and can I do better?)} which overcomes the loss associated with having both readout ports open~\footnote{This performs better than the idler readout separately with the signal readout port open.}. %~\footnote{By inspection, this smaller idler readout rate is less tolerant to changes in the pump power\jam{(why?)}, but this can be addressed by pump laser power stabilisation.}.
% although it does not achieve the kilohertz sensitivity target anywhere.\jam{(is there a problem with low readout rates?)}\jam{(experiment with low signal readout rate as well)}
Therefore, nondegenerate internal squeezing could feasibly use incoherently combined readout to better improve broadband 0.1--4~kHz\jam{(check range)} gravitational-wave detection. %, motivated by the same astrophysical applications as Section~\ref{sec:signalRO_broadband}. % and improving the existing 100~Hz sensitivity. %But, signal readout is more useful for kilohertz 1--4~kHz detection\jam{(is there a physical explanation why?)}.
Exploring coherently combined readout, including the variational readout suggested in Section~\ref{sec:idlerRO_pump_phase}, for the possibility of even greater improvement is left to future work and will be discussed in the next chapter. % with the ultimate goal of optimising the readout combination angles at each frequency against some astrophysical metric. --> this has been explained already 


% does opening the signal readout port improve the idler readout broadband detection application?
% But, for broadband sensitivity from 100-1000~Hz, idler readout does appear suitable and better than the signal readout\jam{(quantify performance)}, as shown in Fig.~\ref{fig:nIS_signal_vs_idler_ROs}, where the signal loss from opening the signal readout port decreases the peak but broadens the bandwidth as shown in Fig.~\ref{fig:nIS_idlerRO_tolerance_Tlb}. Since the signal readout was already promising for broadband detection, therefore the idler readout is worth considering for broadband detection. %, since it achieves better than $10^{-24}\text{Hz}^{-1/2}$ from 100--500~Hz with realistic loss and 50~Hz idler and 500~Hz signal readout rates\jam{(is there an astrophysical target for 100-1000~Hz?)}\jam{(can I do better?)}.
% suggestion of incoherent readout scheme
% An incoherent readout scheme could be viable that uses the envelope of the signal and idler sensitivity curves, e.g.\ with equal readout rates of 500~Hz it uses the idler below 4~kHz and the signal above that. This beats the Mizuno limit because that limit applies separately to each measurement but not to the incoherent combination\jam{(check)}~\cite{}. A potential problem with incoherently combined readout is that with both readout ports open both readout schemes experience a high amount of loss from the other port and I have shown that idler loss limits the signal readout. I will later address whether this prevents the incoherently combined readout from being viable at realistic loss levels. 

% Since the ideal idler readout rate is low, this configuration will be sensitive to the pump power, but this can be addressed by pump laser amplitude stabilisation~\cite{}. 
% But this is an issue worth over-coming\jam{(how? laser amplitude FCS?)} because of the possible broadband sensitivity. % from idler or incoherently combined readout. 
% the sensitivity target at $\sim$~2~kHz\jam{(check)} and still have sensitivity better than $10^{-24}\text{Hz}^{-1/2}$ at 100~Hz\jam{(check)}.
% Over the band, this readout has sensitivity better than $2\times10^{-24}\text{Hz}^{-1/2}$\jam{(can I do better?)} from $\sim$~90--3000~Hz which is a large bandwidth\jam{(quantify and watch out for logarithmic scale, do I need two values here, just give the latter)}
% Compared to the detector without squeezing, e.g. with 50~kHz signal readout rate in Fig.~\ref{fig:nIS_sens_target} which has good\jam{(quantify)} sensitivity around 100~Hz and would be a good candidate for a future detector\jam{(quantify and justify, cite parameter set?)}, the incoherently combined readout achieves a sensitivity below $2\times10^{-24}\text{Hz}^{-1/2}$\jam{(can I do better?)} from $\sim$~90--3000~Hz which is a large bandwidth\jam{(quantify, watch out for logarithmic scale)} compared to $\sim$~50--1000~Hz.
% The idler readout has large bandwidth here because the signal readout causes signal loss in the idler readout.

% nondegenerate is less sensitive to losses and could potentially use idler/combined readout to get ahead
% goal of detection is the same as before
% There are also a couple of other benefits to using the idler readout in this all-optical configuration: (1) potentially increased quantum efficiency by choosing the idler frequency and (2) the relative ease of measuring the optical idler mode compared to the mechanical idler mode of the optomechanical analogue.
% give a sense that this exploration is not yet finished but is quite promising
% combined (coherent) readout is even more interesting than swapping between signal and idler readout because the correlations can be used to potentially achieve even greater sensitivity
% Motivated by the performance of incoherently combined readout,
% , along with the rest of the configuration parameters of a more detailed model, 


% David Ottaway: the frequencies where the idler is improving low freq sensitivity might fit into the band of InGaAs photodetectors where the sensitivity at 2 um can be improved (currently this is a problem i.e low Quantum efficiency of photodetectors at 2um) - Vaishali
% In practice, the parameter set might be different for idler readout.
% In this model, I have assumed that the quantum efficiency of the photodetector~\footnote{Which measures how much photocurrent is created by an incident photon at a given frequency. Low quantum efficiency means that the photocurrent created is closer to the electronic and dark noise floors and the sensitivity of the detector is worse.\jam{(check)}} is frequency independent, but in practice, this is not true and the quantum efficiency for the signal at angular frequency $\omega_0$ and idler at $\omega_0+\Delta$ can be different. Since the frequency difference $\Delta$ does not appear in the result in Eq.~\ref{eq:nIS_Xpd}, because I use the Interaction Picture and assume $\Delta\ll\omega_0$\jam{(but why does it not matter physically?)}, the idler frequency can be chosen to best suit the photodetector technology, e.g.\ current photodiodes have low quantum efficiency at $2 \mu\text{m}$~\cite{}, so with the signal mode at $2 \mu\text{m}$ for the LIGO~Voyager parameter set that I use the idler wavelength could be chosen to optimise the readout. Given current technology, this could lead to a sensitivity improvement of\jam{... (find this out)}~\cite{} but I omit this from the results below\jam{(why?)}.

% To improve kilohertz sensitivity from 1--4~kHz, the idler readout does not appear well-suited, as shown in Fig.~\ref{fig:nIS_idlerRO_ideal_losses} with the signal readout port closed. These results are for 500~Hz idler readout rate, where increasing the readout rate reduces the peak\jam{(why? is this shown elsewhere?)}. The peak is narrow because the signal readout port is closed and the signal loss is low compared to Fig.~\ref{fig:nIS_idlerRO_tolerance_Tlb}. The peak sensitivity does not achieve the kilohertz target from Section~\ref{sec:nIS_kHz} for realistic or ideal\jam{(why change the arm loss given the noise breakdown?)} losses\jam{(quantify)}. I experimented with what parameters would be necessary to achieve the target and found that\jam{... would be required (find this out), which is more stringent than the ideal conditions required for signal readout in Section~\ref{sec:nIS_sigRO_feasibility}}. Introducing 10~dB external squeezing does achieve the target with realistic losses, however, it is over a narrower band than the signal readout\jam{(quantify)}. Therefore, compared to the signal readout in Fig.~\ref{fig:nIS_sens_target}, the idler readout's kilohertz sensitivity is not as promising because it is over less of the 1--4~kHz band and cannot be achieved without external squeezing for these parameters.



%%%%%%%%%%%%%%%%%%%%%%%%%%%%%%%%%%%%%%%%%%
\section{Chapter summary}
% I understand idler readout's threshold, stability, limits, tolerance to losses, and performance against readout rates.
% this chapter is an additional result: idler readout might be useful for broadband detection
% be explicit somewhere about the next steps for studying the combined readout --> done

In this chapter, I have explored how measuring the idler mode changes the performance of nondegenerate internal squeezing. %I have also discussed how the two readouts can be combined to achieve greater sensitivity
% During my exploration of this configuration, I have found some exciting
% Firstly, I reviewed idler readout and how the signal and idler at the photodetector can be combined coherently or incoherently. % and defined the idler readout to maximise the gravitational-wave signal response.
Firstly, I characterised the stability, threshold, high arm loss limit, and performance of idler readout, %I found that all of the readouts have the same stability and threshold and that they reduce to an OPO in the high arm loss limit.
and showed that idler readout, compared to signal readout, improves ``low'' frequency sensitivity (around 100~Hz) at the cost of ``high'' frequency sensitivity (around 1~kHz).
Then, I showed that idler readout is limited by signal loss followed by detection loss.  %I also showed that it is sensitive to pump power at low idler readout rates and to pump phase independently of readout rate.
Finally, for gravitational-wave detection, I showed that the most promising application of idler readout at present is to use incoherently combined signal and idler readouts for broadband detection from 0.1--4~kHz\jam{(check)}, possibly with variational readout of the idler readout to squeeze the radiation-pressure noise. % but that kilohertz sensitivity from 1--4~kHz is improved more by signal readout. 

\jam{(do I need to show more criticism of my methodology and approach in this chapter?)}


