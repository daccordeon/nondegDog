\chapter{Conclusions}
\label{chp:future_work_and_conclusions}
% return to aims: does nIS better improve kHz sensitivity? have I filled the gap in the literature?
% interpret results and bring together what contribution they give
% synthesis and critical analysis --> talk about your results beyond the plots, what do they mean?

In this thesis, I have studied nondegenerate internal squeezing to see if it could improve the kilohertz sensitivity of future gravitational-wave detectors.
% key results: I have characterised nIS with separate signal and idler readouts, particularly its optical loss tolerance and squeezing threshold; nIS is more tolerant to loss than dIS; nIS is a valid all-optical alternative to sWLC; nIS signal readout can achieve the kilohertz target at peak sensitivity but not across 1--4~kHz for LIGO~Voyager without external squeezing or increased circulating power; nIS could also be used for broadband 100--4000~Hz detection using incoherently combined signal and idler readouts
Firstly, I have modelled nondegenerate internal squeezing using an analytic, Hamiltonian method used widely in the literature, and I have characterised its different readouts, squeezing threshold, stability, optical loss tolerance, and high and low loss limits. \jam{(should I go into any of these, e.g.\ which loss dominates which readout?)} Secondly, I have compared its feasibility with realistic optical losses to existing proposals to improve kilohertz sensitivity and found that it is an all-optical alternative to stable optomechanical filtering and a more loss-resistant alternative to degenerate internal squeezing. Finally, I have shown that using signal readout can achieve the astrophysical, kilohertz sensitivity target at peak sensitivity but not across the 1--4~kHz band for a detector like LIGO~Voyager without external squeezing or increased circulating power. And I have shown that using incoherently combined signal and idler readouts can achieve broadband 100--4000~Hz \jam{(check range)} sensitivity improvement.
% interpretation: nIS is a loss-resistant, all-optical configuration that can improve interferometer sensitivity without increased circulating power. it is most suitable for improving shot noise--limited sensitivity from 100--4000~Hz, including kilohertz sensitivity from 1--4~kHz.
In summary, I have found that nondegenerate internal squeezing is a loss-resistant, all-optical configuration that improves shot noise--limited sensitivity and could be used for broadband (100--4000~Hz) or kilohertz (1--4~kHz) gravitational-wave detection.

% contribution (aim 2: gap in the literature):  the only understanding of nIS in the literature is through the optomechanical analogue, and although the behaviour is largely the same, I have characterised it separately with loss in every mode. I contribute its threshold, limits, comparison to existing proposals including the optomechanical analogue, and feasibility for gravitational-wave detection (where the different states of optical and mechanical loss technology matter).  but this goes beyond GW-detection to general cavity-based metrology as well
This thesis contributes a full \jam{(see limitations?)} characterisation of nondegenerate internal squeezing that was previously understood only through the optomechanical analogue, stable optomechanical filtering. Although the lossless Hamiltonians of these systems are equivalent, the feasibility of using an optical versus a mechanical idler mode, respectively, is different. I have found that the optical loss required to match the low mechanical loss is at least as realistic if not more so, and combining signal and idler readout is likely easier using the all-optical configuration. Moreover, I have characterised elements of the lossy configuration that were not previously understood, such as the squeezing threshold using my notion of singularity threshold and the different response to each optical loss.
% by improving kHz sensitivity, astrophysics is improved...
By improving the possible sensitivity for future gravitational-wave detectors, this work is part of the effort to detect and study new astrophysical sources, e.g.\ the detection of kilohertz gravitational-waves from the post-merger remnant of a binary neutron-star merger might lead to further constraining the neutron star equation-of-state and a better understanding of exotic states of matter.
The results in this thesis also apply beyond gravitational-wave detection to general cavity-based metrology and describe a configuration that is potentially useful for other applications such as axion detection.


% results are promising but not explosive
% sensitivity target is hard to achieve broadband at current power (and without external squeezing as well etc.), maybe possible for a Voyage-like detector to achieve it at some peak frequency
% nIS is resistant to detection and signal losses, but vulnerable to idler loss (unless we exploit a combined readout and use the leakage through the SRM?)
% nIS certainly warrants more attention as an all-optical alternative to sWLC for future detectors + there is a lot of benefit to looking into combined readout
% singularity threshold definition works well here -- can we directly show that it corresponds to the gain=loss threshold, also pump depletion models exist to test this


\section{Limitations}
% is your approach good or to be avoided?
% be very critical!

% only quantum noise

% sensitivity target conditions on EoS

% trawl through the thesis and find all approximations used, do a big round up of them and why each might be problematic

% lack of physical understanding, e.g.\ behaviour of signal versus idler readout --> perhaps a more abstract approach could see past the complications of all of the interferometer parameters to address the underlying structure, e.g.\ study the normal modes of the three coupled oscillators

% single-mode and resolved sideband (where used?) approximations, see Li2021


\section{Future work}
\label{sec:future_work}

% understanding singularity threshold

% study other applications thoroughly, e.g.\ axion detection

% understanding idler readout and particularly combined readout
% optimum readout (e.g.\ Weiner filter for integrated sensitivity from 10 Hz to 10 kHz)
% tunability between idler and signal readouts, frequency-dependent combination/readout angles

% add external squeezing ? caves's amplifier, see if it can be achieved 1--4~kHz

% Finesse numerical modelling, for verification and extension past limitations above -- e.g. multimode, detuning, etc.. numerical modelling of pump depletion?

% \subsubsection*{Back-action evasion in non-degenerate internal squeezing} Is it easier for sWLC?

% Carl Bender: investigate the exceptional point of lossless sWLC by checking the bounded-from-below condition on the Hamiltonian transformed into continuous co-ordinate (quasi-quadrature?) space as to whether it is accidental or true, if true then the square-root singularity is what is enhancing sensitivity. Moreover, the optical system, if above threshold is also modelled, could be used to experimentally demonstrate the change in the g.s. energy as it is adiabatically transported onto the other sheets (see Carl's example), which would be very exciting to witness. Alternatively, could check the type of exceptional point by varying the coupling constants and calculating the eigenvalues and see if they just cross on the real axis but stay on it. --> cite personal communication

% idler sloshing frequency

