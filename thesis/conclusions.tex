\chapter{Conclusions}
\label{chp:future_work_and_conclusions}
% return to aims: does nIS better improve kHz sensitivity? have I filled the gap in the literature?
% interpret results and bring together what contribution they give
% synthesis and critical analysis --> talk about your results beyond the plots, what do they mean?

In this thesis, I have studied nondegenerate internal squeezing to see if it could improve the kilohertz sensitivity of future gravitational-wave detectors.
% key results: I have characterised nIS with separate signal and idler readouts, particularly its optical loss tolerance and squeezing threshold; nIS is more tolerant to loss than dIS; nIS is a valid all-optical alternative to sWLC; nIS signal readout can achieve the kilohertz target at peak sensitivity but not across 1--4~kHz for LIGO~Voyager without external squeezing or increased circulating power; nIS could also be used for broadband 100--4000~Hz detection using incoherently combined signal and idler readouts
Firstly, I have modelled nondegenerate internal squeezing using an analytic, Hamiltonian method used widely in the literature, and I have characterised its different readouts, squeezing threshold, stability, optical loss tolerance, and high and low loss limits. \jam{(should I go into any of these, e.g.\ which loss dominates which readout?)} Secondly, I have compared its feasibility with realistic optical losses to existing proposals to improve kilohertz sensitivity and found that it is an all-optical alternative to stable optomechanical filtering and is comparably if not more loss-resistant than degenerate internal squeezing. Finally, I have shown that using signal readout can achieve the kilohertz sensitivity target for a typical post-merger signal from a binary neutron-star merger at peak sensitivity but not across the 1--4~kHz band, where I calculated the sensitivity for a detector like LIGO~Voyager without external squeezing or increased circulating power. I have also shown that using incoherently combined signal and idler readouts can improve broadband 100--4000~Hz \jam{(check range)} sensitivity \jam{(should I mention increased radiation-pressure noise?)}.
% interpretation: nIS is a loss-resistant, all-optical configuration that can improve interferometer sensitivity without increased circulating power. it is most suitable for improving shot noise--limited sensitivity from 100--4000~Hz, including kilohertz sensitivity from 1--4~kHz.
In summary, I have found that nondegenerate internal squeezing is a loss-resistant, all-optical configuration that improves shot noise--limited sensitivity and could be used for broadband (100--4000~Hz) or kilohertz (1--4~kHz) gravitational-wave detection.

% contribution (aim 2: gap in the literature):  the only understanding of nIS in the literature is through the optomechanical analogue, and although the behaviour is largely the same, I have characterised it separately with loss in every mode. I contribute its threshold, limits, comparison to existing proposals including the optomechanical analogue, and feasibility for gravitational-wave detection (where the different states of optical and mechanical loss technology matter).  but this goes beyond GW-detection to general cavity-based metrology as well
This thesis contributes a characterisation of nondegenerate internal squeezing that was previously understood only through the optomechanical analogue, stable optomechanical filtering. Although the lossless Hamiltonians of these two systems are equivalent, the feasibility of using an optical versus a mechanical idler mode, respectively, is different. I have found that the optical loss required to match the low mechanical loss assumed for stable optomechanical filtering is at least as realistic and that combining signal and idler readouts is likely easier using the all-optical configuration. Moreover, I have characterised elements of the lossy configuration that were not previously understood, e.g.\ the squeezing threshold and the response to each optical loss.
% by improving kHz sensitivity, astrophysics is improved...
By improving the possible sensitivity for future gravitational-wave detectors, this work is part of the effort to detect and study new astrophysical sources, e.g.\ the detection of kilohertz gravitational-waves from the post-merger remnant of a binary neutron-star merger might lead to further constraining the neutron star equation-of-state and a better understanding of exotic states of matter.
The results in this thesis also apply beyond gravitational-wave detection to general cavity-based metrology and describe a configuration that is potentially useful for other applications, e.g.\ axion detection to constrain theories of dark matter~\cite{}.

% results are promising but not explosive
% sensitivity target is hard to achieve broadband at current power (and without external squeezing as well etc.), maybe possible for a Voyage-like detector to achieve it at some peak frequency
% nIS is resistant to detection and signal losses, but vulnerable to idler loss (unless we exploit a combined readout and use the leakage through the SRM?)
% nIS certainly warrants more attention as an all-optical alternative to sWLC for future detectors + there is a lot of benefit to looking into combined readout
% singularity threshold definition works well here -- can we directly show that it corresponds to the gain=loss threshold, also pump depletion models exist to test this


\section{Limitations of this work}
% is your approach good or to be avoided? --> the Hamiltonian method is robust and good, the question is whether my assumptions are good
% be very critical!

\jam{(Add in stability assumption for singularity threshold: that the only thing that causes the system to become unstable is the amplifier reaching threshold.)}
The work in this thesis is limited from making stronger claims about the best configuration for future detectors by the simplifying assumptions used. Although the Hamiltonian method I used is widely used in the literature and therefore is the approach I recommend for any future work, some of the assumptions that I have made are not generally valid.
Firstly, some of my assumptions are generally valid, e.g.\ assuming uncorrelated vacuum and $\Delta\ll\omega_0$, or ignore phenomena that affect Michelson interferometers uniformly and therefore do not distinguish between configurations, e.g.\ assuming that circulating power cannot increase and that any incident gravitational wave is polarised to achieve the maximal signal response. These assumptions should be maintained in future work.
Secondly, some of my assumptions limit the frequency range in which the estimated sensitivity is accurate such as to where quantum noise is the dominant noise source, to above where \jam{(what frequency?)} the horizontally free-falling mass assumption breaks, and below the arm cavity free-spectral range of $37.5$~kHz to maintain the single-mode approximation \jam{(and the resolved sideband approximation?)}. These assumptions are to be appropriately avoided in any future work that wants to predict the behaviour of nondegenerate internal squeezing outside the 100--4000~Hz frequency range that I study. Similarly, the no-pump-depletion and semi-classical pump approximations should be dropped to predict the behaviour above threshold and more accurately predict it at high ratios to threshold, e.g.\ at $95\%$ threshold. A pump depletion model could also replace the singularity threshold definition with a more physical one related directly to gains and losses. 
Finally, some of my assumptions about what is feasible for a future detector are not guaranteed given the unknowns of future technological progress and astrophysical understanding. This includes what optical and mechanical losses are realistic and what sensitivity is required to detect the gravitational-wave sources that are yet to be detected. These assumptions are hard to improve upon and should be updated with the best understanding of the time.
In summary, the results that I have shown are valid within the parameter range stated but not otherwise, and any future work should remove the appropriate assumptions that limit extending the results.
\jam{(have I been critical enough? are there other limitations that I've missed that aren't explicit assumptions?)}

\jam{(do I need to say this?)}
A problem with my approach that is more difficult to address is the lack of physical intuition for why the sensitivity of a given configuration behaves the way it does. Beyond explaining the general features of the signal and the quantum noise, which I have done, the model provides little explanation for why, for example, the idler readout is better at low frequencies than the signal readout \jam{(check)}. I do not believe that there is an alternative method that would make these facts more evident, but this might limit the amount of physics to be gained directly from further modelling.

% trawl through the thesis and find all approximations used, do a big round up of them and why each might be problematic
	% uncorrelated vacuum
	% Delta << omega_0
	% circulating power of 3 MW for LIGO Voyager
	% maximal detector response to GW polarisation

	% only quantum noise
	% horizontally free falling mass
	% semi-classical, no pump depletion
	% single-mode (assume IFO tuned, PDC process modes are resonant) and resolved sideband (where used?) approximations, see Li2021

	% symmetric detection loss --> not worth mentioning?
	% no added optical loss by adding squeezer,

% singularity threshold is an unconvincing workaround that could be replaced with a pump-depletion model
% sensitivity target conditions on EoS
% lack of physical understanding, e.g.\ behaviour of signal versus idler readout --> perhaps a more abstract approach could see past the complications of all of the interferometer parameters to address the underlying structure, e.g.\ study the normal modes of the three coupled oscillators


\section{Future work}
\label{sec:future_work}

The results in this thesis are promising and suggest that nondegenerate internal squeezing warrants further consideration. There are three main avenues for future work studying (1) miscellaneous extensions of the system beyond the above limitations, (2) coherently combined signal and idler readout, and (3) potential experiment of the PT-symmetry.

% study other applications thoroughly, e.g.\ axion detection
% add external squeezing ? caves's amplifier, see if it can be achieved 1--4~kHz
% Finesse numerical modelling, for verification and extension past limitations above -- e.g. multimode, detuning, etc.. numerical modelling of pump depletion?
Firstly, to better judge the feasibility of nondegenerate internal squeezing for gravitational-wave detection, the above limitations should be addressed. This could include adding the effects of external squeezing (injected~\cite{} and Caves's amplification~\cite{}), pump depletion~\cite{}, multiple cavity modes~\cite{liEnhancingInterferometerSensitivity2021,}, detuning~\cite{}, or other noise sources than quantum noise such as thermal noise at the test masses~\cite{}. These additions would give a more realistic estimate of the performance of a future detector. Other necessary additions might also arise when studying other applications than gravitational-wave detection \jam{(an example of what needs to be considered for axion detection?)}. These additions might necessitate changing to numerical modelling in an optics simulation tool such as Finesse~\cite{}, which could also be used for verification of the present results \jam{(do I need to explain why the current results are only analytic)}. Then, the sensitivity could be optimised to discover the best parameters for a more realistic future detector than what I have presented and determine whether such a detector could meet the astrophysical sensitivity targets, e.g.\ achieve the kilohertz target from 1--4~kHz.
% understanding singularity threshold
% \subsubsection*{Back-action evasion in non-degenerate internal squeezing} Is it easier for sWLC?
% idler sloshing frequency
There are also other possible extensions such as using pump depletion to verify singularity threshold, coupling the arm mode to a negative-effective-mass mechanical oscillator to compare to the back-action evasion studied for the optomechanical analogue~\cite{}, and coupling the idler to the arm mode to examine if nondegenerate then becomes degenerate internal squeezing using coherently combined readout.

% understanding combined readout
% optimum readout (e.g.\ Weiner filter for integrated sensitivity from 10 Hz to 10 kHz)
% tunability between idler and signal readouts, frequency-dependent combination/readout angles
Secondly, coherently combined readout might lead to better sensitivity than signal, idler, and incoherently combined readouts because of the addition of the signal-idler correlations which I have derived but not studied. For example, the correlations can produce squeezing as seen in Fig.~\ref{fig:nIS_idlerRO_tolerance_phi} \jam{(discuss phase variation more)}. As explained in Section~\ref{sec:nIS_idlerRO_reduction_to_OPOs}, coherently combined readout should be studied using a similar process as the idler readout in Chapter~\ref{chp:idler_readout}. Additionally, once the tolerance of the readout scheme to losses and the combination angles is understood, the sensitivity should be optimised by changing the combination angles at each frequency, forming a Weiner filter for the sensitivity~\cite{}. Since the choice of angles can recover signal or idler readout, the Weiner filter is as least as sensitive as either readout separately or optimally incoherently combined. % The coherently combined readout cannot generally recover incoherently combined readout, however, the optimal incoherently combined readout is the envelope of the separate readouts, which coherently combined readout can recover \jam{(check)}.
Therefore, coherently combined readout should be studied to find the best sensitivity that nondegenerate internal squeezing can achieve.

% Carl Bender: investigate the exceptional point of lossless sWLC by checking the bounded-from-below condition on the Hamiltonian transformed into continuous co-ordinate (quasi-quadrature?) space as to whether it is accidental or true, if true then the square-root singularity is what is enhancing sensitivity. Moreover, the optical system, if above threshold is also modelled, could be used to experimentally demonstrate the change in the g.s. energy as it is adiabatically transported onto the other sheets (see Carl's example), which would be very exciting to witness. Alternatively, could check the type of exceptional point by varying the coupling constants and calculating the eigenvalues and see if they just cross on the real axis but stay on it. --> cite personal communication
% requires pump-depletion. even if PT-symmetry is true in lossless case, then it is unclear that it would be possible to show it experimentally because losses make the system open and not Hermitian nor PT-symmetric
Finally, the PT-symmetry of lossless nondegenerate internal squeezing at threshold could be used to demonstrate a theoretical result from PT-symmetry theory~\cite{CMBPersonalCommunication}. This requires that the Exceptional Point at threshold, mentioned in Section~\ref{sec:sWLC}, is not accidental~\cite{}, i.e.\ that the degenerate real eigenvalue pair at threshold becomes a complex eigenvalue pair above threshold. I have not shown this because it would require a pump-depletion model~\cite{}. It would also mean that the Hamiltonian is not Hermitian above threshold which is a situation studied in PT-symmetry theory~\cite{}. To determine if this is the case, either the boundedness-from-below of the Hamiltonian could be checked~\cite{} or the eigenvalues could be explicitly calculated~\cite{}. Supposing that this is the case, which might not be true, then there is a branch point in the complex eigenvalue plane at the Exceptional Point and PT-symmetry theory predicts that the ground state energy on the other Riemann sheet(s) that the branch cut connects to could be different to the starting sheet~\cite{}. This could be tested experimentally with signal readout by starting with the ground state of the system with the squeezer off, adiabatically transporting it above threshold, across the branch cut onto the second sheet, and back below threshold to where the squeezer is off again. Then, comparing the ground state energy before and after the process would test the PT-symmetry theory. I do not know whether PT-symmetry theory can still predict something about the lossy system where the losses break the PT-symmetry \jam{(check)}, but it is worth further investigation.
\jam{(does this make any sense?)}

% final remark?
This is only some of the potential future research motivated by the promising results about nondegenerate internal squeezing in this thesis.
\jam{(I wanted a final remark, will this do?)}

