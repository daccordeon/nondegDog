\chapter{Conclusions}
\label{chp:future_work_and_conclusions}
% return to aims
% is your approach good or to be avoided?
% synthesis and critical analysis --> talk about your results beyond the plots, what do they mean?



% results are promising but not explosive <-- don't say this, hype the results!
% sensitivity target is hard to achieve broadband at current power (and without external squeezing as well etc.), maybe possible for a Voyage-like detector to achieve it at some peak frequency
% nIS is resistant to detection and signal losses, but vulnerable to idler loss (unless we exploit a combined readout and use the leakage through the SRM?)
% nIS certainly warrants more attention as an all-optical alternative to sWLC for future detectors + there is a lot of benefit to looking into combined readout
% singularity threshold definition works well here -- can we directly show that it corresponds to the gain=loss threshold, also pump depletion models exist to test this



\section{Limitations}
% be very critical!

% sensitivity target conditions on EoS

% trawl through the thesis and find all approximations used, do a big round up of them and why each might be problematic

% lack of physical understanding, e.g.\ behaviour of signal versus idler readout --> perhaps a more abstract approach could see past the complications of all of the interferometer parameters to address the underlying structure, e.g.\ study the normal modes of the three coupled oscillators


\section{Future work}
\label{sec:future_work}

% understanding singularity threshold

% the techniques of detection are ultimately not limited to gravitational waves. broader quantum metrology can benefit

% understanding idler readout and particularly combined readout
% optimum readout (e.g.\ Weiner filter for integrated sensitivity from 10 Hz to 10 kHz)
% tunability between idler and signal readouts, frequency-dependent combination/readout angles

% add external squeezing ? caves's amplifier

% Finesse numerical modelling, for verification and extension past limitations above -- e.g. multimode, detuning, etc.. numerical modelling of pump depletion?

% \subsubsection*{Back-action evasion in non-degenerate internal squeezing} Is it easier for sWLC?

% Carl Bender: investigate the exceptional point of lossless sWLC by checking the bounded-from-below condition on the Hamiltonian transformed into continuous co-ordinate (quasi-quadrature?) space as to whether it is accidental or true, if true then the square-root singularity is what is enhancing sensitivity. Moreover, the optical system, if above threshold is also modelled, could be used to experimentally demonstrate the change in the g.s. energy as it is adiabatically transported onto the other sheets (see Carl's example), which would be very exciting to witness. Alternatively, could check the type of exceptional point by varying the coupling constants and calculating the eigenvalues and see if they just cross on the real axis but stay on it. --> cite personal communication

% idler sloshing frequency

