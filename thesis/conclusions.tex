\chapter{Conclusions and future work}
\label{chp:future_work_and_conclusions}
% return to aims: does nIS better improve kHz sensitivity? have I filled the gap in the literature?
% interpret results and bring together what contribution they give
% synthesis and critical analysis --> talk about your results beyond the plots, what do they mean?

In this thesis, I have investigated nondegenerate internal squeezing from the perspectives of general quantum metrology	and gravitational-wave detection.
% studied nondegenerate internal squeezing to see if it could improve the kilohertz sensitivity of future gravitational-wave detectors.
% key results: I have characterised nIS with separate signal and idler readouts, particularly its optical loss tolerance and squeezing threshold; nIS is more tolerant to loss than dIS; nIS is a valid all-optical alternative to sWLC; nIS signal readout can achieve the kilohertz target at peak sensitivity but not across 1--4~kHz for LIGO~Voyager without external squeezing or increased circulating power; nIS could also be used for broadband 100--4000~Hz detection using incoherently combined signal and idler readouts
Firstly, I have developed a model of nondegenerate internal squeezing using an analytic, Hamiltonian method that incorporates optical losses. I validated this model by showing that it reduces to the correct high and low loss limits. I have shown that the configuration is stable and calculated its squeezing threshold. I compared the different possible readout schemes and showed that signal readout is limited by idler loss and idler readout is limited by signal loss.
Secondly, using my model, I have evaluated the feasibility of nondegenerate internal squeezing for gravitational-wave detection in comparison to two existing proposals. I have found that it is a viable all-optical alternative to stable optomechanical filtering and that it is more resistant than degenerate internal squeezing to detection loss. %\jam{(comparably if not more?)}
Finally, I have shown that nondegenerate internal squeezing using signal readout can feasibly improve kilohertz sensitivity, e.g.\ to detect a kilohertz gravitational wave from the remnant of a binary neutron-star merger, without increasing circulating power. %, where I calculated the sensitivity for a detector like LIGO~Voyager without external squeezing or increased circulating power.
I have also shown that using incoherently combined signal and idler readouts can improve broadband 0.1--4~kHz sensitivity. %\jam{(should I mention increased radiation-pressure noise?)}.
% interpretation: nIS is a loss-resistant, all-optical configuration that can improve interferometer sensitivity without increased circulating power. it is most suitable for improving shot noise--limited sensitivity from 100--4000~Hz, including kilohertz sensitivity from 1--4~kHz.
In summary, I have found that nondegenerate internal squeezing is a detection loss-resistant configuration that improves quantum noise--limited sensitivity and could be used for kilohertz (1--4~kHz) or broadband (0.1--4~Hz) gravitational-wave detection as an all-optical alternative to existing proposals.

% contribution (aim 2: gap in the literature):  the only understanding of nIS in the literature is through the optomechanical analogue, and although the behaviour is largely the same, I have characterised it separately with loss in every mode. I contribute its threshold, limits, comparison to existing proposals including the optomechanical analogue, and feasibility for gravitational-wave detection (where the different states of optical and mechanical loss technology matter).  but this goes beyond GW-detection to general cavity-based metrology as well
This thesis \emph{characterises nondegenerate internal squeezing} which was previously understood only in the lossless case and by analogy to stable optomechanical filtering. Although the Hamiltonians of nondegenerate internal squeezing and stable optomechanical filtering are theoretically equivalent under a certain mapping of optical to mechanical modes, I have shown that, in practice, the feasibility of each configuration is different and that the loss requirements of nondegenerate internal squeezing are at least as realistic as stable optomechanical filtering. %Specifically, I have found that the optical loss required to match the low mechanical loss assumed for stable optomechanical filtering is at least as realistic as that low mechanical loss. %Also, combining signal and idler readouts is likely easier using nondegenerate internal squeezing. 
Moreover, I have characterised aspects of nondegenerate internal squeezing that were not present in the literature to date but are essential to a thorough understanding, e.g.\ the squeezing threshold and the tolerance to the different sources of optical loss.
% by improving kHz sensitivity, astrophysics is improved...

% results are promising but not explosive
% sensitivity target is hard to achieve broadband at current power (and without external squeezing as well etc.), maybe possible for a Voyage-like detector to achieve it at some peak frequency
% nIS is resistant to detection and signal losses, but vulnerable to idler loss (unless we exploit a combined readout and use the leakage through the SRM?)
% nIS certainly warrants more attention as an all-optical alternative to sWLC for future detectors + there is a lot of benefit to looking into combined readout
% singularity threshold definition works well here -- can we directly show that it corresponds to the gain=loss threshold, also pump depletion models exist to test this


\section{Future work}
\label{sec:future_work}

% \jam{(Search through thesis to check if all mentions of future work show up here.)}

The results in this thesis indicate several possible avenues of future research into nondegenerate internal squeezing. %Four main directions are (1) extensions of the model, (2) coherently combined readout, (3) connections to PT-symmetry theory, and (4) an experimental demonstration of the results.

% \subsubsection{Extensions of the model} %\section{Limitations of this work}
% is your approach good or to be avoided? --> the Hamiltonian method is robust and good, the question is whether my assumptions are good
% be very critical!

% \jam{(have I been critical enough? are there other limitations that I've missed that aren't explicit assumptions?)}
The \emph{model in this thesis could be extended} to make stronger claims about the best configuration for future detectors by removing some of the simplifying assumptions used.
% Although the Hamiltonian method I use is widely used in the literature and therefore is the approach I recommend for future work,
% Some of the assumptions that I have made in the model  are not generally valid.
% Firstly, some of my assumptions are generally valid, e.g.\ assuming uncorrelated vacuum and $\Delta\ll\omega_0$, or ignore phenomena that affect Michelson interferometers uniformly and therefore do not distinguish between configurations, e.g.\ assuming that circulating power cannot increase and that any incident gravitational wave is polarised to achieve the maximal signal response. These assumptions should be maintained in future work.
Some of my assumptions limit the frequency range in which the estimated sensitivity is accurate such as to below the arm cavity free spectral range of $37.5$~kHz to maintain the single-mode approximation~\footnote{Other such assumptions include assuming that quantum noise is the dominant noise source and that the test masses are horizontally free-falling which restrict the frequencies to be above 100~Hz\jam{(give frequencies?)}.}. These assumptions could be avoided to predict the behaviour of nondegenerate internal squeezing outside the 0.1--4~kHz\jam{(check)} frequency range that I study, e.g.\ by using a multi-mode model~\cite{liEnhancingInterferometerSensitivity2021}.
Similarly, the semi-classical pump and no-pump-depletion approximations should be dropped to predict the behaviour above threshold and more accurately at 95--$100\%$ threshold~\cite{walls_1995}. A pump depletion model could also validate my singularity threshold technique. % with a more physical one related directly to gains and losses. 
There are other possible extensions to enrich understanding such as a more thorough stability analysis~\cite{liEnhancingInterferometerSensitivity2021}, verifying the high loss limit using a transfer matrix method (e.g.\ in Ref.~\cite{korobkoQuantumExpanderGravitationalwave2019}), and checking if nondegenerate internal squeezing with coherently combined readout reduces to degenerate internal squeezing when the idler is coupled to the arm mode.

When these extended models are used to judge the \emph{feasibility of future detectors}, my assumptions about what losses are realistic and what sensitivity is required should be examined given the unknowns of future technological and astrophysical progress. These assumptions are hard to improve upon and should be updated with the best understanding of the time.
% In summary, the results that I have shown are valid within the parameter range stated but not otherwise, and future work should remove the appropriate assumptions that limit extending the results.
% To better judge the feasibility of nondegenerate internal squeezing for gravitational-wave detection, the above improvements could be made along with adding the effects of external squeezing, detuning~\cite{}, and other noise sources than quantum noise such as thermal noise at the test masses~\cite{}. %These additions would give a more realistic estimate of the performance, including the sensitivity, stability, and threshold, of a future detector. 
The extensions to the model might necessitate using numerical modelling in an optics simulation tool such as \code{FINESSE}~\cite{finesse}~\footnote{This would require the addition of a nondegenerate internal squeezer component similar to the degenerate internal squeezer component (\code{nle}) currently available on the \code{PyKat} developer branch but otherwise could be modelled using the standard library~\cite{BROWN_PYKAT}.\jam{(does Finesse have this component already?)}}. This could also be used for validation of the analytic model of nondegenerate internal squeezing, however, it is not a priority since I have already partially validated my model by showing that it reduces to the correct limits. % ``Finesse~\cite{}, which CAN(?) also be used'' -MJ
Using either an analytic or numerical model, the sensitivity could then be optimised to discover the best parameters for a realistic future detector and determine whether such a detector could meet the astrophysical sensitivity targets. %, e.g.\ achieve the kilohertz target from 1--4~kHz.

%\jam{(do I need to say this?)} A problem with my approach that is more difficult to address is the lack of physical intuition for why the sensitivity of a given configuration behaves the way it does. Beyond explaining the general features of the signal and the quantum noise, which I have done, the model provides little explanation for why, for example, the idler readout is better at low frequencies than the signal readout\jam{(check)}. I do not believe that there is an alternative method that would make these facts more evident, but this might limit the amount of physics to be gained directly from further modelling.

% trawl through the thesis and find all approximations used, do a big round up of them and why each might be problematic
	% uncorrelated vacuum
	% Delta << omega_0
	% circulating power of 3 MW for LIGO Voyager
	% maximal detector response to GW polarisation

	% only quantum noise
	% horizontally free falling mass
	% semi-classical, no pump depletion
	% single-mode (assume IFO tuned, PDC process modes are resonant) and resolved sideband (where used?) approximations, see Li2021

	% symmetric detection loss --> not worth mentioning?
	% no added optical loss by adding squeezer,

% singularity threshold is an unconvincing workaround that could be replaced with a pump-depletion model
% sensitivity target conditions on EoS
% lack of physical understanding, e.g.\ behaviour of signal versus idler readout --> perhaps a more abstract approach could see past the complications of all of the interferometer parameters to address the underlying structure, e.g.\ study the normal modes of the three coupled oscillators

% study other applications thoroughly, e.g.\ axion detection
% add external squeezing ? caves's amplifier, see if it can be achieved 1--4~kHz
% Finesse numerical modelling, for verification and extension past limitations above -- e.g. multimode, detuning, etc.. numerical modelling of pump depletion?
%Other necessary additions might also arise when studying other applications than gravitational-wave detection\jam{(an example of what needs to be considered for axion detection?)}. 
% understanding singularity threshold
% \subsubsection*{Back-action evasion in non-degenerate internal squeezing} Is it easier for sWLC?
% idler sloshing frequency
%\jam{(Add in stability assumption for singularity threshold: that the only thing that causes the system to become unstable is the amplifier reaching threshold.)}

% \subsubsection{Coherently combined readout}

% understanding combined readout
% optimum readout (e.g.\ Wiener filter for integrated sensitivity from 10 Hz to 10 kHz)
% tunability between idler and signal readouts, frequency-dependent combination/readout angles
\emph{Coherently combined readout} (as defined in Section~\ref{sec:nIS_idlerRO_model}) might lead to better sensitivity than the other readout schemes because of the signal-idler correlations which I have derived but not studied. This is motivated by promising results for the coherently combined readout of stable optomechanical filtering~\cite{liEnhancingInterferometerSensitivity2021}. To characterise the coherently combined readout, I would first examine the effects of realistic optical loss, the readout rates, pump power, pump phase, and the readout angles $(\psi_0,\psi_1,\psi_2)$ on the signal and noise responses. Then, I would find a Wiener filter for the sensitivity, i.e.\ the readout angles to maximise the sensitivity at each frequency. Since the choice of angles can recover signal or idler readout, this Wiener filter would be at least as sensitive as either readout separately or incoherently combined~\footnote{Variational readout restricted to only combining the signal quadratures could determine the optimum signal readout and, similarly, the optimum idler readout.}. Finally, I would compare this optimum sensitivity to the astrophysical targets to judge the feasibility of gravitational-wave detection.
 % The coherently combined readout cannot generally recover incoherently combined readout, however, the optimal incoherently combined readout is the envelope of the separate readouts, which coherently combined readout can recover\jam{(check)}.
In summary, coherently combined readout should be studied to find the best possible sensitivity using nondegenerate internal squeezing.
% Similarly, for the existing readout schemes, the parameters of the configuration 

% \subsubsection{Connection to PT-symmetry theory}

% Carl Bender: investigate the exceptional point of lossless sWLC by checking the bounded-from-below condition on the Hamiltonian transformed into continuous co-ordinate (quasi-quadrature?) space as to whether it is accidental or true, if true then the square-root singularity is what is enhancing sensitivity. Moreover, the optical system, if above threshold is also modelled, could be used to experimentally demonstrate the change in the g.s. energy as it is adiabatically transported onto the other sheets (see Carl's example), which would be very exciting to witness. Alternatively, could check the type of exceptional point by varying the coupling constants and calculating the eigenvalues and see if they just cross on the real axis but stay on it. --> cite personal communication
% requires pump-depletion. even if PT-symmetry is true in lossless case, then it is unclear that it would be possible to show it experimentally because losses make the system open and not Hermitian nor PT-symmetric
The \emph{PT-symmetry} of lossless nondegenerate internal squeezing at threshold could also be further investigated~\cite{CMBPersonalCommunication}. In particular, it could be checked whether the PT-symmetry is responsible for the enhanced sensitivity.
% used to demonstrate a theoretical result from PT-symmetry theory
PT-symmetry theory only predicts there to be enhanced sensitivity if the Exceptional Point, see Section~\ref{sec:sWLC}, is not accidental, i.e.\ that the degenerate real eigenvalue pair at threshold becomes a complex eigenvalue pair above threshold~\cite{liu2016metrology,hodaei2017enhanced}. This has not been shown and would mean that the Hamiltonian is not Hermitian above threshold which is a situation studied in PT-symmetry theory~\cite{el2018non}. To determine if this is the case using a pump-depletion model, either the boundedness-from-below of the Hamiltonian could be checked or the eigenvalues could be explicitly calculated.
PT-symmetry could also be investigated in the lossy case, where it is expected to break, and with radiation pressure included (as in Ref.~\cite{liBroadbandSensitivityImprovement2020} for the optomechanical analogue). %, where it is necessary to couple a negative--effective mass mechanical oscillator to the arm mode to maintain PT-symmetry by comparison to the back-action evasion studied for the optomechanical analogue~\cite{liBroadbandSensitivityImprovement2020}.

% Supposing that this is the case, which might not be true, then there is a branch point in the complex eigenvalue plane at the Exceptional Point and PT-symmetry theory predicts that the ground state energy on the other Riemann sheet(s) that the branch cut connects to could be different to the starting sheet~\cite{}.
% This could be tested experimentally with signal readout by starting with the ground state of the system with the squeezer off, adiabatically transporting it above threshold, across the branch cut onto the second sheet, and back below threshold to where the squeezer is off again. Then, comparing the ground state energy before and after the process would test the PT-symmetry theory. I do not know whether PT-symmetry theory can still predict something about the lossy system where the losses break the PT-symmetry\jam{(check)}, but it is worth further investigation.\jam{(does this make any sense?)}

% \subsubsection{Table-top experimental demonstration}

Finally, the model in this thesis could be used to design an \emph{experiment to demonstrate nondegenerate internal squeezing} and enhanced interferometer sensitivity. This would require using the parameters and losses realistic to a contemporary, table-top squeezing experiment (e.g.\ Ref.~\cite{sudbeck2020demonstration}). % rather than a future gravitational-wave detector.


\subsubsection{Final word}
The future work suggested above is only some of the potential research motivated by the promising results (paper in preparation) about nondegenerate internal squeezing in this thesis.
By improving the possible sensitivity for future gravitational-wave detectors, this thesis is part of the effort to detect and study new astrophysical sources, e.g.\ detecting kilohertz gravitational-waves from the remnant of a binary neutron-star merger might lead to better understanding exotic states of matter.
The results in this thesis also apply beyond gravitational-wave detection to general quantum metrology and potentially using nondegenerate internal squeezing in other applications, e.g.\ axion detection to constrain theories of dark matter~\cite{MARSH20161,PhysRevX.9.021023,liBroadbandSensitivityImprovement2020}.


