\chapter{Conclusions}
\label{chp:future_work_and_conclusions}
% return to aims




% results are promising but not explosive <-- don't say this, hype the results!
% sensitivity target is hard to achieve broadband at current power (and without external squeezing as well etc.), maybe possible for a Voyage-like detector to achieve it at some peak frequency
% nIS is resistant to detection and signal losses, but vulnerable to idler loss (unless we exploit a combined readout and use the leakage through the SRM?)
% nIS certainly warrants more attention as an all-optical alternative to sWLC for future detectors + there is a lot of benefit to looking into combined readout
% singularity threshold definition works well here -- can we directly show that it corresponds to the gain=loss threshold, also pump depletion models exist to test this



\section{Limitations}
% limitations

% sensitivity target conditions on EoS

% trawl through the thesis and find all approximations used, do a big round up of them and why each might be problematic


\section{Future work}

% the techniques of detection are ultimately not limited to gravitational waves. broader quantum metrology can benefit

% optimum readout

% add external squeezing ? caves's amplifier

\subsubsection*{Back-action evasion in non-degenerate internal squeezing}



