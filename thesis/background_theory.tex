\chapter{Background theory: quantum noise in interferometric gravitational-wave detectors}

% this chapter has a lot of ideas, need to figure out the logical flow -- look at research proposal and other theses --> talk to supes

% do I need to start as far back as MJ's thesis? i.e. back at the quantisation of light, definition of quadratures etc.?

%%%%%%%%%%%%%%%%%%%%%%%%%%%%%%%%%%%%%%%%%%
\section{Quantum noise}

% HUP

\subsection{Optical loss}
%  source of loss, realistic values (now and in a few decades time, 100 years time etc.)

% \subsubsection{Technological limit: optical loss}


\section{Squeezing}


% \section{Squeezed cavity (OPO) - analytic model}

Although this derivation is widely available in the literature~\cite{}, I include it here to demonstrate the Hamiltonian method that I use throughout the rest of the thesis. \jam{Should this be included in an appendix?}


\subsection{Degenerate OPO}

% OPO = squeezed cavity with no seed light

\subsubsection{Threshold}


\subsection{Nondegenerate OPO}


\subsubsection{Threshold and idler loss}


\subsection{External squeezing in interferometric gravitational-wave detectors}

% Generally speaking, the vacuum entering the main port of an arbitrary quantum noise--limited detector can be squeezed externally and injected via a Faraday isolator (a directional beam-splitter) to lower the quantum noise.

\subsubsection{Technological limit: pump power} 
% i.e. threshold

%%%%%%%%%%%%%%%%%%%%%%%%%%%%%%%%%%%%%%%%%%
\section{Chapter summary}

Beyond better external squeezing, there are many \jam{(are there?)} existing proposals to improve the kilohertz sensitivity of gravitational-wave detectors. In the next two chapters, I examine two of the front-runners: degenerate internal squeezing and stable optomechanical filtering.
\jam{(justify somewhere that these are the front-runners and that there are not other configurations that I should be considering -- literature review!)}

