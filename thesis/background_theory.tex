\chapter{Background theory of quantum noise in interferometric gravitational-wave detectors} % Background: theory of quantum noise in interferometric gravitational-wave detectors
\label{chp:background_theory}

% set up EVERYTHING the audience needs to know
% set up the problem sufficiently and specifically, the problems with current proposals are detailed in the next chapter

% this chapter has a lot of ideas, need to figure out the logical flow -- look at research proposal and other theses --> talk to supes

% do I need to start as far back as MJ's thesis? i.e. back at the quantisation of light, definition of quadratures etc.?
% supervisors say no for honours thesis

%%%%%%%%%%%%%%%%%%%%%%%%%%%%%%%%%%%%%%%%%%
% chapter introduction

In this chapter, I will review the necessary physics to describe the quantum noise that affects interferometric gravitational-wave detectors. I will use the conventional treatment of quantum noise widely available in the quantum optics literature~\cite{Danilishin,MiaoQCRB}, which builds from the Heisenberg Uncertainty Principle to the quantum noise response of a detector. 
I will then introduce the technique of squeezing, which can favourably manipulate the quantum noise. Then, demonstrate the Hamiltonian modelling of squeezing that I will continue to use throughout this thesis. I will derive the response of the well-studied optical parametric oscillator (OPO) in both the degenerate and nondegenerate cases. These models will be expanded upon by adding additional modes in later chapters, so it is well-worth carefully understanding the base model. Finally, I will explain how squeezing is currently used in Advanced LIGO to lower the quantum noise~\cite{} and set-up the motivation for the more complex uses of squeezing in the next chapter and in my work.

% I will first review the background theory of quantum noise and squeezing in Chapter~\ref{chp:background_theory}, to have the tools to describe the quantum noise response of a detector. In this chapter, I will demonstrate the analytic, Hamiltonian modelling, that I will then use throughout the thesis, on the simple, well-known cases of degenerate and nondegenerate optical parametric oscillators (OPOs, i.e.\ squeezed cavities)~\cite{}. 

%%%%%%%%%%%%%%%%%%%%%%%%%%%%%%%%%%%%%%%%%%
\section{Quantum noise}
\label{sec:qnoise}
% HUP

Quantum noise, at its most general, refers to any source of noise caused by the quantum-mechanical, fundamental uncertainties in the state of a detector. This covers many different phenomena, from vacuum fluctuations to back-action caused by a measurement~\cite{}. For the light in a cavity-based detector, the amplitude and phase of a particular mode are conjugate quantities related by the Heisenberg Uncertainty Principle, meaning that they can never be simultaneously, exactly known. This uncertainty in the amplitude and phase of the light in different parts of the detector will appear as quantum noise in the final measurement. %Throughout this thesis, I will just focus on uncertainty in the light incident on the photodetector and not consider the complications of readout. % oddly placed sentence

To express these uncertainties formally, I use the Heisenberg Picture (and later the Interaction Picture) used throughout the literature~\cite{Danilishin,}. Let the time-domain creation and annihilation operators of a particular cavity mode (e.g.\ the resonant mode in the single-mode approximation) be $\hat{a}$ and $\hat{a}^\dag$, respectively, where the time dependence is implicit, these operators obey $[\hat{a},\hat{a}^\dag]=1$ and $\hat{a}\lvert\emptyset\rangle=0$ where $\lvert\emptyset\rangle$ is the vacuum state. Let the time-domain quadrature of this mode be $\hat{X}_\theta=\frac{e^{-i \theta}\hat{a}+e^{i \theta}\hat{a}}{\sqrt{2}}$ and let $\hat{X}_1=\hat{X}_{\theta=0},\; \hat{X}_2=\hat{X}_{\theta=\frac{\pi}{2}}$ such that $[\hat{X}_1,\hat{X}_2]=i$. Therefore, the Heisenberg Uncertainty Principle for the amplitude quadrature $\hat{X}_1$ and phase quadrature $\hat{X}_2$ states that $\sigma_{X_1}\sigma_{X_2}\geq\frac{1}{2}\abs{\ev{[\hat{X}_1,\hat{X}_2]}}=\frac{1}{2}$ where the uncertainty is $\sigma_\mathcal{O}=\sqrt{\ev{\hat{\mathcal{O}}^2}-\ev{\hat{\mathcal{O}}}^2}$ and $\ev{\hat{\mathcal{O}}}=\langle\emptyset\lvert\hat{\mathcal{O}}\rvert\emptyset\rangle$ is the vacuum expectation value~\cite{}. That these quadratures are related to the amplitude and the phase of the light, respectively, matters only to understanding but can indeed be shown~\cite{Danilishin}. More importantly, these time-domain quadratures are Hermitian and therefore observable~\footnote{Which is what is meant by ``measuring the phase'' of the light. Even though absolute phase is not real, the phase quadrature is observable.} which is used to ultimately measure the gravitational-wave signal. Their related uncertainties, however, mean that any measurement that depends on both of the quadratures must contend with there always being some quantum noise present. This is the case for gravitational-wave interferometry: the quantum noise can be reduced but never eliminated. 
As an aside, the exact method of readout of the quadratures is not of concern to this thesis, but it suffices to say that a readout scheme like homodyne readout will suffice for all applications shown here~\cite{}.

It is worth also understanding the Fourier-domain counterparts to the quadratures, since these contain spectral information about the gravitational-wave signal and Fourier transforms are the primary method of solving Heisenberg equations-of-motion throughout this thesis. Let $\hat{\mathcal{O}}(\Omega) = \int_{-\infty}^\infty \frac{\mathrm{d}t}{\sqrt{2}} \hat{\mathcal{O}} e^{-i\Omega t}$ where the time-dependence is implicit in $\hat{\mathcal{O}}$. Then, let $\hat{X}_\theta(\Omega)=\frac{e^{-i \theta}\hat{a}(\Omega)+e^{i \theta}\hat{a}^\dag(-\Omega)}{\sqrt{2}}$ where careful attention should be made to the sign of the last argument which was flipped in order to accommodate the $e^{-i\Omega t}$ in the Fourier transform. While the time-domain quadratures are Hermitian, $\hat{X}^\dag=\hat{X}$, the Fourier-domain quadratures are not and instead obey $\hat{X}^\dag(\Omega)=\hat{X}(-\Omega)$. Although not Hermitian, these Fourier-domain quantities are indirectly observable either through their time-domain counterparts or through measurements that can derive their real and imaginary parts, ultimately made possible due to the condition $\hat{X}^\dag(\Omega)=\hat{X}(-\Omega)$~\cite{sourcecitedinReid}. This detail will not be considered further and I will refer to these quadratures as being observable. 
In the Fourier domain, the time-domain variances $V_\mathcal{O}=\sigma_\mathcal{O}^2$ are replaced by (single-sided) spectral densities $S_\mathcal{O}(\Omega)\delta(\Omega-\Omega')=\ev{\hat{\mathcal{O}}(\Omega)\circ\hat{\mathcal{O}}^\dag(\Omega')}$ where $A\circ B=\dfrac{1}{2}(A\cdot B+B\cdot A)$. For $\hat{a}(\Omega)$ corresponding to an uncorrelated vacuum state, let $\ev{\hat{X}_i(\Omega)\circ\hat{X}_j^\dag(\Omega')}=\delta_{i,j}\delta(\Omega-\Omega')$ where $\delta_{i,j}$ is the Kronecker delta, and therefore $S_X^\text{vac}=1$ can be shown from the commutation relations above~\cite{Danilishin}. Finally, there is a subtlety in these spectral density expressions in that I have assumed that the quadrature operators are corresponding to the time-domain quantum fluctuations of the amplitude and phase of the light around its time-independent classical expectation value, i.e.\ I am really considering Fourier transforms of $\delta\hat{X}(t)=\hat{X}(t)-\ev{\hat{X}}$ where I have made the time-dependences explicit for clarity. However, I will leave the $\delta$'s implicit to compactify notation.


	% At high frequencies, the sensitivity of current detectors is limited by quantum noise arising from the quantum nature of light~\cite{danilishinQuantumMeasurementTheory2012}. By the Heisenberg Uncertainty Principle, the amplitude and phase of a quantum state of light (including the vacuum) cannot be exactly known at the same time. Uncertainty in the amplitude of the light within the interferometer leads to quantum radiation pressure noise, while uncertainty in its phase leads to so-called “shot noise”. 

\subsection{Optical loss}
%  source of loss, realistic values (now and in a few decades time, 100 years time etc.)

\begin{figure}
	\centering
	% \includegraphics[width=\textwidth]{}
	\caption{Beamsplitter model of loss.}
	\label{fig:beamsplitter_loss}
\end{figure}

% So, quantum noise is just caused by uncertainties in the state of detector
Quantum noise from fundamental uncertainty the light of a cavity-based detector can be thought of as entering the detector through every open port and lossy optic~\cite{}. So far, the explanation introduced above is that quantum noise is caused by fundamental uncertainties in the state of the detector, so what does the previous statement mean? % too colloquial?
The key insight is that any state of light in the detector can be written in terms of transformations of the vacuum state, and so any uncertainty in the state of the light can be expressed as vacuum fluctuations (corresponding to $S_X^\text{vac}=1$ above) viewed under such transformations~\cite{}. Therefore, the source of the quantum noise is ultimately the vacuum, which can be thought to surround the detector and enter whenever the system is opened to its surroundings~\footnote{This language is potentially confusing, because the vacuum does not travel and therefore can not enter, but it is shorthand for considering how the system interacts with the surrounding fields -- how light can leak out of and into the detector.}. An open port is the term for such a point where light can enter and leave the detector, such as at the photodetector where the detector is necessarily always open~\cite{}. This is worth emphasising, measurement requires the detector to be coupled to the environment and guarantees a port for the vacuum -- this notion of quantum noise is not well-defined for a closed system. Soon, I will express this notion mathematically, but first it needs to be explained why lossy optics should also be a source of quantum noise.

% fluctuation-dissipation theorem
Optical loss refers to the energy lost from a light field upon propagation or interaction with an optic, e.g. if $90\%$ of the power incident on a mirror is reflected but only $5\%$ transmitted, then the remaining $5\%$ of the power has been lost to the environment. The chief mechanism for optical loss is dissipation into the thermal bath, i.e.\ heating of the mirror~\cite{}. The Fluctuation-Dissipation Theorem states that any such dissipation into the thermal bath is accompanied by the introduction of uncorrelated noise into the light field~\cite{}. As such, the loss mechanism in an optic can be modelled by a beamsplitter that releases energy into the environment and creates a new open port for vacuum to enter through, as shown in Fig.~\ref{fig:beamsplitter_loss}. In this Beamsplitter Model of Optical Loss, all optics are perfect (i.e.\ the power incident equals the sum of the power reflected and transmitted) but additional open ports are added throughout the detector -- so-called loss ports. It suffices to have a single loss port in every cavity, i.e.\ for every mode, since the uncorrelated vacuum from multiple sources of loss sum in quadrature and is equivalent to a single vacuum source from a lossier port -- which should be clearer later. %is or are?
This means that any other open ports in the detector, besides any used for measurement, can also be rolled into the loss ports. Therefore, the quantum noise in a cavity-based detector comes from vacuum entering at the measurement device and from loss ports in every mode.

Using this notion of quantum noise, I can write down the quantum noise response of a cavity-based detector. %how arbitrary is this?
Let the quadrature measured at the photodetector (PD) of an arbitrary detector be $\hat{X}_\text{PD}(\Omega)$. This will be determined by the noise quadratures $\hat{X}_i^\text{vac}(\Omega)$ of the vacuum input at each open port and source of optical loss indexed by $i$, the signal $\tilde{h}(\Omega)$, and the detector's linear response to each of these, expressed by the noise $R_i(\Omega)$ and signal $T(\Omega)$ transfer functions (where the frequency dependence will often be implicit henceforth): $$\hat X_\text{PD}(\Omega)=\sum_i R_i \hat X_i^\text{vac}(\Omega) + T \tilde h(\Omega).$$
These transfer functions capture the behaviour of the detector, including the impact on the optical quantum noise by coupling to non-optical modes such as the mechanical mode of a suspended mirror -- more on this later. 
The quantum noise measured by the detector is then simply $\hat X_\text{PD}(\Omega)|_{\tilde h=0}$, i.e.\ the output with the signal turned off. This quantum noise can be characterised by its spectral density, $S_X$, in $\text{Hz}^{-1/2}$ which can be simplified assuming uncorrelated vacuum inputs (which is reasonable for all purposes in this thesis~\cite{}) to the sum of squares: $$S_X(\Omega)=\sum_i \abs{R_i}^2(\Omega).$$ 
Since the transfer functions $R_{i_1}, R_{i_2}$ for two loss ports in the same cavity are the same up to the respective loss rates $\sqrt\gamma_{i_1}, \sqrt\gamma_{i_2}$, the two ports can be collapsed to a single loss port with loss rate $\sqrt{\gamma_{i_1}+\gamma_{i_2}}$ since $$\abs{R_{i_1}}^2+\abs{R_{i_2}}^2=\abs{\sqrt{\gamma_{i_1}}R}^2+\abs{\sqrt{\gamma_{i_2}}R}^2=\abs{\sqrt{\gamma_{i_1}+\gamma_{i_2}}R}^2.$$ This explains the claim made before that only one loss port is needed for each mode; an explanation of the loss rates $\gamma$ will be given later in the next section.

% \subsubsection{Technological limit: optical loss}, just give some example values


\subsection{Quantum noise response of a interferometric gravitational-wave detector}

This understanding of quantum noise can be instantiated for a interferometric detector. 
For an interferometer, the position of the test mass has a fundamental uncertainty of its own, but this uncertainty is fully captured through the amplitude uncertainty of the light and the radiation pressure coupling to the test mass~\cite{}. As mentioned above, the quantum noise transfer function of the detector accounts for the addition of the mechanical mode through solving for its impact on the optical mode. 
This impact can be understood physically: the amplitude of the arm cavity mode (i.e.\ its amplitude quadrature) affects the radiation pressure on the test masses, quantum fluctuations in the amplitude quadrature lead to uncertainty in the position of the test masses and therefore quantum radiation pressure noise in the measurement of the gravitational-wave signal~\cite{Danilishin?}. Similarly, uncertainty in the phase of the light (its phase quadrature) incident on a photodetector affects the counting statistics of the photodetector and therefore leads to uncertainty in the intensity and ultimately so-called ``shot noise'' in the measurement of the signal~\cite{}.  \jam{(What is a counting statistic, explain this better)} %\jam{(phase isn't real, clarify this)} phase quadrature is measurable
And so, (1) for an interferometer, the final measurement is affected by uncertainties in both quadratures of the light, but at different points in the detector (this will be clearer when I give a formal model in the next chapter), and (2) the quantum noise can be divided into the effects of shot noise, associated with the photodetector and the loss ports of the detector, and radiation pressure noise, associated with the test mass mechanical mode. \jam{(Clarify the association of loss ports with shot noise)}


% Similarly, the positions and momenta of the test masses in a gravitational-wave detector are conjugate quantities. This introduces strange consequences of detection, e.g.\ as the position measurement of a test mass improves, its momentum becomes increasingly uncertain which couples through radiation pressure to the arm cavity mode~\cite{}. \jam{(follow up on this, what does this cause?)} 
% These consequences place limits on the quantum noise--limited sensitivity, i.e.\ the signal to quantum noise ratio, of a detector that cannot be surpassed by classical means which will be explained further later.


% . These limits, e.g.\ the Standard Quantum Limit and Quantum Cramer-Rao Bound~\cite{},
%SQL

% QCRB


	% At high frequencies, shot noise is the dominant source of quantum noise and limits the sensitivity of current detectors. Therefore, reducing shot noise at high frequencies is the principal way to improve high-frequency sensitivity.





\section{Squeezing}

% just a paragraph about sidebands, I do not go further into this because I do not use the formalism (preferring the operator+Interaction Picture which achieves the same effect of studying frequency offsets)


	% An established technology to reduce shot noise at high frequencies is squeezing~\cite{danilishinQuantumMeasurementTheory2012,chuaQuantumEnhancementKm2015}. Squeezing manipulates the quantum state of light and the vacuum to trade-off phase uncertainty for amplitude uncertainty; this decreases shot noise by increasing quantum radiation pressure noise~\footnote{Squeezing only improves sensitivity if the interferometer readout is suitably designed to exploit the trade-off between shot noise and radiation pressure noise.}. Squeezed states of light can be created in a crystal with nonlinear polarisability by parametric down-conversion which annihilates a photon at a pump frequency and creates two entangled photons with ``squeezed" uncertainties. To conserve energy, the sum of the frequencies of the created photons must equal the pump frequency. This process is degenerate if the frequencies of the created photons are equal and is nondegenerate otherwise.
	% The fact that nondegenerate squeezing involves three distinct frequencies (the pump and the two created frequencies) rather than two is an important change in the symmetry of the system that I will revisit later.
	% An example application of squeezing to reduce shot noise is the injection of squeezed vacuum from an external, degenerate squeezer which is used to approximately halve the shot noise in current detectors~\cite{tseQuantumEnhancedAdvancedLIGO2019}. The squeezed vacuum is injected behind the signal-recycling mirror via a Faraday isolator placed outside the signal-recycling cavity as shown in the left panel of Fig.~\ref{fig:coupled_cavities}.



\subsection{Degenerate OPO}
% analytic model to demonstrate Hamiltonian method

Although this derivation is widely available in the literature~\cite{}, I include it here to demonstrate the Hamiltonian method that I use throughout the rest of the thesis. \jam{Should this be included in an appendix?}

% OPO = squeezed cavity with no seed light

% clarify measurement of non-Hermitian X(Omega) and normal operators, set up covariance 

\subsubsection{Threshold}


\subsection{Nondegenerate OPO}
% succicnt analytic model, to show how to calculate covariances

\subsubsection{Threshold and idler loss}

\subsubsection{Reality of covariance measurements}

\subsubsection{Recovering squeezing}

\subsection{External squeezing in interferometric gravitational-wave detectors}

% Generally speaking, the vacuum entering the main port of an arbitrary quantum noise--limited detector can be squeezed externally and injected via a Faraday isolator (a directional beam-splitter) to lower the quantum noise.

\subsubsection{Technological limit: pump power} 
% i.e. threshold

%%%%%%%%%%%%%%%%%%%%%%%%%%%%%%%%%%%%%%%%%%
\section{Chapter summary}



