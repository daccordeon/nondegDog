\chapter{Background theory of quantum noise in interferometric gravitational-wave detectors} % Background: theory of quantum noise in interferometric gravitational-wave detectors
\label{chp:background_theory}

% set up EVERYTHING the audience needs to know
% set up the problem sufficiently and specifically, the problems with current proposals are detailed in the next chapter

% this chapter has a lot of ideas, need to figure out the logical flow -- look at research proposal and other theses --> talk to supes

% do I need to start as far back as MJ's thesis? i.e. back at the quantisation of light, definition of quadratures etc.? --> supervisors say no for honours thesis

%%%%%%%%%%%%%%%%%%%%%%%%%%%%%%%%%%%%%%%%%%
% chapter introduction

\jam{(Fix section and subsection hierarchy)}

In this chapter, I will review the necessary physics to describe the quantum noise that affects interferometric gravitational-wave detectors. I will use the conventional treatment of quantum noise widely available in the quantum optics literature~\cite{Danilishin,MiaoQCRB}, which builds from the Heisenberg Uncertainty Principle to the quantum noise response of a detector. 
I will then introduce the technique of squeezing, which can favourably manipulate the quantum noise, using a Hamiltonian model that I will continue to use throughout this thesis. I will derive the response of the well-studied optical parametric oscillator (OPO) in both the degenerate and nondegenerate cases. These models will be expanded upon by adding additional modes in later chapters, so it is well-worth carefully understanding the base model. Finally, I will explain how squeezing is currently used in Advanced LIGO to lower the quantum noise~\cite{} and set up the motivation for the more complex uses of squeezing in the next chapter and my work.

% I will first review the background theory of quantum noise and squeezing in Chapter~\ref{chp:background_theory}, to have the tools to describe the quantum noise response of a detector. In this chapter, I will demonstrate the analytic, Hamiltonian modelling, that I will then use throughout the thesis, on the simple, well-known cases of degenerate and nondegenerate optical parametric oscillators (OPOs, i.e.\ squeezed cavities)~\cite{}. 

%%%%%%%%%%%%%%%%%%%%%%%%%%%%%%%%%%%%%%%%%%
\section{Quantum noise}
\label{sec:qnoise}
% HUP

Quantum noise, at its most general, refers to any source of noise caused by the quantum-mechanical, fundamental uncertainties in the state of a detector. This covers many different phenomena, from vacuum fluctuations to back-action caused by a measurement~\cite{}. For the light in a cavity-based detector, the amplitude and phase of a particular mode are conjugate quantities related by the Heisenberg Uncertainty Principle, meaning that they can never be simultaneously, exactly known~\cite{}. This uncertainty in the amplitude and phase of the light in different parts of the detector will appear as quantum noise in the final measurement. %Throughout this thesis, I will just focus on uncertainty in the light incident on the photodetector and not consider the complications of readout. % oddly placed sentence

To express these uncertainties formally, I use the Heisenberg Picture used throughout the literature~\cite{Danilishin,}. Let the time-domain creation and annihilation operators of a particular cavity mode (the resonant mode in the single-mode approximation) be $\hat{a}$ and $\hat{a}^\dag$, respectively, where the time dependence is implicit. These operators obey $[\hat{a},\hat{a}^\dag]=1$ and $\hat{a}\lvert\emptyset\rangle=0$ where $\lvert\emptyset\rangle$ is the vacuum state. Given a Hamiltonian $\hat H$, these Heisenberg operators evolve according to the Heisenberg equation-of-motion $\dot{\hat{a}}=-\frac{i}{\hbar} [\hat a, \hat H]$ and its conjugate equation, where $\hbar$ is the reduced Plank constant. Let the time-domain quadrature of this mode be $\hat{X}_\theta=\frac{e^{-i \theta}\hat{a}+e^{i \theta}\hat{a}}{\sqrt 2}$ and let $\hat{X}_1=\hat{X}_{\theta=0},\; \hat{X}_2=\hat{X}_{\theta=\frac{\pi}{2}}$ such that $[\hat{X}_1,\hat{X}_2]=i$. Therefore, the Heisenberg Uncertainty Principle for the amplitude quadrature $\hat{X}_1$ and phase quadrature $\hat{X}_2$ states~\cite{} that $\sigma_{X_1}\sigma_{X_2}\geq\frac{1}{2}\abs{\ev{[\hat{X}_1,\hat{X}_2]}}=\frac{1}{2}$ \jam{(check this, does it disagree with the vacuum value being 1?)} where the uncertainty is $\sigma_\mathcal{O}=\sqrt{\ev{\hat{\mathcal{O}}^2}-\ev{\hat{\mathcal{O}}}^2}$ and $\ev{\hat{\mathcal{O}}}=\langle\emptyset\lvert\hat{\mathcal{O}}\rvert\emptyset\rangle$ is the vacuum expectation value. These quadratures are only loosely related to the amplitude and the phase of the light, respectively~\cite{Danilishin}. More importantly, these time-domain quadratures are Hermitian and therefore observable~\footnote{Which is what is meant by ``measuring the phase'' of the light. Even though absolute phase is not real, the phase quadrature is observable.} which is used to ultimately measure the gravitational-wave signal. Their related uncertainties, however, mean that any measurement that depends on both quadratures must contend with there always being some quantum noise present. This is the case for gravitational-wave interferometry; the quantum noise can be reduced but never eliminated.
The exact method of detection of the quadratures is not of concern to this thesis, but it suffices to say that a scheme like homodyne readout will suffice for all applications that I consider~\cite{}.

The Fourier-domain counterparts to the quadratures matter since they contain spectral information about the gravitational-wave signal~\footnote{Transient gravitational-wave signals have durations on the order of seconds. The response of an interferometer is on the time-scale of the round-trip time of the arms, which is on the order of $10~\mu\text{s}$. Therefore, the steady-state approximation to the interferometer, required for the Fourier transform, is justified.}. % and Fourier transforms are the primary method of solving Heisenberg equations-of-motion throughout this thesis. 
Let $\hat{\mathcal{O}}(\Omega) = \int_{-\infty}^\infty \frac{\mathrm{d}t}{\sqrt{2}} \hat{\mathcal{O}} e^{-i\Omega t}$ where the time-dependence is implicit in $\hat{\mathcal{O}}$ unless the argument $\Omega$ is shown. Then, let $\hat{X}_\theta(\Omega)=\frac{e^{-i \theta}\hat{a}(\Omega)+e^{i \theta}\hat{a}^\dag(-\Omega)}{\sqrt{2}}$ where the sign of the last argument was flipped to account for the $e^{-i\Omega t}$ in the Fourier transform. While the time-domain quadratures are Hermitian, $\hat{X}^\dag=\hat{X}$, the Fourier-domain quadratures are not and instead obey $\hat{X}^\dag(\Omega)=\hat{X}(-\Omega)$. Although not Hermitian, these Fourier-domain quantities are indirectly observable, either through their time-domain counterparts or through measurements that can derive their real and imaginary parts, made possible due to the ``reality'' condition $\hat{X}^\dag(\Omega)=\hat{X}(-\Omega)$~\cite{sourcecitedinReid}. For the sake of brevity, I will refer to these quadratures as being observable henceforth. 
In the Fourier domain, the time-domain variances $V_\mathcal{O}=\sigma_\mathcal{O}^2$ are replaced by (single-sided) spectral densities (which I will also call variances) $S_\mathcal{O}(\Omega)\delta(\Omega-\Omega')=\ev{\hat{\mathcal{O}}(\Omega)\circ\hat{\mathcal{O}}^\dag(\Omega')}$ where $A\circ B=\frac{1}{2}(A\cdot B+B\cdot A)$. For $\hat{a}(\Omega)$ corresponding to an uncorrelated vacuum state, let $\ev{\hat{X}_i(\Omega)\circ\hat{X}_j^\dag(\Omega')}=\delta_{i,j}\delta(\Omega-\Omega')$ where $\delta_{i,j}$ is the Kronecker delta, and therefore $S_X^\text{vac}=1$~\footnote{Because the units of $\delta(\Omega-\Omega')$ and $\hat X(\Omega)$ are time by their integrals over $\Omega$ and $\hat X$ being unitless, therefore the units of $S_X$ are time or $1/\text{Hz}$ \jam{(therefore there is a units error in the vacuum value?)}.} can be shown from the commutation relations above~\cite{Danilishin}. Finally, there is a hidden assumption in these spectral density expressions: I have assumed that the quadrature operators are corresponding to the time-domain quantum fluctuations of the amplitude and phase of the light around its time-independent classical expectation value, i.e.\ I am considering Fourier transforms of $\delta\hat{X}(t)=\hat{X}(t)-\ev{\hat{X}}$ where I have made the time-dependences explicit for clarity. However, I will leave the $\delta$'s implicit to compactify notation, henceforth.


	% At high frequencies, the sensitivity of current detectors is limited by quantum noise arising from the quantum nature of light~\cite{danilishinQuantumMeasurementTheory2012}. By the Heisenberg Uncertainty Principle, the amplitude and phase of a quantum state of light (including the vacuum) cannot be exactly known at the same time. Uncertainty in the amplitude of the light within the interferometer leads to quantum radiation pressure noise, while uncertainty in its phase leads to so-called “shot noise”. 

\subsection{Optical loss and decoherence}
%  source of loss, realistic values (now and in a few decades time, 100 years time etc.)
\label{sec:optical_loss_background}

\begin{figure}
	\centering
	% \includegraphics[width=\textwidth]{}
	\caption{Beamsplitter model of optical loss. The incident field loses energy into the surroundings and uncorrelated vacuum is introduced. The Fluctuation-Dissipation Theorem states that these two rates are the same.}
	\label{fig:beamsplitter_loss}
\end{figure}

% So, quantum noise is just caused by uncertainties in the state of detector
Quantum noise from fundamental uncertainty in the light of a cavity-based detector can be thought of as entering the detector through every open port and lossy optic~\cite{}. %So far, the explanation introduced above is that quantum noise is caused by fundamental uncertainties in the state of the detector, so what does the previous statement mean? % too colloquial?
Any state of light in the detector can be written in terms of transformations of the vacuum state, and so any uncertainty in the state of the light can be expressed as vacuum fluctuations (corresponding to $S_X^\text{vac}=1$ above) viewed under such transformations~\cite{}. Therefore, the source of the quantum noise is ultimately the vacuum, which can be thought to surround the detector and enter whenever the system is opened to its surroundings~\footnote{This language is potentially confusing, because the vacuum does not propagate, but it is shorthand for considering how the system interacts with the surrounding fields -- how light can leak out of and into the detector in the form of real and virtual photons.}~\cite{}. An open port refers to where light can enter and leave the detector, such as at the photodetector where the detector is necessarily always open. This is worth emphasising, measurement requires the detector to be coupled to the environment and guarantees a port for the vacuum. % -- this notion of quantum noise is not well-defined for a closed system \jam{(this is misleading)}. 
I will, shortly, express this notion mathematically, but first I will explain why lossy optics are also a source of quantum noise.

% fluctuation-dissipation theorem
Optical loss refers to the energy lost from a light field upon propagation or interaction with an optic, e.g. if $50\%$ of the power incident on a mirror is reflected but only $45\%$ is transmitted, then the remaining $5\%$ of the power is lost to the environment. The chief mechanism for optical loss is dissipation into the thermal bath of the propagation medium and the optics, e.g.\ heating of a mirror~\cite{}. The Fluctuation-Dissipation Theorem states that any such dissipation into the thermal bath is accompanied by the introduction of uncorrelated noise into the light field~\cite{}. As such, the loss mechanism in an optic can be modelled by a beamsplitter that releases energy into the environment and creates a new open port for vacuum to enter through, as shown in Fig.~\ref{fig:beamsplitter_loss}. In this model of optical loss, all optics are perfect (i.e.\ the power incident equals the sum of the power reflected and transmitted) but additional open ports are added throughout the detector -- so-called loss ports. It suffices to have a single loss port in every cavity, i.e.\ for every mode, since the uncorrelated vacuum from multiple sources of loss, e.g.\ the propagation medium plus each optic in the cavity, sum in quadrature and is equivalent to a single vacuum source from a lossier port. % -- which should be clearer later. %is or are?
This means that any other open ports in the detector, besides any used for measurement, can also be collapsed into the loss ports. I will assume uncorrelated vacuum from every source of loss in this thesis, which is reasonable~\cite{}. Therefore, in this model, the quantum noise in a cavity-based detector comes from the vacuum entering at the measurement device and from loss ports in every mode.

% decoherence and the effect of mixing with the vacuum
% This mathematical formalism is useful but it leaves the effect of the loss ports vague 
The physical effect of optical loss is to decohere the state of the light -- reduce correlations and pull the quantum noise variance back towards its vacuum value of $S_X=1$. This will be demonstrated later, but the effect can be understood using the above notion of optical loss. The reflected light from a loss port is some linear combination of the vacuum and the incident light~\footnote{The Fluctuation-Dissipation Theorem states that the rates of energy lost and vacuum entering are equal, meaning that this combination is normalised: $\hat X_\text{out}=\sqrt{1-R}\hat X_\text{in}+\sqrt{R}\hat X_\text{vac}$, as shown in Fig.~\ref{fig:beamsplitter_loss}.}. This means that the quantum noise of the reflected light is a weighted average of the incident light and the vacuum value of $1$, i.e.\ more loss pulls the quantum noise towards $1$. The correlations of the light are also pulled towards the vacuum, which is uncorrelated, and therefore the correlations are reduced towards $0$. These correlations, such as between quadratures $\hat{X}_i, \hat{X}_j$, are given by the covariances $\ev{\hat{X}_i(\Omega)\circ\hat{X}_j^\dag(\Omega')}$, which is not real but is indirectly observable like the quadratures themselves~\cite{}. That optical loss decoheres the state is important to techniques that seek to reduce the quantum noise by introducing correlations, such as some of those explored in this thesis.

Finally, using this notion of quantum noise, I can write down the quantum noise response and sensitivity of a cavity-based detector. %how arbitrary is this?
Let the quadrature measured at the photodetector (PD) of an arbitrary detector be $\hat{X}_\text{PD}(\Omega)$. This will be determined by the noise quadratures $\hat{X}_i^\text{vac}(\Omega)$ of the vacuum input at each open port and source of optical loss indexed by $i$, the signal $\tilde{h}(\Omega)$, and the detector's linear response to each of these, expressed by the noise $R_i(\Omega)$ and signal $T(\Omega)$ transfer functions (where the frequency dependence will often be implicit henceforth): \begin{equation}\label{eq:transfer_fns_background}\hat X_\text{PD}(\Omega)=\sum_i R_i \hat X_i^\text{vac}(\Omega) + T \tilde h(\Omega).\end{equation}
These transfer functions capture the entire behaviour of the detector. %, including the impact on the optical quantum noise by coupling to non-optical modes such as the mechanical mode of a suspended mirror. % -- more on this later. 
The total quantum noise measured by the detector is simply $\hat X_\text{PD}(\Omega)|_{\tilde h=0}$, i.e.\ the output with the signal turned off. This quantum noise can be characterised by its spectral density, $S_X$, which can be simplified assuming uncorrelated vacuum inputs to the sum of squares: \begin{equation}S_X(\Omega)=\sum_i \abs{R_i}^2(\Omega).\end{equation} 
As an aside, since the transfer functions $R_{i_1}, R_{i_2}$ for two loss ports in the same cavity are the same up to the respective loss rates $\sqrt\gamma_{i_1}, \sqrt\gamma_{i_2}$, the two ports can be collapsed to a single loss port with loss rate $\sqrt{\gamma_{i_1}+\gamma_{i_2}}$ since \begin{equation}\abs{R_{i_1}}^2+\abs{R_{i_2}}^2=\abs{\sqrt{\gamma_{i_1}}R}^2+\abs{\sqrt{\gamma_{i_2}}R}^2=\abs{\sqrt{\gamma_{i_1}+\gamma_{i_2}}R}^2.\end{equation} This explains the claim made before that only one loss port is needed for each mode; an expression for the loss rates $\gamma$ will be given later. 
The sensitivity of the detector is defined as the signal-to-noise ratio in the measurement~\cite{}, but the gravitational-wave literature convention~\cite{} is to plot the inverse, the noise-to-signal ratio given by $S_h = \frac{S_X}{\abs{T}^2}$ (or more precisely $\sqrt S_h$). The units of sensitivity can be found by noting that $h$ is the unitless gravitational-wave strain and $\hat a, \hat a^\dag$ are unitless~\cite{} implies that $\hat X$ is unitless. In the Fourier domain, $\hat X(\Omega), \tilde h(\Omega)$ acquire units of time from the $\text{d}t$ in the Fourier transform. Therefore, the transfer functions in Eq.~\ref{eq:transfer_fns_background} are unitless and $S_X$ has units of time \jam{(this disagrees with the sum of squares expression, the Kronecker deltas must also come with something with units of time?)}. Therefore, $\sqrt S_h$ has units of $\text{Hz}^{-1/2}$~\cite{} \jam{(check that this is the right justification)}. 
%When plotting, I will use the amplitude spectral density, like the gravitational-wave literature~\cite{}, rather than the Fourier transform, and this means that $\sqrt S_h$ has units of $\text{Hz}^{-1/2}$~\cite{}. \jam{(check this)}




% \subsubsection{Technological limit: optical loss}, just give some example values

\subsection{Quantum noise response of an interferometric gravitational-wave detector}
\label{sec:qnoise_GW_IFO}

\begin{figure}
	\centering
	% \includegraphics[width=\textwidth]{}
	\caption{Quantum noise response of the interferometer shown in Fig.~\ref{fig:coupled_cavity_approx} to quantum radiation pressure noise, shot noise, and total quantum noise. At kilohertz, shot noise is the dominant form of quantum noise. The Standard Quantum Limit (SQL) is shown \jam{(should I display the SQL on a sensitivity curve as well?)}. The parameters of Advanced~LIGO were used~\cite{}. \jam{(update this appropriately)}} % in the signal-normalised curve \jam{(does this need to be a three-panel N, S, NSR plot?)}.} 
	\label{fig:simplifed_QN_response_conventional}
\end{figure}

\jam{(This section talks about the quantum noise response but does not derive it, should I derive the two-mode Michelson QN response function here instead of through dIS?)}

An interferometric detector has two sources of quantum noise: (1) quantum radiation pressure noise and (2) shot noise. 
% mention back-action as QPRN, mention free-falling mass in horizontal direction approximation
Firstly, the position of the test mass has some fundamental uncertainty which affects the optical mode. %, and this is captured through the quantum noise transfer function of the detector. 
This contribution to the quantum noise is known as quantum radiation pressure noise~\cite{}. 
Physically, the amplitude quadrature of the arm cavity mode couples to the test mass through radiation pressure~\cite{}, therefore, uncertainty in the position of the test masses leads to fluctuations in the amplitude quadrature, and vice versa~\cite{Danilishin?}. The response of the detector to this radiation pressure noise is related to the resonance behaviour of the mechanical mode, i.e.\ the resonances of the multi-stage suspension that holds the test mass~\cite{}. Although the suspension chain in a real detector has a non-trivial response due to the coupling of different mechanical modes, this response becomes simpler above the main pendulum resonance(s)~\cite{}. And because the kilohertz frequencies that I am interested in lie above these resonances, I will approximate the test mass as free-falling horizontally, i.e.\ that it is resonant at DC, $0$~Hz. This approximation is commonly made in the literature when studying the kilohertz response of interferometric detectors~\cite{}. The result of this approximation is that the quantum radiation pressure noise response of an interferometric detector is given by \jam{(look up radiation pressure formula)}~\cite{}, decreasing as $f^{-2}$ with increasing frequency $f$.

Secondly, the rest of the quantum noise comes from the optical mode itself, e.g.\ from the vacuum from each loss port. Physically, this quantum noise manifests as so-called ``shot noise'' at the photodetector -- uncertainty in the number of photons incident on the photodetector~\cite{}. While quantum radiation pressure noise appears (with the gravitational-wave signal) in the amplitude quadrature in the arms, shot noise appears in all quadratures as it is associated with the vacuum \jam{(I am now confused why shot noise is associated with the phase quadrature? why is it conjugate to QRPN?)}. The shot noise response of an interferometric detector is given by $S_X(\Omega) = 1$ \jam{(should show calculation? from dIS model with $\chi=0$)}, flat with frequency, which will be shown later. Therefore, shot noise is the dominant source of quantum noise at kilohertz because quantum radiation pressure noise falls off while shot noise remains flat, as shown in Fig.~\ref{fig:simplifed_QN_response_conventional}, and so shot noise is the dominant noise source overall at kilohertz~\cite{}. This thesis focuses on improving the shot noise--limited sensitivity as it is the most efficient way to improve kilohertz sensitivity. 

% Similarly, uncertainty in the phase of the light (its phase quadrature) incident on a photodetector affects the counting statistics of the photodetector and therefore leads to uncertainty in the intensity and ultimately so-called ``shot noise'' in the measurement of the signal~\cite{}.  \jam{(What is a counting statistic, explain this better)} %\jam{(phase isn't real, clarify this)} phase quadrature is measurable
% And so, (1) for an interferometer, the final measurement is affected by uncertainties in both quadratures of the light, but at different points in the detector (this will be clearer when I give a formal model in the next chapter), and (2) the quantum noise can be divided into the effects of shot noise, associated with the photodetector and the loss ports of the detector, and radiation pressure noise, associated with the test mass mechanical mode. \jam{(Clarify the association of loss ports with shot noise, what is shot noise only?)}

% Similarly, the positions and momenta of the test masses in a gravitational-wave detector are conjugate quantities. This introduces strange consequences of detection, e.g.\ as the position measurement of a test mass improves, its momentum becomes increasingly uncertain which couples through radiation pressure to the arm cavity mode~\cite{}. \jam{(follow up on this, what does this cause?)} 
% These consequences place limits on the quantum noise--limited sensitivity, i.e.\ the signal to quantum noise ratio, of a detector that cannot be surpassed by classical means which will be explained further later.

	% At high frequencies, shot noise is the dominant source of quantum noise and limits the sensitivity of current detectors. Therefore, reducing shot noise at high frequencies is the principal way to improve high-frequency sensitivity.

\subsubsection{Limits on improving the quantum noise}

\jam{(need to read up on SQL and QCRB to write this section properly.)}
% SQL
% review QCRB again? but did Mizuno limit on the integrated sensitivity in Introduction?
The factors limiting improving the quantum noise--limited sensitivity of gravitational-wave detectors were discussed in Chapter~\ref{chp:introduction}. Since the signal transfer function falls off at kilohertz, because of the arm cavity resonance, and the quantum radiation pressure noise starts dominating the quantum noise below $10$~Hz, there is a window of detection around $100$~Hz where the sensitivity is greatest~\cite{}. In particular, in this window, there is a point where shot noise and radiation pressure noise are equal, shown in Fig.~\ref{fig:simplifed_QN_response_conventional}.  %The trade-off between the two \jam{(need to explain this tradeoff somewhere above, formulae should help)} means that 
 %changing interferometer parameters (except for circulating power)
This is known as the Standard Quantum Limit (SQL) for interferometric detection and is an instance of a more general result about quantum noise~\cite{}. Beating the Standard Quantum Limit is not possible using classical techniques \jam{(explain why?)} and without increasing the circulating power. This is similar to the Mizuno limit on the integrated sensitivity described in Chapter~\ref{chp:introduction}, which comes from the Quantum Cramer-Rao Bound, a more fundamental limit on the quantum noise based on the total, theoretically-available information~\cite{MiaoQCRB2015}. However, these two limits are both able to be beaten \jam{(need to explain how SQL is easier to beat than QCRB)} by quantum techniques~\cite{MiaoQCRB2015,}, such as squeezing, which motivates the study of such technologies to improve the sensitivity.
% \jam{(this paragraph should motivate the need for squeezing it is almost there but misses many technical details)}
% These limits on improving the quantum noise motivate the need for technologies like squeezing which can get past them through manipulating the quantum state of the interferometer. 

% leave optical loss values in LIGO to external squeezing


\section{Squeezing}
\label{sec:squeezing_background}

\begin{figure}
	\centering
	% \includegraphics[width=\textwidth]{}
	\caption{Ball-and-stick illustration of squeezing. The ball (ellipse) represents the quantum noise around the coherent amplitude represented by the stick. The ellipse's semi-axes lengths represent the uncertainty in each quantity. Squeezing the ellipse changes these lengths such that their product is preserved (or increased).}
	\label{fig:ballandstick_simple}
\end{figure}

% return to HUP
% ball-and-stick plot
Squeezing refers to a broad range of technologies that all manipulate the quantum noise through exchanging uncertainty in one quantity for its conjugate quantity, while still obeying the Heisenberg Uncertainty Principle~\cite{}. For example, decreasing uncertainty in the amplitude quadrature by some factor $e^r$ by increasing uncertainty in the phase quadrature by the same factor still satisfies the Heisenberg Uncertainty Principle
\begin{equation}
\sigma_{X_1}\sigma_{X_2}\geq\frac{1}{2}\implies (\frac{\sigma_{X_1}}{e^r}) (e^r\sigma_{X_2})\geq\frac{1}{2}\label{eq:HUP_squeezed}.
\end{equation} 
A simple illustration of this is shown in Fig~\ref{fig:ballandstick_simple} where the trade-off of uncertainties is visualised as a literal squeezing of the noise ellipse, where the product of the semi-major and semi-minor axes is maintained (or increased) to obey the Heisenberg Uncertainty Principle. Squeezing is advantageous when the noise in one quantity matters more than the other, such as in interferometric detectors where, e.g.\ with homodyne readout, one quadrature is measured at the photodetector and noise in the other quadrature does not affect the measurement~\cite{}. The full story is more complicated, e.g.\ squeezing to decrease shot noise increases quantum radiation pressure noise in the final measurement, but this will be explained later in Section~\ref{sec:external_squeezing}. Squeezing can also be explained using sideband theory~\cite{}, but I do not discuss sidebands in this thesis.

% mention non-Gaussian squeezing, what is it actually, a different Hamiltonian?

\begin{figure}
	\centering
	% \includegraphics[width=\textwidth]{}
	\caption{Parametric down-conversion, showing the degenerate and nondegenerate process. In either case, the process conserves energy and the created photons are squeezed and entangled (i.e.\ the variances and covariances of the created fields have been changed).}
	\label{fig:PDC_deg_and_nondeg}
\end{figure}

% production of squeezed states, PDC, give a citation to answer Ilya's question
% Although I will model the crystal as a ``black-box'' that contributes some term to the Hamiltonian
Optical squeezing can be achieved via a variety of techniques~\cite{}, but in this thesis, I will focus on the production of squeezed states by shining light through nonlinear crystals~\cite{}. Consider a crystal with a quadratic (therefore, nonlinear) polarisability, i.e.\ when exposed to an electric field $\vec E$ it produces an electric field $\varepsilon_0 \sum_{n=0}^\infty \chi^{(n)} {\vec E}^n$~\cite{ ref in Kirk thesis?} \jam{(check this expression, what is ${\vec E}^n$?)} where the quadratic co-efficient dominates $\abs{\chi^{(2)}}\gg\abs{\chi^{(n)}},\; n\neq2$~\footnote{$\chi^{(2)}$ is used to label the squeezer crystal, henceforth.}. In such a crystal, the process of parametric down-conversion can occur~\cite{}, where a photon of a \jam{(particular?)} pump frequency $\omega_p$ is annihilated to create two entangled, squeezed photons at lower frequencies $\omega_0,\; \omega_0+\Delta$ such that $\omega_p=2\omega_0+\Delta$ to conserve energy~\cite{}. The frequency difference $\Delta$, chosen by how the crystal is pumped, is typically small in comparison to the other frequencies in the system, and when $\Delta=0$ the produced photons are degenerate which dramatically changes the mode structure of the system; the energy level structure of the process is shown in Fig.~\ref{fig:PDC_deg_and_nondeg}. I will show that the photons produced in parametric down-conversion are squeezed in the next section. %, but that they are entangled follows from their shared point of creation. 
\jam{(but why does quadratic nonlinearity make PDC happen? - Ilya's question)}. Although this process can create photons of many different frequencies \jam{(can it?)}, I use a single-mode approximation throughout this thesis that only considers the process involving a particular pump frequency $2\omega_0+\Delta$, carrier frequency $\omega_0$, and frequency difference $\Delta$. This is justified by the same reasoning as in Section~\ref{sec:coupled_cavity_approximation} \jam{(do I need to say it again?)}, %, the cavities select their resonant frequencies and all other frequencies can be ignored~\cite{}. 
which means that I assume that wherever the squeezer crystal is placed, the pump frequency and the two produced frequencies (called the signal~\footnote{There is potential confusion between the signal mode of light created by the squeezer and the light in the detector that contains the gravitational-wave signal. %E.g.\ when talking about the signal transfer function, it can be that the light in the signal mode is measured to estimate the gravitational-wave signal, although the signal transfer function is named for the latter. 
These terms are too prevalent in the literature~\cite{} not to use them, but I have tried to be clear whenever necessary.} $\omega_0$ and idler $\omega_0+\Delta$ frequencies) are resonant.

Before I construct a Hamiltonian model of squeezing, I need to make three clarifications. % to reduce confusion. 
% although one can imagine a green laser beam incident on a crystal squeezing two slightly differently coloured red lasers also incident
Firstly, squeezing is often used to squeeze the vacuum of virtual photons. % at the signal and idler (or just the signal if degenerate) modes. % which can be initially confusing. 
This corresponds to the coherent amplitudes of the incident modes being zero, i.e.\ the ellipse in Fig.~\ref{fig:ballandstick_simple} would be centred at the origin. This means that I need to distinguish between the squeezed vacuum from a squeezer and true vacuum, where the ellipse in Fig.~\ref{fig:ballandstick_simple} is circular, centred at the origin, and corresponds to the $S_X=1$ value. 
% can these latter general comments (frequency, dB) be moved into dOPO section where they will make more sense?
Secondly, throughout this thesis, I consider the quantum noise response of a detector and talk about the quantum noise being squeezed (suppressed) or ``anti-squeezed'' (amplified) at different frequencies. This should not be confused with the down-conversion occurring with different frequencies, rather, the squeezer always interacts with the same frequencies of the single-mode approximation but the response considers the spectra of their respective quadratures (i.e.\ the distinction is between the frequency annihilated by ${\hat a}$ and its Fourier spectrum with respect to frequency $\Omega$). Finally, in the squeezing literature, there are a few ways to quantify the amount of squeezing produced~\cite{}. For example, if $S_X$ is squeezed by a factor $e^r$, then typically the quantum noise response $\sqrt S_X$ is plotted logarithmically in amplitude-decibels (dB) as $20 \log_{10}(\sqrt {S_X e^{-r}})=20 \log_{10}(\sqrt S_X) - 20\log_{10}(e^{r/2})$ with the vacuum ($S_X=1$) at $0$~dB. However, I reserve the term ``squeezing factor'' for the parameter $\chi$ seen shortly in the Hamiltonian model, instead of these factors $e^r, e^{r/2}, 20\log_{10}(e^{r/2})$ or various other, related quantities in the literature~\cite{}.

% just a paragraph about sidebands?, I do not go further into this because I do not use the formalism (preferring the operator+Interaction Picture which achieves the same effect of studying frequency offsets)


	% An established technology to reduce shot noise at high frequencies is squeezing~\cite{danilishinQuantumMeasurementTheory2012,chuaQuantumEnhancementKm2015}. Squeezing manipulates the quantum state of light and the vacuum to trade-off phase uncertainty for amplitude uncertainty; this decreases shot noise by increasing quantum radiation pressure noise~\footnote{Squeezing only improves sensitivity if the interferometer readout is suitably designed to exploit the trade-off between shot noise and radiation pressure noise.}. Squeezed states of light can be created in a crystal with nonlinear polarisability by parametric down-conversion which annihilates a photon at a pump frequency and creates two entangled photons with ``squeezed" uncertainties. To conserve energy, the sum of the frequencies of the created photons must equal the pump frequency. This process is degenerate if the frequencies of the created photons are equal and is nondegenerate otherwise.
	% The fact that nondegenerate squeezing involves three distinct frequencies (the pump and the two created frequencies) rather than two is an important change in the symmetry of the system that I will revisit later.


\subsection{Degenerate OPO}
% this is quite long for a subsection?

\begin{figure}
	\centering
	% \includegraphics[width=\textwidth]{}
	\caption{Degenerate optical parametric oscillator (OPO), all modes in the analytic model are labelled. The result of the squeezer is that the vacuum incident on the OPO at the readout port is reflected squeezed (its quantum noise variances and covariances changed), which is shown by the noise ellipses. The optical loss in the system either occurs intra-cavity or in the detection chain.}
	\label{fig:dOPO_config}
\end{figure}

A degenerate optical parametric oscillator (OPO) consists of a squeezer crystal inside of an optical cavity, operating in the degenerate regime (i.e.\ producing two squeezed photons at the same frequency for each pump photon annihilated), as shown in Fig.~\ref{fig:dOPO_config}. The optical cavity is used to increase the average number of passes of the squeezer that the light makes and therefore increase the squeezing. Vacuum enters the cavity through the main, readout port (connected to the photodetector) and through an intra-cavity loss port, shown in Fig.~\ref{fig:dOPO_config}. This is the quintessential squeezing configuration and is well-studied in the literature~\cite{}. But I examine this configuration here for two main reasons: (1) to show that the photons produced in parametric down-conversion are squeezed and (2) to demonstrate the analytic, Hamiltonian model that I will use throughout the rest of this thesis (this approach is used widely in the literature~\cite{}).  
% In Fig.~\ref{fig:dOPO_config}, the pump laser is shown incident orthogonally on the crystal to the beam path, this is only somewhat realistic~\cite{} and spatial aspects of the configuration are not included in the model. 
To simplify the model, I make the single-mode approximation and assume that the pump laser down-converts into the cavity's (fundamental) resonant mode.
% OPO = squeezed cavity with no seed light
% In the literature, a distinction is sometimes made between when the squeezed field is ``seeded'' or not, i.e.\ whether the degenerate OPO produces squeezed vacuum or squeezed light, and in the latter case the OPO is instead called an optical parametric amplifier (OPA). Since I am interested in the quantum noise response, i.e.\ the response to vacuum noise, this is indeed the former case. 
% But since I am interested in the quantum noise response of this configuration, i.e.\ how vacuum noise appears in the final measurement, I do not need to worry about this distinction. % is this even necessary to bring up then?
To further simplify the model, I ignore the dynamics of the pump laser by approximating it as an un-depletable reservoir of photons. Formally, this means that I make a semi-classical approximation to the pump mode and then approximate its coherent amplitude as constant (the ``no pump depletion'' assumption). This approximation is widely used in the literature~\cite{} and I will later justify what parameter range it is valid in.

\subsubsection{Analytic model}
\label{sec:dOPO_model}
% analytic model to demonstrate Hamiltonian method

I derive the quantum noise response of a degenerate OPO using a Hamiltonian approach. Let the modes of this system be as shown in Fig.~\ref{fig:dOPO_config}: $\hat u$ for the pump at frequency $2\omega_0$, $\hat b$ for the signal cavity mode at frequency $\omega_0$, $\hat n^L_b$ for the vacuum entering the intra-cavity loss port, $\hat B_\text{in}$ for the vacuum entering the readout port, $\hat B_\text{out}$ for the signal mode leaving the readout port, $\hat n^L_\text{PD}$ for the vacuum entering the detection loss port, and $\hat B_\text{PD}$ for the signal mode incident on the photodetector, i.e.\ after the detection loss port. I want to derive $\hat B_\text{PD}$ as a function of the vacuum inputs, and to do so I will first solve for $\hat b$ and then propagate the solution to the photodetector. 
% ~\footnote{Spatial propagation is not part of this model, I mean that I will use the appropriate input/output relations~\cite{} to find the detected mode.}
The Hamiltonian of this system is given by $\hat H = \hat H_0 + \hat H_I + \hat H_\gamma$, where $\hat H_0$ gives the decoupled dynamics of $\hat u$ and $\hat b$, $\hat H_I$ gives the parametric down-conversion in the squeezer crystal, and $\hat H_\gamma$ gives the coupling through the readout and each loss port, which are given by~\cite{} \jam{(fill in readout term, look at Korobko?)}
\begin{align}
\hat H_0 &= \hbar \omega_0 \hat b^\dag \hat b + \hbar 2 \omega_0 \hat u^\dag \hat u\\
\hat H_I &= i \hbar \frac{x}{2} e^{i\phi} \hat u (\hat b^\dag)^2 + \text{h.c.}\\
\hat H_\gamma &= \int \ldots .
\end{align}
Where $x$ is the real coupling constant of the down-conversion, $\phi$ is the phase of the pump laser, $\text{h.c.}$ represents the Hermitian conjugate of the rest of the expression~\footnote{This represents the reverse process, second-harmonic generation~\cite{}, where two $\omega_0$ photons combine to make a $2\omega_0$ photon}, and $\gamma_b, \gamma^b_R$ are the coupling rates through the intra-cavity loss port and the readout port, respectively. These coupling rates are related to the transmission through their respective ports $T_{l,b}, T_R$ and the round-trip time for the cavity $\tau = \frac{L_\text{rt}}{c}$, where $L_\text{rt}$ is the round-trip length of the cavity, by $\gamma = -\frac{1}{2\tau}\log(1-T)$~\cite{}. The detection loss in the output chain of optics will be included later in the derivation, instead of explicitly in the Hamiltonian. 

For this Hamiltonian, the Heisenberg-Langevin~\footnote{The Langevin input/output terms come from $H_\gamma$, but fundamentally this is still the Heisenberg equation-of-motion.} equation-of-motion~\cite{GardinerCollete} for $\hat b$, given by $\dot{\hat b}=-\frac{i}{\hbar}[\hat b,\hat H]$ and the bosonic commutation relations~\footnote{$[\hat Q_i,\hat Q_j^\dag]=\delta_{i,j}$ where the annihilation operator of field $i$ is $\hat Q_i$.}, is 
\begin{equation}
\dot{\hat{b}}= -i\omega_0 \hat b+x e^{i\phi} \hat u\hat b^\dag - \gamma^b_\mathrm{tot} \hat{b} + \sqrt{2\gamma^b_R}\hat{B}_\mathrm{in} + \sqrt{2\gamma_b}\hat{n}^L_b.
\end{equation}
Where $\gamma^b_\text{tot}=\gamma^b_\text{R}+\gamma_b$ is the total loss rate from the cavity. I make the semi-classical pump and no pump depletion approximations which treat $\hat u$ as some constant, coherent amplitude $u$ (determined by the classical pump power~\cite{}). Let the ``squeezer parameter'' be $\chi = x u$. Changing from the Heisenberg Picture to the Interaction Picture, i.e.\ separating out the dynamics of $\hat H_0$ onto the states and leaving $\hat H - \hat H_0$ to evolve the operators, removes the $-i\omega_0 \hat b$ term from the equation-of-motion leaving %like entering a rotating frame at $\hat b \mapsto e^{-i\omega_0 t} \hat b$ for each operator
\begin{equation}
\label{eq:dOPO_pre_FT}
\dot{\hat{b}}= \chi e^{i\phi}\hat b^\dag - \gamma^b_\mathrm{tot} \hat{b} + \sqrt{2\gamma^b_R}\hat{B}_\mathrm{in} + \sqrt{2\gamma_b}\hat{n}^L_b.
\end{equation}
\jam{(check that rotating-wave approximation is not required?)}
Since I am interested in the quantum noise, I take the fluctuating components of each of these operators (leaving the $\delta \hat b\mapsto\hat b$ implicit as mentioned in Section~\ref{sec:coupled_cavity_approximation}) to ignore the classical dynamics, but each of the input modes $\hat{B}_\mathrm{in}, \hat{n}^L_b$ are vacuum meaning that the equation-of-motion is the same for the fluctuating components. \jam{(That the classical amplitude for $\hat b$ remains zero is not obvious, i.e.\ time-average is still zero, and the squeezer starts lasing at threshold so I need to clarify.)}

Eq.~\ref{eq:dOPO_pre_FT} can be solved algebraically via Fourier transform %(where $\partial_t \mapsto -i \Omega$):
\begin{equation}
\label{eq:dOPO_FT_initial}
(\gamma^b_\mathrm{tot}-i \Omega)\hat b(\Omega)=\chi e^{i\phi}\hat b^\dag(-\Omega)  + \sqrt{2\gamma^b_R}\hat{B}_\mathrm{in}(\Omega) + \sqrt{2\gamma_b}\hat{n}^L_b(\Omega).
\end{equation}
Where $\Omega$ is the angular frequency offset from $\omega_0$ due to the Interaction Picture. % Algebraically, there are a few ways to arrive at the result, but I will demonstrate the linear algebra solution. % since it will be a useful method in the nondegenerate case where there are twice as many operators. 
Let $\vec{\hat{Q}}(\Omega)=[\hat{Q}(\Omega),\hat{Q}^\dag(-\Omega)]^\text{T}$ for each pair of annihilation/creation operators $\hat Q, \hat Q^\dag$ and take Hermitian conjugate and $\Omega\mapsto -\Omega$ of Eq.~\ref{eq:dOPO_FT_initial} to find
\begin{align}
\label{eq:dOPO_FT_vectorised}
\vec{\hat b}(\Omega)&=\text{M}_b^{-1}\left(\sqrt{2\gamma^b_R}\vec{\hat{B}}_\mathrm{in}(\Omega) + \sqrt{2\gamma_b}\vec{\hat{n}}^L_b(\Omega)\right)\\
\text{M}_b&=(\gamma^b_\mathrm{tot}-i \Omega)\text{I}-\chi \begin{bsmallmatrix}0 & e^{i\phi} \\e^{-i\phi} & 0\end{bsmallmatrix}.
\end{align}
Where $\text{I}$ \jam{(formatting unclear)} is the 2 by 2 identity matrix. At the readout port, the input/output relation~\cite{} is 
\begin{equation}
\label{eq:dOPO_IO_readout}\vec{\hat{B}}_\mathrm{out}(\Omega)=\vec{\hat{B}}_\mathrm{in}(\Omega)-\sqrt{2\gamma^b_R}\vec{\hat b}(\Omega).
\end{equation} 
And at the photodetector, using the beamsplitter model of detection loss with reflectivity $R_\text{PD}\in(0,1)$, the light is 
\begin{equation}
\label{eq:dOPO_IO_PD}\vec{\hat{B}}_\mathrm{PD}(\Omega)=\sqrt{1-R_\text{PD}}\vec{\hat{B}}_\mathrm{out}(\Omega)+\sqrt{R_\text{PD}}\vec{\hat n}^L_\text{PD}(\Omega).
\end{equation} 
From which the quadratures of the light can be found in terms of the input fields. Let $\Gamma = \frac{1}{\sqrt2} \begin{bsmallmatrix}1 & 1 \\-i & i\end{bsmallmatrix}$ such that the quadratures associated with some annihilation operator $\hat{Q}(\Omega)$ are given by $\vec{\hat{X}}_Q(\Omega)=[\hat{X}_{Q,1}(\Omega),\hat{X}_{Q,2}(\Omega)]^\text{T}=\Gamma \vec{\hat{Q}}(\Omega)$~\footnote{Note the difference to the vectorisation $\vec{\hat{Q}}(\Omega)$, $\vec{\hat{X}}_Q(\Omega)$ is a vector of two separate quadratures each of which obey $\hat{X}^\dag(-\Omega)=\hat{X}(\Omega)$.}. Putting Eqs.~\ref{eq:dOPO_FT_vectorised}~\ref{eq:dOPO_IO_readout}~\ref{eq:dOPO_IO_PD} together, I find the measured quadratures to be
\begin{align}
\label{eq:dOPO_PD_as_fn_of_vac}
\vec{\hat X}_\mathrm{PD}(\Omega)&=\text{R}_\text{in}\vec{\hat X}_\mathrm{in}(\Omega)+\text{R}^L_b\vec{\hat X}^L_b(\Omega)+\text{R}^L_\text{PD}\vec{\hat X}^L_\text{PD}(\Omega)\\
\text{R}_\text{in}&=\sqrt{1-R_\text{PD}}\Gamma\left(\text{I}-2\gamma^b_R\text{M}_b^{-1}\right)\Gamma^{-1}\\
\text{R}^L_b&=-\sqrt{1-R_\text{PD}}\Gamma 2\sqrt{\gamma^b_R \gamma_b}\text{M}_b^{-1}\Gamma^{-1}\\
\text{R}^L_\text{PD}&=\sqrt{R_\text{PD}} \text{I}.
\end{align}
I.e.\ given by transfer matrices $\text{R}_\text{in},\text{R}^L_b, \text{R}^L_\text{PD}$, as a linear combination of the three (uncorrelated) vacuum sources.

This defines the quantum noise response of the degenerate OPO, where the total quantum noise measured at the photodetector is described by the spectral density matrix (an extension of the spectral density $S_X$ described in Section~\ref{sec:optical_loss_background}) \begin{equation}\label{eq:dOPO_Sx_abstract}\text{S}_X(\Omega)=\text{R}_\text{in} \text{R}_\text{in}^\dag+\text{R}^L_b {\text{R}^L_b}^\dag+\text{R}^L_\text{PD}{\text{R}^L_\text{PD}}^\dag.\end{equation} Where the uncorrelated vacuum assumption was used to simplify 
\begin{equation}\label{eq:total_noise_matrix}
(\text{S}_X)_{i,j}(\Omega)\delta(\Omega-\Omega')=\ev{(\vec{\hat X}_\text{PD})_i(\Omega)\circ(\vec{\hat X}_\text{PD})_j^\dag(\Omega')}.
\end{equation}
Where $\ev{\ldots}$ is the vacuum expectation value, as before. The diagonal elements of $\text{S}_X$ are the (Fourier domain) variances (labelled $S_X$ in Section~\ref{sec:optical_loss_background}) of ${\hat X}_{\mathrm{PD},1}(\Omega), {\hat X}_{\mathrm{PD},2}(\Omega)$ and the off-diagonal elements (where $\text{S}_X$ is manifestly Hermitian by Eq.~\ref{eq:dOPO_Sx_abstract}) give the covariance between the two quadratures.

This demonstrates the Hamiltonian method that I will use throughout the rest of this thesis, with later configurations just adding more terms/modes to the Hamiltonian. % -- including those associated with the signal. % Overall, the method is robust \jam{(why?)} and easy to perform computationally 

% Results
\subsubsection{Demonstrating squeezing}

Having now found the quantum noise response, I can demonstrate that parametric down-conversion leads to squeezing the variances of the measured quadratures $(\text{S}_X)_{1,1}, (\text{S}_X)_{2,2}$. Computing $\text{S}_X$ requires matrix algebra which I perform using Wolfram Mathematica~\cite{} throughout this thesis. Doing so shows that (in agreement with previous results~\cite{}): 
\begin{equation}\label{eq:dOPO_full_freedom}
\text{S}_X(\Omega)=\left[
\begin{array}{cc}
 1+\frac{(1-R_\text{PD})4 \gamma^b_R \chi  \left(2 \gamma^b_\text{tot} \chi +\left({\gamma^b_\text{tot}}^2+\chi ^2+\Omega ^2\right) \cos (\phi )\right)}{\left({\gamma^b_\text{tot}}^2-\chi ^2\right)^2+2 \Omega ^2 \left({\gamma^b_\text{tot}}^2+\chi ^2\right)+\Omega ^4} 
 & \frac{(1-R_\text{PD})4 \gamma^b_R \chi  \left({\gamma^b_\text{tot}}^2+\chi ^2+\Omega ^2\right) \sin (\phi )}{\left({\gamma^b_\text{tot}}^2-\chi ^2\right)^2+2 \Omega ^2 \left({\gamma^b_\text{tot}}^2+\chi ^2\right)+\Omega ^4} \\
 \frac{(1-R_\text{PD})4 \gamma^b_R \chi  \left({\gamma^b_\text{tot}}^2+\chi ^2+\Omega ^2\right) \sin (\phi )}{\left({\gamma^b_\text{tot}}^2-\chi ^2\right)^2+2 \Omega ^2 \left({\gamma^b_\text{tot}}^2+\chi ^2\right)+\Omega ^4} 
 & 1+\frac{(1-R_\text{PD})4 \gamma^b_R \chi  \left(2 \gamma^b_\text{tot} \chi -\left({\gamma^b_\text{tot}}^2+\chi ^2+\Omega ^2\right) \cos (\phi )\right)}{\left({\gamma^b_\text{tot}}^2-\chi ^2\right)^2+2 \Omega ^2 \left({\gamma^b_\text{tot}}^2+\chi ^2\right)+\Omega ^4} \\
\end{array}
\right].\end{equation}
The general form of $\text{S}_X(\Omega)$ is common to all configurations in this thesis: all elements are rational functions (fractions of polynomials) of $\Omega$ and $\chi$, diagonal elements are often simpler when put in the form $1 + \Delta S$ since they are perturbations from the vacuum value of $1$, and off-diagonal elements (although real here, they can be complex in general) obey the Hermitiancy of $\text{S}_X$. If the squeezer is turned off, i.e.\ $\chi=0$, then the output is vacuum, $\text{S}_X=\text{I}$. Looking at the Hamiltonian, only $\phi$ (and the sign of $\chi$, although I will keep $\chi$ non-negative and vary the phase using $\phi$) will break the symmetry between the quadratures \jam{(clarify this, Hamiltonian is not written in terms of the quadratures, perhaps should be re-expressed)} -- observe that $(\text{S}_X)_{2,2}=(\text{S}_X)_{1,1}|_{\chi\mapsto-\chi}=(\text{S}_X)_{1,1}|_{\phi\mapsto\phi+\pi}$. When $\phi=0$, then the covariance vanishes and the variances simplify to \begin{equation} \label{eq:dOPO_fixed_phase}
\text{S}_X(\Omega)=\left[
\begin{array}{cc}
 1+\frac{(1-R_\text{PD})4 \gamma^b_R \chi}{\left({\gamma^b_\text{tot}}-\chi\right)^2+\Omega ^2}
 & 0 \\
 0
 & 1-\frac{(1-R_\text{PD})4 \gamma^b_R \chi}{\left({\gamma^b_\text{tot}}+\chi\right)^2+\Omega ^2} \\
\end{array}
\right].\end{equation} From this expression, by inspection for $\chi>0$, one quadrature ($\hat X_1$) has increased uncertainty while the other quadrature ($\hat X_2$) has its uncertainty decreased below the vacuum value of $1$, these are the anti-squeezed (amplified) and squeezed quadratures, respectively. % Although using squeezing to decrease the quantum noise is the common application, anti-squeezing is also useful when the signal can be amplified as well -- to be discussed much later. 
In the lossless case, i.e.\ $R_\text{PD}=0, \gamma_b=0, \gamma^b_\text{tot}=\gamma^b_R$, the results satisfy the Heisenberg Uncertainty Principle as an equality \begin{equation}\label{eq:dOPO_HUP_sat}(\text{S}_X)_{1,1}(\text{S}_X)_{2,2}=\left(1+\frac{4 \gamma^b_R \chi}{\left(\gamma^b_R-\chi\right)^2+\Omega ^2}\right)\left(1-\frac{4 \gamma^b_R \chi}{\left(\gamma^b_R+\chi\right)^2+\Omega ^2}\right)=1.\end{equation} And in the lossy case, as an inequality $(\text{S}_X)_{1,1}(\text{S}_X)_{2,2}\geq1$ \jam{(show/cite this?)} which is an indication that optical losses decohere the system and can increase the overall uncertainty product \jam{(how? optical loss pulls to vacuum value of 1, right?)}. While understanding physically why parametric down-conversion produces squeezing is beyond the scope of this thesis~\footnote{See Ref.~\cite{} for an explanation.}, this derivation has shown it mathematically.

There remain a few effects visible in this simpler model that will return in later models and are therefore worth examining, namely the effects of: (1) increasing the squeezer parameter $\chi$, (2) intra-cavity loss $\gamma_b$ versus detection loss $R_\text{PD}$, and (3) changing the pump phase $\phi$. I discuss each of these in turn. % is this necessary to say?
% use of colon is grammatically incorrect?

\begin{figure}
	\centering
	% \includegraphics[width=\textwidth]{}
	\caption{Degenerate OPO quantum noise response, showing squeezing and anti-squeezing, the effect of intra-cavity versus detection loss, and the notion of threshold. Showing variances (left panel) and covariance (right panel) against frequency.}
	\label{fig:dOPO_variances}
\end{figure}


\subsubsection{Threshold}
\label{sec:dOPO_threshold}
% in the abstract, a squeezing technology might produce any amount of squeezing, the energetic reality of gain and loss in a NL crystal creates threshold
% pump depletion only matters near and above threshold
Firstly, the effect of the squeezer parameter $\chi$. In the lossless case and with $\phi=0$, by Eq.~\ref{eq:dOPO_HUP_sat}, one can see that increasing $\chi$ from zero increases the difference from the vacuum value for both quadratures, i.e.\ the amount of (anti)squeezing, as shown in Fig.~\ref{fig:dOPO_variances}. %  symmetrically for squeezing and anti-squeezing when $\sqrt S_X$ is plotted logarithmically \jam{(explain this more once plot present)}. 
Increasing $\chi$ further, observe that the DC ($\Omega=0$) value of the anti-squeezed quadrature of the quantum noise is given by $(\text{S}_X)_{1,1}=1+\frac{4 \gamma^b_R \chi}{\left(\gamma^b_R-\chi\right)^2}$ which is singular at squeezer parameter $\chi_\text{thr}=\gamma^b_R$, and at $\chi_\text{thr}$ the squeezed quadrature is zero at DC, as shown in Fig.~\ref{fig:dOPO_variances}. %(to satisfy the Heisenberg Uncertainty Principle exactly)
This value $\chi_\text{thr}$ is known as the threshold~\cite{} of the degenerate OPO. % and serves as a boundary of this model's physicality, specifically of the assumption of no pump depletion. 
Threshold can be understood from the gains and losses inside the cavity: the squeezer creates photons in the cavity mode at a rate $\chi$ which are lost from the cavity at a rate $\gamma^b_R$, and when $\chi=\chi_\text{thr}$ the gain and loss balance and the state of the system changes as beyond $\chi_\text{thr}$ net photons appear in the cavity mode. Like a phase transition, beyond $\chi_\text{thr}$ this means that the OPO starts lasing -- the output field has a non-zero coherent amplitude~\cite{}. % stimulated emission is less than sponteaneous emission, decoherence > coherence
In the lossy model, threshold can be shown to be $\chi_\text{thr}=\gamma^b_\text{tot}$~\cite{} which makes sense from the gain-loss argument~\footnote{That it is independent of pump phase also makes sense by inspection of the denominator in Eq.~\ref{eq:dOPO_full_freedom}.} and recovers the lossless value in the appropriate limit. However, above, at, or just below threshold the loss of energy from the pump mode is significant and the approximation that it is an un-depletable reservoir is no longer valid~\cite{}. Therefore, all models of squeezers in cavity detectors in this thesis, which use the no pump depletion assumption as a simplification, have squeezer parameter bound to $\chi\in(0,\chi_\text{thr})$. % (although, as seen in the lossless case, this can still allow arbitrarily large squeezing/anti-squeezing). 
Although I have introduced threshold here as a limitation of the assumptions I have made, in reality, we do not want to operate squeezers above threshold since the created coherent amplitude mode can damage sensitive optics such as photodetectors designed to sense vacuum fluctuations~\cite{}. Therefore, staying below threshold is satisfactory and the no pump depletion assumption is justified. Moreover, in experiments, to avoid the system wandering above threshold some safety margin is often introduced, e.g.\ remaining below $70\%$ threshold as a rule-of-thumb~\cite{} and pushing above $90\%$ only in a well-controlled system~\cite{}.  % set up singularity threshold?

\subsubsection{Effect of losses}
\jam{(Should these be subsubsections or not? Unify style with rest of thesis.)} % Threshold is warranted but these just look like general results.)}
% effect of different losses

Secondly, the effect of detection loss $R_\text{PD}$ versus intra-cavity loss $\gamma_b$. Understanding the tolerance of a detector to various losses is important when judging its feasibility and the results here will apply generally. By Eq.~\ref{eq:dOPO_full_freedom}, detection loss $R_\text{PD}\in(0,1)$ gives a weighted average of the lossless variance and the vacuum, and therefore increasing $R_\text{PD}$ simply pulls the variances towards the constant $1$, as shown in Fig.~\ref{fig:dOPO_variances}. This creates a difference between the squeezed and anti-squeezed quadratures, however, since the on-threshold, DC value of the squeezed quadrature changes from $0$ to $R_\text{PD}$ but the anti-squeezed quadrature remains at $\infty$, i.e.\ the singularity in the anti-squeezed quadrature is more robust than the zero in the squeezed quadrature. This indicates that the detection loss has increased the product in the Heisenberg Uncertainty Principle to being $>1$ at DC, and indeed at all frequencies \jam{(cite/check this?)}. Meanwhile, introducing intra-cavity loss $\gamma_b$ acts like damping an oscillator~\cite{}, which decreases the quality factor of the resonance peak, since the peak of each variance moves towards vacuum and broadens~\footnote{This means that there are frequencies for which the squeezing is improved by loss.}. Similarly to detection loss, intra-cavity loss affects the on-threshold, DC value of the squeezed quadrature but not the anti-squeezed quadrature. Since intra-cavity loss increases the total loss rate $\gamma^b_\text{tot}$, it also increases threshold $\chi_\text{thr}$ which contributes to the decrease in performance. This effect is not particularly interesting experimentally, however, as pump power could be increased to accommodate the loss. Therefore, henceforth I will compare the response at different losses with the ratio to threshold $\chi/\chi_\text{thr}$ fixed, i.e.\ I will calculate threshold in each case and change $\chi$ accordingly unless stated otherwise. 
% decoherences of co-variances --> need a figure?
The effect of losses on the covariance is similar, detection loss pulls the whole curve towards zero while intra-cavity loss reduces the peak and broadens the response, as shown in Fig.~\ref{fig:dOPO_variances}.

\subsubsection{Effect of pump phase}
% effect of pump phase

\begin{figure}
	\centering
	% \includegraphics[width=\textwidth]{}
	\caption{Degenerate OPO quantum noise response, showing squeezing and anti-squeezing variances versus frequency for different pump phases $\phi \in (0,\pi)$.}
	\label{fig:dOPO_variances_pump_phase}
\end{figure}

Finally, the effect of pump phase $\phi$. As mentioned above, the phase of the pump mode is what breaks the symmetry between the quadratures. From Eq.~\ref{eq:dOPO_full_freedom}, $(\text{S}_X)_{2,2}=(\text{S}_X)_{1,1}|_{\phi\mapsto\phi+\pi}$ and vice versa -- the quadratures switch under $\phi\mapsto\phi+\pi$, as shown in Fig.~\ref{fig:dOPO_variances_pump_phase}. For other values of $\phi$, the anti-squeezing and squeezing occur on a different set of orthonormal basis vectors for the 2D space of quadratures: $\hat{X}_{\theta=\phi}$ and $\hat{X}_{\theta=\phi+\pi/2}$ respectively \jam{(fix this, is it $\phi\mapsto\phi+\pi$ or $\phi\mapsto\phi+\pi/2$?)} -- a rotation of the noise ellipse in Fig.~\ref{fig:ballandstick_simple}. 
% And continuously changing $\phi\in(0,\pi)$ \jam{($\pi/2$?)} smoothly connects the squeezing and anti-squeezing behaviour. 
Fixing the pump phase and performing a change-of-basis of the quadratures at the photodetector confirms that this is the case~\cite{}.


% summary of dOPO, is this necessary?
% The degenerate OPO provides a simple, well-studied configuration to witness many of the effects of placing a squeezer inside of a cavity. I have also demonstrated the method and some of the assumptions used throughout this thesis to model detectors. However, before I can move on to discussing how squeezing is used currently in interferometric detectors and how proposals have suggested to exploit it further for future detectors, I need to expand the above model and discuss the effects of including an additional mode -- the idler.


\subsection{Nondegenerate OPO}
\label{sec:nOPO}
% succinct analytic model, to show how to calculate covariances (requires 4x4 model)

\begin{figure}
	\centering
	% \includegraphics[width=\textwidth]{}
	\caption{Nondegenerate OPO configuration is the same as the degenerate OPO in Fig.~\ref{fig:dOPO_config} but with an idler mode, frequencies of each mode are labelled. Vacuum at the signal and idler frequencies incident on the OPO is reflected anti-squeezed, shown by the noise ellipses which are symmetric up to the loss rates, and correlated. The intra-cavity loss and readout rate can be different for the signal and idler by using dichroic optics.}
	\label{fig:nOPO_config}
\end{figure}

A nondegenerate optical parametric oscillator (OPO) is the same configuration as a degenerate OPO, except that the squeezer performs nondegenerate parametric down-conversion as shown in Fig.~\ref{fig:nOPO_config}: splitting the pump at frequency $2\omega_0+\Delta$ down into two separate, squeezed, and entangled modes at $\omega_0$ (the signal) and $\omega_0+\Delta,\; \Delta\neq0$ (the idler). Similarly to the degenerate case, this configuration is well-studied in the literature~\cite{}. I show an abridged derivation of its quantum noise response here because the main configuration later in this thesis is fundamentally a nondegenerate OPO plus an additional mode and so that derivation will build on this one. Moreover, much of the behaviour of the nondegenerate OPO is inherited by that configuration. % -- to be seen later.

\subsubsection{Analytic model}

I follow the same Hamiltonian method as Section~\ref{sec:dOPO_model} to derive the quantum noise response and will abridge many of the steps here as they are exactly the same. Let the modes of the system be as described in Section~\ref{sec:dOPO_model} but with the pump mode $\hat u$ at $2\omega_0+\Delta$, and let $\hat c$ annihilate the idler cavity mode at $\omega_0+\Delta$ (assumed on-resonance by the single-mode approximation \jam{(how can the signal, idler, and pump all be on resonance? why, in nIS, can the idler not be resonant in the arms?)}) with associated input vacuum fields $\hat n^L_c, \hat C_\text{in}, \hat n^L_\text{c,PD}$ and output fields $\hat C_\text{out}, \hat C_\text{PD}$ that are direct analogues of the signal mode's $\hat n^L_b, \hat B_\text{in}, \hat n^L_\text{b,PD}, \hat B_\text{out}, \hat B_\text{PD}$, respectively. The Hamiltonian of this system is $\hat H = \hat H_0+\hat H_I+\hat H_\gamma$ with \jam{(fill in Langevin Hamiltonian)}
\begin{align}
\hat H_0 &= \hbar \omega_0 \hat b^\dag \hat b + \hbar (\omega_0+\Delta) \hat c^\dag \hat c + \hbar (2\omega_0+\Delta) \hat u^\dag \hat u\\
\hat H_I &= \hbar \frac{x}{2} e^{i\phi} \hat u \hat b^\dag \hat c^\dag + \text{h.c.}\\
\hat H_\gamma &= \int \ldots .
\end{align}
% Where $\phi$ is $\pi/2$ ahead of the phase in the previous section 
\jam{(explain why $\phi$ is $\pi/2$ ahead of the phase in the previous section ?)} 
Where $\gamma^c_\text{tot}=\gamma^c_R+\gamma_c$ are the idler's coupling rates through the different ports, determined by the transmittivities $T_{R,c}, T_{l,c}$ for the readout and loss port, respectively. Although $\Delta$ is assumed small compared to $\omega_0$, dichroic optics can have different transmittivities for the signal and idler modes~\cite{}, but I will default to assuming symmetric loss between the modes. The Heisenberg-Langevin equations-of-motion for $\hat b, \hat c$ can then be found as before, where I again: (1) assume the semi-classical and no pump depletion approximations to the pump mode $\hat u\mapsto u=2\chi/x$, (2) enter the Interaction Picture to ignore the effect of $\hat H_0$, and (3) take fluctuating components but leave the $\delta \hat Q(t)$ implicit in the notation,
\begin{equation}\begin{cases}\label{eq:nOPO_EoM}
\dot{\hat{b}}=-i\chi e^{i\phi}\hat{c}^\dagger - \gamma^b_\mathrm{tot} \hat{b} + \sqrt{2\gamma^b_R}\hat{B}_\mathrm{in} + \sqrt{2\gamma_b}\hat{n}^L_b\\
\dot{\hat{c}}=-i\chi e^{i\phi}\hat{b}^\dagger - \gamma^c_\mathrm{tot} \hat{c} + \sqrt{2\gamma^c_R}\hat{C}_\mathrm{in} + \sqrt{2\gamma_c}\hat{n}^L_c.
\end{cases}\end{equation}
To solve these equations-of-motion, I take Fourier transforms and collapse them into one vector equation for $\vec{\hat b}(\Omega)=[\hat b(\Omega), \hat b^\dag(-\Omega), \hat c(\Omega), \hat c^\dag(-\Omega)]^\text{T}$ with similar vectorisation for each port,
\begin{align}
\text{M}_b\vec{\hat b}(\Omega)&=\sqrt{2}\begin{bsmallmatrix}
\sqrt{\gamma^b_R} & 0 & 0 & 0 \\
0 & \sqrt{\gamma^b_R} & 0 & 0 \\
0 & 0 & \sqrt{\gamma^c_R} & 0 \\
0 & 0 & 0 & \sqrt{\gamma^c_R}
\end{bsmallmatrix}\vec{\hat B}_\mathrm{in}(\Omega) + \sqrt{2}\begin{bsmallmatrix}
\sqrt{\gamma_b} & 0 & 0 & 0 \\
0 & \sqrt{\gamma_b} & 0 & 0 \\
0 & 0 & \sqrt{\gamma_c} & 0 \\
0 & 0 & 0 & \sqrt{\gamma_c}
\end{bsmallmatrix}\vec{\hat n}^L_b(\Omega)\\
\text{M}_b&=\begin{bsmallmatrix}
\gamma^b_\mathrm{tot} & 0 & 0 & 0 \\
0 & \gamma^b_\mathrm{tot} & 0 & 0 \\
0 & 0 & \gamma^c_\mathrm{tot} & 0 \\
0 & 0 & 0 & \gamma^c_\mathrm{tot} 
\end{bsmallmatrix}-i\Omega \text{I}+\chi \begin{bsmallmatrix}
0 & 0 & 0 & i e^{i\phi} \\
0 & 0 & -i e^{-i\phi} & 0 \\
0 & i e^{i\phi} & 0 & 0 \\
-i e^{-i\phi} & 0 & 0 & 0
\end{bsmallmatrix}.
\end{align}
Where $\text{I}$ is now the 4 by 4 identity matrix. The input/output relations at the readout port and at the detection loss beamsplitter~\footnote{I assume that the detection loss $R_\text{PD}$ is symmetric between the signal and idler, although this is not necessary.}, are given by formulae similar to Eqs.~\ref{eq:dOPO_IO_readout}~\ref{eq:dOPO_IO_PD}
\begin{align}
\label{eq:nOPO_IO_relations}
\vec{\hat{B}}_\mathrm{out}(\Omega)&=\vec{\hat{B}}_\mathrm{in}(\Omega)-\sqrt{2}\begin{bsmallmatrix}
\sqrt{\gamma^b_R} & 0 & 0 & 0 \\
0 & \sqrt{\gamma^b_R} & 0 & 0 \\
0 & 0 & \sqrt{\gamma^c_R} & 0 \\
0 & 0 & 0 & \sqrt{\gamma^c_R}
\end{bsmallmatrix}\vec{\hat b}(\Omega)\\
\vec{\hat{B}}_\mathrm{PD}(\Omega)&=\sqrt{1-R_\text{PD}}\vec{\hat{B}}_\mathrm{out}(\Omega)+\sqrt{R_\text{PD}}\vec{\hat n}^L_\text{PD}(\Omega).
\end{align}
Using these, I find the quadratures at the photodiode to be
\begin{align}
\vec{\hat X}_\mathrm{PD}(\Omega)&=\text{R}_\text{in}\vec{\hat X}_\mathrm{in}(\Omega)+\text{R}^L_b\vec{\hat X}^L_b(\Omega)+\text{R}^L_\text{PD}\vec{\hat X}^L_\text{PD}(\Omega)\\
\text{R}_\text{in}&=\sqrt{1-R_\text{PD}}\Gamma(\text{I}-\sqrt 2\begin{bsmallmatrix}
\sqrt{\gamma^b_R} & 0 & 0 & 0 \\
0 & \sqrt{\gamma^b_R} & 0 & 0 \\
0 & 0 & \sqrt{\gamma^c_R} & 0 \\
0 & 0 & 0 & \sqrt{\gamma^c_R}
\end{bsmallmatrix}\text{M}_b^{-1}\sqrt{2}\begin{bsmallmatrix}
\sqrt{\gamma^b_R} & 0 & 0 & 0 \\
0 & \sqrt{\gamma^b_R} & 0 & 0 \\
0 & 0 & \sqrt{\gamma^c_R} & 0 \\
0 & 0 & 0 & \sqrt{\gamma^c_R}
\end{bsmallmatrix})\Gamma^{-1}\\
\text{R}^L_b&=-\sqrt{1-R_\text{PD}}\Gamma\sqrt 2\begin{bsmallmatrix}
\sqrt{\gamma^b_R} & 0 & 0 & 0 \\
0 & \sqrt{\gamma^b_R} & 0 & 0 \\
0 & 0 & \sqrt{\gamma^c_R} & 0 \\
0 & 0 & 0 & \sqrt{\gamma^c_R}
\end{bsmallmatrix}\text{M}_b^{-1}\sqrt{2}\begin{bsmallmatrix}
\sqrt{\gamma_b} & 0 & 0 & 0 \\
0 & \sqrt{\gamma_b} & 0 & 0 \\
0 & 0 & \sqrt{\gamma_c} & 0 \\
0 & 0 & 0 & \sqrt{\gamma_c}
\end{bsmallmatrix}\Gamma^{-1}\\
\text{R}^L_\text{PD}&=\sqrt{R_\text{PD}} \text{I}\\
\Gamma&= \frac{1}{\sqrt2}\begin{bsmallmatrix}
1 & 1 & 0 & 0 \\
-i & i & 0 & 0 \\
0 & 0 & 1 & 1 \\
0 & 0 & -i & i
\end{bsmallmatrix}.
\end{align}
Where $\vec{\hat X}=[\hat X_{b,1},\hat X_{b,2},\hat X_{c,1},\hat X_{c,2}]^\text{T}$ now contains the quadratures of both signal and idler.

The total quantum noise response is given by the above matrices and Eq.~\ref{eq:dOPO_Sx_abstract}, where uncorrelated vacuum between the signal and idler modes is assumed (and is valid~\cite{} throughout this thesis), computing the matrix algebra (which agrees with the literature~\cite{}) shows that 
\begin{equation}\label{eq:nOPO_full_freedom}
\text{S}_X(\Omega)=\begin{bsmallmatrix}
1+\frac{8 \gamma^b_R {\gamma^c_\text{tot}} (1-R_\text{PD}) \chi ^2}{\left({\gamma^b_\text{tot}} {\gamma^c_\text{tot}}-\chi ^2\right)^2+\Omega ^2 \left({\gamma^b_\text{tot}}^2+{\gamma^c_\text{tot}}^2+2 \chi ^2\right)+\Omega ^4} & 0 & S_{3,1}^* & S_{4,1}^* \\
0 & S_{1,1} & S_{4,1}^* & -S_{3,1}^* \\
-\frac{\sin (\phi )4 (1-R_\text{PD}) \chi  \sqrt{\gamma^b_R \gamma^c_R}  \left(\chi ^2+\Omega^2+\gamma^b_\text{tot}\gamma^c_\text{tot}+i\Omega\left(\gamma^c_\text{tot}-\gamma^b_\text{tot}\right)\right)}{\left({\gamma^b_\text{tot}} {\gamma^c_\text{tot}}-\chi ^2\right)^2+\Omega ^2 \left({\gamma^b_\text{tot}}^2+{\gamma^c_\text{tot}}^2+2 \chi ^2\right)+\Omega ^4} & S_{4,1} & 1+\frac{8 {\gamma^b_\text{tot}} \gamma^c_R (1-R_\text{PD}) \chi ^2}{\left({\gamma^b_\text{tot}} {\gamma^c_\text{tot}}-\chi ^2\right)^2+\Omega ^2 \left({\gamma^b_\text{tot}}^2+{\gamma^c_\text{tot}}^2+2 \chi ^2\right)+\Omega ^4} & 0\\
\frac{\cos (\phi ) 4 (1-R_\text{PD}) \chi  \sqrt{\gamma^b_R \gamma^c_R} \left(\chi ^2+\Omega^2+\gamma^b_\text{tot}\gamma^c_\text{tot}+i\Omega\left(\gamma^c_\text{tot}-\gamma^b_\text{tot}\right)\right)}{\left({\gamma^b_\text{tot}} {\gamma^c_\text{tot}}-\chi ^2\right)^2+\Omega ^2 \left({\gamma^b_\text{tot}}^2+{\gamma^c_\text{tot}}^2+2 \chi ^2\right)+\Omega ^4} & -S_{3,1} & 0 & S_{3,3}
\end{bsmallmatrix}.\end{equation} 
\jam{(would be nice to partition into block matrices, also, may be too small)} 
Where $\text{S}_X$ is divided into four 2 by 2 blocks: in the upper-left the signal variances and signal-signal covariance (which is zero) form the equivalent of the signal-only degenerate case matrix in Eq.~\ref{eq:dOPO_full_freedom}, in the bottom-right the idler variances and idler-idler covariance (also zero), and in the off-diagonal blocks the signal-idler covariances which are not-zero but all closely related. Some similarities to the degenerate case in Eq.~\ref{eq:dOPO_full_freedom} are seen: all expressions are rational functions, the variances are simplest in the form $1+\chi \cdot(\ldots)$ where the difference from the vacuum vanishes when the squeezer is off, and the covariances obey the Hermitiancy of $\text{S}_X$ and also vanish when the squeezer is off. 

\subsubsection{Results: differences from the degenerate case, threshold, and losses}
\label{sec:nOPO_results}

\begin{figure}
	\centering
	% \includegraphics[width=\textwidth]{}
	\caption{Nondegenerate OPO quantum noise response, showing anti-squeezing in the signal and idler variances, the signal-idler covariances versus frequency, the effect of different losses, and threshold.}
	\label{fig:nOPO_variances}
\end{figure}

% effect of losses, non-effect of pump phase
% resolution of the mystery of zero idler loss
There are two main differences to the degenerate case: (1) all variances are now anti-squeezed (amplified) and do not depend on the pump phase, and (2) while the signal-signal and idler-idler covariances are zero, the two modes are correlated. These differences must come from the change in the degeneracy of the parametric down-conversion.
This explains the change in the covariances, the photons created in the process are entangled regardless of what field they belong to, but why no squeezing is seen for the nondegenerate case will only be clear later~\cite{}.
That the variances are independent of pump phase follows from there only being anti-squeezing, so the noise ellipse in Fig.~\ref{fig:nOPO_config} is symmetric up to the loss rates.

Despite these differences, the effects of (1) the squeezer parameter $\chi$ and (2) the different losses are similar to the degenerate case, as shown in Fig.~\ref{fig:nOPO_variances} where the overall shape of the anti-squeezed curves is similar to Fig.~\ref{fig:dOPO_variances}. 
Firstly, considering the DC ($\Omega=0$) response of the variances, the anti-squeezing is flat near DC, increases with $\chi$, and is singular at threshold $\chi_\text{thr}=\sqrt{\gamma^b_\text{tot}\gamma^c_\text{tot}}$, the geometric mean of the total loss rates for each mode. This means that anti-squeezing in the signal mode is limited by losses in the idler mode and vice versa, meaning that the squeezer cannot be turned on and remain below threshold without some idler loss being present. % -- a feature that will persist later. 
From the gain-loss understanding of threshold, each of the signal and idler experiences the same gain $\chi$~\footnote{There is a factor of two difference between this value of $\chi$ and the degenerate case.} but separate losses, and threshold occurs, not when both experience more gain than loss, but when the two do so on-average. The effect of this threshold is similar to the degenerate case: it bounds the squeezer parameter to prevent lasing~\cite{}.  \jam{(why doesn't lasing start in one mode and not the other when losses are asymmetric?)}
Secondly, the different losses behave similarly as the degenerate case: the detection loss scales the variances and covariances back towards their vacuum values of $1$ and $0$, respectively, while the intra-cavity losses reduce and broaden the peak of the response, shown in Fig.~\ref{fig:nOPO_variances}. Like threshold, the idler losses affect the signal variances and vice versa, but there are some differences, e.g.\ $\gamma^b_R {\gamma^c_\text{tot}}$ in the numerator of $S_{1,1}$ in Eq.~\ref{eq:nOPO_full_freedom} means that idler loss is less impactful than signal loss to the signal variance \jam{(I am claiming this by eye, check this with plots as it disagrees with my conclusion that nIS is tolerant to signal loss but not idler loss (also check nIS signal transfer function))}. 


\subsubsection{Recovering squeezing}
\label{sec:nOPO_combined_readout}
% is this directly relevant to the thesis? yes, to the combined readout preliminary work.
% \subsubsection{Reality of covariance measurements}% already spoken about this?
% % clarify measurement of non-Hermitian X(Omega) and normal operators, set up covariance 
% % cite Schori

\begin{figure}
	\centering
	% \includegraphics[width=\textwidth]{}
	\caption{Nondegenerate OPO with combined readout, i.e.\ measurement of the combined quadrature in Eq.~\ref{eq:Scom_nOPO_eg}, showing how the nondegenerate case can use the correlations between the signal and idler to recover squeezing and behave like the degenerate case.}
	\label{fig:nOPO_combined_readout}
\end{figure}

% These features of the nondegenerate OPO reviewed so far are well-known~\cite{} and useful to know to recognise them in later configurations, but 
One remaining feature of the nondegenerate OPO that is worth discussing for its historical significance and to motivate some of the later \jam{(preliminary)} work in this thesis is the idea of making a combined measurement of the signal and idler modes. The motivation for combining the measurement is that if the signal is measured but the idler mode leaks out of the cavity through the readout port, then that loss decoheres the signal measurement. But if that leaking-out idler mode could be measured then the readout loss would be useful. Consider a linear combination of the four quadratures at the photodetector: \begin{equation}\hat X_\text{com}=\begin{bsmallmatrix}\cos(\psi_2)\cos(\psi_0) & \cos(\psi_2)\sin(\psi_0) & \sin(\psi_2)\cos(\psi_1) & \sin(\psi_2)\sin(\psi_1)\end{bsmallmatrix}\vec{\hat X}_\text{PD}.\end{equation} \jam{(replace with vector Xcom?)}
Where the combination has been normalised such that the vacuum variance remains $1$~\footnote{In experiments~\cite{Schori2001}, the combination is often not normalised and so should be compared to a higher vacuum value, e.g.\ $2$ for $\hat X_{b,1} + \hat X_{c,1}$.}. This operator is observable and its variance can be measured by coherently~\footnote{As opposed to incoherent readout where the fields are detected separately and at best the envelope of the two responses can be obtained~\cite{}.} combining the different modes~\cite{Schori2001}. \jam{(how do you physically combine the two modes if they are at different frequencies, do you just expose the PD to both?)} 
For example, for $\psi_2=\pi/4,\psi_1=\psi_0=0$, the combined variance is
\begin{align}\label{eq:Scom_nOPO_eg}
S_\text{com}&=\frac{1}{2}S_{1,1}+\frac{1}{2}S_{3,3}+\frac{1}{2}(S_{1,3}+S_{3,1})\\
&=\frac{1}{2}S_{1,1}+\frac{1}{2}S_{3,3}+\text{Re}[S_{3,1}].\end{align} 
Where the Hermitiancy of $\text{S}$ has been used.
% EPR significance <-- single sentence, not relevant to thesis
By separately measuring $S_{1,1}$ and $S_{3,3}$~\footnote{Since $[\hat b, \hat c]=0$ \jam{(does this assume a particular size of $\Delta$?)}, the quadratures of signal and idler commute and are therefore simultaneously observable \jam{(although correlated)}, which allows for incoherent readout.}, the combined variance $S_\text{com}$ can be used to measure the real part of the covariance~\cite{}. Similarly, a second measurement with complex linear coefficients can be made to measure the imaginary part~\cite{}. The correlations between the modes from a nondegenerate OPO have been measured in this way to demonstrate quantum entanglement in tests of the Einstein-Podolsky-Rosen (EPR) paradox~\cite{Reid1985,Schori2001,EPR19..}. This association, although not directly relevant to this thesis, is historically important~\cite{} and means that nondegenerate squeezing is often referred to as ``EPR squeezing'' in the literature~\cite{}. %, which is worth being aware of. 

% optimising the combination angles to minimise the combined variance is known as a Weiner filter for the noise
% Something more relevant to this thesis is that. 
% Importantly, 
The combination angles of the combined variance can be optimised, e.g.\ to maximise anti-squeezing or squeezing. The pump phase $\phi$ can also be optimised, but there is some redundancy in $\phi, \psi_0, \psi_1$ since only the relative phase matters. Consider minimising the quantum noise at the photodetector, i.e.\ the combined variance, by changing the combination angles. This is known as constructing a Weiner filter for the quantum noise~\cite{}. Although each variance from the nondegenerate OPO is anti-squeezed, the correlations mean that the combined variance can go below the vacuum value, e.g.\ $\text{Re}[S_{3,1}]$ can be sufficiently large and negative in Eq.~\ref{eq:Scom_nOPO_eg}. This optimisation \jam{(check this, over three angles looks more general than the working in vol 1)} converges on a minimum variance squeezed state with variance given by Eq.~\ref{eq:Scom_nOPO_eg}~\cite{}. Surprisingly, by examining the behaviour when the combination angles are fixed and the pump phase is varied, this variance reduces to that of a degenerate OPO when the losses are symmetric between signal and idler~\cite{}, as shown in Fig.~\ref{fig:nOPO_combined_readout}. In this sense, a nondegenerate OPO can recover the squeezing of a degenerate OPO. Physically, this behaviour is explained by either case producing the same correlations between the output photons from the down-conversion, but that in the nondegenerate case the modes associated with these photons have to be recombined at the output to compare to the degenerate case. This also explains why squeezing appears in the degenerate case but not immediately in the nondegenerate case. %(and balanced: $\psi_2=\pi/4$, symmetric losses)
I will return to the idea of using a combined signal-idler readout later in this thesis, but the freedom and motivation to do so come from this application. %, in Chapter~\ref{chp:idler_readout} to maximise the signal-to--quantum noise ratio \jam{(check that I do)},


\subsection{External squeezing in interferometric gravitational-wave detectors}
\label{sec:external_squeezing}

\begin{figure}
	\centering
	% \includegraphics[width=\textwidth]{}
	\caption{External squeezing configuration. The Faraday isolator is essential to this method but contributes a large portion of the detection and injection losses~\cite{}. The quantum noise ellipses and signal arrows (where the height of the arrow represents the strength of the gravitational-wave signal in the light) show the effect of external squeezing on the sensitivity.}
	% squeezing ellipse and signal arrow plot (+ show the effect of optical loss: injection, detection, intracavity)
	\label{fig:extSqz_config}
\end{figure}

\begin{figure}
	\centering
	% \includegraphics[width=\textwidth]{}
	\caption{Sensitivity curve showing the effect of frequency-independent and frequency-dependent external squeezing on an interferometric detector. The frequency-dependent noise ellipses are shown.}
	\label{fig:extSqz_sensitivity}
\end{figure}

\jam{(Should there be an empty interferometer model proper before this or does the introduction suffice?)}

% Having examined the essential squeezing configurations (the OPOs), 
I will now discuss the benefits of using squeezing in interferometric detectors -- a topic that will remain the focus for the rest of the thesis. In this section, I discuss the existing application of squeezing, external squeezing, used in gravitational-wave detectors today~\cite{} as well as the proposed improvements to this scheme. 
In external squeezing, the vacuum into the readout port (the signal-recycling mirror) $\vec{\hat B}_\text{in}$, is squeezed to reduce the quantum noise in the measurement~\cite{}. This is accomplished by a degenerate OPO separate from the interferometer; the squeezed vacuum from the external squeezer is injected via a Faraday isolator~\cite{} (i.e.\ a directional beamsplitter) behind the signal-recycling mirror, as shown in Fig.~\ref{fig:extSqz_config}. This closes the readout port to the vacuum and replaces it with the squeezed vacuum. The squeezed vacuum from the external squeezer experiences some loss in the injection chain, including \jam{(mostly?)} at the Faraday isolator, which can be collectively attributed to an injection loss port with reflectivity $R_\text{inj}$. This injection loss and the intra-cavity loss in the external cavity limit how much squeezing can be produced. However, the Faraday isolator also increases the detection losses~\footnote{I.e.\ the Faraday isolator causes loss into and out of the readout port.}, it dominates the current detection losses in Advanced LIGO (which total to around $R_\text{PD}=10\%$ compared to total intra-cavity losses of $15\%$ \jam{this seems way too high, check this})~\cite{}. The result is summarised in the noise ellipses and signal arrows in Fig.~\ref{fig:extSqz_config}, where external squeezing is designed to decrease the overall shot noise ellipse at the photodetector by more than the decrease in gravitational-wave signal due to the increased detection loss. Indeed, external squeezing is beneficial, in Advanced LIGO the shot noise--limited sensitivity is improved by a factor of $2$ at $100$~Hz~\cite{tseQuantumEnhancedAdvancedLIGO2019}, shown in Fig.~\ref{fig:extSqz_sensitivity}, using external squeezing.  
% final sentence in this paragraph missing?

I consider three questions about external squeezing: (1) why not squeeze all of the vacuum ports using the same method, (2) what is the effect of quantum radiation-pressure noise (notice the careful use of shot noise in the above paragraph), and (3) should I consider external squeezing in my work?
% why can't you squeeze the other ports
% Firstly, the simplicity of external squeezing comes from it only affecting one vacuum port and not affecting the signal, beyond the increased detection loss. 
Firstly, because external squeezing only squeezes one port, it cannot reduce the shot noise below that associated with the intra-cavity and detection losses, which motivates considering whether these other losses can also be squeezed externally. Although the model has each of these other losses associated with a single port, in reality, they are distributed between the propagation medium and absorption~\footnote{And the vacuum from absorption (propagation) losses cannot be squeezed externally because it comes from the optic (medium) itself by the Fluctuation-Dissipation theorem~\cite{} rather than from a single outside direction.} and transmission \jam{(check terminology)} losses at each optic. Therefore, unlike the readout port at the signal-recycling mirror, the vacuum into these unified loss ports cannot be squeezed externally~\cite{}. 
% from the start of the squeezing section: ``The full story is more complicated, e.g.\ squeezing to decrease shot noise increases quantum radiation pressure noise in the final measurement, but this will be explained at the end of this section when I cover frequency-dependent external squeezing.''
Secondly, as mentioned in Section~\ref{sec:qnoise_GW_IFO}, the quantum radiation-pressure noise in is the opposite quadrature to the shot noise~\cite{} \jam{(fix this, I do not understand why shot noise is not all quadratures)} but in the same quadrature as the gravitational-wave signal since they are both associated with the test mass mechanical mode. Therefore, when the shot noise quadrature of the input vacuum is squeezed externally, the radiation-pressure noise quadrature is anti-squeezed and the radiation-pressure noise is increased in the final measurement. Conversely, the external squeezer's pump phase could be chosen such that the shot noise is anti-squeezed and the radiation-pressure noise is squeezed. Depending on the frequencies of interest either of these could be beneficial; Advanced~LIGO uses shot noise squeezing because shot noise is dominant at and above $100$~Hz as shown in Fig.~\ref{fig:simplifed_QN_response_conventional}. The designs for future detectors (such as LIGO~Voyager~\cite{}, a planned third-generation detector) include improvements to external squeezing through reducing injection losses and using filter cavities to achieve frequency-dependent~\footnote{I.e.\ with pump phase $\phi(\Omega)$.} squeezing~\cite{}. Frequency-dependent squeezing reduces quantum noise at all frequencies by rotating the noise ellipse optimally \jam{(should I give the arctan formula for $\phi(\Omega)$?)}: squeezing radiation-pressure noise at low frequencies (below $10$~Hz), performing no squeezing at the Standard Quantum Limit \jam{(this disagrees with the curves where the whole curve drops uniformly, explain this, e.g.\ look at $\text{S}_X$ with $\phi=\pi/2$)}, and squeezing shot noise at high frequencies (at and above $100$~Hz)~\cite{}, as shown in Fig.~\ref{fig:extSqz_sensitivity}. 
% mention how external squeezing could be added to each of the later models but why I don't (and that I have experimented with this in some preliminary work), and while it may be that it matters more to some configurations than others (e.g. whether squeezing or anti-squeezing)
Finally, frequency-dependent external squeezing is included in many future detectors' designs~\cite{} because it is applicable and beneficial to all quantum noise--limited interferometers~\cite{} \jam{(check this)}, assuming that the added detection loss by including the Faraday isolator does not out-weigh the injected squeezing, which is achievable with current technology~\cite{}. Although the internal mode structure of the detector is arbitrary, the necessary photodetector is always associated with a vacuum port (see Section~\ref{sec:optical_loss_background}) and therefore a frequency-dependent external squeezer can improve the quantum noise by squeezing that vacuum. Even if other squeezers are present in the model, the other vacuum inputs are still uncorrelated with the squeezed vacuum input and so the effect on the measured variance $S_X$ is simply to change the normalisation of the readout vacuum term (i.e.\ the co-efficient of $\text{R}_\text{in} \text{R}_\text{in}^\dag$ in Eq.~\ref{eq:dOPO_Sx_abstract}) from $1$ to the squeezed variance. But this universal applicability also means that the performance of the different configurations need not be judged with the inclusion of external squeezing, since broadly speaking the benefits are similar~\footnote{Technically, the response of each configuration against the readout port loss versus the other losses needs to be considered because if Configuration A has low intra-cavity loss but high readout port loss response and Configuration B has the opposite, then Configuration A will benefit more from external squeezing since it will reduce the dominant loss. However, this is not particularly interesting and is a simple enough addition to future work that I do not include it in this thesis.}. Therefore, to simplify comparing models, I do not include external squeezing. % but note that it should be considered for any realisation of them.

% separate optical losses: intracavity and detection
	% 10% detection, 15% internal in aLIGO (why is internal so high?) -- from Korobko talk

% Generally speaking, the vacuum entering the main port of an arbitrary quantum noise--limited detector can be squeezed externally and injected via a Faraday isolator (a directional beam-splitter) to lower the quantum noise.

	% An example application of squeezing to reduce shot noise is the injection of squeezed vacuum from an external, degenerate squeezer which is used to approximately halve the shot noise in current detectors~\cite{tseQuantumEnhancedAdvancedLIGO2019}. The squeezed vacuum is injected behind the signal-recycling mirror via a Faraday isolator placed outside the signal-recycling cavity as shown in the left panel of Fig.~\ref{fig:coupled_cavities}.


\subsubsection{Caves's amplifier}
\label{sec:cavess_amp}

\begin{figure}
	\centering
	% \includegraphics[width=\textwidth]{}
	\caption{Caves's amplifier configuration with noise ellipse and signal arrows to show the change in sensitivity, as in Fig.~\ref{fig:extSqz_config}. For large amplification, the detection loss is negligible.}
	\label{fig:Cavess_amplifier}
\end{figure}

% motivate investigation of nIS, also dIS using anti-squeezing
% check whether the benefits of anti-squeezing have been mentioned yet
A different external change to improve the quantum noise--limited sensitivity of an interferometer is to add a Caves's amplifier before the detection chain~\cite{}, a single-pass squeezer or an OPO \jam{(check if single-pass squeezer or OPO)} placed after the signal-recycling mirror in some manner as to not amplify the main input vacuum~\footnote{Using the Faraday isolator from external squeezing is problematic because the amplifier should be placed before the main sources of detection loss.}, shown in Fig.~\ref{fig:Cavess_amplifier} \jam{(how is it placed?)}. This configuration is worth mentioning %to understand the difference between placing the squeezer external and internal to the interferometer and
to motivate the use of anti-squeezing to improve loss tolerance. 
In this configuration, the output variance from the interferometer is anti-squeezed such that its quantum noise is well above \jam{(how far?)} the vacuum level, and the gravitational-wave signal is equally amplified because the variance as a whole is amplified. %~\footnote{This is where the difference lies between external and internal squeezing: the interaction with the signal (see the next chapter).}.
Therefore, the amplifier does not affect the signal-to-noise ratio immediately, but upon experiencing the detection loss, the ratio improves because the addition of vacuum affects the noise less than before. 
%the signal and the noise both decrease whereas without the amplifier if the noise was below vacuum level it would have increased \jam{(this is not the only effect, if QN is above vacuum, amplification still helps)}.
To quantify this, let the Caves's amplification be $A_\text{Caves}$ \jam{(explain what this is in terms of squeezer parameter?)}, the output noise variance and signal transfer function be $S$ and $\abs{T}$, respectively, and the detection loss be $R_\text{PD}$, then the improvement in signal-to-noise ratio at the photodetector is
\begin{align}\text{signal-to-noise} = \frac{\sqrt{1-R_\text{PD}}\abs{T}}{\sqrt{R_\text{PD}+(1-R_\text{PD})S}}&\mapsto\frac{A_\text{Caves}\sqrt{1-R_\text{PD}}\abs{T}}{\sqrt{R_\text{PD}+A_\text{Caves}^2(1-R_\text{PD})S}}\\
&=\frac{\sqrt{1-R_\text{PD}}\abs{T}}{\sqrt{R_\text{PD}/A_\text{Caves}^2+(1-R_\text{PD})S}}\\
&\xrightarrow[A_\text{Caves}\rightarrow\infty]{} \frac{T}{\sqrt S}.
\end{align}
That is, in the limit of large amplification, the detection loss is negligible, which is also shown in Fig.~\ref{fig:Cavess_amplifier}. Although, losses and threshold mean that arbitrarily large amplification is not possible. \jam{(What about the interaction of the amplifier and radiation pressure noise?)} Similarly to external squeezing, because the Caves's amplifier is universally applicable, I will not include it in my models~\footnote{Where again, technically, the improvement to two configurations is not necessarily the same since it depends on whether the detection loss dominates the noise spectrum.}. There are two things to remember from this configuration: (1) the external amplifier only affects signal and noise equally because they approach it equally (both from the black-box \jam{(colloquial?)} of the interferometer readout) and (2) detection losses can be addressed by amplification, i.e.\ anti-squeezing.

% anything to say for the end of the section? no, just use the summary.

%%%%%%%%%%%%%%%%%%%%%%%%%%%%%%%%%%%%%%%%%%
\section{Chapter summary}

In this chapter, I have revised the theory of quantum noise necessary to describe the benefits of squeezing for gravitational-wave interferometers and assess the feasibility of the configurations in the following chapters. Firstly, I set up the mathematical formalism of quantum noise, discussed how optical loss leads to decoherence, and described the quantum noise response of an interferometer. Secondly, I introduced the idea of squeezing and demonstrated the analytic, Hamiltonian modelling that I will use throughout this thesis. I explained the notion of threshold, the effects of different optical losses on the quantum noise response, and the effect of pump phase, for both degenerate and nondegenerate optical parametric oscillators (OPOs) -- the quintessential squeezing configurations. Finally, I explained how external squeezing is currently used in Advanced~LIGO and how the use of squeezers externally (including frequency-dependent external squeezing and Caves's amplification) is of universal interest to the improvement of interferometers, and therefore I will not include them in my models when comparing configurations that would benefit roughly equally from them. % and the designs of future detectors.


