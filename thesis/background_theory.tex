\chapter{Background theory of quantum noise and squeezing in gravitational-wave detectors} % Background: theory of quantum noise in interferometric gravitational-wave detectors
% change title?
\label{chp:background_theory}

% set up EVERYTHING the audience needs to know
% set up the problem sufficiently and specifically, the problems with current proposals are detailed in the next chapter

% this chapter has a lot of ideas, need to figure out the logical flow -- look at research proposal and other theses --> talk to supes

% do I need to start as far back as MJ's thesis? i.e. back at the quantisation of light, definition of quadratures etc.? --> supervisors say no for honours thesis

%%%%%%%%%%%%%%%%%%%%%%%%%%%%%%%%%%%%%%%%%%
% chapter introduction

% \jam{(Change chapter title to: ``Quantum noise and squeezing''?)}
In this chapter, I will review the necessary physics to describe quantum noise and squeezing. Firstly, in Section~\ref{sec:QM_nature_of_light}, I review the quantum-mechanical nature of light and introduce squeezing using the conventional mathematical formalism that starts from the Heisenberg Uncertainty Principle. In Section~\ref{sec:Hamiltonian_modelling}, I then show how squeezing can be understood using Hamiltonian models of degenerate and nondegenerate optical parametric oscillators as I use similar Hamiltonian modelling later in my work. %Demonstrating the properties of squeezing and how to derive them using these models is the focus of this chapter.
Finally, in Section~\ref{sec:qnoise_GW_IFO}, I explain the application of squeezing states to improve the quantum noise--limited sensitivity of current gravitational-wave detectors.
% will motivate squeezing by explaining how it is used in current gravitational-wave detectors to improve the quantum noise--limited sensitivity. % and set up the motivation for the more complex uses of squeezing in the next chapter and my work.

% I will use the conventional treatment of quantum noise widely available in the quantum optics literature~\cite{danilishinQuantumMeasurementTheory2012,miaoFundamentalQuantumLimit2017}, which builds from the Heisenberg Uncertainty Principle to the quantum noise response of a detector. 
% I will then introduce the technique of squeezing, which can favourably manipulate the quantum noise, using a Hamiltonian model that I will continue to use throughout this thesis. I will derive the response of the well-studied optical parametric oscillator (OPO) in both the degenerate and nondegenerate cases. These models will be expanded upon by adding additional modes in later chapters, so it is well-worth carefully understanding the base model. 

% ``Introduction : here I want to know three things.
	% (1) What is the point of the chapter? The key things that this 1 paragraph should contain is that you want to introduce the quantum mechanical nature of light and introduce the mathematical framework for squeezing (beginning of the chapter max 3 pages).
	% (2) What else is needed to understand squeezing? degenerate and nondegenerate squeezing (middle of the chapter)- this should infact be the focus of the chapter.
	% (3) Why do I care? because I need to understand the relevance to current GWDs i.e. external squeezing.'' - VA

% I will first review the background theory of quantum noise and squeezing in Chapter~\ref{chp:background_theory}, to have the tools to describe the quantum noise response of a detector. In this chapter, I will demonstrate the analytic, Hamiltonian modelling, that I will then use throughout the thesis, on the simple, well-known cases of degenerate and nondegenerate optical parametric oscillators (OPOs, i.e.\ squeezed cavities)~\cite{}. 

%%%%%%%%%%%%%%%%%%%%%%%%%%%%%%%%%%%%%%%%%%
\section{Quantum-mechanical nature of light}
\label{sec:QM_nature_of_light}
% \section{Quantum noise}
% \label{sec:qnoise}

% ``Quantum nature of light (beginning of the chapter max 3 pages) --> Include things like quantisation of E-field, operators, quantised Hamiltonian, Heisenberg Uncertainity principle, Fourier domain operators, vaccuum states, squeezed states (the last 2 with pictures). All these can be paragraphs with (or without) headings because that improves readability. Currently I find it hard to read because its blocky unseparated text everywhere. Here, you can also include nonlinear interactions in a crystal if you like but I would keep this to 2 paragraphs.'' - VA

Quantum noise in a detector is caused by the quantum-mechanical, fundamental uncertainties in the state of the light in the detector~\cite{PhysRevD.23.1693,corbitt_2003}. %, e.g.\ the amplitude of the light hitting the photodetector~\cite{}. %This covers many different phenomena, from vacuum fluctuations to back-action caused by a measurement~\cite{}.
% For example, the amplitude and phase of a particular cavity mode are conjugate quantities related by the Heisenberg Uncertainty Principle, meaning that they can never be simultaneously, exactly known which causes quantum noise in the measurement~\cite{}. %This uncertainty in the amplitude and phase of the light in different parts of the detector will appear as quantum noise in the final measurement. %Throughout this thesis, I will just focus on uncertainty in the light incident on the photodetector and not consider the complications of readout. % oddly placed sentence
To understand the quantum noise caused by quantum uncertainty in the light of a cavity-based detector, I quantise the electromagnetic field following the conventional formalism (e.g.\ see Refs.~\cite{danilishinQuantumMeasurementTheory2012,walls_1995} throughout this chapter). Let the creation and annihilation operators\jam{(is it clear that I am talking about these because I care about the quantum behaviour?)} of a resonant mode of an optical cavity be $\hat{a}^\dag$ and $\hat{a}$, respectively, that act on the Hilbert space of states of the light inside the cavity such that $\hat{a}\lvert\emptyset\rangle=0$ where $\lvert\emptyset\rangle$ is the vacuum state. These bosonic operators obey the commutation relation $[\hat{a},\hat{a}^\dag]=1$. 
In the Heisenberg Picture with their time-dependence implicit, these operators evolve according to the Heisenberg equation-of-motion $\dot{\hat{a}}=-\frac{i}{\hbar} [\hat a, \hat H]$, and its conjugate equation, given a quantised, time-independent Hamiltonian $\hat H$ and the reduced Plank constant $\hbar$. In the absence of any interaction, the mode created by $\hat a^\dag$ evolves according to the harmonic oscillator Hamiltonian $\hat H_0=\hbar\omega(\hat{a}^\dag\hat{a}+\frac{1}{2})$\jam{(check)} where $\omega$ is the angular frequency of the mode corresponding to the energy $\hbar\omega$ per each occupation of the mode~\footnote{I will ignore the vacuum energy ($\frac{\hbar\omega}{2}$) by re-normalisation or entering the Interaction Picture to ignore $\hat H_0$ when evolving the operators.}.

% \subsubsection{Quadratures and the Heisenberg Uncertainty Principle}
% quadratures and HUP
Consider the linear combinations $\hat{X}_\theta=\frac{1}{\sqrt 2}(e^{-i \theta}\hat{a}+e^{i \theta}\hat{a}^\dag)$ for $\theta\in[0,2\pi)$ and let the ``amplitude'' $\hat{X}_1=\hat{X}_{\theta=0}$ and ``phase'' $\hat{X}_2=\hat{X}_{\theta=\frac{\pi}{2}}$ \emph{quadratures}~\footnote{Although they are only associated to the amplitude and the phase of the electric field, respectively, by convention~\cite{danilishinQuantumMeasurementTheory2012}.} be defined such that they span these combinations and obey $[\hat{X}_1,\hat{X}_2]=i$. The annihilation/creation operators can be converted to the quadrature picture by the matrix $\Gamma = \frac{1}{\sqrt2} \begin{bsmallmatrix}1 & 1 \\-i & i\end{bsmallmatrix}$ such that $(\hat{X}_1,\hat{X}_2)^\text{T}=\Gamma(\hat{a},\hat{a}^\dag)^\text{T}$. %\jam{(I care about them principally because they are an observable way to get the state of the light.)
Unlike $\hat a$ and $\hat a^\dag$, these quadrature operators are observable because they are Hermitian and therefore they can be used to measure the state of a detector (and, ultimately, a gravitational wave) which is why I am interested in them~\footnote{The exact method of measuring the quadratures at the photodetector is not of concern, but a balanced homodyne readout scheme will suffice throughout this thesis~\cite{danilishinQuantumMeasurementTheory2012}.}. The uncertainty in a measurement of $\hat{X}_1$ is given by $\sigma_{X_1}=\sqrt{\ev{\hat{X}_1^2}-\ev{\hat{X}_1}^2}$ where $\ev{\hat{X}_1}=\langle\emptyset\lvert\hat{X}_1\rvert\emptyset\rangle$. However, the Heisenberg Uncertainty Principle states that
\begin{equation}\label{eq:HUP_time_domain}
\sigma_{X_1}\sigma_{X_2}\geq\frac{1}{2}\abs{\ev{[\hat{X}_1,\hat{X}_2]}}=\frac{1}{2}.
\end{equation}
Therefore, the quadratures can never be simultaneously known exactly and there is always uncertainty in at least one of them. % that can lead to quantum noise. % and a precise measurement of $\hat X_1$ will increase uncertainty in $\hat X_2$~\cite{}.
% Therefore, the Heisenberg Uncertainty Principle for the amplitude quadrature $\hat{X}_1$ and phase quadrature $\hat{X}_2$ states~\cite{} that $\sigma_{X_1}\sigma_{X_2}\geq\frac{1}{2}\abs{\ev{[\hat{X}_1,\hat{X}_2]}}=\frac{1}{2}$\jam{(check this, does it disagree with the vacuum value being 1?)} where the uncertainty is $\sigma_\mathcal{O}=\sqrt{\ev{\hat{\mathcal{O}}^2}-\ev{\hat{\mathcal{O}}}^2}$ and $\ev{\hat{\mathcal{O}}}=\langle\emptyset\lvert\hat{\mathcal{O}}\rvert\emptyset\rangle$ is the vacuum expectation value. These quadratures are only loosely related to the amplitude and the phase of the light, respectively~\cite{danilishinQuantumMeasurementTheory2012}. More importantly, these time-domain quadratures are Hermitian and therefore observable~\footnote{Which is what is meant by ``measuring the phase'' of the light. Even though absolute phase is not real, the phase quadrature is observable.} which is used to ultimately measure the gravitational-wave signal.
% This means that any measurement that depends on both quadratures must contend with there always being some quantum noise present. This is the case for gravitational-wave interferometry; the quantum noise can be reduced but never eliminated. %The exact method of detection of the quadratures is not of concern to this thesis, but it suffices to say that a scheme like homodyne readout will suffice for all applications that I consider~\cite{}.

% \subsubsection{Fourier-domain quadratures and the vacuum state}
% FT (brief!)

The \emph{Fourier-domain} counterparts to the above time-domain quadratures matter since they contain spectral information about a gravitational-wave signal~\footnote{Transient gravitational-wave signals have durations on the order of seconds, but the response of an interferometer is on the time-scale of the round-trip time of the arms (e.g.\ $\sim10~\mu\text{s}$)~\cite{bond_2010}. Therefore, the steady-state approximation to the interferometer required for the Fourier transform is valid.}. % and Fourier transforms are the primary method of solving Heisenberg equations-of-motion throughout this thesis. 
Let $\hat{\mathcal{O}}(\Omega) = \int_{-\infty}^\infty \frac{1}{\sqrt{2}} \hat{\mathcal{O}} e^{-i\Omega t}\mathrm{d}t$ be the Fourier transform of $\hat{\mathcal{O}}$ with the Fourier frequency $\Omega$. Then, the Fourier-domain quadratures are $\hat{X}_\theta(\Omega)=\frac{1}{\sqrt{2}}(e^{-i \theta}\hat{a}(\Omega)+e^{i \theta}\hat{a}^\dag(-\Omega))$ where the sign of the last argument was flipped to account for the $e^{-i\Omega t}$ in the Fourier transform~\footnote{Although the Fourier-domain quadratures are not Hermitian, they are indirectly observable, either through their time-domain counterparts or through measurements that can derive their real and imaginary parts separately which is possible due to the condition $\hat{X}^\dag(\Omega)=\hat{X}(-\Omega)$~\cite{SCHUMAKER1986317}. Therefore, I will treat them as being observable.}.
In the Fourier domain, the time-domain variance $\sigma_\mathcal{O}^2$ becomes a (single-sided) spectral density $S_\mathcal{O}(\Omega)\delta(\Omega-\Omega')=\ev{\hat{\mathcal{O}}(\Omega)\circ\hat{\mathcal{O}}^\dag(\Omega')}$~\footnote{Here, $A\circ B=\frac{1}{2}(A\cdot B+B\cdot A)$.}.
The spectral densities of the quadratures obey a similar Heisenberg Uncertainty Principle as Eq.~\ref{eq:HUP_time_domain}, $\sqrt{S_{X_1}S_{X_2}}\geq1$. 
This inequality is achieved as an equality by the uncorrelated vacuum state $\lvert\emptyset\rangle$ since for the associated quadratures $\ev{\hat{X}_i(\Omega)\circ\hat{X}_j^\dag(\Omega')}=\delta_{i,j}\delta(\Omega-\Omega')$ where $\delta_{i,j}$ is the Kronecker delta and therefore $\sqrt{S_{X_i}^\text{vac}}=1$ for $i=1,2$~\cite{danilishinQuantumMeasurementTheory2012}\jam{(check that this is unitless)}. This means that there is equal uncertainty in each of the quadratures $\hat{X}_\theta(\Omega)$ and can be visualised as the vacuum state having a ``noise ellipse'' with equal semi-axes lengths of one in the $(\hat{X_1}(\Omega),\hat{X_2}(\Omega))$ plane as shown in Fig.~\ref{fig:ballandstick_simple}, where the radius of the noise ellipse in a direction $\theta$ represents the uncertainty in $\hat{X}_\theta(\Omega)$.

% For $\hat{a}(\Omega)$ corresponding to an uncorrelated vacuum state, let $\ev{\hat{X}_i(\Omega)\circ\hat{X}_j^\dag(\Omega')}=\delta_{i,j}\delta(\Omega-\Omega')$ where $\delta_{i,j}$ is the Kronecker delta, and therefore $S_X^\text{vac}=1$~\footnote{Because the units of $\delta(\Omega-\Omega')$ and $\hat X(\Omega)$ are time by their integrals over $\Omega$ and $\hat X$ being unitless, therefore the units of $S_X$ are time or $1/\text{Hz}$\jam{(therefore there is a units error in the vacuum value?)}.} can be shown from the commutation relations above~\cite{danilishinQuantumMeasurementTheory2012}.

% Finally, there is a hidden assumption in these spectral density expressions: I have assumed that the quadrature operators are corresponding to the time-domain quantum fluctuations of the amplitude and phase of the light around its time-independent classical expectation value, i.e.\ I am considering Fourier transforms of $\delta\hat{X}(t)=\hat{X}(t)-\ev{\hat{X}}$ where I have made the time-dependences explicit for clarity. However, I will leave the $\delta$'s implicit to compactify notation, henceforth.

	% At high frequencies, the sensitivity of current detectors is limited by quantum noise arising from the quantum nature of light~\cite{danilishinQuantumMeasurementTheory2012QuantumMeasurementTheory2012}. By the Heisenberg Uncertainty Principle, the amplitude and phase of a quantum state of light (including the vacuum) cannot be exactly known at the same time. Uncertainty in the amplitude of the light within the interferometer leads to quantum radiation pressure noise, while uncertainty in its phase leads to so-called “shot noise”. 

\subsection{Squeezing}
\label{sec:squeezing_background}

\begin{figure}
	\centering
	\includegraphics[angle=-90,width=0.9\textwidth]{ball_and_stick.pdf}
	\caption{``Ball-and-stick'' illustration of squeezing. The red ellipse represents the uncertainty around the coherent amplitude represented by the end of the blue stick; the coherent amplitude is zero for the vacuum state but otherwise, the uncertainties in the vacuum state are the same as the coherent state. The lengths of the ellipse's semi-axes represent the uncertainty $\sqrt{S_{X_i}}$ in each quadrature $\hat X_i(\Omega)$. Squeezing the noise, for either the vacuum or coherent state, changes these lengths such that their product is preserved, or increased if optical losses are present, to obey the Heisenberg Uncertainty Principle.}
	\label{fig:ballandstick_simple}
\end{figure}
\begin{figure}[ht]
	\centering
	\includegraphics[angle=-90,width=0.6\textwidth]{PDC.pdf}
	\caption{Parametric down-conversion, showing the energy levels of the degenerate (left panel) and nondegenerate (right panel) processes. In either case, the process conserves energy, i.e.\ $\hbar\omega$ for angular frequency $\omega$, and the created photons are squeezed and entangled.}
	\label{fig:PDC_deg_and_nondeg}
\end{figure}

% return to HUP
% ball-and-stick plot
Squeezing refers to a broad range of technologies that decrease uncertainty in a desired quantity by increasing uncertainty in its conjugate, less desired quantity while still obeying the Heisenberg Uncertainty Principle~\cite{Andersen_2016}. For example, decreasing (``squeezing'') uncertainty in the amplitude quadrature by some factor $e^{-r}<1$ by increasing (``anti-squeezing'') uncertainty in the phase quadrature, $\pi/2$ away in the quadrature basis, by $e^r>1$ still satisfies the Heisenberg Uncertainty Principle $(\frac{S_{X_1}}{e^r}) (e^rS_{X_2})\geq1$.
% \begin{equation}
% S_{X_1}S_{X_2}\geq1\implies (\frac{S_{X_1}}{e^r}) (e^rS_{X_2})\geq1\label{eq:HUP_squeezed}.
% \end{equation} 
% If a vacuum or coherent state is squeezed in this manner to reduce the noise in one quantity below the vacuum value of one, then the
This can be expressed as a ``squeezing operator'' trading the uncertainties of a vacuum or coherent state~\footnote{I will only consider squeezed vacuum states in this thesis.} and squeezing their noise ellipses in Fig~\ref{fig:ballandstick_simple}~\cite{danilishinQuantumMeasurementTheory2012}~\footnote{This correlates the quadratures that lie off the semi-axes of the noise ellipse, e.g.\ $\hat{X}_\theta$ and $\hat{X}_{\theta'}$ for $\theta\neq0,\pi/2$ for $0<\theta<\pi$ and $\theta'$ similar.}.
% A illustration of this is shown in Fig~\ref{fig:ballandstick_simple} where the trade-off of uncertainties is visualised as a literal squeezing of the noise ellipse, where the product of the semi-major and semi-minor axes is maintained (or increased) to obey the Heisenberg Uncertainty Principle.
% Squeezing is advantageous when the noise in one quantity affects the measurement more than the other. %, and this is true in gravitational-wave detectors which I will discuss in Section~\ref{sec:external_squeezing}.
% where, e.g.\ with homodyne readout, one quadrature is measured at the photodetector and noise in the other quadrature does not affect the measurement~\cite{}. The full story is more complicated, e.g.\ squeezing to decrease shot noise increases quantum radiation pressure noise in the final measurement, but this will be explained later in Section~\ref{sec:external_squeezing}.
Squeezing can also be explained using sideband theory~\cite{danilishinQuantumMeasurementTheory2012} which I do not discuss.
% mention non-Gaussian squeezing, what is it actually, a different Hamiltonian?
When a squeezed state encounters optical loss, the squeezing is reduced as the decoherence of the state reduces correlations and pulls the uncertainties back towards their vacuum values of one, however, the anti-squeezed and squeezed uncertainties are affected differently which increases their product in the Heisenberg Uncertainty Principle~\footnote{This will be explained later as optical loss mixing the light with the vacuum as shown in Fig.~\ref{fig:beamsplitter_loss}.}. %, i.e.\ makes the noise ellipse closer to a circle with radius one around the coherent amplitude

% \subsubsection{Squeezing using non-linear interactions in a crystal}
% production of squeezed states, PDC, give a citation to answer Ilya's question
% Although I will model the crystal as a ``black-box'' that contributes some term to the Hamiltonian
% https://www.rp-photonics.com/nonlinear_polarization.html
Optical squeezing can be achieved via a variety of technologies, but I will focus on the production of squeezed states using \emph{nonlinear crystals}~\cite{Andersen_2016}. In a crystal with a quadratic polarisability $\chi^{(2)}$~\footnote{When classically exposed to an electric field $\vec E$ it produces an electric field with $i\text{th}$ component $\varepsilon_0 (\sum_j \chi_{ij}^{(1)} E_j + \sum_{j,k} \chi_{ijk}^{(2)} E_j E_k)$ where $\chi^{(1)}$ is the linear polarisability.}, the process of parametric down-conversion can occur\jam{(why does quadratic nonlinearity make PDC happen? - Ilya's question)}, where a photon of a\jam{(particular?)} pump (angular) frequency $\omega_p$ is annihilated to create two entangled, squeezed photons at the ``signal''~\footnote{There is potential confusion later between the signal mode of light created by the squeezer and the light in the detector that contains the gravitational-wave signal. I will clarify wherever necessary.} $\omega_0$ and ``idler'' $\omega_0+\Delta$ frequencies such that $\omega_p=2\omega_0+\Delta$ to conserve energy~\cite{Klyshko_1976}. The frequency difference $\Delta$, chosen by how the crystal is pumped, is small in comparison to the other frequencies in the system, and when $\Delta=0$ the produced photons are energetically degenerate which changes the mode structure of the system; the energy level structure of this process is shown in Fig.~\ref{fig:PDC_deg_and_nondeg}. %I will prove shortly that this process squeezes the vacuum at the produced frequencies. %, but that they are entangled follows from their shared point of creation.
These degenerate and nondegenerate processes can be used to manipulate the quantum noise differently. % in different ways which I will explore below.


%This is justified by the same reasoning as in Section~\ref{sec:coupled_cavity_approximation}\jam{(do I need to say it again?)}, %, the cavities select their resonant frequencies and all other frequencies can be ignored~\cite{}. 
% This means that I assume that wherever the squeezer crystal is placed, the pump frequency and the two produced frequencies (called the signal~\footnote{There is potential confusion between the signal mode of light created by the squeezer and the light in the detector that contains the gravitational-wave signal, but I have tried to be clear whenever necessary.} $\omega_0$ and idler $\omega_0+\Delta$ frequencies) are resonant.  %E.g.\ when talking about the signal transfer function, it can be that the light in the signal mode is measured to estimate the gravitational-wave signal, although the signal transfer function is named for the latter. 
% These terms are too prevalent in the literature~\cite{} not to use them,

% I make two brief clarifications.
% % Before I construct a Hamiltonian model of squeezing, I need to make three clarifications. % to reduce confusion. 
% % although one can imagine a green laser beam incident on a crystal squeezing two slightly differently coloured red lasers also incident
% Firstly, squeezing is often used to squeeze the vacuum of virtual photons. % at the signal and idler (or just the signal if degenerate) modes. % which can be initially confusing. 
% This corresponds to the coherent amplitudes of the incident modes being zero, i.e.\ the ellipse in Fig.~\ref{fig:ballandstick_simple} would be centred at the origin. This means that I need to distinguish between the squeezed vacuum from a squeezer and true vacuum, where the ellipse in Fig.~\ref{fig:ballandstick_simple} is circular, centred at the origin, and corresponds to the $S_X=1$ value. 
% can these latter general comments (frequency, dB) be moved into dOPO section where they will make more sense?
% Secondly, throughout this thesis, I consider the quantum noise response of a detector and talk about the quantum noise being squeezed (suppressed) or ``anti-squeezed'' (amplified) at different frequencies. This should not be confused with the down-conversion occurring with different frequencies, rather, the squeezer always interacts with the same frequencies of the single-mode approximation but the response considers the spectra of their respective quadratures (i.e.\ the distinction is between the frequency annihilated by ${\hat a}$ and its Fourier spectrum with respect to frequency $\Omega$).

% In the squeezing literature, there are a few ways to quantify the amount of squeezing produced~\cite{}. For example, if $S_X$ is squeezed by a factor $e^r$, then typically the quantum noise response $\sqrt S_X$ is plotted logarithmically in amplitude-decibels (dB) as $20 \log_{10}(\sqrt {S_X e^{-r}})=20 \log_{10}(\sqrt S_X) - 20\log_{10}(e^{r/2})$ with the vacuum ($S_X=1$) at $0$~dB. However, I reserve the term ``squeezing factor'' for the parameter $\chi$ seen shortly in the Hamiltonian model, instead of these factors $e^r, e^{r/2}, 20\log_{10}(e^{r/2})$ or various other, related quantities in the literature~\cite{}.

% just a paragraph about sidebands?, I do not go further into this because I do not use the formalism (preferring the operator+Interaction Picture which achieves the same effect of studying frequency offsets)

	% An established technology to reduce shot noise at high frequencies is squeezing~\cite{danilishinQuantumMeasurementTheory2012QuantumMeasurementTheory2012,chuaQuantumEnhancementKm2015}. Squeezing manipulates the quantum state of light and the vacuum to trade-off phase uncertainty for amplitude uncertainty; this decreases shot noise by increasing quantum radiation pressure noise~\footnote{Squeezing only improves sensitivity if the interferometer readout is suitably designed to exploit the trade-off between shot noise and radiation pressure noise.}. Squeezed states of light can be created in a crystal with nonlinear polarisability by parametric down-conversion which annihilates a photon at a pump frequency and creates two entangled photons with ``squeezed" uncertainties. To conserve energy, the sum of the frequencies of the created photons must equal the pump frequency. This process is degenerate if the frequencies of the created photons are equal and is nondegenerate otherwise.
	% The fact that nondegenerate squeezing involves three distinct frequencies (the pump and the two created frequencies) rather than two is an important change in the symmetry of the system that I will revisit later.


\section{Hamiltonian models of squeezing}
\label{sec:Hamiltonian_modelling}

% ``Cavity equations of motion --> Degenerate internal squeezing. How do I go from the langevin equations of motion to squeezing? there's no need to put every equation under the sky here, but the main equations about recovering squeezing. How about 2 plots here/ Possibly Frequency vs squeezing levels, pump parameter vs squeezing. This has been done to death everywhere. Use references as cited in Yap or Mansell or Chua or Sarre or Gould theses, they'll help you cut down the text as well. If you wanted, you can include a sentence or two (but not paragraphs each!) about the limits of this model, a little bit on effect of losses, what does threshold mean etc .
% Same for nondegenerate squeezing'' - VA

% To better understand squeezing and show that it can be produced via the process of parametric down-conversion, I will now model the quintessential squeezing configurations of degenerate and nondegenerate optical parametric oscillators (OPOs). 
Cavity-based squeezing is currently used to generate squeezed states for gravitational-wave detectors~\cite{aasietal2013}. Here, I will introduce the Hamiltonian modelling of cavity-based squeezing that I will use in my work.

\subsection{Degenerate OPO}
% this is quite long for a subsection?

\begin{figure}
	\centering
	\includegraphics[angle=-90,width=\textwidth]{OPOs_config.pdf}
	\caption{Degenerate (left panel) and nondegenerate (right panel) optical parametric oscillator (OPO) configurations with all modes labelled. Intra-cavity loss (e.g.\ $\gamma_b$) occurs via the mechanism in Fig.~\ref{fig:beamsplitter_loss} henceforth. The non-linear crystal is labelled with $\chi^{(2)}$ to represent the quadratic polarisability. %, it (anti)squeezes the vacuum input into the cavity.
	% The degenerate OPO squeezes the vacuum. %as shown by the noise ellipse from Fig.~\ref{fig:ballandstick_simple}.
	% The nondegenerate OPO anti-squeezes all quadratures of the vacuum of the signal and idler and correlates the two modes.
	}
	\label{fig:OPOs_config}
\end{figure}

A degenerate optical parametric oscillator (OPO) consists of a non-linear crystal operating the degenerate down-conversion process inside of an optical cavity as shown in Fig.~\ref{fig:OPOs_config}~\cite{PhysRevA.30.1386}. The cavity increases the number of passes of the squeezer that the light makes and therefore increase the squeezing\jam{(I think it is important to say why the cavity is added)}. The vacuum state entering the cavity through the readout port exits squeezed by the crystal and its quadratures are measured at a photodetector.
% But I examine this configuration here for two main reasons: (1) to show that the photons produced in parametric down-conversion are squeezed and (2) to demonstrate the analytic, Hamiltonian model that I will use throughout the rest of this thesis (this approach is used widely in the literature~\cite{}).  
% In Fig.~\ref{fig:OPOs_config}, the pump laser is shown incident orthogonally on the crystal to the beam path, this is only somewhat realistic~\cite{} and spatial aspects of the configuration are not included in the model. 
% Although this process can create photons of many different frequencies\jam{(check)}, I will study the operation of this crystal inside optical cavities and therefore make a single-mode approximation to select only the resonant cavity modes and consider a particular pump frequency $2\omega_0+\Delta$, carrier frequency $\omega_0$, and frequency difference $\Delta$. 

% OPO = squeezed cavity with no seed light
% In the literature, a distinction is sometimes made between when the squeezed field is ``seeded'' or not, i.e.\ whether the degenerate OPO produces squeezed vacuum or squeezed light, and in the latter case the OPO is instead called an optical parametric amplifier (OPA). Since I am interested in the quantum noise response, i.e.\ the response to vacuum noise, this is indeed the former case. 
% But since I am interested in the quantum noise response of this configuration, i.e.\ how vacuum noise appears in the final measurement, I do not need to worry about this distinction. % is this even necessary to bring up then?

\subsubsection{Analytic model}
\label{sec:dOPO_model}
% analytic model to demonstrate Hamiltonian method

\begin{figure}
	\centering
	\includegraphics[angle=-90,width=0.7\textwidth]{BS_loss_model.pdf}
	\caption{Beamsplitter model of optical loss. The light $\hat X_\text{in}$ loses energy into the environment at some rate and uncorrelated vacuum $\hat X_\text{vac}$ is introduced at some rate; the Fluctuation-Dissipation Theorem states that these two rates $R_l$ are equal. The transmission $T_l$ and reflection $R_l$ coefficients obey $T_l+R_l=1$ to conserve energy which is proportional to the quadrature squared. I use the equivalent ``loss port'' convention $R_l\leftrightarrow T_l$ for the intra-cavity losses in Fig.~\ref{fig:OPOs_config}.}
	\label{fig:beamsplitter_loss}
\end{figure}

% I model the degenerate OPO using the Hamiltonian method from Refs.~\ref{,}.
% To simplify the model, I make a single-mode approximation to the light in the cavity and assume that it is only in the cavity's fundamental resonant mode given by annihilation operator $\hat b$ at (angular) frequency $\omega_0$ and interacts with the pump mode $\hat u$ at $2\omega_0$ which is also resonant. This approximation is valid when considering frequencies within the first cavity resonance, which is what I consider~\cite{}.
% Let the other modes of this system be as shown in Fig.~\ref{fig:OPOs_config}: $\hat B_\text{in}$ for the vacuum entering the readout port, $\hat B_\text{out}$ for the mode leaving the readout port, and $\hat B_\text{PD}$ for the mode incident on the photodetector~\footnote{I ignore spatial propagation of these external modes because it does not affect the noise.}.
% I will derive $\hat B_\text{PD}$ as a function of the readout and loss vacuum inputs, which defines the noise response of the system. %, and to do so I will first solve for $\hat b$ and then find the solution at the photodetector. 
% ~\footnote{Spatial propagation is not part of this model, I mean that I will use the appropriate input/output relations~\cite{} to find the detected mode.}

The \emph{Hamiltonian} of this system is given by $\hat H = \hat H_0 + \hat H_I + \hat H_\gamma$, where $\hat H_0$ gives the decoupled dynamics of the pump mode with annihilation operator $\hat u$ at $2\omega_0$ and the cavity mode $\hat b$ at frequency $\omega_0$~\footnote{I make a single-mode approximation to the light in the cavity and assume that it is only in mode $\hat b$ and interacts with the pump mode which is also resonant. This approximation is valid when considering frequencies within the cavity's bandwidth~\cite{walls_1995}.}, $\hat H_I$ gives the parametric down-conversion in the crystal~\footnote{As well as the reverse process of ``second-harmonic generation'' where two $\omega_0$ photons combine to create a $2\omega_0$ photon.}, and $\hat H_\gamma$ gives the coupling to the vacuum entering the readout port ($\hat B_\text{in}$) and describes the intra-cavity loss~\cite{korobkoQuantumExpanderGravitationalwave2019}, \jam{(should Bout appear?)}
\begin{align}
\hat H_0 &= \hbar \omega_0 \hat b^\dag \hat b + \hbar 2 \omega_0 \hat u^\dag \hat u\\
\hat H_I &= i \hbar \frac{g}{2} (e^{i\phi} \hat u (\hat b^\dag)^2 - e^{-i\phi} \hat u^\dag \hat b^2)\\
\hat H_\gamma &= i\hbar \sqrt{2}\int \biggl( \sqrt{\gamma^b_R}\left(\hat{b}^\dag(\Omega)\hat{B}_\text{in}(\Omega)-\hat{b}(\Omega)\hat{B}_\text{in}^\dag(\Omega)\right) + \sqrt{\gamma_b}\left(\hat{b}^\dag(\Omega)\hat n^L_b(\Omega)-\hat{b}(\Omega)\hat{n}^{L\dag}_b(\Omega)\right)\biggr)\text{d}\Omega .\nonumber
\end{align} % g = x in notes.
Here, $g$ is the real coupling constant of the down-conversion, $\phi$ is the phase of the pump mode, and $\gamma^b_R,\gamma_b$ are the readout and intra-cavity loss rates, respectively~\footnote{These coupling rates are given by $\gamma = -\frac{1}{2\tau}\log(1-T)$ where $\tau = \frac{2L}{c}$ is the round-trip time of the length $L$ cavity and $T$ is the transmission through the readout port ($T_{R,b}$) or the intra-cavity loss port ($T_{l,b}$).}. %The detection loss will be included later.
% I include optical losses in the model to represent the energy lost from the light to the thermal bath of the propagation medium and the optics~\cite{}.
The Fluctuation-Dissipation Theorem states that the intra-cavity loss of energy to the thermal bath of the propagation medium and the optics is accompanied by the introduction of uncorrelated noise~\cite{landau_lifshitz_1980}. Therefore, I model the optical loss with a beamsplitter that releases energy into the environment and creates another open port for vacuum~\footnote{By which I mean vacuum fluctuations, i.e.\ the sea of virtual photons, henceforth.} to enter through as shown in Fig.~\ref{fig:beamsplitter_loss}. It suffices to have a single loss mechanism inside the cavity coupled to vacuum $\hat n^L_b$~\cite{danilishinQuantumMeasurementTheory2012}. ``Detection'' loss can also occur at the photodetector and in the output chain of optics but I omit it from this model.

The \emph{Heisenberg-Langevin equation-of-motion}~\footnote{The Langevin input/output terms come from $H_\gamma$ and describe the Heisenberg equation-of-motion for an open system.}~\cite{PhysRevA.31.3761,PhysRevA.30.1386} for $\hat b$ given the bosonic commutation relations~\footnote{$[\hat Q_i,\hat Q_j^\dag]=\delta_{i,j}$ where the annihilation operator of the $i\text{th}$ bosonic mode is $\hat Q_i$.}, is 
\begin{equation}\label{eq:dOPO_initial_EoM}
\dot{\hat{b}}= -i\omega_0 \hat b+g e^{i\phi} \hat u\hat b^\dag - \gamma^b_\mathrm{tot} \hat{b} + \sqrt{2\gamma^b_R}\hat{B}_\mathrm{in} + \sqrt{2\gamma_b}\hat{n}^L_b.
\end{equation}
Here, $\gamma^b_\text{tot}=\gamma^b_\text{R}+\gamma_b$ is the total loss rate from the cavity.
I ignore the dynamics of the pump mode $\hat u$ by making a semi-classical approximation to it with coherent amplitude $u$ determined by the classical pump power that I assume is constant. This ``no pump depletion'' assumption is widely used in the literature and I will later justify what parameter range it is valid in~\cite{walls_1995}. In the Interaction Picture, i.e.\ separating the simple dynamics of $\hat H_0$ onto the states, and $\hat H_I + \hat H_\gamma$ onto the operators, I can ignore the $-i\omega_0 \hat b$ term from Eq.~\ref{eq:dOPO_initial_EoM} which leaves %like entering a rotating frame at $\hat b \mapsto e^{-i\omega_0 t} \hat b$ for each operator
\begin{equation}
\label{eq:dOPO_pre_FT}
\dot{\hat{b}}= \chi e^{i\phi}\hat b^\dag - \gamma^b_\mathrm{tot} \hat{b} + \sqrt{2\gamma^b_R}\hat{B}_\mathrm{in} + \sqrt{2\gamma_b}\hat{n}^L_b.
\end{equation}\jam{(check that rotating-wave approximation is not required?)}
Here, $\chi = g u$, the ``squeezer parameter''~\footnote{Not to be confused with the polarisability (e.g.\ $\chi^{(2)}$).}, is the gain rate of photons in the cavity mode\jam{(introducing threshold here is awkward because I can't talk about the variance)}. 
To find the quantum noise, I take the fluctuating components of each of these operators implicitly, i.e.\ $\delta \hat b\mapsto\hat b$ where $\delta \hat b(t)=\hat b(t)-\ev{\hat b}_t$, to ignore the classical dynamics described by the time-average $\ev{\hat b}_t$\jam{(check)} but each of the input modes $\hat{B}_\mathrm{in}, \hat{n}^L_b$ are vacuum and therefore the equation is the same for the fluctuating components since their time-averages are zero~\footnote{Assuming that the squeezer remains below ``threshold'' which I explain shortly.}. %\jam{(That the classical amplitude for $\hat b$ remains zero is not obvious, i.e.\ time-average is still zero, and the squeezer starts lasing at threshold so I need to clarify.)}

In the \emph{Fourier domain}\jam{($\Omega$ is the angular frequency offset from $\omega_0$ due to the Interaction Picture?)}, Eq.~\ref{eq:dOPO_pre_FT} can be solved algebraically~\cite{PhysRevA.30.1386} %(where $\partial_t \mapsto -i \Omega$):
% \begin{equation}
% \label{eq:dOPO_FT_initial}
% (\gamma^b_\mathrm{tot}-i \Omega)\hat b(\Omega)=\chi e^{i\phi}\hat b^\dag(-\Omega)  + \sqrt{2\gamma^b_R}\hat{B}_\mathrm{in}(\Omega) + \sqrt{2\gamma_b}\hat{n}^L_b(\Omega). 
% \end{equation} %I take the Hermitian conjugate and $\Omega\mapsto -\Omega$ of Eq.~\ref{eq:dOPO_FT_initial} to find
\begin{equation}
\label{eq:dOPO_FT_vectorised}
\vec{\hat b}(\Omega)=\text{M}_b^{-1}\left(\sqrt{2\gamma^b_R}\vec{\hat{B}}_\mathrm{in}(\Omega) + \sqrt{2\gamma_b}\vec{\hat{n}}^L_b(\Omega)\right),\quad\text{M}_b=(\gamma^b_\mathrm{tot}-i \Omega)\text{I}-\chi \begin{bsmallmatrix}0 & e^{i\phi} \\e^{-i\phi} & 0\end{bsmallmatrix}.
\end{equation}
% Here, $\Omega$ is the angular frequency offset from $\omega_0$ due to the Interaction Picture, % Algebraically, there are a few ways to arrive at the result, but I will demonstrate the linear algebra solution. % since it will be a useful method in the nondegenerate case where there are twice as many operators. 
Here, $\vec{\hat{Q}}(\Omega)=(\hat{Q}(\Omega),\hat{Q}^\dag(-\Omega))^\text{T}$ for each pair of annihilation/creation operators $(\hat Q, \hat Q^\dag)$, and $\text{I}$ is the 2 by 2 identity matrix.
Using the input/output relation at the readout port~\cite{PhysRevA.31.3761}, % and the beamsplitter model of detection loss with reflectivity $R_\text{PD}\in(0,1)$ as in Fig.~\ref{fig:beamsplitter_loss},
the light incident on the photodetector ($\hat B_\text{PD}$) is~\footnote{I ignore the spatial propagation of these external modes because it does not affect the noise.}\jam{(why is there not a reflection/transmission term for the readout port for the in field?)}
% \begin{equation}
% \label{eq:dOPO_IO_readout}\vec{\hat{B}}_\mathrm{out}(\Omega)=\vec{\hat{B}}_\mathrm{in}(\Omega)-\sqrt{2\gamma^b_R}\vec{\hat b}(\Omega).
% \end{equation} 
% And at the photodetector,\jam{using the beamsplitter model of detection loss with reflectivity $R_\text{PD}\in(0,1)$ (explain!)}, the light is 
\begin{equation}
\label{eq:dOPO_IO}\vec{\hat{B}}_\mathrm{PD}(\Omega)=\vec{\hat{B}}_\mathrm{in}(\Omega)-\sqrt{2\gamma^b_R}\vec{\hat b}(\Omega).
\end{equation} 
Using $\Gamma$ to convert to quadratures $\vec{\hat{X}}_Q(\Omega)=(\hat{X}_{Q,1}(\Omega),\hat{X}_{Q,2}(\Omega))^\text{T}=\Gamma \vec{\hat{Q}}(\Omega)$~\footnote{This is different to the vectorisation $\vec{\hat{Q}}(\Omega)$; $\vec{\hat{X}}_Q(\Omega)$ is a vector of quadratures each of which obeys $\hat{X}^\dag(-\Omega)=\hat{X}(\Omega)$.} and putting Eqs.~\ref{eq:dOPO_FT_vectorised}~\ref{eq:dOPO_IO} together, I find the output quadratures in terms of the input vacuum quadratures to be
\begin{align}
\label{eq:dOPO_PD_as_fn_of_vac}
\vec{\hat X}_\mathrm{PD}(\Omega)&=\text{R}_\text{in}\vec{\hat X}_\mathrm{in}(\Omega)+\text{R}^L_b\vec{\hat X}^L_b(\Omega).\\
\text{R}_\text{in}&=\Gamma\left(\text{I}-2\gamma^b_R\text{M}_b^{-1}\right)\Gamma^{-1} \\
\text{R}^L_b&=-2\sqrt{\gamma^b_R \gamma_b}\Gamma\text{M}_b^{-1}\Gamma^{-1}%\\\text{R}^L_\text{PD}&=\sqrt{R_\text{PD}} \text{I}.
\end{align}
% In terms of transfer matrices $\text{R}_\text{in},\text{R}^L_b, \text{R}^L_\text{PD}$ from each vacuum source.
% I.e.\ given by transfer matrices $\text{R}_\text{in},\text{R}^L_b, \text{R}^L_\text{PD}$, as a linear combination of the three (uncorrelated) vacuum sources.

This defines the quantum noise response of the degenerate OPO, where the \emph{total quantum noise}~\footnote{Which I also call the noise response henceforth.} measured at the photodetector is described by the matrix $\text{S}_X$ of spectral densities 
\begin{equation}\label{eq:total_noise_matrix}
(\text{S}_X)_{i,j}(\Omega)\delta(\Omega-\Omega')=\ev{(\vec{\hat X}_\text{PD})_i(\Omega)\circ(\vec{\hat X}_\text{PD})_j^\dag(\Omega')}.
\end{equation}
Here, $\ev{\ldots}$ is the vacuum expectation value. This can be found using the (valid) assumption of uncorrelated vacuum inputs to be
\begin{equation}\label{eq:dOPO_Sx_abstract}
\text{S}_X(\Omega)=\text{R}_\text{in} \text{R}_\text{in}^\dag+\text{R}^L_b {\text{R}^L_b}^\dag. %+\text{R}^L_\text{PD}{\text{R}^L_\text{PD}}^\dag.
\end{equation}
The diagonal elements of $\text{S}_X$ are the Fourier-domain variances $S_{X_i}$ of the light at the photodetector and the off-diagonal elements give the covariances, i.e.\ correlations, between the two quadratures~\footnote{Which obey the Hermitiancy of $\text{S}_X$ from Eq.~\ref{eq:total_noise_matrix}. Like the Fourier-domain quadratures themselves, these covariances are not real but are indirectly observable~\cite{reidDemonstrationEinsteinPodolskyRosenParadox1989}.}.

% This demonstrates the Hamiltonian method that I will use throughout this thesis, with later configurations just adding more modes to the Hamiltonian. % -- including those associated with the signal. % Overall, the method is robust\jam{(why?)} and easy to perform computationally 

\subsubsection{Demonstrating squeezing}

% Results
\begin{figure}
	\centering
	\includegraphics[width=\textwidth]{dOPO_noise_chi.pdf}
	\caption{Degenerate OPO noise response, i.e.\ $\sqrt{S_{X_i}}$ versus frequency, showing anti-squeezing ($i=1$) and squeezing ($i=2$). Left panel: different squeezer parameters up to threshold with no loss. On threshold, the anti-squeezed quadrature is singular and the squeezed quadrature is zero. Right panel: different intra-cavity loss $T_{l,b}$ for a fixed ratio of $95\%$ threshold calculated for each loss. For both quadratures, intra-cavity loss pulls the peak variance towards the vacuum value of one (0~dB) but also broadens the cavity resonance -- although here the loss is unrealistically high. The anti-squeezed quadrature is more tolerant of losses as explained in the text. For a 1~m cavity with readout transmission $T_{R,b}=0.1$ and $\phi=0$.}
	\label{fig:dOPO_variances}
\end{figure}

I now demonstrate that parametric down-conversion squeezes the variances of the measured quadratures.
Computing Eq.~\ref{eq:dOPO_Sx_abstract} using matrix algebra~\footnote{Which I perform using Wolfram Mathematica~\cite{mathematica} throughout this thesis.} shows that, for $\phi=0$~\footnote{The pump phase $\phi$ breaks the symmetry between the quadratures by selecting which quadrature is squeezed; $\phi=0$ corresponds to the angle of the noise ellipse shown in Fig.~\ref{fig:ballandstick_simple}.}, the covariances vanish and the variances simplify to
% \begin{equation}\label{eq:dOPO_full_freedom}
% \text{S}_X(\Omega)=\left[
% \begin{array}{cc}
%  1+\frac{(1-R_\text{PD})4 \gamma^b_R \chi  \left(2 \gamma^b_\text{tot} \chi +\left({\gamma^b_\text{tot}}^2+\chi ^2+\Omega ^2\right) \cos (\phi )\right)}{\left({\gamma^b_\text{tot}}^2-\chi ^2\right)^2+2 \Omega ^2 \left({\gamma^b_\text{tot}}^2+\chi ^2\right)+\Omega ^4} 
%  & \frac{(1-R_\text{PD})4 \gamma^b_R \chi  \left({\gamma^b_\text{tot}}^2+\chi ^2+\Omega ^2\right) \sin (\phi )}{\left({\gamma^b_\text{tot}}^2-\chi ^2\right)^2+2 \Omega ^2 \left({\gamma^b_\text{tot}}^2+\chi ^2\right)+\Omega ^4} \\
%  \frac{(1-R_\text{PD})4 \gamma^b_R \chi  \left({\gamma^b_\text{tot}}^2+\chi ^2+\Omega ^2\right) \sin (\phi )}{\left({\gamma^b_\text{tot}}^2-\chi ^2\right)^2+2 \Omega ^2 \left({\gamma^b_\text{tot}}^2+\chi ^2\right)+\Omega ^4} 
%  & 1+\frac{(1-R_\text{PD})4 \gamma^b_R \chi  \left(2 \gamma^b_\text{tot} \chi -\left({\gamma^b_\text{tot}}^2+\chi ^2+\Omega ^2\right) \cos (\phi )\right)}{\left({\gamma^b_\text{tot}}^2-\chi ^2\right)^2+2 \Omega ^2 \left({\gamma^b_\text{tot}}^2+\chi ^2\right)+\Omega ^4} \\
% \end{array}
% \right].\end{equation}
% The general form of $\text{S}_X(\Omega)$ is common to squeezing configurations: all elements are rational functions (fractions of polynomials) of $\Omega$ and $\chi$, the output is uncorrelated vacuum when the squeezer is turned off $\text{S}_X|_{\chi=0}=\text{I}$, and off-diagonal elements obey the Hermitiancy of $\text{S}_X$ that can be proved from Eq.~\ref{eq:dOPO_Sx_abstract}~\cite{}.
% Looking at the Hamiltonian, only $\phi$ (and the sign of $\chi$, although I will keep $\chi$ non-negative and vary the phase using $\phi$) will break the symmetry between the quadratures\jam{(clarify this, Hamiltonian is not written in terms of the quadratures, perhaps should be re-expressed)} -- observe that $(\text{S}_X)_{2,2}=(\text{S}_X)_{1,1}|_{\chi\mapsto-\chi}=(\text{S}_X)_{1,1}|_{\phi\mapsto\phi+\pi}$.
\begin{equation} \label{eq:dOPO_fixed_phase}
\text{S}_X(\Omega)=\left[
\begin{array}{cc}
 1+\frac{4 \gamma^b_R \chi}{\left({\gamma^b_\text{tot}}-\chi\right)^2+\Omega ^2}
 & 0 \\
 0
 & 1-\frac{4 \gamma^b_R \chi}{\left({\gamma^b_\text{tot}}+\chi\right)^2+\Omega ^2} \\
\end{array}
\right].
\end{equation}
Therefore, by turning the squeezer on, $\chi>0$, one quadrature ($\hat X_1$) has increased uncertainty while the other quadrature ($\hat X_2$) has uncertainty below the vacuum value of $1$; these are the anti-squeezed and squeezed quadratures, respectively\jam{(is there a physical explanation why this occurs?)}, as shown in Fig.~\ref{fig:dOPO_variances}. The squeezing/anti-squeezing curves eventually converge to the vacuum value in frequency (around the cavity bandwidth $\gamma^b_R$) as the cavity goes off-resonance. %~\cite{}\jam{(check: Korobko ``vacuum is supported'' argument)}. % Although using squeezing to decrease the quantum noise is the common application, anti-squeezing is also useful when the signal can be amplified as well -- to be discussed much later. 
 % and increase the overall uncertainty product\jam{(how? optical loss pulls to vacuum value of 1, right?)}. 
%While understanding physically why parametric down-conversion produces squeezing is beyond the scope of this thesis~\footnote{See Ref.~\cite{} for an explanation.}, this derivation has shown it mathematically.

There are two further aspects of the squeezing shown in Fig.~\ref{fig:dOPO_variances} to discuss: (1) the limit on increasing $\chi$ and (2) the effect of optical loss.
% remain a few effects visible in this simpler model that will return in later models and are therefore worth examining, namely the effects of (1) increasing the squeezer parameter $\chi$, (2) intra-cavity loss $\gamma_b$ versus detection loss $R_\text{PD}$, and (3) changing the pump phase $\phi$. I discuss each of these in turn. % is this necessary to say?

% \subsubsection{Threshold}
\label{sec:dOPO_threshold}

% in the abstract, a squeezing technology might produce any amount of squeezing, the energetic reality of gain and loss in a NL crystal creates threshold
% pump depletion only matters near and above threshold
Fig.~\ref{fig:dOPO_variances} shows that increasing $\chi$ from zero increases the difference from the vacuum value for both quadratures~\footnote{If the vacuum variance $S_X=1$ is squeezed by a factor $e^r$, then I quantify the squeezing in amplitude-decibels (dB) as $20 \log_{10}(\sqrt{S_X e^{-r}})$ compared to the vacuum at 0~dB.}. %  symmetrically for squeezing and anti-squeezing when $\sqrt S_X$ is plotted logarithmically\jam{(explain this more once plot present)}. 
The $\Omega=0$ (DC~\footnote{DC (direct current) referring to frequency zero behaviour comes from electrical engineering.}) value of the anti-squeezed quadrature from Eq.~\ref{eq:dOPO_fixed_phase} is singular at $\chi_\text{thr}=\gamma^b_\text{tot}$ and the squeezed quadrature is zero in the lossless case, as shown in Fig.~\ref{fig:dOPO_variances}. %(to satisfy the Heisenberg Uncertainty Principle exactly)
This value $\chi_\text{thr}$, the \emph{``threshold''} of the degenerate OPO, % and serves as a boundary of this model's physicality, specifically of the assumption of no pump depletion. 
can be understood as the balance of gains and losses inside the cavity: the squeezer creates photons in the cavity mode at a rate $\chi$ which are lost at a rate $\gamma^b_\text{tot}$, and when $\chi=\chi_\text{thr}$ the gain and loss balance and, like a phase transition, beyond $\chi_\text{thr}$ the OPO starts lasing with a non-zero coherent amplitude at the output~\cite{walls_1995}. % stimulated emission is less than sponteaneous emission, decoherence > coherence
The no-pump-depletion assumption breaks at threshold\jam{(fix to ``the threshold'' throughout?)} as it implies that is no limit to the amount of energy transferred from the pump which makes the system unstable~\footnote{Below threshold, no energy is lost from the pump mode since the squeezed output remains at vacuum.}. 
% In the lossy model, threshold can be shown to be $\chi_\text{thr}=\gamma^b_\text{tot}$~\cite{} which makes sense from the gain-loss argument~\footnote{That it is independent of pump phase also makes sense by inspection of the denominator in Eq.~\ref{eq:dOPO_full_freedom}.} and recovers the lossless value in the appropriate limit.
% However, above, at, or just below threshold the loss of energy from the pump mode is significant and the no-pump-depletion approximation is not valid~\cite{}. Therefore, all models of squeezers in cavity detectors in this thesis, which use the no pump depletion assumption as a simplification, have squeezer parameter bound to $\chi\in(0,\chi_\text{thr})$. % (although, as seen in the lossless case, this can still allow arbitrarily large squeezing/anti-squeezing). 
% Although I have introduced threshold here as a limitation of the assumptions I have made,
While it is experimentally possible to operate above threshold (e.g.\ in Ref.~\cite{martinelli2001classical}), however, it is not necessary for generating squeezed vacuum states for gravitational-wave detection~\cite{aasietal2013}, and, therefore, I will not consider the behaviour above threshold; this makes the no-pump-depletion assumption valid in this thesis~\footnote{Moreover, in the application to gravitational-wave detection, some margin to threshold, e.g.\ $5\%$, is maintained so that the system does not stray above threshold.}.
% However, I am not interested here in the behaviour above threshold since the lasing can damage sensitive photodetectors\jam{(not true! ``To generate squeezed vacuum for GWD it is not necessary to operate above threshold and although it is experimentally possible~\cite{} I will not consider behaviour above threshold in this thesis.'')} designed to sense vacuum fluctuations or used in a gravitational-wave detector~\cite{}. Therefore, staying below threshold by some margin, e.g.\ $5\%$, is satisfactory and the no-pump-depletion assumption is valid for this thesis~\cite{}. % that otherwise breaks above, at, or just below threshold since the loss of energy from the pump mode is significant.
% ~\footnote{As a rule-of-thumb below $70\%$ threshold, but in a well-controlled system 90--95~$\%$ can be reliably, safely achieved~\cite{}.}
% Moreover, in experiments, to avoid the system wandering above threshold some safety margin is often introduced, e.g.\ remaining below $70\%$ threshold as a rule-of-thumb~\cite{} and pushing above $90\%$ only in a well-controlled system~\cite{}.  % set up singularity threshold?

% \subsubsection{Effect of optical losses}
% effect of different losses

%optical loss decoheres the state of the light, i.e.\ reduces correlations and pulls the variances back towards the vacuum value of $S_{X_i}=1$, because it mixes the light with the vacuum as shown in Fig.~\ref{fig:beamsplitter_loss}.
\emph{Optical loss} decoheres the state of light because it mixes it with the vacuum state as shown in Fig.~\ref{fig:beamsplitter_loss}.
By Eq.~\ref{eq:dOPO_fixed_phase}, in the lossless case, the Heisenberg Uncertainty Principle is satisfied as an equality $\sqrt{(\text{S}_X)_{1,1}(\text{S}_X)_{2,2}}=1$, and in the lossy case, as an inequality (i.e.\ $>1$) because optical losses decohere the system and the squeezed uncertainty is increased more than the anti-squeezed uncertainty is decreased because the latter is further from the vacuum value of one~\footnote{For example, loss does not affect the singularity of the on-threshold anti-squeezed quadrature but does affect the zero of the squeezed quadrature, therefore increasing the product of the uncertainties.}.
% However, the squeezed uncertainty is increased more than the anti-squeezed uncertainty is decreased because the latter is further from the vacuum value of one which increases the uncertainty product~\cite{}.
% of detection loss $R_\text{PD}$ versus intra-cavity loss $\gamma_b$. Understanding the tolerance of a detector to various losses is important when judging its feasibility and the results here will apply generally.
% Considering the different losses, by Eq.~\ref{eq:dOPO_fixed_phase}, detection loss $R_\text{PD}$ simply scales the lossless variances towards the constant vacuum value of one as shown in Fig.~\ref{fig:dOPO_variances}. %This affects the on-threshold, DC values of the quadratures differently as the singularity in the anti-squeezed quadrature is robust to loss while the than the zero in the squeezed quadrature.
% This creates a difference between the squeezed and anti-squeezed quadratures, however, since the on-threshold, DC value of the squeezed quadrature changes from $0$ to $R_\text{PD}$ but the anti-squeezed quadrature remains at $\infty$, i.e.\ the singularity in the anti-squeezed quadrature is more robust than the zero in the squeezed quadrature. This indicates that the detection loss has increased the product in the Heisenberg Uncertainty Principle to being $>1$ at DC, and indeed at all frequencies\jam{(cite/check this?)}.
Introducing intra-cavity loss $\gamma_b$ acts like damping an oscillator, i.e.\ the peaks in Fig.~\ref{fig:dOPO_variances} move towards vacuum but broaden, their quality factor decreasing~\footnote{This means that there are frequencies for which the squeezing is improved by loss because the cavity resonance broadens.}.
% Similarly to detection loss, intra-cavity loss affects the on-threshold, DC value of the squeezed quadrature but not the anti-squeezed quadrature.
The intra-cavity loss also increases threshold $\chi_\text{thr}=\gamma^b_\text{tot}$ which decreases the performance for a fixed pump power. Since this effect can be mitigated experimentally by increasing the pump power, henceforth, I will compare different losses with the same ratio to threshold $\chi/\chi_\text{thr}$, i.e.\ I will normalise to the different threshold in each case. % and change $\chi$ accordingly unless stated otherwise. 
% decoherences of co-variances --> need a figure?
The effect of losses on the covariances between the other quadratures is similar, i.e.\ intra-cavity loss reduces the peak and broadens the response.
 % detection loss pulls the whole curve towards zero while

% \subsection{Optical loss and decoherence}
% %  source of loss, realistic values (now and in a few decades time, 100 years time etc.)
% \label{sec:optical_loss_background}

% So, quantum noise is just caused by uncertainties in the state of detector
% Quantum noise from fundamental uncertainty in the light of a cavity-based detector can be thought of as entering the detector through every open port and lossy optic~\cite{}. %So far, the explanation introduced above is that quantum noise is caused by fundamental uncertainties in the state of the detector, so what does the previous statement mean? % too colloquial?
% Any state of light in the detector can be written in terms of transformations of the vacuum state, and so any uncertainty in the state of the light can be expressed as vacuum fluctuations (corresponding to $S_X^\text{vac}=1$ above) viewed under such transformations~\cite{}. Therefore, the source of the quantum noise is ultimately the vacuum, which can be thought to surround the detector and enter whenever the system is opened to its surroundings~\footnote{This language is potentially confusing because the vacuum does not propagate, but it is shorthand for considering how the system interacts with the surrounding field of virtual photons.}~\cite{}. An open port refers to where light can enter and leave the detector, such as at the photodetector where the detector is necessarily always open. This is worth emphasising, measurement requires the detector to be coupled to the environment and guarantees a port for the vacuum. % -- this notion of quantum noise is not well-defined for a closed system\jam{(this is misleading)}. 
% I will, shortly, express this notion mathematically, but first I will explain why lossy optics are also a source of quantum noise.

% % fluctuation-dissipation theorem
% Optical loss refers to the energy lost from a light field upon propagation or interaction with an optic, e.g. if $50\%$ of the power incident on a mirror is reflected but only $45\%$ is transmitted, then the remaining $5\%$ of the power is lost to the environment. The chief mechanism for optical loss is dissipation into the thermal bath of the propagation medium and the optics, e.g.\ heating of a mirror~\cite{}. The Fluctuation-Dissipation Theorem states that any such dissipation into the thermal bath is accompanied by the introduction of uncorrelated noise into the light field~\cite{}. As such, the loss mechanism in an optic can be modelled by a beamsplitter that releases energy into the environment and creates a new open port for vacuum to enter through, as shown in Fig.~\ref{fig:beamsplitter_loss}. In this model of optical loss, all optics are perfect (i.e.\ the power incident equals the sum of the power reflected and transmitted) but additional open ports are added throughout the detector -- so-called loss ports. It suffices to have a single loss port in every cavity, i.e.\ for every mode, since the uncorrelated vacuum from multiple sources of loss, e.g.\ the propagation medium plus each optic in the cavity, sum in quadrature and is equivalent to a single vacuum source from a lossier port. % -- which should be clearer later. %is or are?
% This means that any other open ports in the detector, besides any used for measurement, can also be collapsed into the loss ports. I will assume uncorrelated vacuum from every source of loss in this thesis, which is reasonable~\cite{}. Therefore, in this model, the quantum noise in a cavity-based detector comes from the vacuum entering at the measurement device and from loss ports in every mode.

% decoherence and the effect of mixing with the vacuum
% This mathematical formalism is useful but it leaves the effect of the loss ports vague 
% The physical effect of optical loss is to decohere the state of the light -- reduce correlations and pull the quantum noise variance back towards its vacuum value of $S_X=1$. This will be demonstrated later, but the effect can be understood using the above notion of optical loss. The reflected light from a loss port is some linear combination of the vacuum and the incident light~\footnote{The Fluctuation-Dissipation Theorem states that the rates of energy lost and vacuum entering are equal, meaning that this combination is normalised: $\hat X_\text{out}=\sqrt{1-R}\hat X_\text{in}+\sqrt{R}\hat X_\text{vac}$, as shown in Fig.~\ref{fig:beamsplitter_loss}.}. This means that the quantum noise of the reflected light is a weighted average of the incident light and the vacuum value of $1$, i.e.\ more loss pulls the quantum noise towards $1$. The correlations of the light are also pulled towards the vacuum, which is uncorrelated, and therefore the correlations are reduced towards $0$. These correlations, such as between quadratures $\hat{X}_i, \hat{X}_j$, are given by the covariances $\ev{\hat{X}_i(\Omega)\circ\hat{X}_j^\dag(\Omega')}$, which is not real but is indirectly observable like the quadratures themselves~\cite{}. That optical loss decoheres the state is important to techniques that seek to reduce the quantum noise by introducing correlations, such as some of those explored in this thesis.

% Finally, using this notion of quantum noise, I can write down the quantum noise response and sensitivity of a detector. %how arbitrary is this?
% Let the quadrature measured at the photodetector (PD) of an arbitrary detector be $\hat{X}_\text{PD}(\Omega)$. This will be determined by the noise quadratures $\hat{X}_i^\text{vac}(\Omega)$ of the vacuum input at each open port and source of optical loss indexed by $i$, the signal $\tilde{h}(\Omega)$, and the detector's linear response to each of these, expressed by the noise $R_i(\Omega)$ and signal $T(\Omega)$ transfer functions (where the frequency dependence will often be implicit henceforth): \begin{equation}\label{eq:transfer_fns_background}\hat X_\text{PD}(\Omega)=\sum_i R_i \hat X_i^\text{vac}(\Omega) + T \tilde h(\Omega).\end{equation}
% These transfer functions capture the entire behaviour of the detector. %, including the impact on the optical quantum noise by coupling to non-optical modes such as the mechanical mode of a suspended mirror. % -- more on this later. 
% The total quantum noise measured by the detector is simply $\hat X_\text{PD}(\Omega)|_{\tilde h=0}$, i.e.\ the output with the signal turned off. This quantum noise can be characterised by its spectral density, $S_X$, which can be simplified assuming uncorrelated vacuum inputs to the sum of squares: \begin{equation}S_X(\Omega)=\sum_i \abs{R_i}^2(\Omega).\end{equation} 
% As an aside, since the transfer functions $R_{i_1}, R_{i_2}$ for two loss ports in the same cavity are the same up to the respective loss rates $\sqrt\gamma_{i_1}, \sqrt\gamma_{i_2}$, the two ports can be collapsed to a single loss port with loss rate $\sqrt{\gamma_{i_1}+\gamma_{i_2}}$ since \begin{equation}\abs{R_{i_1}}^2+\abs{R_{i_2}}^2=\abs{\sqrt{\gamma_{i_1}}R}^2+\abs{\sqrt{\gamma_{i_2}}R}^2=\abs{\sqrt{\gamma_{i_1}+\gamma_{i_2}}R}^2.\end{equation} This explains the claim made before that only one loss port is needed for each mode; an expression for the loss rates $\gamma$ will be given later. 

% The units of sensitivity can be found by noting that $h$ is the unitless gravitational-wave strain and $\hat a, \hat a^\dag$ are unitless~\cite{} implies that $\hat X$ is unitless. In the Fourier domain, $\hat X(\Omega), \tilde h(\Omega)$ acquire units of time from the $\text{d}t$ in the Fourier transform. Therefore, the transfer functions in Eq.~\ref{eq:transfer_fns_background} are unitless and $S_X$ has units of time\jam{(this disagrees with the sum of squares expression, the Kronecker deltas must also come with something with units of time?)}. Therefore, $\sqrt S_h$ has units of $\text{Hz}^{-1/2}$~\cite{}\jam{(check that this is the right justification)}. 
%When plotting, I will use the amplitude spectral density, like the gravitational-wave literature~\cite{}, rather than the Fourier transform, and this means that $\sqrt S_h$ has units of $\text{Hz}^{-1/2}$~\cite{}.\jam{(check this)}



% \subsubsection{Effect of pump phase}
% effect of pump phase

% \begin{figure}
% 	\centering
% 	% \includegraphics[width=\textwidth]{}
% 	\caption{Degenerate OPO quantum noise response, showing squeezing and anti-squeezing variances versus frequency for different pump phases $\phi \in (0,\pi)$.}
% 	\label{fig:dOPO_variances_pump_phase}
% \end{figure}

% Finally, the effect of pump phase $\phi$. As mentioned above, the phase of the pump mode is what breaks the symmetry between the quadratures. From Eq.~\ref{eq:dOPO_full_freedom}, $(\text{S}_X)_{2,2}=(\text{S}_X)_{1,1}|_{\phi\mapsto\phi+\pi}$ and vice versa -- the quadratures switch under $\phi\mapsto\phi+\pi$~\cite{}. %, as shown in Fig.~\ref{fig:dOPO_variances_pump_phase}.
% For other values of $\phi$, the anti-squeezing and squeezing occur on a different set of orthonormal basis vectors for the 2D space of quadratures: $\hat{X}_{\theta=\phi}$ and $\hat{X}_{\theta=\phi+\pi/2}$ respectively\jam{(fix this, is it $\phi\mapsto\phi+\pi$ or $\phi\mapsto\phi+\pi/2$?)} -- a rotation of the noise ellipse in Fig.~\ref{fig:ballandstick_simple}. 
% % And continuously changing $\phi\in(0,\pi)$\jam{($\pi/2$?)} smoothly connects the squeezing and anti-squeezing behaviour. 
% Fixing the pump phase and performing a change-of-basis of the quadratures at the photodetector confirms that this is the case~\cite{}.

% summary of dOPO, is this necessary?
% The degenerate OPO provides a simple, well-studied configuration to witness many of the effects of placing a squeezer inside of a cavity. I have also demonstrated the method and some of the assumptions used throughout this thesis to model detectors. However, before I can move on to discussing how squeezing is used currently in interferometric detectors and how proposals have suggested to exploit it further for future detectors, I need to expand the above model and discuss the effects of including an additional mode -- the idler.


\subsection{Nondegenerate OPO}
\label{sec:nOPO}
% succinct analytic model, to show how to calculate covariances (requires 4x4 model)

% \begin{figure}
% 	\centering
% 	% \includegraphics[width=\textwidth]{}
% 	\caption{\jam{(Combine side-by-side with dOPO config?)} Nondegenerate OPO configuration is the same as the degenerate OPO but with an idler mode, frequencies of each mode are labelled. Vacuum at the signal and idler frequencies incident on the OPO is reflected anti-squeezed, shown by the noise ellipses which are symmetric up to the loss rates, and correlated. The intra-cavity loss and readout rate can be different for the signal and idler by using dichroic optics.}
% 	\label{fig:nOPO_config}
% \end{figure}

A nondegenerate optical parametric oscillator (OPO) is the same configuration as a degenerate OPO except that the crystal performs nondegenerate parametric down-conversion as shown in Fig.~\ref{fig:OPOs_config}: splitting the pump at frequency $2\omega_0+\Delta$ down into two separate, squeezed, and entangled modes at $\omega_0$ (the signal) and $\omega_0+\Delta,\; \Delta\neq0$ (the idler). %Similarly to the degenerate case, this configuration is well-studied in the literature~\cite{}. % I show an abridged derivation of its quantum noise response here because the main configuration later in this thesis is fundamentally a nondegenerate OPO plus an additional mode and so that derivation will build on this one. Moreover, much of the behaviour of the nondegenerate OPO is inherited by that configuration. % -- to be seen later.

\subsubsection{Analytic model}

% I show a model of the nondegenerate OPO here because the model in my work is quite similar.
% Using the same Hamiltonian method as Section~\ref{sec:dOPO_model}, the quantum noise response can be derived, where the steps are all similar and available in the literature~\cite{,}. Let the modes of the system be as described in Section~\ref{sec:dOPO_model} but with the pump mode $\hat u$ at $2\omega_0+\Delta$, and let $\hat c$ annihilate the idler cavity mode at $\omega_0+\Delta$ (assumed on-resonance by the single-mode approximation\jam{(how can the signal, idler, and pump all be on resonance?)}) with associated input vacuum fields $\hat n^L_c, \hat C_\text{in}, \hat n^L_\text{c,PD}$ and output fields $\hat C_\text{out}, \hat C_\text{PD}$ that are direct analogues of the signal mode's $\hat n^L_b, \hat B_\text{in}, \hat n^L_\text{b,PD}, \hat B_\text{out}, \hat B_\text{PD}$, respectively.
The Hamiltonian of this system is $\hat H = \hat H_0+\hat H_I+\hat H_\gamma$ with~\cite{schoriNarrowbandFrequencyTunable2002,reidDemonstrationEinsteinPodolskyRosenParadox1989}
\begin{align}
\hat H_0 &= \hbar \omega_0 \hat b^\dag \hat b + \hbar (\omega_0+\Delta) \hat c^\dag \hat c + \hbar (2\omega_0+\Delta) \hat u^\dag \hat u\\
\hat H_I &= \hbar \frac{g}{2} \left(e^{i\phi} \hat u \hat b^\dag \hat c^\dag + e^{-i\phi} \hat u^\dag \hat b \hat c \right)\\
\hat H_\gamma &= i\hbar \sqrt{2}\int \biggl( \sqrt{\gamma^b_R}\left(\hat{b}^\dag(\Omega)\hat{B}_\text{in}(\Omega)-\hat{b}(\Omega)\hat{B}_\text{in}^\dag(\Omega)\right) + \sqrt{\gamma_b}\left(\hat{b}^\dag(\Omega)\hat n^L_b(\Omega)-\hat{b}(\Omega)\hat{n}^{L\dag}_b(\Omega)\right) \nonumber \\
& \hspace*{1.55cm} + \sqrt{\gamma^c_R}\left(\hat{c}^\dag(\Omega)\hat C_\text{in}(\Omega)-\hat{c}(\Omega)\hat C_\text{in}(\Omega)\right) + \sqrt{\gamma_c}\left(\hat{c}^\dag(\Omega)\hat n^L_c(\Omega)-\hat{c}(\Omega)\hat{n}^{L\dag}_c(\Omega)\right)\biggr)\text{d}\Omega \nonumber.
\end{align}
Here, the pump mode $\hat u$ is now at $2\omega_0+\Delta$, $\hat c$ is the idler cavity mode at $\omega_0+\Delta$~\footnote{Also assumed on resonance by the single-mode approximation\jam{(the signal is not the fundamental FSR)}.} with analogous input/output fields to the signal mode $\hat b$,
% Where $\phi$ is $\pi/2$ ahead of the phase in the previous section 
%\jam{(explain why $\phi$ is $\pi/2$ ahead of the phase in the previous section ?)} 
and $\gamma^c_R,\gamma_c$ are the idler's coupling rates through the readout and loss ports, respectively~\footnote{Using dichroic optics can mean that these are different to the signal mode.}. %, but I will default to assuming symmetric loss between the modes.
Using the \emph{same Hamiltonian method} as Section~\ref{sec:dOPO_model}, the Heisenberg-Langevin equations-of-motion can then be found, where I again: (1) assume the semi-classical and no pump depletion approximations to the pump mode $\hat u\mapsto u=2\chi/g$, (2) enter the Interaction Picture to ignore $\hat H_0$, and (3) take fluctuating components but leave the $\delta \hat Q(t)$ implicit in the notation,
\begin{equation}\begin{cases}\label{eq:nOPO_EoM}
\dot{\hat{b}}=-i\chi e^{i\phi}\hat{c}^\dagger - \gamma^b_\mathrm{tot} \hat{b} + \sqrt{2\gamma^b_R}\hat{B}_\mathrm{in} + \sqrt{2\gamma_b}\hat{n}^L_b\\
\dot{\hat{c}}=-i\chi e^{i\phi}\hat{b}^\dagger - \gamma^c_\mathrm{tot} \hat{c} + \sqrt{2\gamma^c_R}\hat{C}_\mathrm{in} + \sqrt{2\gamma_c}\hat{n}^L_c.
\end{cases}\end{equation}
Here, $\gamma^c_\text{tot}=\gamma^c_R+\gamma_c$.
Similarly to the degenerate case, I take Fourier transforms and find the vector equation for $\vec{\hat d}(\Omega)=(\hat b(\Omega), \hat b^\dag(-\Omega), \hat c(\Omega), \hat c^\dag(-\Omega))^\text{T}$ which now combines the signal and idler modes, with similar vectorisation for each mode, then I solve that equation algebraically. Using the input/output relation at the readout port~\cite{PhysRevA.31.3761}, I then find the signal and idler quadratures at the photodetector.

% Similarly to the degenerate case, I take Fourier transforms and find the vector equation for $\vec{\hat b}(\Omega)=[\hat b(\Omega), \hat b^\dag(-\Omega), \hat c(\Omega), \hat c^\dag(-\Omega)]^\text{T}$ which now combines the signal and idler modes, with similar vectorisation for each mode, then I solve the equation algebraically
%\jam{(cut off-diagonal elements like nIS?)}
% \begin{align}\label{eq:nOPO_b_sol}
% \vec{\hat b}(\Omega)&=\text{M}_b^{-1}\left(\sqrt{2}\begin{bsmallmatrix}
% \sqrt{\gamma^b_R} & 0 & 0 & 0 \\
% 0 & \sqrt{\gamma^b_R} & 0 & 0 \\
% 0 & 0 & \sqrt{\gamma^c_R} & 0 \\
% 0 & 0 & 0 & \sqrt{\gamma^c_R}
% \end{bsmallmatrix}\vec{\hat B}_\mathrm{in}(\Omega) + \sqrt{2}\begin{bsmallmatrix}
% \sqrt{\gamma_b} & 0 & 0 & 0 \\
% 0 & \sqrt{\gamma_b} & 0 & 0 \\
% 0 & 0 & \sqrt{\gamma_c} & 0 \\
% 0 & 0 & 0 & \sqrt{\gamma_c}
% \end{bsmallmatrix}\vec{\hat n}^L_b(\Omega)\right)\\
% \text{M}_b&=\begin{bsmallmatrix}
% \gamma^b_\mathrm{tot} & 0 & 0 & 0 \\
% 0 & \gamma^b_\mathrm{tot} & 0 & 0 \\
% 0 & 0 & \gamma^c_\mathrm{tot} & 0 \\
% 0 & 0 & 0 & \gamma^c_\mathrm{tot} 
% \end{bsmallmatrix}-i\Omega \text{I}+\chi \begin{bsmallmatrix}
% 0 & 0 & 0 & i e^{i\phi} \\
% 0 & 0 & -i e^{-i\phi} & 0 \\
% 0 & i e^{i\phi} & 0 & 0 \\
% -i e^{-i\phi} & 0 & 0 & 0
% \end{bsmallmatrix}.
% \end{align}
% Where $\text{I}$ is now the 4 by 4 identity matrix. The input/output relations at the readout port and at the detection loss beamsplitter~\footnote{I assume that the detection loss $R_\text{PD}$ is symmetric between the signal and idler, although this is not necessary.} are similar to Eq.~\ref{eq:dOPO_IO},
% \begin{align}
% \label{eq:nOPO_IO_relations}
% \vec{\hat{B}}_\mathrm{PD}(\Omega)=\sqrt{1-R_\text{PD}}\left(\vec{\hat{B}}_\mathrm{in}(\Omega)-\sqrt{2}\begin{bsmallmatrix}
% \sqrt{\gamma^b_R} & 0 & 0 & 0 \\
% 0 & \sqrt{\gamma^b_R} & 0 & 0 \\
% 0 & 0 & \sqrt{\gamma^c_R} & 0 \\
% 0 & 0 & 0 & \sqrt{\gamma^c_R}
% \end{bsmallmatrix}\vec{\hat b}(\Omega)\right)+\sqrt{R_\text{PD}}\vec{\hat n}^L_\text{PD}(\Omega).
% \end{align}
% Substituting Eq.~\ref{eq:nOPO_b_sol} into the above equation and using $\Gamma=\frac{1}{\sqrt2}\begin{bsmallmatrix}
% 1 & 1 & 0 & 0 \\
% -i & i & 0 & 0 \\
% 0 & 0 & 1 & 1 \\
% 0 & 0 & -i & i
% \end{bsmallmatrix}$ to convert to quadratures, I find the signal and idler quadratures at the photodetector to be~\cite{}
% \begin{align}
% \vec{\hat X}_\mathrm{PD}(\Omega)&=\text{R}_\text{in}\vec{\hat X}_\mathrm{in}(\Omega)+\text{R}^L_b\vec{\hat X}^L_b(\Omega)+\text{R}^L_\text{PD}\vec{\hat X}^L_\text{PD}(\Omega)\\
% \text{R}_\text{in}&=\sqrt{1-R_\text{PD}}\Gamma(\text{I}-\sqrt 2\begin{bsmallmatrix}
% \sqrt{\gamma^b_R} & 0 & 0 & 0 \\
% 0 & \sqrt{\gamma^b_R} & 0 & 0 \\
% 0 & 0 & \sqrt{\gamma^c_R} & 0 \\
% 0 & 0 & 0 & \sqrt{\gamma^c_R}
% \end{bsmallmatrix}\text{M}_b^{-1}\sqrt{2}\begin{bsmallmatrix}
% \sqrt{\gamma^b_R} & 0 & 0 & 0 \\
% 0 & \sqrt{\gamma^b_R} & 0 & 0 \\
% 0 & 0 & \sqrt{\gamma^c_R} & 0 \\
% 0 & 0 & 0 & \sqrt{\gamma^c_R}
% \end{bsmallmatrix})\Gamma^{-1}\\
% \text{R}^L_b&=-\sqrt{1-R_\text{PD}}\Gamma\sqrt 2\begin{bsmallmatrix}
% \sqrt{\gamma^b_R} & 0 & 0 & 0 \\
% 0 & \sqrt{\gamma^b_R} & 0 & 0 \\
% 0 & 0 & \sqrt{\gamma^c_R} & 0 \\
% 0 & 0 & 0 & \sqrt{\gamma^c_R}
% \end{bsmallmatrix}\text{M}_b^{-1}\sqrt{2}\begin{bsmallmatrix}
% \sqrt{\gamma_b} & 0 & 0 & 0 \\
% 0 & \sqrt{\gamma_b} & 0 & 0 \\
% 0 & 0 & \sqrt{\gamma_c} & 0 \\
% 0 & 0 & 0 & \sqrt{\gamma_c}
% \end{bsmallmatrix}\Gamma^{-1}\\
% \text{R}^L_\text{PD}&=\sqrt{R_\text{PD}} \text{I}.
% \end{align}
% Where $\vec{\hat X}=[\hat X_{b,1},\hat X_{b,2},\hat X_{c,1},\hat X_{c,2}]^\text{T}$. % now contains the quadratures of both signal and idler.

The \emph{total quantum noise response} ($\text{S}_X$) is again given by Eq.~\ref{eq:dOPO_Sx_abstract} and found from analogous matrices to those in Eq.~\ref{eq:total_noise_matrix}~\cite{schoriNarrowbandFrequencyTunable2002}~\footnote{Which now apply to the vector of signal and idler quadratures $\vec{\hat X}=(\hat X_{b,1},\hat X_{b,2},\hat X_{c,1},\hat X_{c,2})^\text{T}$.}, where uncorrelated vacuum between the signal and idler modes is assumed, \jam{(too small)}
%Computing the matrix algebra shows that~\cite{}\jam{(too small, split elements up)}
% The total quantum noise response is given by the above matrices and Eq.~\ref{eq:dOPO_Sx_abstract}, where uncorrelated vacuum between the signal and idler modes is assumed. Computing the matrix algebra shows that~\cite{}\jam{(too small, split elements up)}
% \begin{equation}\label{eq:nOPO_full_freedom}
% \text{S}_X(\Omega)=\begin{bsmallmatrix}
% 1+\frac{8 \gamma^b_R {\gamma^c_\text{tot}} \chi ^2}{\left({\gamma^b_\text{tot}} {\gamma^c_\text{tot}}-\chi ^2\right)^2+\Omega ^2 \left({\gamma^b_\text{tot}}^2+{\gamma^c_\text{tot}}^2+2 \chi ^2\right)+\Omega ^4} & 0 & S_{3,1}^* & S_{4,1}^* \\
% 0 & S_{1,1} & S_{4,1}^* & -S_{3,1}^* \\
% -\frac{\sin (\phi )4 \chi  \sqrt{\gamma^b_R \gamma^c_R}  \left(\chi ^2+\Omega^2+\gamma^b_\text{tot}\gamma^c_\text{tot}+i\Omega\left(\gamma^c_\text{tot}-\gamma^b_\text{tot}\right)\right)}{\left({\gamma^b_\text{tot}} {\gamma^c_\text{tot}}-\chi ^2\right)^2+\Omega ^2 \left({\gamma^b_\text{tot}}^2+{\gamma^c_\text{tot}}^2+2 \chi ^2\right)+\Omega ^4} & S_{4,1} & 1+\frac{8 {\gamma^b_\text{tot}} \gamma^c_R  \chi ^2}{\left({\gamma^b_\text{tot}} {\gamma^c_\text{tot}}-\chi ^2\right)^2+\Omega ^2 \left({\gamma^b_\text{tot}}^2+{\gamma^c_\text{tot}}^2+2 \chi ^2\right)+\Omega ^4} & 0\\
% \frac{\cos (\phi ) 4 \chi  \sqrt{\gamma^b_R \gamma^c_R} \left(\chi ^2+\Omega^2+\gamma^b_\text{tot}\gamma^c_\text{tot}+i\Omega\left(\gamma^c_\text{tot}-\gamma^b_\text{tot}\right)\right)}{\left({\gamma^b_\text{tot}} {\gamma^c_\text{tot}}-\chi ^2\right)^2+\Omega ^2 \left({\gamma^b_\text{tot}}^2+{\gamma^c_\text{tot}}^2+2 \chi ^2\right)+\Omega ^4} & -S_{3,1} & 0 & S_{3,3}
% \end{bsmallmatrix}.\end{equation} 
{\small
\begin{align}\label{eq:nOPO_full_freedom}
(\text{S}_X)_{1,1}&=(\text{S}_X)_{2,2}=1+\frac{8 \gamma^b_R {\gamma^c_\text{tot}} \chi ^2}{\left({\gamma^b_\text{tot}} {\gamma^c_\text{tot}}-\chi ^2\right)^2+\Omega ^2 \left({\gamma^b_\text{tot}}^2+{\gamma^c_\text{tot}}^2+2 \chi ^2\right)+\Omega ^4}\\
(\text{S}_X)_{3,1}&=-(\text{S}_X)_{4,2}= -\frac{\sin (\phi )4 \chi  \sqrt{\gamma^b_R \gamma^c_R}  \left(\chi ^2+\Omega^2+\gamma^b_\text{tot}\gamma^c_\text{tot}+i\Omega\left(\gamma^c_\text{tot}-\gamma^b_\text{tot}\right)\right)}{\left({\gamma^b_\text{tot}} {\gamma^c_\text{tot}}-\chi ^2\right)^2+\Omega ^2 \left({\gamma^b_\text{tot}}^2+{\gamma^c_\text{tot}}^2+2 \chi ^2\right)+\Omega ^4}\nonumber\\
(\text{S}_X)_{3,3}&=(\text{S}_X)_{4,4}=\left((\text{S}_X)_{1,1}\right)|_{\hat{b}\leftrightarrow \hat{c}}\qquad (\text{S}_X)_{4,1}=(\text{S}_X)_{3,2}= -\cot(\phi)(\text{S}_X)_{3,1}\nonumber.
\end{align}}%
Here, $\text{S}_X$ is now divided into four two-by-two blocks: in the upper-left the signal variances and signal-signal covariance (which is zero), in the bottom-right the idler variances (equal to the signal variances under the exchange of rates $\gamma^b_R\leftrightarrow\gamma^c_R$ etc.) and idler-idler covariance (also zero), and in the off-diagonal blocks the signal-idler covariances which are not-zero but all closely related. Like the degenerate case, all expressions are rational functions, the variances are perturbations from the vacuum value of one, and the covariances obey the Hermitiancy of $\text{S}_X$ and vanish when the squeezer is off. 

\subsubsection{Results} %: differences from the degenerate case, threshold, and losses}
\label{sec:nOPO_results}

\begin{figure}
	\centering
	\includegraphics[width=\textwidth]{nOPO_noise.pdf}
	\caption{Nondegenerate OPO noise response, showing anti-squeezing in the signal mode, i.e.\ $\sqrt{(\text{S}_X)_{1,1}}$. The behaviour is similar to the anti-squeezed quadrature of the degenerate OPO in Fig.~\ref{fig:dOPO_variances} because the cavity resonance is the same. Left panel: different squeezer parameters up to threshold with no loss. Right panel: different symmetric intra-cavity losses $T_{l,b}=T_{l,c}$ for fixed ratio of $95\%$ threshold. For a 1~m cavity with symmetric readout transmission $T_{R,b}=T_{R,c}=0.1$. The noise in the idler mode is similar~\cite{schoriNarrowbandFrequencyTunable2002}.}
	\label{fig:nOPO_variances}
\end{figure}

% effect of losses, non-effect of pump phase
% resolution of the mystery of zero idler loss
There are two main differences to the degenerate case: all the variances are now anti-squeezed~\footnote{Which also makes them independent of the pump phase because no quadrature is squeezed.}, which will be explained later, and the signal and idler modes are correlated instead of the signal mode being correlated with itself~\footnote{Because the photons created in the parametric down-conversion are entangled regardless of what field they belong to.}.
% These differences must come from the change in the degeneracy of the parametric down-conversion.
% This explains the change in the covariances, the photons created in the process are entangled regardless of what field they belong to, but why no squeezing is seen for the nondegenerate case will only be clear later~\cite{}.
% The variances are independent of pump phase $\phi$ because there is only anti-squeezing and that the inflated noise ellipse is symmetric up to the loss rates.
However, the effects of (1) the squeezer parameter $\chi$ and (2) the intra-cavity loss are similar to the anti-squeezed quadrature of the degenerate case as shown in Fig.~\ref{fig:nOPO_variances}, where threshold is now $\chi_\text{thr}=\sqrt{\gamma^b_\text{tot}\gamma^c_\text{tot}}$ by Eq.~\ref{eq:nOPO_full_freedom}~\cite{schoriNarrowbandFrequencyTunable2002,martinelli2001classical}~\footnote{The signal and idler experience the same gain $\chi$, which is half the gain of the degenerate case\jam{(why? doesn't lasing start in one mode and not the other when losses are asymmetric?)}.}. %, the geometric mean of the total loss rates for each mode. %This means that anti-squeezing in the signal mode is limited by losses in the idler mode and vice versa, meaning that the squeezer cannot be turned on and remain below threshold without some idler loss being present. % -- a feature that will persist later. 
 %The experimental effect of threshold is similar to the degenerate case: it bounds the squeezer parameter to prevent lasing~\cite{}. 
% Secondly, the intra-cavity loss reduces and broadens the peak of the response in in Fig.~\ref{fig:nOPO_variances} for the same reasons as the degenerate case~\cite{}. %the detection loss scales the variances towards the vacuum while 
% Like threshold, the idler losses affect the signal variances and vice versa, but there are some differences, e.g.\ $\gamma^b_R {\gamma^c_\text{tot}}$ in the numerator of $S_{1,1}$ in Eq.~\ref{eq:nOPO_full_freedom} means that idler loss is less impactful than signal loss to the signal variance\jam{(I am claiming this by eye, check this with plots as it disagrees with my conclusion that nIS is tolerant to signal loss but not idler loss (also check nIS signal transfer function))}. 

\subsubsection{Recovering squeezing}
\label{sec:nOPO_combined_readout}
% is this directly relevant to the thesis? yes, to the combined readout preliminary work.
% \subsubsection{Reality of covariance measurements}% already spoken about this?
% % clarify measurement of non-Hermitian X(Omega) and normal operators, set up covariance 
% % cite Schori

% \begin{figure}
% 	\centering
% 	% \includegraphics[width=\textwidth]{}
% 	\caption{Nondegenerate OPO with combined readout, i.e.\ measurement of the combined quadrature in Eq.~\ref{eq:Scom_nOPO_eg}, showing how the nondegenerate case can use the correlations between the signal and idler to recover squeezing and behave like the degenerate case.}
% 	\label{fig:nOPO_combined_readout}
% \end{figure}

% These features of the nondegenerate OPO reviewed so far are well-known~\cite{} and useful to know to recognise them in later configurations, but 
% One remaining feature of the nondegenerate OPO that is worth discussing for its historical significance and to motivate some of the later\jam{(preliminary)} work in this thesis is the idea of making a combined measurement of the signal and idler modes.
To explain why only anti-squeezing is seen in the nondegenerate case, I consider making a \emph{coherently combined measurement of the signal and idler} modes. The motivation is that if the signal is measured, then the idler readout rate $\gamma^c_R$ decoheres the signal measurement by Eq.~\ref{eq:nOPO_full_freedom}\jam{(check)}, but if the idler mode is also measured then this readout rate would be useful, and vice versa. Consider a particular coherent~\footnote{As opposed to an incoherent combination where the fields are detected separately and then their variances added.}, linear combination of the signal and idler quadratures at the photodetector, \jam{(how do you physically combine the two modes if they are at different frequencies, do you just expose the PD to both?)} 
% \begin{equation}\label{eq:Xcom_eg}
$\hat X_\text{com}(\Omega)=\frac{1}{\sqrt{2}}(\hat X_{\text{PD};b,1}+\hat X_{\text{PD};c,1})$~\cite{schoriNarrowbandFrequencyTunable2002}.
% \end{equation} 
Then the combined variance is %\ given $\psi_2=\pi/4,\psi_1=\psi_0=0$,
% \begin{equation}\label{eq:Scom_nOPO_eg}
$S_\text{com}=\frac{1}{2}(\text{S}_X)_{1,1}+\frac{1}{2}(\text{S}_X)_{3,3}+\text{Re}[(\text{S}_X)_{3,1}]$.
% \end{equation} 
Here, the Hermitiancy of $\text{S}_X$ has been used and the vacuum value is still one~\footnote{In experiments~\cite{schoriNarrowbandFrequencyTunable2002}, the combination is not normalised and so should be compared to a higher vacuum value, e.g.\ $2$ for $\hat X_{b,1} + \hat X_{c,1}$.}~\footnote{Measuring $S_\text{com}$ has also been used to calculate the correlations between the signal and idler $(\text{S}_X)_{3,1}$ and demonstrate quantum entanglement in tests of the Einstein-Podolsky-Rosen (EPR) paradox~\cite{PhysRev.47.777,reidDemonstrationEinsteinPodolskyRosenParadox1989,schoriNarrowbandFrequencyTunable2002}. Although not relevant to this thesis, this historical association means that nondegenerate squeezing is commonly referred to as ``EPR squeezing''.}. 
% EPR significance <-- single sentence, not relevant to thesis
% By separately measuring $\text{S}_{1,1}$ and $\text{S}_{3,3}$~\footnote{Since $[\hat b, \hat c]=0$\jam{(does this assume a particular size of $\Delta$?)}, the quadratures of signal and idler commute and are therefore simultaneously observable.}, the combined variance $S_\text{com}$ can be used to measure the real part of the covariance~\cite{}. Similarly, a second measurement with complex linear coefficients can be made to measure the imaginary part~\cite{}.
Although each variance, e.g.\ $(\text{S}_X)_{1,1}$, is anti-squeezed, the correlation $\text{Re}[S_{3,1}]$ can be sufficiently negative such that the combined variance is squeezed overall. The above choice of linear combination achieves the minimum variance which is equivalent, when the signal and idler losses are the same, to the squeezed variance from a degenerate OPO in Eq.~\ref{eq:dOPO_fixed_phase}~\cite{ma_2017}~\footnote{This is the Wiener filter for the quantum noise.}.
% The combination angles of the combined variance can be optimised, e.g.\ to maximise anti-squeezing or squeezing. The pump phase $\phi$ can also be optimised, but there is redundancy in $\phi, \psi_0, \psi_1$ since only the relative phase matters. Consider minimising the quantum noise at the photodetector, i.e.\ the combined variance, by changing the combination angles. This is known as constructing a Weiner filter for the quantum noise~\cite{}. Although each variance from the nondegenerate OPO is anti-squeezed, the correlations mean that the combined variance can go below the vacuum value, e.g.\ $\text{Re}[S_{3,1}]$ can be sufficiently large and negative in Eq.~\ref{eq:Scom_nOPO_eg}. This optimisation\jam{(check this, over three angles looks more general than the working in vol 1)} converges on a minimum variance squeezed state with variance given by Eq.~\ref{eq:Scom_nOPO_eg}~\cite{}. Surprisingly, by examining the behaviour when the combination angles are fixed and the pump phase is varied, this variance reduces to that of a degenerate OPO when the losses are symmetric between signal and idler~\cite{}. %, as shown in Fig.~\ref{fig:nOPO_combined_readout}.
% In this sense, a nondegenerate OPO can recover the squeezing of a degenerate OPO.
This explains that the degenerate and nondegenerate OPOs produce the same correlations between the photons from the down-conversion, but that in the nondegenerate case the modes are different and have to be coherently combined to see the resulting squeezing. %This also explains why squeezing appears in the degenerate case but not immediately in the nondegenerate case. %(and balanced: $\psi_2=\pi/4$, symmetric losses)
% I will return to the idea of using a combined signal-idler readout later in this thesis, but the freedom and motivation to do so come from this application. %, in Chapter~\ref{chp:idler_readout} to maximise the signal-to--quantum noise ratio\jam{(check that I do)},

% I have now explained the general features of degenerate and nondegenerate squeezing which will reappear throughout this thesis as will the Hamiltonian modelling that I have used to explain them.


\section{Quantum noise in gravitational-wave detectors}
\label{sec:qnoise_GW_IFO}

% ``Quantum noise in gravitational wave detectors --> Photon counting noise, backaction noise, degenerate squeezed state injection'' - VA

\begin{figure}[ht]
	\centering
	\includegraphics[width=0.6\textwidth]{QN_SQL_ext_sqz.pdf}
	\caption{\jam{(Add frequency-dependent noise ellipses to external squeezing?)} Quantum noise response of the detector shown in Fig.~\ref{fig:DRFPMI} to radiation-pressure noise, shot noise, and total quantum noise. I use the parameters in Tab.~\ref{tab:dIS_parameters} (explained later) but with no losses. At kilohertz, shot noise is the dominant source of quantum noise. The point where the two sources of quantum noise balance is shown and is related to the Standard Quantum Limit but I will not discuss it here because it is weaker than the Mizuno limit and can also be beaten by squeezing~\cite{miaoFundamentalQuantumLimit2017}.
	The total noise with 10~dB injected, frequency-dependent external squeezing is also shown, and 10~dB squeezing is measured because there are not losses here. Since the signal response does not change, this uniformly improves the quantum noise--limited sensitivity.	
	%\jam{(update this appropriately)} %\jam{(Combine with Fig.~\ref{fig:simplifed_QN_response_conventional})} Sensitivity\jam{(change to QN?)} curve showing the effect of injected, frequency-dependent external squeezing on an interferometric detector. %The sensitivity is uniformly improved at all frequencies because the noise is squeezed and the signal response is not affected. The frequency-dependent noise ellipses are shown.
	} % in the signal-normalised curve\jam{(does this need to be a three-panel N, S, NSR plot?)}.} 
	\label{fig:simplifed_QN_response_conventional}
\end{figure}

% \jam{(Check this section against other theses. Add references)}
% To explain how squeezing is currently used to improve the quantum noise--limited sensitivity of gravitational-wave detectors, I will first discuss the quantum noise in such detectors.
The quantum noise in measuring the gravitational-wave signal with a detector like that shown in Fig.~\ref{fig:DRFPMI} comes from two sources: (1) quantum shot noise and (2) quantum radiation-pressure noise~\cite{PhysRevD.23.1693,corbitt_2003}.

% \subsubsection{Quantum shot noise}

The \emph{quantum noise} in the arrival time of the photons, i.e.\ the phase of the light, incident on the photodetector is called ``shot noise''~\cite{PhysRevD.23.1693}. This affects the measurement because a gravitational wave is detected by how it changes the phase of the light. %\jam{(how does it affect a quadrature measurement? if there are few photons then it is more prone to the random arrival times, GWD changes the phase accrued by each photon)}
Specifically, the Poissonian behaviour of the arrival time of each photon leads to noise that is frequency-independent (i.e.\ ``white'') and a measured signal-to--shot noise ratio~$\propto\sqrt{P_\text{circ}}$ which improves with increased circulating power ($P_\text{circ}$)~\cite{corbitt_2003}~\footnote{The known proportionality constant is not important to a conceptual understanding.}.
% \begin{equation}
% $\sqrt{S_h(\Omega)}\propto{P_\text{circ}}^{-1/2}$~\cite{}~\footnote{The known proportionality constant is not important to a conceptual understanding.}\jam{(check)}. % convert from displacement noise (see Miao thesis 2.1)
% \end{equation}
% Shot noise comes from the optical mode itself, specifically, from the fundamental uncertainty in the quadratures as described by the Heisenberg Uncertainty Principle in Eq.~\ref{eq:HUP_time_domain}\jam{(use frequency-domain)}.
In the Hamiltonian modelling, the vacuum entering the readout and loss ports produces noise with the vacuum value of one uniformly across all frequencies and quadratures and, in particular, noise in the phase quadrature of the vacuum entering the readout port becomes shot noise in the measured phase quadrature at the photodetector. Although the shot noise remains at one, the above shot noise--limited sensitivity improves with power because the signal response does. 
% Physically, this quantum noise manifests as so-called ``shot noise'' at the photodetector -- uncertainty in counting the number of photons incident on the photodetector~\cite{}. While quantum radiation pressure noise appears (with the gravitational-wave signal) in the amplitude quadrature in the arms, shot noise appears in all quadratures as it is associated with the vacuum\jam{(I am now confused why shot noise is associated with the phase quadrature? why is it conjugate to QRPN?)}. The shot noise response of an interferometric detector is given by $S_X(\Omega) = 1$\jam{(should show calculation? from dIS model with $\chi=0$)}, flat with frequency, which will be shown later.
% mention back-action as QPRN, mention free-falling mass in horizontal direction approximation

% \subsubsection{Quantum radiation-pressure noise}

The optomechanical interaction with the test mass mechanical modes causes \emph{quantum radiation-pressure noise}~\cite{PhysRevD.23.1693}.
The fluctuating amplitude of the light, i.e.\ the number of photons, incident on a suspended optic produces a fluctuating force due to radiation pressure~\footnote{This can also be interpreted as back-action noise by making a precise measurement of the position $\hat x$ at earlier times~\cite{danilishinQuantumMeasurementTheory2012}.\jam{(check)}} which becomes noise in the displacement of the optic and therefore in the propagation phase of the reflected light. %\jam{(``Additional displacement noise via the susceptibility of the mirror'' - VBA, means what? mechanical $\chi$ susceptibility gives 1/f^2 from the suspended optic)}
This means that noise in the amplitude quadrature of the vacuum entering the readout port becomes radiation-pressure noise in the measured phase quadrature at the photodetector.
This noise follows the mechanical resonance(s) of the test masses can be approximated above their resonant frequencies (e.g.\ above $\sim10$~Hz~\cite{PhysRevLett.116.131103}) as though the test masses are free-falling horizontally (i.e.\ are harmonic oscillators with a resonant frequency of 0~Hz and mass $M$~\cite{PhysRevLett.116.131103}) with the signal-to-radiation pressure noise ratio~$\propto M\Omega^2/\sqrt{P_\text{circ}}$~\cite{corbitt_2003}.
% \begin{equation}
% $\sqrt{S_h(\Omega)}\propto\sqrt{P_\text{circ}}/(\sqrt{M}\Omega^2)$.\jam{(check)}
% \end{equation}
Here, the noise decreases as $\Omega^{-2}$, is singular at DC ($\Omega=0$) in this approximation, and increases with increased power. %\jam{(Is $1/f^2$ because it only cares about the final stage)}
Therefore, as shown in Fig.~\ref{fig:simplifed_QN_response_conventional}, shot noise is the dominant source of quantum noise at kilohertz and I will focus on reducing shot noise through squeezing.
% Since the position $\hat x$ and momentum $\hat p$ of one of the test masses are described by their own Heisenberg Uncertainty Principle~\cite{}, making a precise measurement of the position $\hat x$, e.g.\ to measure the gravitational-wave strain~\footnote{Technically, the detector only measures the differential position of the test masses but the same principle applies~\cite{}.}, causes back-action noise in $\hat p$ which optomechanically couples to noise in the arm cavity optical mode~\cite{}.
% The radiation-pressure noise depends on the mechanical resonance but can be approximated as though the test masses are free-falling horizontally~\cite{}, i.e.\ resonant at DC, when studying the behaviour at kilohertz because the frequencies are well above the resonant frequencies of the test masses~\cite{}. Under this common approximation, the quantum radiation-pressure noise of an interferometric detector is given by\jam{... (look up radiation pressure formula)}~\cite{}, i.e.\ decreasing as $f^{-2}$ with increasing frequency $f$ and singular at DC.\jam{(Why $1/f^2$ if there are multiple stages of suspension? Why not $1/f^8$?)}

%\jam{(I have cut out the SQL because the Mizuno limit matters more, is it necessary to mention? Talk about circulating power?)} 
%, and so shot noise is the dominant noise source overall at kilohertz~\cite{}. This thesis focuses on improving the shot noise to most efficiently improve the kilohertz sensitivity. 

%the position of the test mass has some fundamental uncertainty which affects the optical mode. %, and this is captured through the quantum noise transfer function of the detector. 
% This contribution to the quantum noise is known as quantum radiation pressure noise~\cite{}. Physically, the amplitude quadrature of the arm cavity mode couples to the test mass through radiation pressure~\cite{}, therefore, uncertainty in the position of the test masses leads to fluctuations in the amplitude quadrature, and vice versa~\cite{danilishinQuantumMeasurementTheory2012?}. The response of the detector to this radiation pressure noise is related to the resonance behaviour of the mechanical mode, i.e.\ the resonances of the multi-stage suspension that holds the test mass~\cite{}. Although the suspension chain in a real detector has a non-trivial response due to the coupling of different mechanical modes, this response becomes simpler above the main pendulum resonance(s)~\cite{}. And because the kilohertz frequencies that I am interested in lie above these resonances, I will approximate the test mass as free-falling horizontally, i.e.\ that it is resonant at DC, $0$~Hz. This approximation is commonly made in the literature when studying the kilohertz response of interferometric detectors~\cite{}. The result of this approximation is that the quantum radiation pressure noise response of an interferometric detector is given by\jam{(look up radiation pressure formula)}~\cite{}, decreasing as $f^{-2}$ with increasing frequency $f$. 

% Similarly, uncertainty in the phase of the light (its phase quadrature) incident on a photodetector affects the counting statistics of the photodetector and therefore leads to uncertainty in the intensity and ultimately so-called ``shot noise'' in the measurement of the signal~\cite{}. \jam{(What is a counting statistic, explain this better)} %\jam{(phase isn't real, clarify this)} phase quadrature is measurable
% And so, (1) for an interferometer, the final measurement is affected by uncertainties in both quadratures of the light, but at different points in the detector (this will be clearer when I give a formal model in the next chapter), and (2) the quantum noise can be divided into the effects of shot noise, associated with the photodetector and the loss ports of the detector, and radiation pressure noise, associated with the test mass mechanical mode.\jam{(Clarify the association of loss ports with shot noise, what is shot noise only?)}

% Similarly, the positions and momenta of the test masses in a gravitational-wave detector are conjugate quantities. This introduces strange consequences of detection, e.g.\ as the position measurement of a test mass improves, its momentum becomes increasingly uncertain which couples through radiation pressure to the arm cavity mode~\cite{}.\jam{(follow up on this, what does this cause?)} 
% These consequences place limits on the quantum noise--limited sensitivity, i.e.\ the signal to quantum noise ratio, of a detector that cannot be surpassed by classical means which will be explained further later.

	% At high frequencies, shot noise is the dominant source of quantum noise and limits the sensitivity of current detectors. Therefore, reducing shot noise at high frequencies is the principal way to improve high-frequency sensitivity.

% \subsubsection{Limits on improving the quantum noise}

%\jam{(need to read up on SQL and QCRB to write this section properly.)}
% % SQL
% % review QCRB again? but did Mizuno limit on the integrated sensitivity in Introduction?
% The factors limiting improving the quantum noise--limited sensitivity of gravitational-wave detectors were discussed in Chapter~\ref{chp:introduction}. Since the signal transfer function falls off at kilohertz, because of the arm cavity resonance, and the quantum radiation pressure noise starts dominating the quantum noise below $10$~Hz, there is a window of detection around $100$~Hz where the sensitivity is greatest~\cite{}. In particular, in this window, there is a point where shot noise and radiation pressure noise are equal, shown in Fig.~\ref{fig:simplifed_QN_response_conventional}.  %The trade-off between the two\jam{(need to explain this tradeoff somewhere above, formulae should help)} means that 
%  %changing interferometer parameters (except for circulating power)
% This is known as the Standard Quantum Limit (SQL) for interferometric detection and is an instance of a more general result about quantum noise~\cite{}. Beating the Standard Quantum Limit is not possible using classical techniques\jam{(explain why?)} and without increasing the circulating power. This is similar to the Mizuno limit on the integrated sensitivity described in Chapter~\ref{chp:introduction}, which comes from a more fundamental limit on the quantum noise based on the total, theoretically-available information~\cite{miaoFundamentalQuantumLimit2017}. However, these two limits are both able to be beaten\jam{(need to explain how SQL is easier to beat than QCRB)} by quantum techniques~\cite{miaoFundamentalQuantumLimit2017,}, such as squeezing, which motivates the study of such technologies to improve the sensitivity.
%\jam{(this paragraph should motivate the need for squeezing it is almost there but misses many technical details)}
% These limits on improving the quantum noise motivate the need for technologies like squeezing which can get past them through manipulating the quantum state of the interferometer. 

% leave optical loss values in LIGO to external squeezing

\subsection{Squeezing in current gravitational-wave detectors}
\label{sec:external_squeezing}

% \begin{figure}
% 	\centering
% 	\includegraphics[width=0.7\textwidth]{ext_squeezing_config.pdf}
% 	\caption{\jam{(Merge with 1.2, add squeezed state arrow instead of OPO for later mention.)} Degenerate external squeezing configuration (top panel) that adds the degenerate OPO in Fig.~\ref{fig:OPOs_config} to the detector in Fig.~\ref{fig:DRFPMI} such that the output of the OPO replaces the vacuum into the readout port of the detector. The Faraday isolator is essential but contributes a large portion\jam{(quantify)} of the detection and injection losses~\cite{}. The noise ellipses and signal arrows (bottom panel) show that external squeezing decreases the noise and does not affect the signal (the height of the arrow).}
% 	% squeezing ellipse and signal arrow plot (+ show the effect of optical loss: injection, detection, intracavity)
% 	\label{fig:extSqz_config}
% \end{figure}

% \begin{figure}
% 	\centering
% 	% \includegraphics[width=\textwidth]{}
% 	\caption{\jam{(Combine with Fig.~\ref{fig:simplifed_QN_response_conventional})} Sensitivity curve showing the effect of frequency-dependent external squeezing on an interferometric detector. The frequency-dependent noise ellipses are shown.}
% 	\label{fig:extSqz_sensitivity}
% \end{figure}

% I can now explain how squeezing is currently used to improve the quantum noise--limited sensitivity of gravitational-wave detectors. As shown in Fig.~\ref{fig:DRFPMI}, degenerate external squeezing squeezes the vacuum entering
The sensitivity of current gravitational-wave detectors is improved by injecting the squeezed vacuum from a external degenerate OPO into the readout port via a Faraday isolator as shown in Fig.~\ref{fig:DRFPMI}~\cite{aasietal2013}~\footnote{Although essential, the Faraday isolator (also known as a circulator or directional beamsplitter) contributes significantly\jam{(quantify)} to the detection and injection losses, but the sensitivity is improved if the squeezing is high enough.}.
As the vacuum entering readout port is the dominant source of quantum noise in current gravitational-wave detectors, squeezing it reduces the shot noise in the measurement which improves the sensitivity because the signal is not affected by external squeezing~\cite{tseQuantumEnhancedAdvancedLIGO2019}.
 % with the amount of possible squeezing limited by the losses in the external OPO and the detector~\cite{}.
% Having examined the essential squeezing configurations (the OPOs), 
% I will now discuss the current benefits of using squeezing in interferometric detectors. In this section, I discuss the existing application of squeezing, external squeezing, used in gravitational-wave detectors today~\cite{} as well as the proposed improvements to this scheme. 
% In external squeezing, the vacuum into the readout port (the signal-recycling mirror) $\vec{\hat B}_\text{in}$, is squeezed to reduce the quantum noise in the measurement~\cite{}. This is accomplished by a degenerate OPO separate from the interferometer; the squeezed vacuum from the external squeezer is injected via a Faraday isolator~\cite{} (i.e.\ a directional beamsplitter) behind the signal-recycling mirror, as shown in Fig.~\ref{fig:extSqz_config}. This closes the readout port to the vacuum and replaces it with the squeezed vacuum.
% The squeezed vacuum from the external squeezer experiences some loss in the injection chain, including\jam{(mostly?)} at the Faraday isolator, which can be collectively attributed to an injection loss port with reflectivity $R_\text{inj}$. This injection loss and the intra-cavity loss in the external cavity limit how much squeezing can be produced. However, the Faraday isolator also increases the detection losses~\footnote{I.e.\ the Faraday isolator causes loss into and out of the readout port.}, it dominates the current detection losses in Advanced LIGO (which total to around $R_\text{PD}=10\%$ compared to total intra-cavity losses of $15\%$\jam{this seems way too high, check this})~\cite{}. The result is summarised in the noise ellipses and signal arrows in Fig.~\ref{fig:extSqz_config}, where external squeezing is designed to decrease the overall shot noise ellipse at the photodetector by more than the decrease in gravitational-wave signal due to the increased detection loss.
External squeezing is used in Advanced LIGO to improve the shot noise--limited sensitivity by a factor of $2$ at $\sim200$~Hz~\cite{aasietal2013,tseQuantumEnhancedAdvancedLIGO2019}~\footnote{Which is equivalent to doubling the circulating power but is far more practical.}.
% final sentence in this paragraph missing?
However, squeezing only the shot noise increases the radiation-pressure noise because they are associated with opposite quadratures of the input squeezed vacuum state. %Although they are measured in the same phase quadrature at the photodetector, the shot noise comes from noise in the input phase quadrature and the radiation-pressure noise comes from noise in the input amplitude quadrature which becomes phase noise by the optomechanical interaction~\cite{}\jam{(this repeats some details from above, is it clear if I remove it?)}.
This means that, although the quantum noise around 100~Hz is improved, it is worsened below 100~Hz~\cite{aasietal2013}.
%\jam{(Why are shot noise and QPRN in different quadratures and therefore squeezing one worsens the other? Look at input quadratures -- shot noise comes from phase quadrature and radiation pressure comes from amplitude quadrature. QRPN causes changes in phase noise because of the moving mirror. SQL FDExtSqz is at 45 and partially squeezes both?)}
% This problem will be solved in future detectors such as LIGO~Voyager~\cite{}\jam{(check)} by injecting frequency-dependent squeezing where the pump phase or orientation of the noise ellipse is optimally rotated using a series of filter cavities to squeeze the dominant source of quantum noise at each frequency~\cite{} as shown in Fig.~\ref{fig:simplifed_QN_response_conventional}.\jam{(dOPO variance has bandwidth s.t. at some point the squeezing falls off, do the filter cavities extend this to all frequencies as well?)}
Current LIGO detectors are undergoing an upgrade to get a broadband improvement in sensitivity by using a series of filter cavities to rotate the injected squeezed states and squeeze the dominant source of quantum noise at each frequency as shown in Fig.~\ref{fig:simplifed_QN_response_conventional}~\cite{Ganapathy_2021}.
This \emph{frequency-dependent external squeezing} is universally applicable~\footnote{Technically, the tolerance of each configuration to the readout port loss versus the other losses affects the sensitivity improvement with external squeezing. However, the readout port is typically the main vacuum source and this is a simple enough addition to future work.} to quantum noise--limited detectors since there is always vacuum entering the readout port and the injected squeezing is uncorrelated with any other squeezing in the detector, and, therefore, I will not include it in my work when comparing different configurations. % because it does not distinguish between them~\footnote{Similarly, I do not include other universal improvements such as a Caves's amplifier~\cite{}.}.


% I consider three questions about external squeezing: (1) why not squeeze all of the vacuum ports using the same method, (2) what is the effect of quantum radiation-pressure noise (notice the careful use of shot noise in the above paragraph), and (3) should I consider external squeezing in my work?
% why can't you squeeze the other ports
% Firstly, the simplicity of external squeezing comes from it only affecting one vacuum port and not affecting the signal, beyond the increased detection loss. 
% Firstly, because external squeezing only squeezes one port, it cannot reduce the shot noise below that associated with the intra-cavity and detection losses, which motivates considering whether these other losses can also be squeezed externally. Although the model has each of these other losses associated with a single port, in reality, they are distributed between the propagation medium and absorption~\footnote{And the vacuum from absorption (propagation) losses cannot be squeezed externally because it comes from the optic (medium) itself by the Fluctuation-Dissipation theorem~\cite{} rather than from a single outside direction.} and transmission\jam{(check terminology)} losses at each optic. Therefore, unlike the readout port at the signal-recycling mirror, the vacuum into these unified loss ports cannot be squeezed externally~\cite{}. 
% from the start of the squeezing section: ``The full story is more complicated, e.g.\ squeezing to decrease shot noise increases quantum radiation pressure noise in the final measurement, but this will be explained at the end of this section when I cover frequency-dependent external squeezing.''
% Secondly, as mentioned in Section~\ref{sec:qnoise_GW_IFO}, the quantum radiation-pressure noise in is the opposite quadrature to the shot noise~\cite{}\jam{(fix this, I do not understand why shot noise is not all quadratures)} but in the same quadrature as the gravitational-wave signal since they are both associated with the test mass mechanical mode. Therefore, when the shot noise quadrature of the input vacuum is squeezed externally, the radiation-pressure noise quadrature is anti-squeezed and the radiation-pressure noise is increased in the final measurement. Conversely, the external squeezer's pump phase could be chosen such that the shot noise is anti-squeezed and the radiation-pressure noise is squeezed. Depending on the frequencies of interest either of these could be beneficial; Advanced~LIGO uses shot noise squeezing because shot noise is dominant at and above $100$~Hz as shown in Fig.~\ref{fig:simplifed_QN_response_conventional}. The designs for future detectors (such as LIGO~Voyager~\cite{}, a planned third-generation detector) include improvements to external squeezing through reducing injection losses and using filter cavities to achieve frequency-dependent~\footnote{I.e.\ with pump phase $\phi(\Omega)$.} squeezing~\cite{}. Frequency-dependent squeezing reduces quantum noise at all frequencies by rotating the noise ellipse optimally\jam{(should I give the arctan formula for $\phi(\Omega)$?)}: squeezing radiation-pressure noise at low frequencies (below $10$~Hz), performing no squeezing at the Standard Quantum Limit\jam{(this disagrees with the curves where the whole curve drops uniformly, explain this, e.g.\ look at $\text{S}_X$ with $\phi=\pi/2$)}, and squeezing shot noise at high frequencies (at and above $100$~Hz)~\cite{}, as shown in Fig.~\ref{fig:simplifed_QN_response_conventional}. 
% mention how external squeezing could be added to each of the later models but why I don't (and that I have experimented with this in some preliminary work), and while it may be that it matters more to some configurations than others (e.g. whether squeezing or anti-squeezing)
% Finally, frequency-dependent external squeezing is included in many future detectors' designs~\cite{} because it is applicable and beneficial to all quantum noise--limited interferometers~\cite{}\jam{(check this)}, assuming that the added detection loss by including the Faraday isolator does not out-weigh the injected squeezing, which is achievable with current technology~\cite{}. Although the internal mode structure of the detector is arbitrary, the necessary photodetector is always associated with a vacuum port (see Section~\ref{sec:optical_loss_background}) and therefore a frequency-dependent external squeezer can improve the quantum noise by squeezing that vacuum. Even if other squeezers are present in the model, the other vacuum inputs are still uncorrelated with the squeezed vacuum input and so the effect on the measured variance $S_X$ is simply to change the normalisation of the readout vacuum term (i.e.\ the co-efficient of $\text{R}_\text{in} \text{R}_\text{in}^\dag$ in Eq.~\ref{eq:dOPO_Sx_abstract}) from $1$ to the squeezed variance. But this universal applicability also means that the performance of the different configurations need not be judged with the inclusion of external squeezing, since broadly speaking the benefits are similar~\footnote{Technically, the response of each configuration against the readout port loss versus the other losses needs to be considered because if Configuration A has low intra-cavity loss but high readout port loss response and Configuration B has the opposite, then Configuration A will benefit more from external squeezing since it will reduce the dominant loss. However, this is not particularly interesting and is a simple enough addition to future work that I do not include it in this thesis.}. Therefore, to simplify comparing configurations, I do not include external squeezing~\footnote{Similarly, I do not include other universal improvements such as a Caves's amplifier~\cite{}.}. % but note that it should be considered for any realisation of them.

% separate optical losses: intracavity and detection
	% 10% detection, 15% internal in aLIGO (why is internal so high?) -- from Korobko talk

% Generally speaking, the vacuum entering the main port of an arbitrary quantum noise--limited detector can be squeezed externally and injected via a Faraday isolator (a directional beam-splitter) to lower the quantum noise.

	% An example application of squeezing to reduce shot noise is the injection of squeezed vacuum from an external, degenerate squeezer which is used to approximately halve the shot noise in current detectors~\cite{tseQuantumEnhancedAdvancedLIGO2019}. The squeezed vacuum is injected behind the signal-recycling mirror via a Faraday isolator placed outside the signal-recycling cavity as shown in the left panel of Fig.~\ref{fig:coupled_cavities}.


% \subsubsection{Caves's amplifier}
% \label{sec:cavess_amp}

% \begin{figure}
% 	\centering
% 	% \includegraphics[width=\textwidth]{}
% 	\caption{Caves's amplifier configuration with noise ellipse and signal arrows to show the change in sensitivity, as in Fig.~\ref{fig:extSqz_config}. For large amplification, the detection loss is negligible.}
% 	\label{fig:Cavess_amplifier}
% \end{figure}

% % motivate investigation of nIS, also dIS using anti-squeezing
% % check whether the benefits of anti-squeezing have been mentioned yet
% A different external change to improve the quantum noise--limited sensitivity of an interferometer is to add a Caves's amplifier before the detection chain~\cite{}, a single-pass squeezer or an OPO\jam{(check if single-pass squeezer or OPO)} placed after the signal-recycling mirror in some manner as to not amplify the main input vacuum~\footnote{Using the Faraday isolator from external squeezing is problematic because the amplifier should be placed before the main sources of detection loss.}, shown in Fig.~\ref{fig:Cavess_amplifier}\jam{(how is it placed?)}. This configuration is worth mentioning %to understand the difference between placing the squeezer external and internal to the interferometer and
% to motivate the use of anti-squeezing to improve loss tolerance. 
% In this configuration, the output variance from the interferometer is anti-squeezed such that its quantum noise is well above\jam{(how far?)} the vacuum level, and the gravitational-wave signal is equally amplified because the variance as a whole is amplified. %~\footnote{This is where the difference lies between external and internal squeezing: the interaction with the signal (see the next chapter).}.
% Therefore, the amplifier does not affect the signal-to-noise ratio immediately, but upon experiencing the detection loss, the ratio improves because the addition of vacuum affects the noise less than before. 
% %the signal and the noise both decrease whereas without the amplifier if the noise was below vacuum level it would have increased\jam{(this is not the only effect, if QN is above vacuum, amplification still helps)}.
% To quantify this, let the Caves's amplification be $A_\text{Caves}$\jam{(explain what this is in terms of squeezer parameter?)}, the output noise variance and signal transfer function be $S$ and $\abs{T}$, respectively, and the detection loss be $R_\text{PD}$, then the improvement in signal-to-noise ratio at the photodetector is
% \begin{align}\text{signal-to-noise} = \frac{\sqrt{1-R_\text{PD}}\abs{T}}{\sqrt{R_\text{PD}+(1-R_\text{PD})S}}&\mapsto\frac{A_\text{Caves}\sqrt{1-R_\text{PD}}\abs{T}}{\sqrt{R_\text{PD}+A_\text{Caves}^2(1-R_\text{PD})S}}\\
% &=\frac{\sqrt{1-R_\text{PD}}\abs{T}}{\sqrt{R_\text{PD}/A_\text{Caves}^2+(1-R_\text{PD})S}}\nonumber\\
% &\xrightarrow[A_\text{Caves}\rightarrow\infty]{} \frac{T}{\sqrt S}.\nonumber
% \end{align}
% ~\footnote{For example, for detection loss $R_\text{PD}$, anti-squeezing gain $g$, and signal $T$ and noise $S_X$ responses at the output, the noise-to-signal ratio is $\sqrt{g^2(1-R_\text{PD})S_X+R_\text{PD}}/(g\sqrt{1-R_\text{PD}}\abs{T})$. In the limit of $g\rightarrow\infty$, this becomes $\sqrt{S_X}/\abs{T}$ which is equivalent to no loss, i.e.\ $R_\text{PD}=0$.}
% That is, in the limit of large amplification, the detection loss is negligible, which is also shown in Fig.~\ref{fig:Cavess_amplifier}. Although, losses and threshold mean that arbitrarily large amplification is not possible.\jam{(What about the interaction of the amplifier and radiation pressure noise?)} Similarly to external squeezing, because the Caves's amplifier is universally applicable, I will not include it in my models~\footnote{Where again, technically, the improvement to two configurations is not necessarily the same since it depends on whether the detection loss dominates the noise spectrum.}. There are two things to remember from this configuration: (1) the external amplifier only affects signal and noise equally because they approach it equally (both from the black-box\jam{(colloquial?)} of the interferometer readout) and (2) detection losses can be addressed by amplification, i.e.\ anti-squeezing.

% anything to say for the end of the section? no, just use the summary.

%%%%%%%%%%%%%%%%%%%%%%%%%%%%%%%%%%%%%%%%%%
\section{Chapter summary}

In this chapter, I have revised the necessary quantum mechanics to explain the benefits of squeezing for gravitational-wave detectors and discuss the different configurations in the following chapters.
Firstly, I set up the mathematical framework of quantum noise in the quadratures of the light and vacuum inside a detector, how these are described by the Heisenberg Uncertainty Principle, and introduced the idea of squeezing the vacuum to reduce the quantum noise by using a non-linear crystal.  %I set up the mathematical formalism of quantum noise, discussed how optical loss leads to decoherence, and described the quantum noise response of an interferometer.
Secondly, I showed how squeezing can be understood using the analytic, Hamiltonian modelling that I will use throughout this thesis. I considered both the degenerate and nondegenerate OPOs and demonstrated squeezing and anti-squeezing. I also explained the threshold of a squeezing configuration and how optical loss decoheres the state and worsens the noise. %, and the effect of pump phase, for both degenerate and nondegenerate optical parametric oscillators (OPOs) -- the quintessential squeezing configurations.
Finally, I explained the quantum noises inside of a gravitational-wave detector and how external squeezing is currently used to improve the quantum noise--limited sensitivity of Advanced~LIGO. % and how the use of an external squeezer is of universal interest to the improvement of detectors, and therefore I will not include them in my models when comparing configurations that would benefit roughly equally from them. % and the designs of future detectors.


