\chapter{Existing proposals for improving kilohertz sensitivity} % Background: existing proposals for improving kilohertz sensitivity
\label{chp:proposals}
%%%%%%%%%%%%%%%%%%%%%%%%%%%%%%%%%%%%%%%%%%
% chapter introduction
% Beyond better external squeezing, there are many \jam{(are there?)} existing proposals to improve the kilohertz sensitivity of gravitational-wave detectors. In the next two chapters, I examine two of the front-runners: degenerate internal squeezing and stable optomechanical filtering.
% demonstrate modelling again, for the third and last time before my model -- much of the maths should be quite familiar by now
% set-up the problem: how do these configs get past the four factors in the introduction and why aren't they good enough for kilohertz sensitivity

In this chapter, I will consider some of the existing configurations that address the problem of increasing kilohertz sensitivity outlined in Section~\ref{sec:intro_factors_limiting_kHz} and that motivate the configuration that I examine. After first summarising the literature, I will focus on two promising proposals: degenerate internal squeezing~\cite{} and stable optomechanical filtering~\cite{}. I will use the tools of the previous chapter to understand and model degenerate internal squeezing and demonstrate how I will approach my work. Then, I will discuss stable optomechanical filtering using the same ideas. %, although I do not present a separate model of it for reasons that will become clear. 
%Stable optomechanical filtering is the optomechanical analogue of my all-optical configuration which implies that there is . 
While promising, these existing proposals have limited tolerance to different losses: optical and mechanical loss, respectively. This motivates my work in the following chapters that focuses on a configuration that is closely related to these proposals but might be able to overcome these limitations and better improve kilohertz sensitivity.


%%%%%%%%%%%%%%%%%%%%%%%%%%%%%%%%%%%%%%%%%%
\section{Literature review}
\label{sec:literature_review}
% should be categorising and judging

% John Close says: don't just list papers, a powerful review identifies the goal of the thesis and puts it in context, ideally quantitatively --> what are the quantities for my thesis, sensitivity?

\jam{(The role of this section is (1) to explain why I look at these two configurations and (2) satisfy the thesis requirement for a literature review, which I need to look up in the honours material. Do I need more citations or critical analysis, perhaps?)}

% intro: beyond external changes (external squeezing and caves's amplifier) to interferometer and increasing circulating power
Various solutions to improve the kilohertz sensitivity of gravitational-wave detectors has been given in the literature~\cite{}. Any proposal to do so without increasing circulating power or degrading existing sensitivity must overcome the Mizuno Limit through some non-classical technique \jam{(how does sWLC avoid Mizuno limit? why is cancelling arm cavity resonance non-classical?)}, as outlined in Section~\ref{sec:intro_factors_limiting_kHz}. Broadband improvement is possible through external modifications to the interferometer such as external squeezing and Caves's amplification, as already discussed, but I am interested in techniques that directly address kilohertz sensitivity. In this review, I will summarise: (1) the existing gravitational-wave detectors around the world, (2) the future, third-generation detectors under conceptualisation and construction \jam{(are they being constructed right now?)}, (3) the proposals that use internal squeezing, (4) the proposals that use optomechanical filtering, and (5) some of the other existing proposals. %However, I will refrain from a thorough technical explanation of these configurations, which will be given later in this chapter. 

\jam{(I will need to do more reading to write this section convincingly. See supervisors for citations. Do I need to really review detectors?)}
% existing detectors
The global network of existing gravitational-wave detectors consists of Advanced~LIGO~\cite{} (Hanford~\cite{} and Livingston~\cite{} sites), Advanced~Virgo~\cite{}, KAGRA~\cite{}, and GEO600~\cite{} \jam{(check this list)}. All of these follow the interferometer design discussed in Section~\ref{sec:intro_IFO} with the only major change relevant for quantum noise \jam{(check this)} being the use of external squeezing in Advanced~LIGO, discussed in Section~\ref{sec:external_squeezing}. I will use Advanced~LIGO to represent this generation, henceforth. 
% 3rd generation detectors: LIGO Voyager, NEMO, Einstein Telscope, Cosmic Explorer --> why I will focus on the Voyager parameter set (no increased circulating power, no change in arm lengths)
The proposed next, third generation of \jam{(ground-based)} detectors consists of LIGO~Voyager~\cite{}, NEMO~\cite{}, Einstein~Telescope~\cite{}, and Cosmic~Explorer~\cite{} \jam{(check 2.5 versus 3 generation)}. These still use the same interferometer design \jam{(with the exception of NEMO~\cite{}?)} \jam{(check this)} but differ from the previous generation in the interferometer parameters, such as the circulating power, arm length, test mass mass, carrier wavelength, and transmissivities of the input test mass and the signal-recycling mirror~\cite{}. 
The later detectors, Einstein~Telescope and Cosmic~Explorer, are proposed to have significantly longer arm lengths and higher circulating powers than LIGO~Voyager or the current generation of detectors. 
I will consider the feasibility of configurations using the LIGO~Voyager parameter set~\cite{} because it will be the first of these future detectors to come online as it is a series of upgrades to the existing Advanced~LIGO detectors~\cite{}. Moreover, variations of the LIGO~Voyager parameter set are often used in the literature to assess the configurations in this chapter~\cite{Li2020,Miao2018,Korobko,}. In particular, I will use \jam{(tabulate these parameters to reference later)} $3$~MW circulating power, $4$~km long arm cavity, $56$~m long signal-recycling cavity, $200$~kg test masses, and $2~\mu\text{m}$ carrier wavelength~\footnote{There is debate about $2$ versus $1.064~\mu\text{m}$ as the preferred carrier wavelength~\cite{}, e.g.\ different photodiode efficiencies~\cite{}. I will not consider this because my models are too simplified to compare the technological constraints.}, but I will vary the transmissivities away from their values of $0.002$ for the input test mass and $0.046$ for the signal-recycling mirror to test the effects of changing the respective coupling rates~\footnote{There is a subtlety here, these parameters might be biased against one configuration over another. I have mitigated this by varying the coupling rates between the modes which appears to well characterise the different classes of parameter sets~\cite{}.}. 

% separate from these defined detectors is a literature of different possible configurations and proposals for these and future detectors.
These future detectors have established science-cases \jam{(colloquial?)} and detailed proposals, but separate from them in the literature is much exploratory work which could be realised in a non-specific future detector, and this is where my work fits into. %~\footnote{It is worth clarifying here that I am not aiming to produce a complete proposal for a future detector either, only explore the space of designs}.
% degenerate internal squeezing
One such proposal is degenerate internal squeezing (also known as a degenerate quantum expander)~\cite{Korobko2019,Adya2020}, which has been considered for NEMO~\cite{}. 
Degenerate internal squeezing was first proposed in Ref.~\cite{Korobko2019} and has been subsequently characterised in Refs.~\cite{Adya2020,Korobko?,}. 
% I will explain this configuration thoroughly later,
This configuration has a squeezer inside the signal-recycling cavity, turning the two-cavity interferometer into an OPO coupled to the arm cavity. Unlike external squeezing, where only the readout port vacuum is squeezed, degenerate internal squeezing also squeezes the intra-cavity noise and the gravitational-wave signal. Although de-amplifying (squeezing) the signal is not ideal, the reduction in the quantum noise is significant enough to produce an improvement in sensitivity. This improvement can occur at kilohertz if the interferometer parameters are chosen correctly~\cite{}. However, degenerate internal squeezing is sensitive to intra-cavity and detection losses, since these losses reduce the signal but amplify the noise towards vacuum. 
Research into degenerate internal squeezing continues today~\cite{}, e.g.\ for its ability to be switched into an anti-squeezing regime in the high loss limit~\footnote{This limit is beyond the expected losses in future detectors, e.g.\ above $50\%$ compared to $10\%$, but it demonstrates that internal anti-squeezing is useful under certain circumstances.}~\cite{KorobkoTalk}, which improves sensitivity similarly to a Caves's amplifier, see Section~\ref{sec:cavess_amp}, except that the amount of anti-squeezing can be different for signal and noise.
% is there any GW community interest in nondegenerate squeezing?
% But nondegenerate internal squeezing, where the internal squeezing is operated nondegenerately, is an alternative that has not yet been thoroughly considered~\cite{} -- I will identify this gap in the literature in Section~\ref{} \jam{(haven't I identified it here?)}. 

% optomechanical filtering: unstable~\cite{Miao2015,Miao2018,Page2018} and stable~\cite{Li2020,Li2021}
\jam{(I should review my research journals for this paragraph)}
Another configuration, popular \jam{(colloquial?)} in the literature is stable optomechanical filtering~\cite{Li2020,Li2021,Miao2015,Miao2018,Page2018}. Optomechanical filtering in the signal-recycling cavity of an interferometer was first proposed in Ref.~\cite{Miao2015} in an unstable configuration. This configuration couples the optical mode in the signal-recycling cavity to a mechanical mode. The optomechanical coupling can be chosen such that signal-recycling (filter) cavity imparts the opposite round-trip phase to the arm cavity at certain frequencies, which broadens the arm cavity resonance (the ``white-light cavity'' idea~\cite{}).
% When the readout occurs via the arm cavity mode, the system is unstable and needs a feedback control system,
The unstable version of this system, which required a feedback control system to stabilise, was further investigated in Refs.~\cite{Miao2018,Page2018,} \jam{(which contributed what?)}. 
This system was made stable in Refs.~\cite{Li2020} by changing the readout to read out the signal-recycling cavity mode instead of the arm cavity mode. The model did not include optical loss but showed kilohertz and broadband improvement. The stable configuration was further investigated in Ref.~\cite{Li2021} \jam{(what was its contribution?)}.
% But if the readout is changed to the signal-recycling (filter) cavity optical mode, then the system becomes stable, which has been investigated in Refs.~\cite{Li2020,Li2021} and produces broadband sensitivity improvement in the lossless case.
However, this system has stringent mechanical requirements to keep the thermal noise low, requiring a high mechanical quality factor and low environmental temperature~\cite{}. Research is underway into different optomechanical techniques to achieve the demands of this configuration, such as optical springs~\cite{}, cat-flap resonators~\cite{}, and \jam{(... look at Miao, Page references)}.

% other configurations + configurations not based on the Michelson interferometer: speed-meters etc.
Finally, there are other proposals for kilohertz improvement~\cite{,,} unrelated to these two that also do not increase the circulating power in the arms, some of which are not even based on the Michelson interferometer~\cite{}. I am not aware of all that has been suggested, but some of the configurations that continue to generate interest in the literature are \jam{..., ..., and ... . (I do not know and need to read up to fill this in!)}. 
% this is just an honours project 
However, the time constraints of my research have only allowed me to consider degenerate internal squeezing and stable optomechanical filtering as they are the most related configurations to my work. %, since nondegenerate internal squeezing is to degenerate internal squeezing as the nondegenerate OPO is to the degenerate OPO, and nondegenerate internal squeezing is an all-optical analogue of the stable optomechanical filtering. % I can't claim that they are frontrunners, what evidence?
Therefore, I consider only them as a result of finite time and relevance. 
% should not be taken as a claim of their superior feasibility to the rest of the literature but as


%%%%%%%%%%%%%%%%%%%%%%%%%%%%%%%%%%%%%%%%%%
\section{Existing proposal 1: degenerate internal squeezing}

\begin{figure}
	\centering
	\includegraphics[angle=-90,width=0.9\textwidth]{dIS_config.pdf}
% squeezing ellipse and signal arrow plot (+ show the effect of optical loss: detection and intracavity)
	\caption{Degenerate internal squeezing configuration (top panel), all modes are labelled, and the squeezer's effect on the output fields (bottom panel) using the noise ellipse and signal arrow representation as in Fig.~\ref{fig:extSqz_config}. The signal-recycling cavity resembles a degenerate OPO and uses the same notation as Fig.~\ref{fig:dOPO_config}. Where the signal $h(t)$ and noises $\hat{B}_\text{in},\hat{n}^L_b,\hat{n}^L_a$ enter the system is shown. The sensitivity is improved by squeezing the noise more than the signal.}
	\label{fig:dIS_config}
\end{figure}

% \begin{figure}
% 	\centering
% 	% \includegraphics[width=\textwidth]{}
% 	\caption{Degenerate internal squeezing noise ellipse and signal arrow, compare to external squeezing in Fig.~\ref{fig:extSqz_ellipse_arrow} and Caves's amplifier in Fig.~\ref{fig:Cavess_amplifier}.}
% 	\label{fig:dIS_ellipse_and_arrow}
% \end{figure}

\jam{(Update once the analytic model is cut.)}

\jam{(Explain why it works.)}
% has been explained many times before, but do it again here because it is the main reference
% how does this configuration beat the Mizuno limit (four factors in the introduction)? --> squeezing
Degenerate internal squeezing consists of a degenerate squeezer placed inside the signal-recycling cavity of an interferometer such that it squeezes the signal mode at the carrier frequency~\cite{}, as shown in Fig.~\ref{fig:dIS_config}. The internal placement means that it squeezes not just the vacuum from the readout port, like external squeezing, but also the noise from the intra-cavity loss ports and the gravitational-wave signal. Although a single-pass of the squeezer squeezes the signal and noise equally~\cite{}, the overall amount of squeezing at the output is different for the signal and noise because of their different origins. The signal comes from the test mass in the arms and the noise comes from each cavity and the readout, as shown in Fig.~\ref{fig:dIS_config}. This means that the signal and noise ``see'' the signal-recycling cavity and the squeezer differently, which changes the average number of single-passes of the squeezer each experiences~\cite{}. 
Therefore, degenerate internal squeezing improves the sensitivity and overcomes the Mizuno limit from Section~\ref{sec:intro_factors_limiting_kHz} by squeezing the noise more than the signal~\cite{}, as shown in the signal arrow and noise ellipse representation in Fig.~\ref{fig:dIS_config}.
%The overall effect is to de-amplify the signal but squeeze the noise more such that the sensitivity improves, as shown in Fig.~\ref{fig:dIS_ellipse_and_arrow}. Where the improvement overcomes the Mizuno Limit from Section~\ref{} through the use of squeezing. %However, this improvement only occurs around the ``sloshing'' frequency~\cite{}, the coupling frequency of the two cavities, which will be shown. 

% In this section, I present a model of degenerate internal squeezing and discuss its behaviour. These results exist in the literature~\cite{Korobko2019}, but I demonstrate them, abridged, here to justify the approach I use in my work and establish what is expected from internal squeezing. In particular, I demonstrate how to consider radiation pressure, the gravitational-wave signal, and stability, which I have not yet explained.
%bridge from the OPOs in the previous chapter to my work on nondegenerate internal squeezing in the next chapter.
%The structure of the results presented here will be mirrored in the next chapter which should make those later results more convincing.  % what do I intend the reader to get out of it?
% \jam{(I do not just show this model/results because I spent time with them. Is it clear what I want the audience to get out of them?)}

	% A proposed technology to further reduce shot noise is degenerate internal squeezing where a degenerate squeezer is included inside the signal-recycling cavity as shown in the left panel of Fig.~\ref{fig:coupled_cavities}~\cite{korobkoQuantumExpanderGravitationalwave2019,adyaQuantumEnhancedKHz2020}.
	% To simplify modelling this configuration, the two coupled optical modes of concern, the differential mode from the arm cavities and the mode in the signal-recycling cavity, are approximated as coming from a pair of coupled cavities as shown in the right panel of Fig.~\ref{fig:coupled_cavities}. The quantum noise enters with the vacuum into the signal-recycling cavity, but the signal arises in the arm cavity and so sees the squeezer differently. This difference leads to no change in sensitivity at low frequencies and improved sensitivity at high frequencies around the resonance of the signal-recycling cavity as shown in Fig.~\ref{fig:sensitivity_curve}. The improvement occurs around the signal-recycling cavity resonance as this is where the most quantum noise field is present inside the cavity and so is where the internal squeezer produces the most squeezing. The resulting noise reduction is sufficient to overcome the de-amplification of the signal due to squeezing. The overall sensitivity improvement from degenerate internal squeezing is limited by optical loss in the interferometer and at the photodetector~\cite{korobkoQuantumExpanderGravitationalwave2019}. 


\begin{comment}
\subsection{Analytic model}
\label{sec:dIS_model}

\jam{(Is showing this model necessary?)}

% I have verified this model against Korobko
I construct a Hamiltonian model of degenerate internal squeezing using the formalism from Chapter~\ref{chp:background_theory}. This model is based on and verified against Ref.~\cite{Korobko2019}. %I include it here to demonstrate the Hamiltonian method that I later use in my work, particularly the radiation pressure, gravitational-wave signal, and the system's stability \jam{(repetition with section intro)}. 
Since degenerate internal squeezing is a degenerate OPO coupled to another cavity, as shown in Fig.~\ref{fig:dIS_config}, I use the degenerate OPO model in Section~\ref{sec:dOPO_model} and add the extra modes associated with the arm cavity. Let the signal-recycling cavity and output modes be labelled as in Section~\ref{sec:dOPO_model} and let the arm cavity mode at carrier frequency $\omega_0$ be $\hat a$ (which is resonant in the single-mode approximation) with an intra-cavity loss port with transmissivity $T_{l,a}$ into vacuum $\hat n^L_a$. Let the gravitational-wave signal $h(t)$ from Section~\ref{sec:gravWaves} be coupled to the arm cavity mode by the test mass mechanical mode given by displacement $\hat x$ and momentum $\hat p$ (approximated as free-falling horizontally as explained in Section~\ref{sec:qnoise_GW_IFO}). 
The Hamiltonian of this system is $\hat H = \hat H_0 + \hat H_I + \hat H_\gamma + \hat H_\text{GW}$ where~\cite{}
\jam{(fill in Langevin Hamiltonian)}
\begin{align}
\hat H_0 &= \hbar \omega_0 \hat a^\dag \hat a + \hbar \omega_0 \hat b^\dag \hat b + \hbar 2\omega_0 \hat u^\dag \hat u\\
\hat H_I &= i\hbar\omega_s(\hat a\hat b^\dag-\hat a^\dag\hat b) +\hbar \frac{x}{2} (e^{i\phi} \hat u (\hat b^\dag)^2 + e^{-i\phi} \hat u^\dag \hat b^2)\\
\hat H_\gamma &= \int \ldots \\
\hat H_\text{GW} &= -\alpha (\hat{x}-L_\mathrm{arm}h(t))\left(\frac{\hat{a}+\hat{a}^\dag}{\sqrt{2}}\right)+\frac{1}{2\mu}\hat{p}^2.
\end{align}
% \beta = \frac{\alpha L_\mathrm{arm}}{\sqrt{2}\hbar}, \beta = \sqrt{\frac{2 P_\text{circ} L_\text{arm} \omega_0}{\hbar c}}, \alpha = \sqrt{2} \hbar/L_arm= \sqrt{\frac{4 P_\text{circ} \omega_0 \hbar}{c  L_\text{arm}}}
% multiple different hamiltonians for RP
% problem with omega_s formula, only valid below one FSR of the arms
Where $\alpha=\sqrt{\frac{2 P_\text{circ} \omega_0 \hbar}{c  L_\text{arm}}}$ \jam{(clarify that this is the Li formula which Korobko disagrees with by a factor of rt2 for reasons unknown. Why do they disagree? I use the Li value for comparison to sWLC but this affects the feasibility of nIS! Ask the supervisors.)} is the coupling strength to the gravitational-wave signal~\cite{}, $\mu=M/4$ is the reduced mass \jam{(Explain reduced mass.)} of the test mass with mass $M$~\cite{}, and $\omega_s\approx c\sqrt{\frac{T_\text{ITM}}{4 L_\text{arm} L_\text{SRM}}}$ is an approximation to the sloshing frequency between the coupled cavities (also known as the coupled cavity pole) which holds when it is below one FSR of the arm cavities \jam{(check this)}~\cite{}. I use $\hat H_\text{GW}$ from Ref.~\cite{Li2020,original source?} which couples the mirror position and gravitational wave length displacement $L_\text{arm} h(t)$ to the amplitude quadrature of the cavity mode~\footnote{Although, perhaps, a more natural formulation would couple the gravitational-wave strain to the mirror position and the mirror position to the cavity mode, this is equivalent~\cite{}.}. 
The Heisenberg-Langevin equations-of-motion for $\hat a, \hat b, \hat x$ and $\hat p$ can be found using the bosonic commutation relations, the canonical commutation relation $[\hat x,\hat p]=i\hbar$~\cite{}, and with all other commutators zero. As in Section~\ref{sec:dOPO_model}, I make semi-classical and no-pump-depletion approximations to the pump mode, enter the Interaction Picture, and take fluctuating components of the operators. The resulting equations are
\begin{equation}\label{eq:dIS_EoM}\begin{cases}
\dot{\hat a}=-\omega_s \hat b- \gamma_a \hat{a} + \sqrt{2\gamma_a}\hat{n}^L_a+\frac{i \alpha}{\hbar\sqrt2}(\hat{x}-L_\mathrm{arm}h(t)) \\
\dot{\hat b}=\omega_s \hat a-i\chi e^{i\phi} \hat b^\dag - \gamma^b_\mathrm{tot} \hat{b} + \sqrt{2\gamma^b_R}\hat{B}_\mathrm{in} + \sqrt{2\gamma_b}\hat{n}^L_b\\
\dot{\hat x}=\frac{1}{\mu}\hat p\\
\dot{\hat p}=\alpha\left(\frac{\hat{a}+\hat{a}^\dag}{\sqrt{2}}\right).
\end{cases}\end{equation}
By entering the Fourier domain, I solve these equations for $\vec{\hat b}(\Omega)=[\hat b(\Omega),\hat b^\dag(-\Omega)]^\text{T}$ in terms of similar vectors of the vacuum sources and signal, i.e.\ $\vec h(\Omega)=[\tilde h(\Omega),\tilde h^*(-\Omega)]^\text{T}$, I find that
\begin{align}
\text{M}_a\vec{\hat a}(\Omega)&=-\omega_s\vec{\hat b}(\Omega) + \sqrt{2\gamma_a}\vec{\hat n}^L_a(\Omega)-i\beta\begin{bsmallmatrix}1 & 0 \\0 & -1\end{bsmallmatrix}\vec h(\Omega) \\
\text{M}_a&= (\gamma_a-i \Omega)\text{I}+\frac{i \rho}{\Omega^2 \sqrt 2}\begin{bsmallmatrix}1 & 1 \\-1 & -1\end{bsmallmatrix} \\
\therefore\text{M}_b\vec{\hat b}(\Omega)&=\sqrt{2\gamma^b_R}\vec{\hat B}_\mathrm{in}(\Omega) + \sqrt{2\gamma_b}\vec{\hat n}^L_b(\Omega)+\omega_s \text{M}_a^{-1}\left(\sqrt{2\gamma_a}\vec{\hat n}^L_a(\Omega)-i\beta\begin{bsmallmatrix}1 & 0 \\0 & -1\end{bsmallmatrix}\vec h(\Omega)\right)\\
\text{M}_b&= (\gamma^b_\mathrm{tot}-i \Omega)\text{I}+\chi \begin{bsmallmatrix}0 & i e^{i\phi}\\-i e^{-i\phi} & 0\end{bsmallmatrix}+\omega_s^2\text{M}_a^{-1}.
\end{align}
Where the re-scaled coupling constants for the gravitational-wave signal and the radiation pressure, respectively~\footnote{As expected, $\mu=M/4\rightarrow\infty$ turns off the radiation pressure. Therefore, I will use $\rho=0$ to turn off the radiation-pressure noise.}, are
\begin{equation}\label{eq:beta_and_rho}
\beta = \frac{\alpha L_\mathrm{arm}}{\sqrt{2}\hbar}=\sqrt{\frac{ P_\text{circ}L_\text{arm} \omega_0 }{c  \hbar}},\quad \rho = \frac{\alpha^2}{\sqrt{2}\hbar\mu}=\frac{\sqrt{2} P_\text{circ} \omega_0}{c \mu L_\text{arm}}.
\end{equation}
\jam{(this disagrees with Korobko since I use the alpha from Li, note this somewhere)}
Using the same input/output relations as Section~\ref{sec:dOPO_model} and $\Gamma=\frac{1}{\sqrt2}\begin{bsmallmatrix}1 & 1 \\-i & i\end{bsmallmatrix}$, I find the quadratures at the photodetector to be
\begin{align}
\vec{\hat X}_\mathrm{PD}(\Omega)&=\text{T}\vec h(\Omega)+\text{R}_\text{in}\vec{\hat X}_\mathrm{in}(\Omega)+\text{R}^L_a\vec{\hat X}^L_a(\Omega)+\text{R}^L_b\vec{\hat X}^L_b(\Omega)+\text{R}^L_\text{PD}\vec{\hat X}^L_\text{PD}(\Omega)\\
\text{T}&=-\sqrt{1-R_\text{PD}}\omega_s(-i\beta)\Gamma \sqrt{2\gamma^b_R}\text{M}_b^{-1}\text{M}_a^{-1}\begin{bsmallmatrix}1 & 0 \\0 & -1\end{bsmallmatrix}\\
\text{R}_\text{in}&=\sqrt{1-R_\text{PD}}\Gamma\left(\text{I}-2\gamma^b_R\text{M}_b^{-1}\right)\Gamma^{-1}\\
\text{R}^L_a&=-\sqrt{1-R_\text{PD}}\omega_s\Gamma 2\sqrt{\gamma^b_R \gamma_a}\text{M}_b^{-1}\text{M}_a^{-1}\Gamma^{-1}\\
\text{R}^L_b&=-\sqrt{1-R_\text{PD}}\Gamma 2\sqrt{\gamma^b_R \gamma_b}\text{M}_b^{-1}\Gamma^{-1}\\
\text{R}^L_\text{PD}&=\sqrt{R_\text{PD}} \text{I}.
\end{align}
Where the noise transfer matrices $\text{R}_\text{in},\text{R}^L_b,\text{R}^L_\text{PD}$ are the same as Eq.~\ref{eq:dOPO_PD_as_fn_of_vac} except for the difference in $\text{M}_b$ which accounts for the arm cavity and the radiation-pressure noise. %, as expected.
The total quantum noise is given by %the spectral density matrix %, i.e.\ shot noise plus quantum radiation-pressure noise,
\begin{equation}
\text{S}_X=\text{R}_\text{in}\text{R}_\text{in}^\dag+\text{R}^L_a{\text{R}^L_a}^\dag+\text{R}^L_b{\text{R}^L_b}^\dag+\text{R}^L_\text{PD}{\text{R}^L_\text{PD}}^\dag.
\end{equation}
This matrix has a similar form to Eq.~\ref{eq:dOPO_full_freedom} for the degenerate OPO but with terms for the radiation pressure noise such that the variances and covariances no longer reduce to 1 and 0, respectively, when the squeezer is off. \jam{(Display the matrix in some shortened form? Maybe in an appendix?)}
The signal transfer function (matrix) is defined with respect to $\tilde h(\Omega)$ ($\vec h(\Omega)$) but since $h(t)$ is real $\tilde h(\Omega)=\tilde h(-\Omega)^*$ and $\vec h(\Omega)=\tilde h(\Omega) \begin{bsmallmatrix}1 \\1\end{bsmallmatrix}$.
% \begin{align}
% \text{T}\begin{bsmallmatrix}1 \\1\end{bsmallmatrix}=\frac{1}{\rho \chi \cos (\phi ) (\ldots)+\Omega ^4 (\ldots)}\begin{bsmallmatrix}4 \beta ^2 \gamma_R (1-R_\text{PD}) \chi ^2 \Omega ^4 \omega_s^2 \left(\gamma_a^2+\Omega ^2\right) \cos ^2(\phi ) \\2 \beta ^2 \gamma_R (1-R_\text{PD}) \Omega ^4 \omega_s^2 \left(\left(\gamma_a^2+\Omega ^2\right) \left(2 {\gamma^b_\text{tot}}^2+\chi ^2+2 \Omega ^2\right)-4 \chi  \sin (\phi ) \left({\gamma^b_\text{tot}} \left(\gamma_a^2+\Omega ^2\right)+\gamma_a \omega_s^2\right)-\chi ^2 \left(\gamma_a^2+\Omega ^2\right) \cos (2 \phi )+4 \omega_s^2 \left(\gamma_a {\gamma^b_\text{tot}}-\Omega ^2\right)+2 \omega_s^4\right)\end{bsmallmatrix}
% \end{align}
Inspecting $\text{T}\begin{bsmallmatrix}1 \\1\end{bsmallmatrix}$, i.e.\ the vector of signal transfer functions to each quadrature, shows that there are two terms: (1) rotates between the quadratures with the pump phase and (2) stays in the second quadrature and never vanishes with the pump phase \jam{(is it worth showing this?)}. I consider measuring the second quadrature at the photodetector since the signal is always there~\footnote{This does not mean that it is necessarily optimal to do so since the profile of the noise between the two quadratures is different to the signal, but it will suffice here~\cite{}. \jam{(What happens if I use $\phi=\pi$ and observe the first quadrature instead?)}}, and therefore the sensitivity ($\sqrt{S_h}$ is the noise-to-signal ratio) is
\begin{equation}
S_h = \frac{(\text{S}_X)_{2,2}}{\abs{(\text{T}\begin{bsmallmatrix}1 \\1\end{bsmallmatrix})_2}^2}.
\end{equation}
\end{comment}


\subsection{Results}
\label{sec:dIS_results}
% priorities: radiation pressure, gravitational-wave signal, stability
% threshold, radiation pressure, and pump phase

\begin{figure}
	\centering
	% \includegraphics[width=\textwidth]{}
	\caption{\jam{(Explain the noise and signal response here, then just use sensitivity later. Say which regime this is for.)} Degenerate internal squeezing's quantum noise (top panel) and gravitational-wave signal (middle panel) responses and sensitivity (bottom panel). Showing the sloshing frequency $\omega_s$, threshold, and the effect of the pump phase \jam{(Split into multiple plots?)}. \jam{(Which parameter set to use? Could show for Korobko and Li, matrix?)}}
	\label{fig:dIS_sensitivity}
\end{figure}
\begin{figure}
	\centering
	% \includegraphics[width=\textwidth]{}
	\caption{\jam{(Use realistic losses)} Degenerate internal squeezing breakdown of noise sources, showing the quantum noise response for each vacuum input, i.e. $\sqrt{(\text{R}_\text{in}\text{R}_\text{in})_{2,2}}$. The radiation-pressure noise, i.e.\ the $f^{-2}$ slope below $100$~Hz, and the squeezing around $\omega_s$ appear in all of the interferometer noise sources and not in the detection loss.}
	\label{fig:dIS_noise_budget}
\end{figure}

\jam{(''Focus on the science, not the equations'' - VA. Consider what order these results should come in.)}
\jam{(Explain the ``two regimes it can operate it I.e. broadening sensitivity (Korobko) or kHz sensitivity (Adya).'' -- VA)}

% comment on general performance, signal
% similarly to external squeezing, improving shot noise worsens radiation-pressure noise.
To better discuss the behaviour of degenerate internal squeezing, particularly its frequency dependence, I consider the results from Ref.~\cite{Korobko2019}~\footnote{But with an added factor of $\sqrt{2}$ to $G_0$ from that reference to match the gravitational-wave coupling constant convention for $\alpha_\text{GW}$ from Ref.~\ref{Li2020} that I use in Chapter~\ref{chp:nIS_analytics}.} for the noise and signal responses and the sensitivity as defined in Section~\ref{sec:optical_loss_background}.
The model of degenerate internal squeezing in Ref.~\cite{Korobko2019} takes the degenerate OPO from Section~\ref{sec:dOPO_model} and adds the arm cavity mode $\hat a$ at carrier frequency $\omega_0$ (which is resonant in the single-mode approximation) coupled to the signal-recycling mode $\hat b$ with coupling rate (called the ``sloshing'' frequency~\cite{}) $\omega_s$ and with an intra-cavity loss port with transmissivity $T_{l,a}$ into vacuum $\hat n^L_a$.
The results are shown in Fig.~\ref{fig:dIS_sensitivity} \jam{(for what parameters?)}.
With the internal squeezer turned off, $\chi=0$, the configuration becomes the interferometer in Section~\ref{sec:coupled_cavity_approximation} and the noise, signal, and sensitivity reduce to Fig.~\ref{fig:simplified_sensitivity}, where the signal transfer function peaks at $\omega_s$ with bandwidth $\gamma^b_R$~\cite{} \jam{(clarify two kinds of bandwidth)}. Turning the squeezer on (1) squeezes the shot noise and de-amplifies the signal around the sloshing frequency $\omega_s$, (2) does not affect the radiation-pressure noise below 100~Hz, and (3) improves the sensitivity around the sloshing frequency and not affect it at lower frequencies. % does not affect RP, do not make this mistake!
Increasing the squeezer parameter $\chi$ further increases the squeezing of the noise and signal but whether this continues to improve the sensitivity will be covered shortly. Compared to external squeezing, the internal squeezer squeezes each of the intra-cavity losses as well as the readout port vacuum, as shown in Fig.~\ref{fig:dIS_noise_budget}. 
Degenerate internal squeezing can be operated in two regimes depending on the choice of $\omega_s$ and $\gamma^b_R$ which determine the frequency range of the sensitivity improvement: (1) broadband sensitivity~\cite{Korobko2019} when $\gamma^b_R$ is large (e.g.\ $L_\text{SRC}$ is short) and the sensitivity is shallowly improved, for example, from around 100 to 10000~Hz \jam{(quantify)} or (2) kilohertz sensitivity~\cite{Adya2020} when $\gamma^b_R$ is small (e.g.\ $L_\text{SRC}$ is long) and the sensitivity is narrowly, but strongly, improved \jam{(quantify)} around $\omega_s$, e.g.\ 2~kHz.
% why improvement only around sloshing frequency
The squeezing of the shot noise and signal is localised to the sloshing frequency because of the resonance structure of the coupled cavity system which changes where the signal-recycling cavity is resonant. At the sloshing frequency, energy is strongly coupled from the arm cavity into the signal-recycling cavity and the signal-recycling cavity mode is resonant~\footnote{The phase acquired upon reflecting off the input test mass depends on whether the arm cavity is resonant and means that the signal-recycling cavity is resonant at frequencies where the arm cavity is not resonant. At the sloshing frequency, the arm cavity is not resonant, as seen in the falling signal response in Fig.~\ref{fig:dIS_sensitivity}.}~\cite{KorobkoThesis}. As the squeezer is only effective when the cavity is resonant~\cite{}, as in Fig.~\ref{sec:dOPO_variances} where the cavity is resonant at DC, the signal and noise are only squeezed around the sloshing frequency~\cite{}. \jam{(I do not understand this)} Away from the sloshing frequency, the cavity is not resonant and degenerate internal squeezing does not affect the sensitivity below kilohertz~\footnote{These results assume that the arm cavity loss is realistically small, e.g.\ $T_{l,a}=100\text{ppm}$, as discussed below.}.

% There is much behaviour to analyse for this configuration~\cite{Korobko2019,Adya2020,KorobkoThesis} that motivates how I analyse the configuration in my work and which are useful to compare to.
To later compare to the behaviour of the configuration in my work \jam{(do I compare all of this behaviour or just the loss tolerance?)}, I will briefly discuss (1) threshold in the lossless case, (2) the optimal squeezing value, (3) the stability of the system, and (4) the system's tolerance to optical loss.
% threshold, lossless threshold but leave lossy threshold to research chapters? ``To not conflate existing knowledge with my work, I leave the lossy threshold to Section~\ref{}.'' just state result and justify later?
Firstly, to find threshold, in the lossless case, $\gamma_a=\gamma_b=R_\text{PD}=0$, the shot noise for $\phi=\pi/2$ is as shown in Fig.~\ref{fig:dIS_sensitivity}~\cite{Korobko2019}.
% \begin{equation}
% \label{eq:dIS_lossless}
% (\text{S}_X)_{2,2}=1-\frac{4 \gamma^b_R \chi \Omega ^2}{\Omega ^2 (\gamma^b_R+\chi )^2+(\Omega ^2-\omega_s^2)^2}.
% \end{equation}
The squeezed noise goes to zero at $\chi_\text{thr}=\gamma^b_R$ and $\Omega=\omega_s$, which defines threshold~\cite{}. In the anti-squeezed quadrature, found by varying the pump phase $\phi\mapsto\phi+\pi/2$ \jam{(check)}, the noise instead diverges at $\Omega=\omega_s$ at threshold. This behaviour is similar to the lossless degenerate OPO in Fig.~\ref{fig:dOPO_variances}, except that the improvement occurs at the sloshing frequency instead of at DC. From the perspective of gains and losses inside the signal-recycling cavity, any light lost to the arms through $\omega_s$ must return since $\gamma_a=0$ and so the loss from the cavity mode remains as $\gamma^b_R$. In the lossy case, the situation is more complicated and the threshold is not quoted in the literature~\cite{}. To address this, I will define a unified notion of threshold in my work in Section~\ref{sec:singularity_threshold}. 

% \begin{figure}
% 	\centering
% 	% \includegraphics[width=\textwidth]{}
% 	\caption{\jam{(Make sure that x-axis is consistent with nIS plot. Is this plot necessary?)} Degenerate internal squeezing sensitivity versus quantum noise, varying the squeezer parameter up to threshold for squeezing and anti-squeezing and different detection losses \jam{(show for other losses as well?)}. The optimal value of the squeezer parameter is shown.}
% 	\label{fig:dIS_optimal_squeezing}
% \end{figure}

\jam{(Paragraph structure is off.)}
% pump phase, foreshadow that maximum of anti-squeezing is not necessarily the minimum of squeezing
% RP is not (anti)squeezed, look at the plots!
% A complication when the system is lossy is that maximising the anti-squeezed quadrature is not necessarily the same as minimising the squeezed quadrature, as shown in Fig.~\ref{fig:dIS_sensitivity}, the two move apart at high arm losses. This is a result of worsening the uncertainty product in the Heisenberg Uncertainty Principle. \jam{(But why does this happen?)}
% optimal squeezing curves against loss, in really high loss should antisqueeze
% difference to caves's amplifier
Secondly, although increasing the squeezer parameter continues to squeeze the quantum noise more, up to threshold, the optimal value for the sensitivity can be below threshold in the lossy case~\cite{Korobko talk}. This is because increasing the squeezing might decrease the signal more than the noise, e.g.\ once the noise is limited by detection loss which is not squeezed by the internal squeezer further squeezing will just decrease the signal. In the high loss regime, it can be that any amount of squeezing is detrimental, which means that it is instead optimal to anti-squeeze internally~\cite{Korobko talk}. Although this is interesting and motivates anti-squeezing, degenerate internal anti-squeezing is only optimal in high loss regimes that are unlike future detectors~\cite{}, unlike using a Caves's amplifier externally. 

% \subsubsection{Stability}
% establish how stability is analysed

\begin{figure}
	\centering
	% \includegraphics[width=\textwidth]{}
	\caption{Degenerate internal squeezing is stable below threshold. Showing imaginary part of the poles of the transfer functions against squeezer parameter, for the lossless and lossy cases, where a positive imaginary part indicates instability.}
	\label{fig:dIS_stability}
\end{figure}

\jam{(Do I need to talk about stability here?)}

Thirdly, configurations must be stable to be feasible. I determine the stability of this configuration by studying the poles of the transfer functions. 
The noise and signal transfer functions are rational functions with related denominators~\cite{Korobko2019}. Let the denominator of the signal transfer function (squared) be $q(\Omega,\chi)$, a polynomial in $\Omega,\chi$. Then, the denominator of the noise transfer function (squared) is $\Omega^4 q(\Omega,\chi) q(\Omega,-\chi)$~\cite{Korobko2019}. Since $\Omega=0$ comes from the horizontally free-falling mass assumption, the remaining zeros of the signal and noise denominators are the shared zeros of $q(\Omega,\chi)$. In the complex $\Omega$ plane, if any of these poles~\footnote{Where the numerator is checked to not also be zero at that point.} have a positive imaginary part, then the system is unstable \jam{(check this, IFT suggests other sign?)}~\cite{}~\footnote{This result is often expressed instead with respect to the Laplace Transform variable $s=i\Omega$, where a positive real part of $s$ indicates instability~\cite{}.}. %This condition is equivalent to others such as \jam{(... the Nyquist criterion?)}.
Therefore, degenerate internal squeezing is stable in the lossless case below threshold~\cite{}, as shown in Fig.~\ref{fig:dIS_stability}. In the lossy case, the system is also stable below threshold which will be shown in Section~\ref{sec:singularity_threshold}.


\subsection{Limitation: tolerance to optical loss}
% Problems with proposal -
\label{sec:dIS_optical_loss}

\begin{figure}
	\centering
	% \includegraphics[width=\textwidth]{}
	\caption{\jam{(Just show sensitivity but explain using signal and noise. Use realistic losses)} Degenerate internal squeezing sensitivity, showing the effect of the different losses. \jam{(Don't need the depth of the results chapters. Combined with: )} Degenerate internal squeezing sensitivity for realistic losses, using the same parameters as Fig.~\ref{fig:dIS_sensitivity} and a squeezer parameter close to threshold. \jam{(Show that detection loss dominates.)}}
	\label{fig:dIS_loss_tolerance}
\end{figure}

\jam{(``Focus on the science not the equations. Once you explain the behavior, explain the effect of losses on this system.'' -- VA. The tolerance to losses is similar for the two regimes of broadband and kilohertz improvement.)}

% intracavity loss behaves differently to dOPO 
Finally, degenerate internal squeezing has different tolerances to the different sources of optical loss, as shown in Fig.~\ref{fig:dIS_loss_tolerance}, but the tolerance is similar for the two regimes of broadband and kilohertz improvement. %, where now the signal's tolerance must be considered.
Firstly, detection losses affect the noise similarly to the degenerate OPO, the noise is uniformly pulled towards the vacuum. But now the effect on the signal also matters, which is pulled towards zero. Secondly, signal-recycling intra-cavity loss behaves differently to the degenerate OPO, as the noise response remains within the lossless envelope, increases radiation-pressure noise, and worsens the signal and noise around the sloshing frequency \jam{(why doesn't the response broaden like before?)}. Finally, arm intra-cavity loss, which has no analogue in the OPO, decreases the peak of the noise and moves the peak frequency away from the sloshing frequency, but improves the radiation-pressure noise \jam{(check this)}. The signal response remains within the lossless signal envelope but worsens the DC response to the signal.

% \begin{figure}
% 	\centering
% 	% \includegraphics[width=\textwidth]{}
% 	\caption{\jam{(Remove this plot)} Degenerate internal squeezing shot noise response in the limit of large arm loss compared to the theoretical limiting degenerate OPO with fully reflective input test mass.}
% 	\label{fig:dIS_limit_dOPO}
% \end{figure}

% \subsubsection{Reduction to degenerate OPO}
% arm cavity loss gives reduction to dOPO
% The behaviour against different parameters will be similar to that seen for the degenerate OPO in Section~\ref{}.
This behaviour against arm intra-cavity loss, in the high arm loss limit, can be understood as the degenerate OPO limit of degenerate internal squeezing, i.e.\ when the arm cavity is removed. Formally taking the limit $\gamma_a\rightarrow\infty$ of the equations-of-motion or the shot noise response in Ref.~\cite{Korobko2019} shows that the system reduces to a degenerate OPO between the signal-recycling mirror and a fully reflective input test mass, which is why the sensitivity away from the sloshing frequency is affected when the arm loss is high. \jam{(demonstrate this result or cite it, also, check if equivalent to sending sloshing frequency to zero)} Although I initially expected the input test mass to instead become another loss port with its original transmissivity, this can be explained as the disappearance of the arm cavity mode altogether, vacuum or otherwise \jam{(check this)}. For $\gamma_a\rightarrow\infty$, the equation-of-motion for $\hat a$~\cite{Korobko2019} becomes $\dot{\hat a}\approx -\gamma_a \hat a$ which quickly decays, and therefore any vacuum $\hat n^L_a$ cannot couple to $\hat b$ because $\hat a$ has vanished.
However, I believe that this is a consequence of the single-mode approximation and that if a ``transfer matrix'' approach~\cite{Finesse,}~\footnote{Not to be confused with the transfer matrices describing the signal and noise responses.} was instead used where the fields at a point are propagated inside the cavities and the cavity modes are not explicit, the limit would be a degenerate OPO with added intra-cavity loss to account for the open input test mass port. This should be easy enough to test but I leave it to future work \jam{(check?)}.
Since realistic arm losses for future detectors are below this high loss regime \jam{(quantify)}, this behaviour is not of concern for applications. 
% This is confusing, I expected the initial test mass to become a loss port, but this can be understood as there no longer being an arm cavity mode, vacuum or otherwise, since it can not make a round-trip.

% \begin{figure}
% 	\centering
% 	% \includegraphics[width=\textwidth]{}
% 	\caption{\jam{(Combine with Fig.~\ref{fig:dIS_loss_tolerance})} Degenerate internal squeezing sensitivity for realistic losses, using the same parameters as Fig.~\ref{fig:dIS_sensitivity} and a squeezer parameter close to threshold. \jam{(Show that detection loss dominates.)}}
% 	\label{fig:dIS_realistic_loss}
% \end{figure}

Compared to the sensitivity improvement in the lossless case, if realistic losses are assumed, then the improvement significantly degrades \jam{(quantify this)}, as shown in Fig.~\ref{fig:dIS_loss_tolerance}. What realistic losses are for future detectors is hard to determine given the unknown progress of future technology, but the literature suggests that, conservatively, arm intra-cavity loss below $T_{l,a}=100$~ppm (parts-per-million), signal-recycling intra-cavity loss below $T_{l,b}=1000$~ppm, and detection loss below $R_\text{PD}=10\%$ are reasonable~\cite{Zhang2021,}. For these realistic losses, detection loss is responsible for most of the degradation seen in Fig.~\ref{fig:dIS_loss_tolerance} since it dominates the noise apart from the readout port vacuum, as seen in Fig.~\ref{fig:dIS_noise_budget} \jam{(check)}. 
% \subsection{Connection to nondegenerate internal squeezing}
% at high enough losses, it becomes optimal to instead anti-squeeze internally, this means that you might as well use nondegenerate internal squeezing because then you can anti-squeeze and potentially exploit the correlations using a combined readout
% conclusions about dIS?
Although degenerate internal squeezing improves the sensitivity, its low \jam{(quantify)} tolerance to optical loss motivates investigating other methods which might improve sensitivity more given the same losses. 
%, such as internal anti-squeezing using nondegenerate internal squeezing~\footnote{Which should anti-squeeze by comparison to the nondegenerate OPO.}. \jam{(What is the optimal squeezing value given realistic losses? I need to rule out degenerate anti-squeezing.)} %This does not mean that degenerate internal squeezing is not useful, it is worth further investigation, especially in low loss applications~\cite{}, but I will consider the nondegenerate case and whether it fares better.


	% Nondegenerate internal squeezing, where the internal squeezer is instead nondegenerate, has
	% been proposed as an alternative to degenerate internal squeezing~\cite{yapadyaPersonalCommunication}, although a comprehensive analysis of nondegenerate internal squeezing is yet to be done~\cite{liBroadbandSensitivityImprovement2020}. Because nondegenerate squeezing results in two entangled photons with different frequencies, these photons will not interfere with each other in the same manner as the degenerate case. Without this interference, nondegenerate internal squeezing increases the signal and the noise instead of decreasing them like the degenerate case. Therefore, nondegenerate internal squeezing is predicted to be more resistant to photodetector loss since the signal amplitude is greater. This project aims to investigate the potential benefits of nondegenerate internal squeezing over degenerate internal squeezing.



%%%%%%%%%%%%%%%%%%%%%%%%%%%%%%%%%%%%%%%%%%
\section{Existing proposal 2: stable optomechanical filtering}
\label{sec:sWLC}
% the point of this section is to explain the existing design, what is wrong with it, and how it connects to nIS

\begin{figure}
	\centering
	\includegraphics[angle=-90,width=0.9\textwidth]{sWLC_config.pdf}
	\caption{Stable optomechanical filtering configuration~\cite{Li2021} with mechanical idler mode $\hat{c}_m$, e.g.\ of a suspended mirror, with mechanical resonance frequency $\omega_m$ coupled to the signal-recycling cavity mode $\hat b$ via radiation pressure. $\hat b$ and $\hat{c}_m$ are nondegenerately, optomechanically squeezed \jam{(check if explained in text)} by a blue-detuned pump at $\omega_0+\omega_m$.}
	\label{fig:sWLC_config}
\end{figure}

\jam{(``No need to go into details about White light cavities – introduce it, mention the challenges for it, cite references.'' -- VA)}

\jam{(Read Li2021 to check if this is still current. Also refer to research journals. Where is blue-detuned pump laser?)}

% explain the design
Stable optomechanical filtering uses the signal-recycling cavity of an interferometer as an optomechanical filter cavity by coupling the signal mode to a mechanical mode, e.g.\ an intermediate~\footnote{The exact position of the mechanical oscillator does not matter in this simplified model, compare Refs.~\cite{Li2020,Li2021} which have different mirror positions but the same Hamiltonian.}, suspended mirror~\cite{}, as shown in Fig.~\ref{fig:sWLC_config}. 
% is it necessary to talk about unstable design? --> not the focus, just give it a mention, Kramers-Kronig relations?
This configuration is dynamically stable~\cite{} and improved upon an unstable design~\cite{} by changing which mode was read out~\footnote{The unstable configuration read out the arm cavity mode directly instead of the signal-recycling cavity mode. The addition of a control system was required to stabilise the naturally unstable system~\cite{}.}. The stable design is more relevant to this thesis because its mode structure is more closely related to my work. %, which I will elaborate on shortly.
The term filter cavity refers to selecting the parameters of a cavity to (de)amplify, i.e.\ filter, certain frequencies at the output~\cite{}. Although the signal-recycling cavity already performs this role, e.g.\ changing the length of the cavity changes the signal transfer function's peak and bandwidth~\footnote{Given the $\omega_s$ and $\gamma^b_R$ formulae in Section~\ref{sec:dIS_model} and their role as peak frequency and bandwidth, respectively.}, the optomechanical coupling changes the resonance behaviour and can be used to more selectively filter~\cite{}. %I use the term in this context to be consistent with the literature~\cite{}. % waste of time explaining this?
% how does this configuration beat the Mizuno limit (four factors in the introduction)? --> cancels arm cavity resonance
% the lossless system does not affect the noise, therefore is not fundamentally squeezing, but the lossy system will antisqueeze % idler loss complicates this later
In particular, the optomechanical filter cavity can be designed to partially counter the arm cavity's resonance which decreases the signal transfer function at kilohertz~\cite{}. %, where the parameter of interest is the optomechanical coupling compared to the optical coupling between the arm and signal-recycling cavities at the input test mass. % Again, without increasing the circulating power in the arms.
This is known as the white-light cavity idea: broaden the resonance by changing the filter cavity's phase response at each frequency to be opposite to that of the arm cavity~\cite{}. \jam{(Why does this avoid the Mizuno limit?)} However, the literature has only considered the case without optical loss~\cite{,} and, in the following chapters, my work will suggest that the behaviour of the system is complicated when optical loss is introduced. %~\footnote{To preview the results, this is because the quantum noise becomes anti-squeezed and therefore the Mizuno limit is beaten by squeezing instead of/along with the countering-the-arm-resonance explanation above.}. \jam{(Should not talk about results here?)}
But, for this section, the configuration should be considered as just changing the signal transfer function through the resonance structure of the interferometer.

% explain the results of the design
% I omit the model of this configuration here, because in the lossless case it is exactly the model in the next chapter, see Section~\ref{}.
I emphasise two aspects of this configuration: (1) its dependence on the optomechanical and optical coupling rates and (2) its vulnerability to mechanical loss. %, and (3) its connection to nondegenerate internal squeezing. Because of this last point,
I omit discussing this configuration further because without optical and mechanical loss it is equivalent to the lossless model in the next chapter, which I will discuss later. 

% \begin{figure}
% 	\centering
% 	% \includegraphics[width=\textwidth]{}
% 	\caption{\jam{(Remove this plot since the results are compared to later. Need to show on-threshold behaviour. Is this figure redundant with the later comparison?)} Stable optomechanical filtering's sensitivity, showing the effect of mechanical loss, the dominant noise source. Optical loss is not included in this model. This figure was generated using the code from Ref.~\cite{} with permission from the authors~\cite{LiPersonalCommunication} \jam{(check this)}, the parameters used are \jam{(... fill this in)} and $T_\text{env}=4$~K and $Q_m=8\times10^9$.}
% 	\label{fig:sWLC_sensitivity}
% \end{figure}

% but I need to talk about their results? in particular, the exceptional point of PT-symmetric, stability, and sensitivity at threshold. maybe do show a lossless, shot-noise only model -- better yet, cite the results in their paper and reference the next section for further explanation?
% exceptional point, stability and PT symmetry
Firstly, in the lossless case, comparing the coupling rates of the three modes shows that when the two coupling rates are equal, the system is special \jam{(colloquial?)}. Let the interaction Hamiltonian of the system be~\cite{} $$\hat{H}_\text{I}=i\hbar\omega_s(\hat{a}\hat{b}^\dag-\hat{a}^\dag\hat{b})+i\hbar\chi_m(\hat{b}^\dag\hat{c}_m^\dag-\hat{b}\hat{c}_m).$$ Where $\hat a, \hat b, \omega_s$ are as in Section~\ref{sec:dIS_model}, $\hat{c}_m$ annihilates the mechanical mode \jam{(explain phonons?)}, and $\chi_m$ is the optomechanical coupling rate. When $\omega_s=\chi_m$, the interaction Hamiltonian becomes invariant under the transformation $\hat a\mapsto\hat{c}_m^\dag, \hat{c}_m\mapsto\hat a^\dag$ which corresponds to the composition of parity, $\hat a\leftrightarrow \hat{c}_m$, and time, $\hat Q\leftrightarrow \hat Q^\dag$, transformations, and that leaves $\hat b$ invariant. This parity-time (PT) symmetry causes other changes in the system~\footnote{Although, the most notable feature of PT-symmetric systems is that quantum mechanics can \jam{(be reformulated to)} also allow non-Hermitian, PT-symmetric Hamiltonians~\cite{}. And since this system is still Hermitian, the importance of PT-symmetry is lessened. \jam{(What did Carl Bender say? --> Hermitiancy is more complicated and requires further examination, see Section~\ref{sec:future_work}.)}}. \jam{(does it?)}
The lossless, PT-symmetric system is borderline stable, with one complex $\Omega$ pole on the real axis, and is at an Exceptional Point of its independent eigenmodes~\cite{Li2020} \jam{(do not just quote Li, I need to revise this)}. Moreover, the shot noise--limited, integrated sensitivity becomes unbounded~\cite{}. With radiation pressure~\footnote{There is a complication with radiation pressure coupling the arm cavity mode to the test mass mechanical mode, as for PT-symmetry to be maintained the filter cavity mechanical mode must be then coupled to a back-action evasion (BAE) mode with negative effective mass~\cite{}. Although this is interesting, I will not consider it in this thesis due to time constraints.}, the integrated sensitivity becomes bounded~\cite{} and although the kilohertz sensitivity improves \jam{(check plot)}, the main improvement is from 100-1000~Hz~\cite{}, as shown for the lossless case \jam{in Fig.~\ref{fig:sWLC_sensitivity} (talk about results without figure)}. %This foreshadows an aspect of my work, that I will discuss later, about broadband versus kilohertz sensitivity improvement. 

	% A recent proposal uses a stable optomechanical filter cavity to avoid this limit and increase high-frequency sensitivity without fully sacrificing low-frequency sensitivity nor increasing the power~\cite{liBroadbandSensitivityImprovement2020}. 
	% Stable optomechanical filtering consists of an auxiliary filter cavity inside the signal-recycling cavity. One of the filter cavity's mirrors is a mechanical oscillator, such as a suspended mirror, driven by a laser whose frequency is appropriate to excite the mechanical mode~\cite{liBroadbandSensitivityImprovement2020}. This design is dynamically stable unlike previous designs for optomechanical filter cavities~\cite{miaoEnhancingBandwidthGravitationalWave2015,pageEnhancedDetectionHigh2018,miaoDesignGravitationalWaveDetectors2018}. This system has a parity-time symmetry between the differential optical mode of the interferometer and the mechanical mode; 


\subsection{Limitation: tolerance to mechanical loss}
% Problems with proposal - 
\label{sec:sWLC_loss}
% thermal noise and mechanical quality factor

Secondly, although stable optomechanical filtering could improve the sensitivity of future detectors, its vulnerability to mechanical loss demands progress beyond current technology~\cite{}. Mechanical loss is damping of the filter cavity mechanical oscillator due to the dissipation of energy into the thermal bath of the mass and its surroundings. This raises the temperature of the mass, increases the thermal noise in its position, becomes radiation pressure noise in the filter cavity mode, and therefore degrades the sensitivity. The thermal noise from mechanical loss dominates the noise of stable optomechanical filtering~\cite{}. The results shown \jam{in Fig.~\ref{fig:sWLC_sensitivity} (talk about results without figure)} assume improvements in the possible environmental temperature and quality factor of the mechanical oscillator. In particular, the ratio of the environmental temperature $T_\text{env}$ to quality factor $Q_m$, must be small~\cite{Miao2015},
\begin{equation}\label{eq:sWLC_mechanical_loss}
\frac{T_\text{env}}{Q_m}\leq\frac{\hbar \gamma_\text{single-cavity}}{8 k_B}.
\end{equation} 
\jam{(Does the same hold for the stable configuration?)}
Where $\gamma_\text{single-cavity}$ is the bandwidth of the Fabry-Perot Michelson interferometer, i.e.\ without the signal-recycling cavity, given by \jam{...~\cite{} (fill this in)}, and $k_B$ is the Boltzmann constant. For example, this is satisfied for the LIGO~Voyager parameter set \jam{(check this, check that the parameter set is consistently named as LIGO Voyager in the thesis)} for $T_\text{env}=4$~K and $Q_m=8\times10^9$~\cite{Li2020}, used \jam{in Fig.~\ref{fig:sWLC_sensitivity} (talk about results without figure)}. But this quality factor is beyond that possible with current technology~\cite{Page2018,,}. If the mechanical loss is higher the bound in Eq.~\ref{eq:sWLC_mechanical_loss}, then the sensitivity is significantly degraded \jam{(quantify ``significantly'')}. Therefore, for stable optomechanical filtering to be feasible, the mechanical loss must improve \jam{(by how much?)}, towards which there is research in the literature~\cite{optical spring,}.

% optical loss --> bridge into next subsection?

	% However, it requires cryogenic (around 4~K) environmental temperature and a higher mechanical quality factor than is currently possible. An all-optical alternative to this optomechanical proposal without these requirements is desirable.
	% It is designed for implementation in later-generation detectors and assumes technological improvements in the near future in arm length, power, optical loss, Brownian noise, and, most stringently, in the thermal noise and quality factor of the mechanical oscillator. These assumptions are necessary to achieve the target sensitivity for astrophysical applications~\cite{miaoDesignGravitationalWaveDetectors2018}. By investigating nondegenerate internal squeezing, I aim to find more realistic requirements for a future detector.


\begin{comment}
\subsection{Connection to nondegenerate internal squeezing}
\label{sec:modal_equivalence}

\begin{figure}
	\centering
	% \includegraphics[width=\textwidth]{}
	\caption{\jam{(Add idler readout)} Mode diagrams of the OPOs, degenerate internal squeezing, stable optomechanical filtering, and nondegenerate internal squeezing. The latter two configurations are modally the same but are optomechanical and all-optical, respectively, which means that their performance might be different given the different losses they encounter. The idler readout scheme and the unstable case of optomechanical filtering are also shown. The parallels between the degenerate OPO and degenerate internal squeezing, and the nondegenerate OPO and nondegenerate internal squeezing, are also shown.}
	\label{fig:mode_diagram}
\end{figure}

\jam{(Break out this section into motivation for nIS either here or at the start of chapter 4 (which might be cleaner).)}

% However, 
% all-optical alternative
Finally, there is an alternative route to progress, to replace the optomechanical interaction with an all-optical one and replace the mechanical loss with optical loss. This is to consider nondegenerate internal squeezing, where the internal squeezer squeezes signal and idler modes and the idler is not resonant in the arms~\footnote{So that the idler mode is not coupled to the arm cavity mode.}. 
% although nIS is closer on first inspection to dIS, the underlying structure is closer to sWLC, but both motivate it.
% equivalent mode structures, just the different noise sources
Although it might seem closer to degenerate internal squeezing than stable optomechanical filtering, the underlying mode structure of nondegenerate internal squeezing is equivalent to the latter, by mapping the idler optical mode and squeezer parameter $\hat c, \chi$ to the mechanical mode and optomechanical coupling $\hat{c}_m, \chi_m$, respectively, in the Hamiltonian~\cite{}, as shown in Fig.~\ref{fig:mode_diagram}. Although this is only true in the case with no optical or mechanical loss, as thermal noise in $\hat{c}_m$ behaves differently to shot noise in $\hat c$~\cite{} \jam{(it does experimentally and the parameter regimes are very different in application, but the Langevin terms are the same)}, it is reasonable to predict that the lossy configurations behave similarly to each other because the abstract dynamics are the same. Therefore, nondegenerate internal squeezing might achieve the sensitivity improvement of stable optomechanical filtering but at more realistic optical loss than the mechanical loss required above.

\jam{(different experimental constraints on optical loss versus mechanical loss)}
\end{comment}

% the lossless system does not affect the noise, therefore is not fundamentally squeezing, but the lossy system will antisqueeze % idler loss complicates this later --> cover this in results if it matters

	% replicating this [PT-]symmetry with an internal squeezer requires the squeezer to be nondegenerate to mimic the distinction between the filter cavity optical mode and the mechanical mode~\cite{liBroadbandSensitivityImprovement2020}. Stable optomechanical filtering improves high-frequency sensitivity by cancelling the effect of the resonance behaviour of the interferometer cavities.
	% Nondegenerate internal squeezing is also motivated by a connection between it and the use of
	% a stable optomechanical filter cavity to improve high-frequency sensitivity~\cite{yapadyaPersonalCommunication,liBroadbandSensitivityImprovement2020}. The Hamiltonians of the two systems are equivalent under some mapping of the creation and annihilation operators of certain optical fields to certain mechanical fields. This connection exploits the fact that the nondegenerate squeezer interacts with three distinct frequencies to introduce a symmetry into the all-optical system that the optomechanical system has. This means that using nondegenerate internal squeezing may achieve the benefits of a stable optomechanical filter cavity without the optomechanical drawbacks of requiring cryogenic (around 4~K~\cite{miaoEnhancingBandwidthGravitationalWave2015}) environmental temperature and high mechanical quality factor. Therefore, understanding stable optomechanical filtering should lead to a better understanding of what nondegenerate internal squeezing might achieve.

%%%%%%%%%%%%%%%%%%%%%%%%%%%%%%%%%%%%%%%%%%
\section{Chapter summary}

% there are solutions in the literature to improving kilohertz sensitivity, but they have their problems, and motivate considering a combined configuration

\jam{(Make this more substantial)}
In this chapter, I have reviewed the literature of proposed configurations that might improve the kilohertz sensitivity of future gravitational-wave detectors and that motivate my work. Firstly, I reviewed the literature of possible configurations to improve kilohertz sensitivity, from which I focussed on two configurations: degenerate internal squeezing and stable optomechanical filtering. Secondly, I examined degenerate internal squeezing and discussed its vulnerability to optical loss. % and how, in the high loss regime, it becomes beneficial to anti-squeeze instead of squeeze internally. %, which suggests that nondegenerate internal squeezing might be more resilient to optical loss than degenerate internal squeezing. 
Finally, I examined stable optomechanical filtering. % which deviates from all other configurations in this thesis by use of a mechanical oscillator instead of an optical squeezer, and avoids the Mizuno limits through partially cancelling the arm cavity resonance.
I explained how PT-symmetry in the lossless limit leads to an Exceptional Point and enhanced sensitivity \jam{(check this)}, but that this sensitivity improvement requires low mechanical loss.
% However, this optomechanical configuration is equivalent to the all-optical nondegenerate internal squeezing, and this equivalence might mean that nondegenerate internal squeezing can achieve the same sensitivity improvement but be more feasible due to the technological differences between optical and mechanical loss.
The low tolerance of these two proposals to realistic losses limits their feasibility for future detectors unless further technological progress is made to lower their respective losses. This motivates investigating configurations that are more resistant to loss.


