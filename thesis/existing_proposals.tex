\chapter{Existing proposals for improving kilohertz sensitivity} % Background: existing proposals for improving kilohertz sensitivity
\label{chp:proposals}
%%%%%%%%%%%%%%%%%%%%%%%%%%%%%%%%%%%%%%%%%%
% chapter introduction
% Beyond better external squeezing, there are many \jam{(are there?)} existing proposals to improve the kilohertz sensitivity of gravitational-wave detectors. In the next two chapters, I examine two of the front-runners: degenerate internal squeezing and stable optomechanical filtering.
% demonstrate modelling again, for the third and last time before my model -- much of the maths should be quite familiar by now
% set-up the problem: how do these configs get past the four factors in the introduction and why aren't they good enough for kilohertz sensitivity

In this chapter, I consider some of the existing configurations that address the problem of increasing kilohertz sensitivity and that motivate the configuration that I examine. I focus on and critically examine two promising proposals: degenerate internal squeezing in Section~\ref{sec:dIS} and stable optomechanical filtering in Section~\ref{sec:sWLC}.
% I will use the tools of the previous chapter to understand degenerate internal squeezing and demonstrate how I will approach my work. Then, I will discuss stable optomechanical filtering using the same ideas. %, although I do not present a separate model of it for reasons that will become clear. 
%Stable optomechanical filtering is the optomechanical analogue of my all-optical configuration which implies that there is . 
I also present the limitations of these two proposals that motivate my work in the following chapters into a configuration that combines their strengths but might be able to overcome their limitations and better improve kilohertz sensitivity. \jam{(is this spoiler-free now?)}
% I will argue that, while promising, these existing proposals have limited tolerance to optical and mechanical loss, respectively. This will motivate my work in the following chapters into a configuration that combines these two proposals but might be able to overcome their limitations and better improve kilohertz sensitivity. 


%%%%%%%%%%%%%%%%%%%%%%%%%%%%%%%%%%%%%%%%%%
\section{Degenerate internal squeezing}
\label{sec:dIS}

\begin{figure}[ht]
	\centering
	\includegraphics[angle=-90,width=0.9\textwidth]{dIS_config.pdf}
% squeezing ellipse and signal arrow plot (+ show the effect of optical loss: detection and intracavity)
	\caption{\jam{(Shorten caption. Do I need to show arm loss in both arms? Do I need to show the GW?)} Degenerate internal squeezing configuration (left panel) with the single-mode approximation, and the squeezer's effect on the measured signal and noise (right panel) using the noise ellipse and signal arrow representation~\cite{} where the height of the arrow represents the signal response.
	The interferometer mirrors are labelled as the ETM (end test mass), ITM (input test mass), and SRM (signal-recycling mirror). The signal-recycling cavity resembles the degenerate OPO in Fig.~\ref{fig:OPOs_config}. The signal enters by moving the end test masses and the noise enters from the readout port and the losses. The sensitivity is improved by squeezing the noise more than the signal. Detection loss $R_\text{PD}$ is included in the model via the beamsplitter convention in Fig.~\ref{fig:beamsplitter_loss}.}
	\label{fig:dIS_config}
\end{figure}

% \jam{(Explain why it works.)}
% has been explained many times before, but do it again here because it is the main reference
% how does this configuration beat the Mizuno limit (four factors in the introduction)? --> squeezing
Degenerate internal squeezing consists of a degenerate squeezer placed inside the signal-recycling cavity of an interferometer such that it squeezes the signal mode~\cite{} as shown in Fig.~\ref{fig:dIS_config}. In this configuration, the vacuum entering the readout port is squeezed, like with external squeezing, but the vacuum from the intra-cavity losses and the gravitational-wave signal are also squeezed, unlike external squeezing.
%Although a single-pass of the squeezer squeezes the signal and noise equally~\cite{},
The overall sensitivity, i.e.\ the signal-to-noise ratio \jam{(explain later that I plot NSR instead?)}, is improved because the signal and noise behave differently, i.e.\ the signal comes from the test mass in the arms and the noise comes predominantly from the vacuum entering the readout port. This means that the signal and noise ``see'' the signal-recycling cavity and the squeezer differently as shown in Fig.~\ref{fig:dIS_config}~\cite{}. %, which changes the average number of single-passes of the squeezer each experiences~\cite{}. 
Therefore, degenerate internal squeezing improves the sensitivity by squeezing the noise more than the signal~\cite{} as shown in Fig.~\ref{fig:dIS_config}.
%The overall effect is to de-amplify the signal but squeeze the noise more such that the sensitivity improves, as shown in Fig.~\ref{fig:dIS_ellipse_and_arrow}. Where the improvement overcomes the Mizuno Limit from Section~\ref{} through the use of squeezing. %However, this improvement only occurs around the ``sloshing'' frequency~\cite{}, the coupling frequency of the two cavities, which will be shown. 


% In this section, I present a model of degenerate internal squeezing and discuss its behaviour. These results exist in the literature~\cite{korobkoQuantumExpanderGravitationalwave2019}, but I demonstrate them, abridged, here to justify the approach I use in my work and establish what is expected from internal squeezing. In particular, I demonstrate how to consider radiation pressure, the gravitational-wave signal, and stability, which I have not yet explained.
%bridge from the OPOs in the previous chapter to my work on nondegenerate internal squeezing in the next chapter.
%The structure of the results presented here will be mirrored in the next chapter which should make those later results more convincing.  % what do I intend the reader to get out of it?
% \jam{(I do not just show this model/results because I spent time with them. Is it clear what I want the audience to get out of them?)}

	% A proposed technology to further reduce shot noise is degenerate internal squeezing where a degenerate squeezer is included inside the signal-recycling cavity as shown in the left panel of Fig.~\ref{fig:coupled_cavities}~\cite{korobkoQuantumExpanderGravitationalwave2019,adyaQuantumEnhancedKHz2020}.
	% To simplify modelling this configuration, the two coupled optical modes of concern, the differential mode from the arm cavities and the mode in the signal-recycling cavity, are approximated as coming from a pair of coupled cavities as shown in the right panel of Fig.~\ref{fig:coupled_cavities}. The quantum noise enters with the vacuum into the signal-recycling cavity, but the signal arises in the arm cavity and so sees the squeezer differently. This difference leads to no change in sensitivity at low frequencies and improved sensitivity at high frequencies around the resonance of the signal-recycling cavity as shown in Fig.~\ref{fig:sensitivity_curve}. The improvement occurs around the signal-recycling cavity resonance as this is where the most quantum noise field is present inside the cavity and so is where the internal squeezer produces the most squeezing. The resulting noise reduction is sufficient to overcome the de-amplification of the signal due to squeezing. The overall sensitivity improvement from degenerate internal squeezing is limited by optical loss in the interferometer and at the photodetector~\cite{korobkoQuantumExpanderGravitationalwave2019}. 

% analytic model
\begin{comment}
	\subsection{Analytic model}
	\label{sec:dIS_model}

	\jam{(Is showing this model necessary?)}

	% I have verified this model against Korobko
	I construct a Hamiltonian model of degenerate internal squeezing using the formalism from Chapter~\ref{chp:background_theory}. This model is based on and verified against Ref.~\cite{korobkoQuantumExpanderGravitationalwave2019}. %I include it here to demonstrate the Hamiltonian method that I later use in my work, particularly the radiation pressure, gravitational-wave signal, and the system's stability \jam{(repetition with section intro)}. 
	Since degenerate internal squeezing is a degenerate OPO coupled to another cavity, as shown in Fig.~\ref{fig:dIS_config}, I use the degenerate OPO model in Section~\ref{sec:dOPO_model} and add the extra modes associated with the arm cavity. Let the signal-recycling cavity and output modes be labelled as in Section~\ref{sec:dOPO_model} and let the arm cavity mode at carrier frequency $\omega_0$ be $\hat a$ (which is resonant in the single-mode approximation) with an intra-cavity loss port with transmissivity $T_{l,a}$ into vacuum $\hat n^L_a$. Let the gravitational-wave signal $h(t)$ from Section~\ref{sec:gravWaves} be coupled to the arm cavity mode by the test mass mechanical mode given by displacement $\hat x$ and momentum $\hat p$ (approximated as free-falling horizontally as explained in Section~\ref{sec:qnoise_GW_IFO}). 
	The Hamiltonian of this system is $\hat H = \hat H_0 + \hat H_I + \hat H_\gamma + \hat H_\text{GW}$ where~\cite{}
	\jam{(fill in Langevin Hamiltonian)}
	\begin{align}
	\hat H_0 &= \hbar \omega_0 \hat a^\dag \hat a + \hbar \omega_0 \hat b^\dag \hat b + \hbar 2\omega_0 \hat u^\dag \hat u\\
	\hat H_I &= i\hbar\omega_s(\hat a\hat b^\dag-\hat a^\dag\hat b) +\hbar \frac{x}{2} (e^{i\phi} \hat u (\hat b^\dag)^2 + e^{-i\phi} \hat u^\dag \hat b^2)\\
	\hat H_\gamma &= \int \ldots \\
	\hat H_\text{GW} &= -\alpha (\hat{x}-L_\mathrm{arm}h(t))\left(\frac{\hat{a}+\hat{a}^\dag}{\sqrt{2}}\right)+\frac{1}{2\mu}\hat{p}^2.
	\end{align}
	% \beta = \frac{\alpha L_\mathrm{arm}}{\sqrt{2}\hbar}, \beta = \sqrt{\frac{2 P_\text{circ} L_\text{arm} \omega_0}{\hbar c}}, \alpha = \sqrt{2} \hbar/L_arm= \sqrt{\frac{4 P_\text{circ} \omega_0 \hbar}{c  L_\text{arm}}}
	% multiple different hamiltonians for RP
	% problem with omega_s formula, only valid below one FSR of the arms
	Where $\alpha=\sqrt{\frac{2 P_\text{circ} \omega_0 \hbar}{c  L_\text{arm}}}$ \jam{(clarify that this is the Li formula which Korobko disagrees with by a factor of rt2 for reasons unknown. Why do they disagree? I use the Li value for comparison to sWLC but this affects the feasibility of nIS! Ask the supervisors.)} is the coupling strength to the gravitational-wave signal~\cite{}, $\mu=M/4$ is the reduced mass \jam{(Explain reduced mass.)} of the test mass with mass $M$~\cite{}, and $\omega_s\approx c\sqrt{\frac{T_\text{ITM}}{4 L_\text{arm} L_\text{SRM}}}$ is an approximation to the sloshing frequency between the coupled cavities (also known as the coupled cavity pole) which holds when it is below one FSR of the arm cavities \jam{(check this)}~\cite{}. I use $\hat H_\text{GW}$ from Ref.~\cite{liBroadbandSensitivityImprovement2020,original source?} which couples the mirror position and gravitational wave length displacement $L_\text{arm} h(t)$ to the amplitude quadrature of the cavity mode~\footnote{Although, perhaps, a more natural formulation would couple the gravitational-wave strain to the mirror position and the mirror position to the cavity mode, this is equivalent~\cite{}.}. 
	The Heisenberg-Langevin equations-of-motion for $\hat a, \hat b, \hat x$ and $\hat p$ can be found using the bosonic commutation relations, the canonical commutation relation $[\hat x,\hat p]=i\hbar$~\cite{}, and with all other commutators zero. As in Section~\ref{sec:dOPO_model}, I make semi-classical and no-pump-depletion approximations to the pump mode, enter the Interaction Picture, and take fluctuating components of the operators. The resulting equations are
	\begin{equation}\label{eq:dIS_EoM}\begin{cases}
	\dot{\hat a}=-\omega_s \hat b- \gamma_a \hat{a} + \sqrt{2\gamma_a}\hat{n}^L_a+\frac{i \alpha}{\hbar\sqrt2}(\hat{x}-L_\mathrm{arm}h(t)) \\
	\dot{\hat b}=\omega_s \hat a-i\chi e^{i\phi} \hat b^\dag - \gamma^b_\mathrm{tot} \hat{b} + \sqrt{2\gamma^b_R}\hat{B}_\mathrm{in} + \sqrt{2\gamma_b}\hat{n}^L_b\\
	\dot{\hat x}=\frac{1}{\mu}\hat p\\
	\dot{\hat p}=\alpha\left(\frac{\hat{a}+\hat{a}^\dag}{\sqrt{2}}\right).
	\end{cases}\end{equation}
	By entering the Fourier domain, I solve these equations for $\vec{\hat b}(\Omega)=[\hat b(\Omega),\hat b^\dag(-\Omega)]^\text{T}$ in terms of similar vectors of the vacuum sources and signal, i.e.\ $\vec h(\Omega)=[\tilde h(\Omega),\tilde h^*(-\Omega)]^\text{T}$, I find that
	\begin{align}
	\text{M}_a\vec{\hat a}(\Omega)&=-\omega_s\vec{\hat b}(\Omega) + \sqrt{2\gamma_a}\vec{\hat n}^L_a(\Omega)-i\beta\begin{bsmallmatrix}1 & 0 \\0 & -1\end{bsmallmatrix}\vec h(\Omega) \\
	\text{M}_a&= (\gamma_a-i \Omega)\text{I}+\frac{i \rho}{\Omega^2 \sqrt 2}\begin{bsmallmatrix}1 & 1 \\-1 & -1\end{bsmallmatrix} \\
	\therefore\text{M}_b\vec{\hat b}(\Omega)&=\sqrt{2\gamma^b_R}\vec{\hat B}_\mathrm{in}(\Omega) + \sqrt{2\gamma_b}\vec{\hat n}^L_b(\Omega)+\omega_s \text{M}_a^{-1}\left(\sqrt{2\gamma_a}\vec{\hat n}^L_a(\Omega)-i\beta\begin{bsmallmatrix}1 & 0 \\0 & -1\end{bsmallmatrix}\vec h(\Omega)\right)\\
	\text{M}_b&= (\gamma^b_\mathrm{tot}-i \Omega)\text{I}+\chi \begin{bsmallmatrix}0 & i e^{i\phi}\\-i e^{-i\phi} & 0\end{bsmallmatrix}+\omega_s^2\text{M}_a^{-1}.
	\end{align}
	Where the re-scaled coupling constants for the gravitational-wave signal and the radiation pressure, respectively~\footnote{As expected, $\mu=M/4\rightarrow\infty$ turns off the radiation pressure. Therefore, I will use $\rho=0$ to turn off the radiation-pressure noise.}, are
	\begin{equation}\label{eq:beta_and_rho}
	\beta = \frac{\alpha L_\mathrm{arm}}{\sqrt{2}\hbar}=\sqrt{\frac{ P_\text{circ}L_\text{arm} \omega_0 }{c  \hbar}},\quad \rho = \frac{\alpha^2}{\sqrt{2}\hbar\mu}=\frac{\sqrt{2} P_\text{circ} \omega_0}{c \mu L_\text{arm}}.
	\end{equation}
	\jam{(this disagrees with Korobko since I use the alpha from Li, note this somewhere)}
	Using the same input/output relations as Section~\ref{sec:dOPO_model} and $\Gamma=\frac{1}{\sqrt2}\begin{bsmallmatrix}1 & 1 \\-i & i\end{bsmallmatrix}$, I find the quadratures at the photodetector to be
	\begin{align}
	\vec{\hat X}_\mathrm{PD}(\Omega)&=\text{T}\vec h(\Omega)+\text{R}_\text{in}\vec{\hat X}_\mathrm{in}(\Omega)+\text{R}^L_a\vec{\hat X}^L_a(\Omega)+\text{R}^L_b\vec{\hat X}^L_b(\Omega)+\text{R}^L_\text{PD}\vec{\hat X}^L_\text{PD}(\Omega)\\
	\text{T}&=-\sqrt{1-R_\text{PD}}\omega_s(-i\beta)\Gamma \sqrt{2\gamma^b_R}\text{M}_b^{-1}\text{M}_a^{-1}\begin{bsmallmatrix}1 & 0 \\0 & -1\end{bsmallmatrix}\\
	\text{R}_\text{in}&=\sqrt{1-R_\text{PD}}\Gamma\left(\text{I}-2\gamma^b_R\text{M}_b^{-1}\right)\Gamma^{-1}\\
	\text{R}^L_a&=-\sqrt{1-R_\text{PD}}\omega_s\Gamma 2\sqrt{\gamma^b_R \gamma_a}\text{M}_b^{-1}\text{M}_a^{-1}\Gamma^{-1}\\
	\text{R}^L_b&=-\sqrt{1-R_\text{PD}}\Gamma 2\sqrt{\gamma^b_R \gamma_b}\text{M}_b^{-1}\Gamma^{-1}\\
	\text{R}^L_\text{PD}&=\sqrt{R_\text{PD}} \text{I}.
	\end{align}
	Where the noise transfer matrices $\text{R}_\text{in},\text{R}^L_b,\text{R}^L_\text{PD}$ are the same as Eq.~\ref{eq:dOPO_PD_as_fn_of_vac} except for the difference in $\text{M}_b$ which accounts for the arm cavity and the radiation-pressure noise. %, as expected.
	The total quantum noise is given by %the spectral density matrix %, i.e.\ shot noise plus quantum radiation-pressure noise,
	\begin{equation}
	\text{S}_X=\text{R}_\text{in}\text{R}_\text{in}^\dag+\text{R}^L_a{\text{R}^L_a}^\dag+\text{R}^L_b{\text{R}^L_b}^\dag+\text{R}^L_\text{PD}{\text{R}^L_\text{PD}}^\dag.
	\end{equation}
	This matrix has a similar form to Eq.~\ref{eq:dOPO_full_freedom} for the degenerate OPO but with terms for the radiation pressure noise such that the variances and covariances no longer reduce to 1 and 0, respectively, when the squeezer is off. \jam{(Display the matrix in some shortened form? Maybe in an appendix?)}
	The signal transfer function (matrix) is defined with respect to $\tilde h(\Omega)$ ($\vec h(\Omega)$) but since $h(t)$ is real $\tilde h(\Omega)=\tilde h(-\Omega)^*$ and $\vec h(\Omega)=\tilde h(\Omega) \begin{bsmallmatrix}1 \\1\end{bsmallmatrix}$.
	% \begin{align}
	% \text{T}\begin{bsmallmatrix}1 \\1\end{bsmallmatrix}=\frac{1}{\rho \chi \cos (\phi ) (\ldots)+\Omega ^4 (\ldots)}\begin{bsmallmatrix}4 \beta ^2 \gamma_R (1-R_\text{PD}) \chi ^2 \Omega ^4 \omega_s^2 \left(\gamma_a^2+\Omega ^2\right) \cos ^2(\phi ) \\2 \beta ^2 \gamma_R (1-R_\text{PD}) \Omega ^4 \omega_s^2 \left(\left(\gamma_a^2+\Omega ^2\right) \left(2 {\gamma^b_\text{tot}}^2+\chi ^2+2 \Omega ^2\right)-4 \chi  \sin (\phi ) \left({\gamma^b_\text{tot}} \left(\gamma_a^2+\Omega ^2\right)+\gamma_a \omega_s^2\right)-\chi ^2 \left(\gamma_a^2+\Omega ^2\right) \cos (2 \phi )+4 \omega_s^2 \left(\gamma_a {\gamma^b_\text{tot}}-\Omega ^2\right)+2 \omega_s^4\right)\end{bsmallmatrix}
	% \end{align}
	Inspecting $\text{T}\begin{bsmallmatrix}1 \\1\end{bsmallmatrix}$, i.e.\ the vector of signal transfer functions to each quadrature, shows that there are two terms: (1) rotates between the quadratures with the pump phase and (2) stays in the second quadrature and never vanishes with the pump phase \jam{(is it worth showing this?)}. I consider measuring the second quadrature at the photodetector since the signal is always there~\footnote{This does not mean that it is necessarily optimal to do so since the profile of the noise between the two quadratures is different to the signal, but it will suffice here~\cite{}. \jam{(What happens if I use $\phi=\pi$ and observe the first quadrature instead?)}}, and therefore the sensitivity ($\sqrt{S_h}$ is the noise-to-signal ratio) is
	\begin{equation}
	S_h = \frac{(\text{S}_X)_{2,2}}{\abs{(\text{T}\begin{bsmallmatrix}1 \\1\end{bsmallmatrix})_2}^2}.
	\end{equation}
\end{comment}


\subsection{Results}
\label{sec:dIS_results}
% priorities: radiation pressure, gravitational-wave signal, stability
% threshold, radiation pressure, and pump phase

\begin{figure}
	\centering
	\includegraphics[width=0.9\textwidth]{dIS_lossless_N_S_NSR.pdf}
	\caption{\jam{(Explain units.)} Degenerate internal squeezing's quantum noise (top panel) and gravitational-wave signal (middle panel) responses and sensitivity (bottom panel) without optical losses. The sensitivity is conventionally shown as the noise-to-signal ratio and, therefore, the goal is to lower the sensitivity curve shown~\cite{} \jam{(do I need to emphasise this again later?)}. I use the parameters in Tab.~\ref{tab:dIS_parameters}. The readout rate $\gamma^b_R$ determines the width of the squeezing peak centred on the sloshing frequency $\omega_s$. %, this value of 5~kHz is at the upper end of the kilohertz sensitivity improvement regime compared to, for example, 90~kHz for broadband sensitivity~\cite{korobkoQuantumExpanderGravitationalwave2019}.
	The radiation-pressure coupling constant $\rho$ (explained in the next chapter) controls whether radiation-pressure noise is seen ($\rho\neq0$) or not ($\rho=0$).
	}
	\label{fig:dIS_sensitivity}
\end{figure}

\begin{table}[]
\centering
\begin{tabular}{@{}ll|ll@{}}
\toprule
carrier wavelength, $\lambda_0$ & 2 $\mu\text{m}$ & sloshing frequency, $\omega_s$ & 5 kHz \\
arm cavity length, $L_\text{arm}$ & 4 km & signal mode transmissivity, $T_{\text{SRM},b}$ & 0.046 \\
signal-recycling cavity length, $L_\text{SRC}$ & 112.4 m & signal readout rate, $\gamma^b_R$ & 5 kHz \\
circulating arm power, $P_\text{circ}$ & 3 MW & arm intra-cavity loss, $T_{l,a}$ & 100 ppm \\
test mass mass, $M$ & 200 kg & signal mode intra-cavity loss, $T_{l,b}$ & 1000 ppm \\
input test mass transmissivity, $T_\text{ITM}$ & 0.0197 & detection loss, $R_\text{PD}$ & $10\%$ \\ \bottomrule
\end{tabular}
\caption{Parameter set based on LIGO~Voyager~\cite{} and realistic future losses~\cite{zhangBroadbandSignalRecycling2021,} in parts-per-million (ppm) but with deviations to achieve a particular sloshing frequency and readout rate. In particular, in the kilohertz improvement regime, the signal-recycling cavity is made longer to make the readout rate comparable to the sloshing frequency and the sensitivity sharply peaked. There is debate about $2$ versus $1.064~\mu\text{m}$ as the preferred carrier wavelength for application~\cite{}.}
\label{tab:dIS_parameters}
\end{table}
% In particular, as a baseline I will use \jam{(tabulate these parameters to reference later)} $3$~MW of circulating power, a $4$~km arm cavity, a $56$~m signal-recycling cavity, $200$~kg test masses, $2~\mu\text{m}$ carrier wavelength~\footnote{T}, $0.002$ transmission for the input test mass, and $0.046$ transmission for the signal-recycling mirror.

\jam{(Check that the linebreaks are appropriate in this sections or use subsubsections.)}

% \jam{(''Focus on the science, not the equations'' - VA. Consider what order these results should come in.)}
% comment on general performance, signal
% similarly to external squeezing, improving shot noise worsens radiation-pressure noise.
Using the Hamiltonian model from Ref.~\cite{korobkoQuantumExpanderGravitationalwave2019}~\footnote{But with an added factor of $\sqrt{2}$ to $G_0$ from that reference to match the convention for the gravitational-wave coupling constant $\alpha_\text{GW}$ from Ref.~\cite{liBroadbandSensitivityImprovement2020} that I use in Chapter~\ref{chp:nIS_analytics}.}, let the interaction Hamiltonian be
\begin{equation}
\hat{H}_\text{I}=i\hbar\omega_s(\hat{a}\hat{b}^\dag-\hat{a}^\dag\hat{b})+\frac{\hbar\chi}{2}(e^{i\phi} (\hat b^\dag)^2 - e^{-i\phi} \hat b^2).
\end{equation}
Here $\hat a$ is the differential arm cavity mode~\footnote{I assume that the Michelson interferometer is tuned such that the light from the arms destructively interferes at the output of the beamsplitter~\cite{}.} at the carrier frequency $\omega_0$ coupled to the signal-recycling cavity signal mode $\hat b$ with coupling rate (called the ``sloshing'' frequency~\cite{}) $\omega_s$ determined by the transmission through the input test mass and the lengths of the two coupled cavities~\cite{}. % \jam{(should I state the approximation here or leave it for the nIS analytics?)}.
The second term in the Hamiltonian is the same as the degenerate OPO in Section~\ref{sec:dOPO_model}~\footnote{Up to the phase of the pump.}.
The model also includes an intra-cavity loss port in the arms with transmissivity $T_{l,a}$ to vacuum $\hat n^L_a$ as shown in Fig.~\ref{fig:dIS_config}~\footnote{Ref.~\cite{korobkoQuantumExpanderGravitationalwave2019} does not include this but I include it in my model for completeness.}. 
Using this model, the sensitivity of the detector to a gravitational-wave strain $h(t)$ is the signal-to-noise ratio $\frac{\abs{T}}{\sqrt{S_X}}$~\cite{} in the measured quadrature $\hat X_\text{PD}(\Omega)=\sum_i R_i \hat X_i^\text{vac}(\Omega) + T \tilde h(\Omega)$, given by the noise $\sqrt{S_X}$ and signal $T$ responses. However, I will plot the noise-to-signal ratio throughout this thesis, i.e.\ $\sqrt{S_h}=\frac{\sqrt{S_X}}{\abs{T}}$ which has units of $\text{Hz}^{-1/2}$ \jam{(why if $S_X$ and T are unitless?)}, as it is the convention in the gravitational-wave literature~\cite{}, and, therefore, smaller values in $\text{Hz}^{-1/2}$ indicate better sensitivity henceforth. 

The resulting noise and signal responses and the sensitivity are shown in Fig.~\ref{fig:dIS_sensitivity} without optical losses and for the parameter set of LIGO~Voyager but with a readout rate of 5~kHz~\footnote{I increase the readout rate from $0.5$ to 5~kHz but keep the sloshing frequency fixed at 5~kHz by shortening the signal-recycling cavity and decreasing the transmission through the input test mass.} as shown in Tab.~\ref{tab:dIS_parameters}.
LIGO~Voyager~\cite{} is a planned series of upgrades to the existing Advanced~LIGO detectors and I use its parameters throughout this thesis, as in the literature~\cite{liBroadbandSensitivityImprovement2020,miaoDesignGravitationalWaveDetectors2018,korobkoQuantumExpanderGravitationalwave2019,}, to represent the next generation of gravitational-wave detectors~\footnote{These parameters might be biased against one configuration over another. I have partially mitigated this by varying the coupling rates between the modes which appears to best characterise the different classes of parameter sets~\cite{}.}.
% From this baseline parameter set, I will change the readout rate through the signal-recycling mirror by varying the length of the signal-recycling cavity~\footnote{I will also vary the input test mass's transmissivity to counteract the effect of changing the signal-recycling cavity's length on the coupling rate to the arm cavity (known as the sloshing frequency~\cite{}).} as there is much variation of this parameter in the literature that I compare to~\cite{liBroadbandSensitivityImprovement2020,miaoDesignGravitationalWaveDetectors2018,korobkoQuantumExpanderGravitationalwave2019,}.
The squeezer parameter $\chi$ affects the sensitivity as shown in Fig.~\ref{fig:dIS_sensitivity} and needs to be optimised for each configuration separately. With the squeezer turned off, the configuration becomes the interferometer in Fig.~\ref{fig:DRFPMI}. %and the noise, signal, and sensitivity reduce to the baseline curves~\cite{,}, where the signal transfer function peaks at $\omega_s$ with bandwidth of the peak $\gamma^b_R$~\cite{}~\footnote{Not to be confused with the overall bandwidth of the signal response which is still the arm cavity free-spectral range.}.
Turning the squeezer on (1) squeezes the shot noise and de-amplifies the signal around the sloshing frequency $\omega_s$, (2) does not affect the radiation-pressure noise below 100~Hz, and therefore (3) improves the sensitivity around the sloshing frequency and not affect it at lower frequencies. 
Threshold in the lossless case is $\chi_\text{thr}=\gamma^b_R$ where the squeezed noise goes to zero at $\Omega=\omega_s$. In the lossy case, the situation is more complicated and the threshold is not quoted in the literature, which I will address in Section~\ref{sec:singularity_threshold}. Since configurations must be stable to be feasible, I also confirm that degenerate internal squeezing is stable below threshold in Appendix~\ref{app:dIS_stability}.
% does not affect RP, do not make this mistake!
% Increasing the squeezer parameter $\chi$ further increases the squeezing of the noise and signal but it might not continue to improve the sensitivity, which I will cover shortly. 
% \jam{(Explain the ``two regimes it can operate it I.e. broadening sensitivity (Korobko) or kHz sensitivity (Adya).'' -- VA)}
Degenerate internal squeezing can be operated in two regimes depending on the choice of the sloshing frequency ($\omega_s$) and the bandwidth of the signal-recycling cavity ($\gamma^b_R$)~\footnote{Not to be confused with the overall bandwidth of the signal response which is from the arm cavities.} which determine the frequency range of the sensitivity improvement: (1) broadband sensitivity~\cite{korobkoQuantumExpanderGravitationalwave2019} when $\gamma^b_R$ is large (e.g.\ $L_\text{SRC}$ is short) and the sensitivity is improved, for example, from around 100 to 10000~Hz or (2) kilohertz sensitivity~\cite{adyaQuantumEnhancedKHz2020} when $\gamma^b_R$ is small (e.g.\ $L_\text{SRC}$ is long) and the sensitivity is narrowly, but strongly, improved around $\omega_s$ (e.g.\ 5~kHz) by more than an order of magnitude at the peak as shown in Fig.~\ref{fig:dIS_sensitivity} \jam{(check)}.

% why improvement only around sloshing frequency
The squeezing of the shot noise and signal is localised to the sloshing frequency because of the resonance structure of the coupled cavity system which changes where the signal-recycling cavity is resonant. At the sloshing frequency, energy is strongly coupled from the arm cavity into the signal-recycling cavity and the signal-recycling cavity mode is resonant~\footnote{The phase acquired upon reflecting off the input test mass depends on whether the arm cavity is resonant and means that the signal-recycling cavity can be chosen to be resonant at frequencies where the arm cavity is not resonant. At the sloshing frequency, the arm cavity is not resonant, as seen in the falling signal response in Fig.~\ref{fig:dIS_sensitivity}.}~\cite{korobkoTamingQuantumNoiseHow2020}. As the squeezer is only effective when the cavity is resonant~\cite{}, as in Fig.~\ref{fig:dOPO_variances} where the cavity is resonant at DC, the signal and noise are only squeezed around the sloshing frequency~\cite{}. \jam{(I do not understand this)} Away from the sloshing frequency, the cavity is not resonant and degenerate internal squeezing does not affect the sensitivity below kilohertz~\footnote{These results assume that the arm cavity loss is realistically small, e.g.\ $T_{l,a}=100\text{ppm}$, as discussed later.}.

% \jam{(Which of these results are necessary to know for the results chapters, none of them?)}
% There is much behaviour to analyse for this configuration~\cite{korobkoQuantumExpanderGravitationalwave2019,adyaQuantumEnhancedKHz2020,korobkoTamingQuantumNoiseHow2020} that motivates how I analyse the configuration in my work and which are useful to compare to.
% To later compare to the behaviour of the configuration in my work \jam{(do I compare all of this behaviour or just the loss tolerance?)}, I will briefly discuss degenerate internal squeezing's (1) threshold, (2) stability, and (3) tolerance to optical loss.
% threshold, lossless threshold but leave lossy threshold to research chapters? ``To not conflate existing knowledge with my work, I leave the lossy threshold to Section~\ref{}.'' just state result and justify later?
% \begin{equation}
% \label{eq:dIS_lossless}
% (\text{S}_X)_{2,2}=1-\frac{4 \gamma^b_R \chi \Omega ^2}{\Omega ^2 (\gamma^b_R+\chi )^2+(\Omega ^2-\omega_s^2)^2}.
% \end{equation}
% Firstly, for the lossless noise response as shown in the top panel of Fig.~\ref{fig:dIS_sensitivity}~\cite{korobkoQuantumExpanderGravitationalwave2019}, the squeezed noise goes to zero at $\chi_\text{thr}=\gamma^b_R$ and $\Omega=\omega_s$ \jam{(do I need to show this?)}, which defines threshold~\cite{}. In the anti-squeezed quadrature, the noise instead diverges at $\Omega=\omega_s$ at threshold. This behaviour is similar to the lossless degenerate OPO in Fig.~\ref{fig:dOPO_variances}, except that the peak occurs at the sloshing frequency instead of at DC. From the perspective of gains and losses inside the signal-recycling cavity, any light lost to the arms must return since the arm loss is zero and, therefore, the net loss from the cavity mode in the steady-state is still $\gamma^b_R$ \jam{(is this argument correct?)}. In the lossy case, the situation is more complicated and the threshold is not quoted in the literature, which I will address in Section~\ref{sec:singularity_threshold}. 

% \begin{figure}
% 	\centering
% 	% \includegraphics[width=\textwidth]{}
% 	\caption{\jam{(Make sure that x-axis is consistent with nIS plot. Is this plot necessary?)} Degenerate internal squeezing sensitivity versus quantum noise, varying the squeezer parameter up to threshold for squeezing and anti-squeezing and different detection losses \jam{(show for other losses as well?)}. The optimal value of the squeezer parameter is shown.}
% 	\label{fig:dIS_optimal_squeezing}
% \end{figure}

% pump phase, foreshadow that maximum of anti-squeezing is not necessarily the minimum of squeezing
% RP is not (anti)squeezed, look at the plots!
% A complication when the system is lossy is that maximising the anti-squeezed quadrature is not necessarily the same as minimising the squeezed quadrature, as shown in Fig.~\ref{fig:dIS_sensitivity}, the two move apart at high arm losses. This is a result of worsening the uncertainty product in the Heisenberg Uncertainty Principle. \jam{(But why does this happen?)}
% optimal squeezing curves against loss, in really high loss should antisqueeze
% difference to caves's amplifier

% \subsubsection{Stability}
% establish how stability is analysed


\subsection{Limitation: tolerance to optical loss}
% Problems with proposal -
\label{sec:dIS_optical_loss}

\begin{figure}
	\centering
	\includegraphics[width=0.7\textwidth]{dIS_loss_tolerance.pdf}
	\caption{Degenerate internal squeezing sensitivity for realistic losses. The dashed lines show the effect on the sensitivity of increasing the loss from the realistic value in Tab.~\ref{tab:dIS_parameters}. Increasing each of the three losses separately shows that arm loss does not affect the sensitivity, signal mode loss decreases the peak sensitivity (e.g.\ diminishes the trough around 4~kHz), and detection loss broadly decreases the sensitivity. This means that detection loss dominates the losses but signal loss would matter around the peak if it was worse than the desired 1000~ppm. I use the parameters in Tab.~\ref{tab:dIS_parameters}.}
	\label{fig:dIS_loss_tolerance}
\end{figure}
% \begin{figure}
% 	\centering
% 	\includegraphics[width=\textwidth]{dIS_noise_budget.pdf}
% 	\caption{\jam{(Cut all these noise budget figures?)} Degenerate internal squeezing breakdown of noise sources, showing the contribution to the total quantum noise response from each vacuum input. The radiation-pressure noise and the squeezing around $\omega_s$ (compared to the noise without squeezing which is not shown) appear in all of the interferometer noise sources and not in the detection loss. The squeezing affects the readout port vacuum more than the intra-cavity noises \jam{(why?)}. The detection loss dominates the other losses except below 10~Hz where the radiation-pressure noise is present and the signal loss dominates, however, except around the peak, the readout port vacuum dominates all the losses. However, Fig.~\ref{fig:dIS_loss_tolerance} provides a better understanding of the tolerance to losses because it also accounts for the signal response.}
% 	\label{fig:dIS_noise_budget}
% \end{figure}

% \jam{(``Focus on the science not the equations. Once you explain the behaviour, explain the effect of losses on this system.'' -- VA.)}

% intracavity loss behaves differently to dOPO 
Degenerate internal squeezing has different tolerances to the different sources of optical loss as shown in Fig.~\ref{fig:dIS_loss_tolerance}. %, but the tolerance is similar for the two regimes of broadband and kilohertz improvement. %, where now the signal's tolerance must be considered.
The squeezer squeezes the vacuum from the intra-cavity losses and the readout port~\cite{}. %, as shown in Fig.~\ref{fig:dIS_noise_budget}. 
The general effects of each loss are that, firstly, detection loss uniformly pulls the noise towards the vacuum and pulls the signal towards zero because it uniformly mixes in vacuum. Secondly, signal intra-cavity loss behaves differently to the degenerate OPO, as the noise response remains within the lossless envelope instead of broadening, the radiation-pressure noise is increased, and the signal and noise are worsened around the sloshing frequency \jam{(why doesn't the response broaden like before?)}. Finally, arm intra-cavity loss diminishes the squeezing of the noise and moves the peak frequency away from the sloshing frequency, worsens the DC response to the signal, but improves the radiation-pressure noise \jam{(check this)}.

%With arm loss, the signal response remains within the lossless signal envelope but worsens the DC response to the signal. 
% \begin{figure}
% 	\centering
% 	% \includegraphics[width=\textwidth]{}
% 	\caption{\jam{(Remove this plot)} Degenerate internal squeezing shot noise response in the limit of large arm loss compared to the theoretical limiting degenerate OPO with fully reflective input test mass.}
% 	\label{fig:dIS_limit_dOPO}
% \end{figure}

% \subsubsection{Reduction to degenerate OPO}
% arm cavity loss gives reduction to dOPO
% The behaviour against different parameters will be similar to that seen for the degenerate OPO in Section~\ref{}.
% The behaviour in the high arm loss limit can be understood as the degenerate OPO limit of degenerate internal squeezing, i.e.\ when the arm cavity is removed. Formally taking the limit $\gamma_a\rightarrow\infty$ of the equations-of-motion in Ref.~\cite{korobkoQuantumExpanderGravitationalwave2019} shows that the system reduces to a degenerate OPO between the signal-recycling mirror and a fully reflective input test mass~\footnote{This is why the sensitivity away from the sloshing frequency is affected when the arm loss is high.}. \jam{(give evidence? check if equivalent to sending sloshing frequency to zero)} Although I initially expected the input test mass to instead become another loss port with its original transmissivity, this can be explained as the disappearance of the arm cavity mode altogether, vacuum or otherwise \jam{(check this)}. For $\gamma_a\rightarrow\infty$, the equation-of-motion for $\hat a$~\cite{korobkoQuantumExpanderGravitationalwave2019} becomes $\dot{\hat a}\approx -\gamma_a \hat a$ which quickly decays, and therefore any vacuum $\hat n^L_a$ cannot couple to $\hat b$, nor any of $\hat b$ couple out through $\omega_s$, because all terms involving $\hat a$ vanish~\footnote{$\hat a$ is implicitly the fluctuating component $\delta \hat a$ throughout this discussion.}.
% However, I suspect that this is a false consequence of the single-mode approximation and that if a ``transfer matrix'' approach~\cite{finesse,}~\footnote{Not to be confused with the transfer matrices describing the signal and noise responses.} was instead used where the fields at a point are propagated inside the cavities and the cavity modes, e.g.\ $\hat a$, are not explicit, the limit would instead be a degenerate OPO with added intra-cavity loss to account for the open input test mass port. I leave verifying this to future work \jam{(check?)}.
% But, since realistic arm losses for future detectors are below this high loss regime \jam{(quantify)}, this behaviour is not of concern for applications. 
% This is confusing, I expected the initial test mass to become a loss port, but this can be understood as there no longer being an arm cavity mode, vacuum or otherwise, since it can not make a round-trip.

With losses included in the model, although increasing the squeezer parameter continues to decrease the noise, the optimal value for the sensitivity can be below threshold~\cite{korobkoCompensatingQuantumDecoherenceTalk2021}. This is because increasing the squeezing might decrease the signal more than the noise, e.g.\ once the noise is limited by detection loss, which is not squeezed by the internal squeezer, further squeezing will only decrease the signal. In the high loss regime, it can be that any amount of squeezing is detrimental and that it is instead optimal to anti-squeeze internally where the sensitivity is improved because the signal is anti-squeezed more than the noise and the losses instead decrease the noise since it is above the vacuum value~\cite{korobkoCompensatingQuantumDecoherenceTalk2021}. However, future detectors~\cite{} do not belong to this high loss regime. %, unlike using a Caves's amplifier. 

% \begin{figure}
% 	\centering
% 	% \includegraphics[width=\textwidth]{}
% 	\caption{\jam{(Combine with Fig.~\ref{fig:dIS_loss_tolerance})} Degenerate internal squeezing sensitivity for realistic losses, using the same parameters as Fig.~\ref{fig:dIS_sensitivity} and a squeezer parameter close to threshold. \jam{(Show that detection loss dominates.)}}
% 	\label{fig:dIS_realistic_loss}
% \end{figure}

If the realistic losses in Tab.~\ref{tab:dIS_parameters}~\footnote{What losses are realistic for future detectors is inexact given the unknown progress of future technology, but the literature suggests that these values are conservative~\cite{zhangBroadbandSignalRecycling2021,}.} are assumed, then the sensitivity improvement significantly degrades to less than a factor of two at the sloshing frequency as shown in Fig.~\ref{fig:dIS_loss_tolerance} compared to the lossless case in Fig.~\ref{fig:dIS_sensitivity} that improved by over an order of magnitude \jam{(check)}. For these realistic losses, detection loss is responsible for most of the degradation seen in Fig.~\ref{fig:dIS_loss_tolerance} since it dominates the noise apart from the readout port vacuum~\cite{}. %, as seen in Fig.~\ref{fig:dIS_noise_budget} \jam{(check)}. 
% \subsection{Connection to nondegenerate internal squeezing}
% at high enough losses, it becomes optimal to instead anti-squeeze internally, this means that you might as well use nondegenerate internal squeezing because then you can anti-squeeze and potentially exploit the correlations using a combined readout
% conclusions about dIS?
Although degenerate internal squeezing improves the sensitivity, its low \jam{(quantify)} tolerance to optical loss motivates investigating other methods which might improve sensitivity more given the same losses. 
%, such as internal anti-squeezing using nondegenerate internal squeezing~\footnote{Which should anti-squeeze by comparison to the nondegenerate OPO.}. \jam{(What is the optimal squeezing value given realistic losses? I need to rule out degenerate anti-squeezing.)} %This does not mean that degenerate internal squeezing is not useful, it is worth further investigation, especially in low loss applications~\cite{}, but I will consider the nondegenerate case and whether it fares better.


	% Nondegenerate internal squeezing, where the internal squeezer is instead nondegenerate, has
	% been proposed as an alternative to degenerate internal squeezing~\cite{yapadyaPersonalCommunication}, although a comprehensive analysis of nondegenerate internal squeezing is yet to be done~\cite{liBroadbandSensitivityImprovement2020}. Because nondegenerate squeezing results in two entangled photons with different frequencies, these photons will not interfere with each other in the same manner as the degenerate case. Without this interference, nondegenerate internal squeezing increases the signal and the noise instead of decreasing them like the degenerate case. Therefore, nondegenerate internal squeezing is predicted to be more resistant to photodetector loss since the signal amplitude is greater. This project aims to investigate the potential benefits of nondegenerate internal squeezing over degenerate internal squeezing.



%%%%%%%%%%%%%%%%%%%%%%%%%%%%%%%%%%%%%%%%%%
\section{Stable optomechanical filtering}
\label{sec:sWLC}
% the point of this section is to explain the existing design, what is wrong with it, and how it connects to nIS

\begin{figure}[ht]
	\centering
	\includegraphics[angle=-90,width=0.9\textwidth]{sWLC_config.pdf}
	\caption{\jam{(Should I include $\chi_m$?)} Stable optomechanical filtering configuration~\cite{liEnhancingInterferometerSensitivity2021} with mechanical idler mode $\hat{c}_m$, e.g.\ of a suspended mirror, at a mechanical resonance frequency of $\omega_m$ coupled to the signal-recycling cavity mode $\hat b$ at $\omega_0$ via radiation pressure. $\hat b$ and $\hat{c}_m$ are driven by a blue-detuned pump mode at $\omega_0+\omega_m$. The interferometer mirrors are labelled as the ETM (end test mass), ITM (input test mass), and SRM (signal-recycling mirror).}
	\label{fig:sWLC_config}
\end{figure}

\jam{(``No need to go into details about White light cavities - introduce it, mention the challenges for it, cite references.'' -- VBA)}

\jam{(Read liEnhancingInterferometerSensitivity2021 to check if this is still current. Also, look at research journals.)}

% explain the design
Stable optomechanical filtering uses a modified signal-recycling cavity compared to the conventional detector shown in Fig.~\ref{fig:DRFPMI}. Here, the signal-recycling cavity acts as an optomechanical filter cavity which is coupled via the signal mode to a mechanical idler mode~\footnote{The exact position and type of the mechanical oscillator does not matter in this simplified model, compare Refs.~\cite{liBroadbandSensitivityImprovement2020,liEnhancingInterferometerSensitivity2021} which have different positions but the same Hamiltonian.} suspended mirror~\cite{}, as shown in Fig.~\ref{fig:sWLC_config}. The mechanical mode and signal mode are driven by a blue-detuned pump mode, and this configuration, where the signal mode is measured, is dynamically stable~\cite{}~\footnote{The previous, unstable configuration measured the arm cavity mode~\cite{miaoEnhancingBandwidthGravitationalWave2015}.}.
% is it necessary to talk about unstable design? --> not the focus, just give it a mention, Kramers-Kronig relations?
% This configuration is dynamically stable~\cite{} and improved upon an unstable design~\cite{} by changing which mode was read out~\footnote{The unstable configuration read out the arm cavity mode directly instead of the signal-recycling cavity mode. The addition of a control system was required to stabilise the naturally unstable system~\cite{}.}. The stable design is more relevant to this thesis because its mode structure is more closely related to my work. %, which I will elaborate on shortly.
The filter cavity amplifies/suppresses certain frequencies given the choice of cavity parameters and can be designed to partially counter the arm cavity's resonance that decreases the signal transfer function at kilohertz~\cite{}. %Although the signal-recycling cavity already filters the signal, e.g.\ the length of the cavity affects the signal transfer function's peak and bandwidth~\footnote{Given the role of $\omega_s$ and $\gamma^b_R$ as peak frequency and bandwidth of the peak, respectively.}, the optomechanical coupling changes the resonance behaviour and can be used to more selectively filter the signal~\cite{}. %I use the term in this context to be consistent with the literature~\cite{}. % waste of time explaining this?
% how does this configuration beat the Mizuno limit (four factors in the introduction)? --> cancels arm cavity resonance
% the lossless system does not affect the noise, therefore is not fundamentally squeezing, but the lossy system will antisqueeze % idler loss complicates this later
 %, where the parameter of interest is the optomechanical coupling compared to the optical coupling between the arm and signal-recycling cavities at the input test mass. % Again, without increasing the circulating power in the arms.
This partially achieves the ``white-light cavity'' idea: to broaden the resonance by changing the filter cavity's phase response at each frequency to be opposite to that of the arm cavity~\cite{}. \jam{(Why does this avoid the Mizuno limit?)} % However, the literature has only considered the case without optical loss~\cite{,} and, in the following chapters, my work will suggest that the behaviour of the system is complicated when optical loss is introduced. %~\footnote{To preview the results, this is because the quantum noise becomes anti-squeezed and therefore the Mizuno limit is beaten by squeezing instead of/along with the countering-the-arm-resonance explanation above.}. \jam{(Should not talk about results here?)}
The behaviour of the configuration is more complicated when optical losses are introduced, but, for this section, it should just be considered as changing the signal response. % through the resonance structure of the interferometer.

% explain the results of the design
% I omit the model of this configuration here, because in the lossless case it is exactly the model in the next chapter, see Section~\ref{}.
I emphasise two aspects of this configuration: (1) its dependence on the optomechanical and optical coupling rates and (2) its vulnerability to mechanical loss. %, and (3) its connection to nondegenerate internal squeezing. Because of this last point,
% \begin{figure}
% 	\centering
% 	% \includegraphics[width=\textwidth]{}
% 	\caption{\jam{(Remove this plot since the results are compared to later. Need to show on-threshold behaviour. Is this figure redundant with the later comparison?)} Stable optomechanical filtering's sensitivity, showing the effect of mechanical loss, the dominant noise source. Optical loss is not included in this model. This figure was generated using the code from Ref.~\cite{} with permission from the authors~\cite{LiPersonalCommunication} \jam{(check this)}, the parameters used are \jam{(... fill this in)} and $T_\text{env}=4$~K and $Q_m=8\times10^9$.}
% 	\label{fig:sWLC_sensitivity}
% \end{figure}
% but I need to talk about their results? in particular, the exceptional point of PT-symmetric, stability, and sensitivity at threshold. maybe do show a lossless, shot-noise only model -- better yet, cite the results in their paper and reference the next section for further explanation?
% exceptional point, stability and PT symmetry
In the lossless case, comparing the coupling rates of the arm and the signal and the signal and the mechanical idler shows that when the two coupling rates are equal, the behaviour is exceptional.
Let the interaction Hamiltonian of the system be~\cite{}~\footnote{This is similar to the Hamiltonian I will use in my work which I will detail in Chapter~\ref{chp:nIS_analytics}.}
\begin{equation}
\hat{H}_\text{I}=i\hbar\omega_s(\hat{a}\hat{b}^\dag-\hat{a}^\dag\hat{b})+i\hbar\chi_m(\hat{b}^\dag\hat{c}_m^\dag-\hat{b}\hat{c}_m).
\end{equation}
Here $\hat a, \hat b, \omega_s$ are the same notation as degenerate internal squeezing, $\hat{c}_m$ annihilates the mechanical mode \jam{(explain phonons?)}, and $\chi_m$ is the optomechanical coupling rate. When this coupling rate $\chi_m$ equals the sloshing frequency ($\omega_s$), the interaction Hamiltonian becomes invariant under the transformation $\hat a\mapsto\hat{c}_m^\dag, \hat{c}_m\mapsto\hat a^\dag$ which corresponds to the composition of parity, $\hat a\leftrightarrow \hat{c}_m$, and time, $\hat a\leftrightarrow \hat a^\dag,\hat {c}_m\leftrightarrow \hat {c}_m^\dag$, transformations, and that leaves $\hat b$ invariant. This parity-time (PT) symmetry causes other changes in the system, 
% Although, the most notable feature of PT-symmetric systems is that quantum mechanics can \jam{(be reformulated to)} also allow non-Hermitian, PT-symmetric Hamiltonians~\cite{}. And since this system is still Hermitian, the importance of PT-symmetry is lessened. \jam{(What did Carl Bender say? --> Hermitiancy is more complicated and requires further examination, see Section~\ref{sec:future_work}.)}}. 
namely, the lossless, PT-symmetric system is borderline stable, with one complex $\Omega$ pole on the real axis, is at an Exceptional Point of its independent eigenmodes~\cite{liBroadbandSensitivityImprovement2020} \jam{(do not just quote Li, I need to revise this)}, and the shot noise--limited, integrated sensitivity becomes unbounded~\cite{}. With radiation pressure included in the model~\footnote{There is a complication with radiation pressure coupling the arm cavity mode to the test mass mechanical mode, as for PT-symmetry to be maintained the filter cavity mechanical mode must then be coupled to a back-action evasion mode with negative effective mass~\cite{}, but I will not consider this for simplicity.}, the integrated sensitivity becomes bounded~\cite{} and although the kilohertz sensitivity improves \jam{(check plot)}, the main improvement is from 100-1000~Hz~\cite{} \jam{(the ``audio-band'')}. %, as shown for the lossless case \jam{in Fig.~\ref{fig:sWLC_sensitivity} (talk about results without figure)}. %This foreshadows an aspect of my work, that I will discuss later, about broadband versus kilohertz sensitivity improvement. 

	% A recent proposal uses a stable optomechanical filter cavity to avoid this limit and increase high-frequency sensitivity without fully sacrificing low-frequency sensitivity nor increasing the power~\cite{liBroadbandSensitivityImprovement2020}. 
	% Stable optomechanical filtering consists of an auxiliary filter cavity inside the signal-recycling cavity. One of the filter cavity's mirrors is a mechanical oscillator, such as a suspended mirror, driven by a laser whose frequency is appropriate to excite the mechanical mode~\cite{liBroadbandSensitivityImprovement2020}. This design is dynamically stable unlike previous designs for optomechanical filter cavities~\cite{miaoEnhancingBandwidthGravitationalWave2015,pageEnhancedDetectionHigh2018,miaoDesignGravitationalWaveDetectors2018}. This system has a parity-time symmetry between the differential optical mode of the interferometer and the mechanical mode; 


\subsection{Limitation: tolerance to mechanical loss}
% Problems with proposal - 
\label{sec:sWLC_loss}
% thermal noise and mechanical quality factor

Stable optomechanical filtering could potentially improve the sensitivity of future detectors but its vulnerability to mechanical loss demands progress beyond current technology~\cite{}. Mechanical loss damps the mechanical mode due to the dissipation of energy into the thermal bath of the mass and its surroundings. This raises the temperature of the mass and increases the thermal noise, which becomes radiation pressure noise in the filter cavity mode, and then degrades the sensitivity. The thermal noise from mechanical loss dominates the losses of stable optomechanical filtering~\cite{}, including the optical losses shown in Fig.~\ref{fig:sWLC_config}. The results in Ref.~\cite{liBroadbandSensitivityImprovement2020} assumes the ratio of the environmental temperature $T_\text{env}$ to quality factor $Q_m$ to be small~\cite{miaoEnhancingBandwidthGravitationalWave2015},
\begin{equation}\label{eq:sWLC_mechanical_loss}
\frac{T_\text{env}}{Q_m}\leq\frac{\hbar \gamma_\text{single-cavity}}{8 k_B}.
\end{equation} 
Here $\gamma_\text{single-cavity}$ is the bandwidth~\cite{} of the Fabry-Perot Michelson interferometer, i.e.\ without the signal-recycling cavity, and $k_B$ is the Boltzmann constant.
The quality factor required to satisfy this bound is $Q_m=8\times10^9$~\cite{liBroadbandSensitivityImprovement2020} which is beyond that possible with current technology~\cite{pageEnhancedDetectionHigh2018,,} and would require improvements, i.e.\ dilution factors, that have not been experimentally demonstrated to date~\cite{pageEnhancedDetectionHigh2018,,}.
Therefore, an all-optical alternative~\footnote{Since all other systems that use optomechanical filtering have the same requirement. \jam{(check/cite?)}} is appealing because the losses required might be more realistic; this is the focus of my work in the following chapters.
% For example, this is satisfied for the LIGO~Voyager parameter set \jam{(check this)} for $T_\text{env}=4$~K and $Q_m=8\times10^9$~\cite{liBroadbandSensitivityImprovement2020}. But this quality factor $Q_m$ is beyond that possible with current technology~\cite{pageEnhancedDetectionHigh2018,,} and if the mechanical loss is higher than the above bound, then the sensitivity significantly degrades \jam{(quantify)}. Therefore, for stable optomechanical filtering to be feasible, the mechanical loss must improve \jam{(by how much?)}, towards which there is research in the literature~\cite{pageEnhancedDetectionHigh2018,,} \jam{(what citations?)}.Alternatively, it could be made feasible by finding a configuration with the same mode structure but different losses, which I will discuss in the next chapter.


% I will not discus this configuration further because without optical and mechanical loss it is equivalent to the lossless model in the next chapter, which I will discuss later. 

% optical loss --> bridge into next subsection?

	% However, it requires cryogenic (around 4~K) environmental temperature and a higher mechanical quality factor than is currently possible. An all-optical alternative to this optomechanical proposal without these requirements is desirable.
	% It is designed for implementation in later-generation detectors and assumes technological improvements in the near future in arm length, power, optical loss, Brownian noise, and, most stringently, in the thermal noise and quality factor of the mechanical oscillator. These assumptions are necessary to achieve the target sensitivity for astrophysical applications~\cite{miaoDesignGravitationalWaveDetectors2018}. By investigating nondegenerate internal squeezing, I aim to find more realistic requirements for a future detector.


\begin{comment}
\subsection{Connection to nondegenerate internal squeezing}
\label{sec:modal_equivalence}

\begin{figure}
	\centering
	% \includegraphics[width=\textwidth]{}
	\caption{\jam{(Add idler readout)} Mode diagrams of the OPOs, degenerate internal squeezing, stable optomechanical filtering, and nondegenerate internal squeezing. The latter two configurations are modally the same but are optomechanical and all-optical, respectively, which means that their performance might be different given the different losses they encounter. The idler readout scheme and the unstable case of optomechanical filtering are also shown. The parallels between the degenerate OPO and degenerate internal squeezing, and the nondegenerate OPO and nondegenerate internal squeezing, are also shown.}
	\label{fig:mode_diagram}
\end{figure}

\jam{(Break out this section into motivation for nIS either here or at the start of chapter 4 (which might be cleaner).)}

% However, 
% all-optical alternative
Finally, there is an alternative route to progress, to replace the optomechanical interaction with an all-optical one and replace the mechanical loss with optical loss. This is to consider nondegenerate internal squeezing, where the internal squeezer squeezes signal and idler modes and the idler is not resonant in the arms~\footnote{So that the idler mode is not coupled to the arm cavity mode.}. 
% although nIS is closer on first inspection to dIS, the underlying structure is closer to sWLC, but both motivate it.
% equivalent mode structures, just the different noise sources
Although it might seem closer to degenerate internal squeezing than stable optomechanical filtering, the underlying mode structure of nondegenerate internal squeezing is equivalent to the latter, by mapping the idler optical mode and squeezer parameter $\hat c, \chi$ to the mechanical mode and optomechanical coupling $\hat{c}_m, \chi_m$, respectively, in the Hamiltonian~\cite{}, as shown in Fig.~\ref{fig:mode_diagram}. Although this is only true in the case with no optical or mechanical loss, as thermal noise in $\hat{c}_m$ behaves differently to shot noise in $\hat c$~\cite{} \jam{(it does experimentally and the parameter regimes are very different in application, but the Langevin terms are the same)}, it is reasonable to predict that the lossy configurations behave similarly to each other because the abstract dynamics are the same. Therefore, nondegenerate internal squeezing might achieve the sensitivity improvement of stable optomechanical filtering but at more realistic optical loss than the mechanical loss required above.

\jam{(different experimental constraints on optical loss versus mechanical loss)}
\end{comment}

% the lossless system does not affect the noise, therefore is not fundamentally squeezing, but the lossy system will antisqueeze % idler loss complicates this later --> cover this in results if it matters

	% replicating this [PT-]symmetry with an internal squeezer requires the squeezer to be nondegenerate to mimic the distinction between the filter cavity optical mode and the mechanical mode~\cite{liBroadbandSensitivityImprovement2020}. Stable optomechanical filtering improves high-frequency sensitivity by cancelling the effect of the resonance behaviour of the interferometer cavities.
	% Nondegenerate internal squeezing is also motivated by a connection between it and the use of
	% a stable optomechanical filter cavity to improve high-frequency sensitivity~\cite{yapadyaPersonalCommunication,liBroadbandSensitivityImprovement2020}. The Hamiltonians of the two systems are equivalent under some mapping of the creation and annihilation operators of certain optical fields to certain mechanical fields. This connection exploits the fact that the nondegenerate squeezer interacts with three distinct frequencies to introduce a symmetry into the all-optical system that the optomechanical system has. This means that using nondegenerate internal squeezing may achieve the benefits of a stable optomechanical filter cavity without the optomechanical drawbacks of requiring cryogenic (around 4~K~\cite{miaoEnhancingBandwidthGravitationalWave2015}) environmental temperature and high mechanical quality factor. Therefore, understanding stable optomechanical filtering should lead to a better understanding of what nondegenerate internal squeezing might achieve.

%%%%%%%%%%%%%%%%%%%%%%%%%%%%%%%%%%%%%%%%%%
\section{Chapter summary}

% there are solutions in the literature to improving kilohertz sensitivity, but they have their problems, and motivate considering a combined configuration

In this chapter, I have reviewed the literature of two proposed configurations that might improve the kilohertz sensitivity of future gravitational-wave detectors and that motivate my work: degenerate internal squeezing and stable optomechanical filtering.
I have explained how they work and their vulnerabilities to losses.
% Firstly, I reviewed the literature of possible configurations to improve kilohertz sensitivity, from which I focussed on two configurations: degenerate internal squeezing and stable optomechanical filtering.
% Firstly, I examined degenerate internal squeezing and discussed its sensitivity, stability, and how it is limited by a low tolerance to optical loss. % and how, in the high loss regime, it becomes beneficial to anti-squeeze instead of squeeze internally. %, which suggests that nondegenerate internal squeezing might be more resilient to optical loss than degenerate internal squeezing. 
% Secondly, I examined stable optomechanical filtering. % which deviates from all other configurations in this thesis by use of a mechanical oscillator instead of an optical squeezer, and avoids the Mizuno limits through partially cancelling the arm cavity resonance.
I have also explained how PT-symmetry in the lossless limit of stable optomechanical filtering, at a particular value of the coupling rates, leads to an Exceptional Point and enhanced sensitivity \jam{(check this)}, but that this sensitivity improvement is limited by requiring low mechanical loss.
% However, this optomechanical configuration is equivalent to the all-optical nondegenerate internal squeezing, and this equivalence might mean that nondegenerate internal squeezing can achieve the same sensitivity improvement but be more feasible due to the technological differences between optical and mechanical loss.
The low tolerance of these two proposals to realistic losses limits their feasibility for future detectors unless further technological progress is made to lower their respective losses. This motivates investigating configurations that are more resistant to loss; the configuration in my work will combine these two proposals to try and improve the tolerance to loss.


