\chapter{Existing proposals for improving kilohertz sensitivity} % Background: existing proposals for improving kilohertz sensitivity
\label{chp:proposals}
%%%%%%%%%%%%%%%%%%%%%%%%%%%%%%%%%%%%%%%%%%
% chapter introduction
% Beyond better external squeezing, there are many \jam{(are there?)} existing proposals to improve the kilohertz sensitivity of gravitational-wave detectors. In the next two chapters, I examine two of the front-runners: degenerate internal squeezing and stable optomechanical filtering.
% demonstrate modelling again, for the third and last time before my model -- much of the maths should be quite familiar by now
% set-up the problem: how do these configs get past the four factors in the introduction and why aren't they good enough for kilohertz sensitivity

In this chapter, I consider some of the existing solutions to solve the problem of increasing kilohertz sensitivity outlined in Section~\ref{}. After first summarising the literature on the matter, I focus on two recent proposals that look promising: degenerate internal squeezing~\cite{} and stable optomechanical filtering~\cite{}. I use the tools of the previous chapter to understand degenerate internal squeezing and demonstrate how I will approach the configuration in my research. Although I discuss stable optomechanical filtering using the same ideas, I do not present a separate model of it for reasons that will become clear. 
%Stable optomechanical filtering is the optomechanical analogue of my all-optical configuration which implies that there is . 
While promising, these existing proposals have problems with their tolerance to different losses: optical and mechanical loss, respectively. This motivates my research in the following chapters into a configuration which is closely related to these proposals but might be able to overcome these problems and better improve kilohertz sensitivity.


%%%%%%%%%%%%%%%%%%%%%%%%%%%%%%%%%%%%%%%%%%
\section{Literature review}

\jam{(Look up what is required of a literature review, need more citations!, all I want out of this is why I am only looking at these two configurations)}

% intro: beyond external changes (external squeezing and caves's amplifier) to interferometer and increasing circulating power
How to improve the kilohertz sensitivity of gravitational-wave detectors has been considered already in the literature~\cite{}. Any proposal to do so without increasing circulating power or degrading existing sensitivity must overcome the Mizuno Limit, outlined in Section~\ref{}, through some non-classical technique \jam{(why is cancelling phase non-classical?)}. Broadband improvement is possible through external modifications to the interferometer such as external squeezing and Caves's amplification (both of which use squeezing to overcome the Mizuno Limit) already discussed, but I am interested in techniques that directly address kilohertz sensitivity. In this review, I will summarise: (1) the existing, first \jam{(do any remain?)} and second generation gravitational-wave detectors around the world, (2) the future third generation detectors under conceptualisation and construction, (3) the existing proposals that use internal squeezing, (4) the existing proposals that use optomechanical filtering, and (5) some of the other existing proposals. %However, I will refrain from a thorough technical explanation of these configurations, which will be given later in this chapter. 

\jam{(Need to do more reading to write this section convincingly + talk to the supes, maybe use a few review articles?)}
% existing detectors
The network of existing (operational) gravitational-wave detectors consists of: Advanced~LIGO~\cite{} (Hanford~\cite{} and Livingston~\cite{} sites), Advanced~Virgo~\cite{}, KAGRA~\cite{}, and GEO600~\cite{}. All of these follow the interferometer design discussed in Section~\ref{} with the only major change relevant for quantum noise \jam{(big claim, check this)} being the use of external squeezing in Advanced~LIGO, discussed in Section~\ref{}. I will consider Advanced~LIGO as the example from this generation, henceforth. 
\jam{(Need to say more for a literature review?)}
% 3rd generation detectors: LIGO Voyager, NEMO, Einstein Telscope, Cosmic Explorer --> why I will focus on the Voyager parameter set (no increased circulating power, no change in arm lengths)
The proposed next, third generation of \jam{(ground-based?)} detectors consists of: LIGO~Voyager~\cite{}, NEMO~\cite{}, Einstein~Telescope~\cite{}, and Cosmic~Explorer~\cite{} \jam{(check this, Voyager vs CE is 2.5 versus 3, no?)}. These still use the same interferometer design (with the exception of NEMO~\cite{}) \jam{(check this)} but differ from the previous generation in a number of interferometer parameters including circulating power, arm length, test mass mass, carrier wavelength, and transmissivities of the input test mass and the signal-recycling mirror~\cite{}. 
The later detectors, Einstein~Telescope and Cosmic~Explorer, are proposed to have significantly longer arm lengths and higher circulating powers than LIGO~Voyager or the current generation of detectors. 
I will consider the feasibility of configurations using the LIGO~Voyager parameter set~\cite{} because it will be the first of these future detectors to come online~\cite{} -- as it is a series of upgrades to the existing Advanced~LIGO detectors \jam{(surely both of them?)}. Moreover, variations of this parameter set have been used in the literature to assess the configurations in this chapter~\cite{Li2020,Miao2018,Adya2020}. In particular I will use: $3$~MW circulating power, $4$~km arms, $56$~m signal-recycling cavity, $200$~kg test masses, and $2~\mu\text{m}$ carrier wavelength~\footnote{There is much debate about using $2$ versus $1.064~\mu\text{m}$ which is summarised in Ref.~\cite{} and not worth explaining here as my models are too simplified to compare the two, e.g.\ with respect to frequency-dependent photodiode efficiencies~\cite{}. \jam{(is there more I can say here?)}}, but will vary the transmissivities away from their values of $0.002$ for the input test mass and $0.046$ for the signal-recycling mirror to test the effects of changing the respective coupling rates~\footnote{There is a subtlety here in that these parameters might be biased against one configuration over another. I have tried to overcome this by changing the coupling rates between the modes which appears to produce the most dramatic change to the sensitivity curve. \jam{(see Fig...?)}}. 
\jam{(Need to say more for a literature review?)}

% separate from these defined detectors is a literature of different possible configurations and proposals for these and future detectors.
The future detectors mentioned above are all those with established science-cases and detailed proposals, separate from them is a literature full of alternative proposals and exploratory work which could be realised in a non-specific future detector -- which is where my work fits into. %~\footnote{It is worth clarifying here that I am not aiming to produce a complete proposal for a future detector either, only explore the space of designs}.
% degenerate internal squeezing
One such proposal is degenerate internal squeezing~\cite{Korobko2019,Adya2020}, which is being considered for NEMO~\cite{}. 
Degenerate internal squeezing was first proposed in Ref.~\cite{Korobko2019} and has been subsequently characterised in Refs.~\cite{Adya2020,Korobko?,}. 
I will explain this configuration thoroughly later, but it suffices to say that internal squeezing places a squeezer inside the signal-recycling cavity, turning the two-cavity system into an OPO coupled to the arm cavity. Unlike external squeezing, where only the readout port vacuum is squeezed, degenerate internal squeezing can also squeeze the intra-cavity noises and the gravitational-wave signal. Although de-amplifying (squeezing) the signal is not ideal, the reduction in the quantum noise is significant enough to produce an improvement in sensitivity (signal-to-noise ratio) -- at kilohertz if the interferometer parameters are chosen correctly~\cite{}. Degenerate internal squeezing is sensitive to intra-cavity and detection losses, however, since these losses reduce the signal but amplify the noise. 
Interest in degenerate internal squeezing continues today~\cite{KorobkoTalk} for its ability to be switched into an anti-squeezing regime in the high loss limit~\footnote{This limit is far beyond the expected losses in future detectors, but the interest in it is to show the ability to optimise degenerate internal squeezing to different configurations.}, similarly to a Caves's amplifier, see Section~\ref{}, except that it interacts with signal and intra-cavity losses. 
% is there any GW community interest in nondegenerate squeezing?
But nondegenerate internal squeezing, where the internal squeezing is operated nondegenerately, is an alternative that has not yet been thoroughly considered~\cite{} -- I will identify this gap in the literature in Section~\ref{} \jam{(haven't I identified it here?)}. 

% optomechanical filtering: unstable~\cite{Miao2015,Miao2018,Page2018} and stable~\cite{Li2020,Li2021}
\jam{(review Miao 2015, research journals)}
Another configuration, even more popular in the literature than internal squeezing, is stable optomechanical filtering~\cite{Li2020,Li2021,Miao2015,Miao2018,Page2018}. Optomechanical filtering in the signal-recycling cavity of an interferometer was first proposed in Ref.~\cite{Miao2015} in an unstable configuration. Again, I will explain this in more detail later, but for now \jam{is this necessary?}, this configuration couples the optical mode in the signal-recycling cavity to a mechanical mode. The optomechanical coupling can be chosen such that the filter cavity that the signal-recycling cavity becomes imparts the opposite round-trip phase to the arm cavity, which broadens the arm cavity resonance (the ``white-light cavity'' idea~\cite{}). When the readout occurs via the mechanical mode, the system is unstable and needs a feedback control system, this unstable system was further investigated in Refs.~\cite{Miao2018,Page2018,}. But if the readout is changed back to the signal-recycling cavity optical mode, then the system becomes stable, which has been investigated in Refs.~\cite{Li2020,Li2021} and produces broadband sensitivity improvement in the lossless case. Both unstable and stable systems have stringent mechanical requirements to keep the thermal noise low, however, requiring high mechanical quality factor and low thermal noise~\cite{}. Research is currently underway into different optomechanical techniques to achieve the demands of this configuration, such as optical springs~\cite{}, cat-flap resonators~\cite{}, and \jam{(... look at Miao, Page references)}.

% other configurations + configurations not based on the Michelson interferometer: speed-meters etc.
Finally, there are still more proposals for kilohertz improvement~\cite{,,}, some of which are not even based on the Michelson interferometer~\cite{}. I am not aware of all that has been suggested, but some of the configurations that continue to generate interest in the literature are \jam{(..., ..., and ... . read up to fill this in!)}. 
% this is just an honours project 
However, the time constraints of my research have only allowed me to consider the two above classes of existing proposals. These are the most related configurations to my own work, since nondegenerate internal squeezing is to degenerate internal squeezing as the nondegenerate OPO is to the degenerate OPO, and nondegenerate internal squeezing is an all-optical analogue of the stable optomechanical filtering. % I can't claim that they are frontrunners, what evidence?
Therefore, that I only consider them should not be taken as a claim of their superior feasibility to the rest of the literature but as a result of finite time and relevance. 


%%%%%%%%%%%%%%%%%%%%%%%%%%%%%%%%%%%%%%%%%%
\section{Existing proposal 1: degenerate internal squeezing}

\begin{figure}
	\centering
	% \includegraphics[width=\textwidth]{}
% squeezing ellipse and signal arrow plot (+ show the effect of optical loss: detection and intracavity)
	\caption{Degenerate internal squeezing configuration, all modes are labelled (the signal-recycling cavity resembles a degenerate OPO and uses the same notation as Fig.~\ref{fig:dOPO_config}). Showing the origin of the signal and noise and the squeezer's effect on them using the noise ellipse and signal arrow representation as in Fig.~\ref{fig:extSqz_config}.}
	\label{fig:dIS_config}
\end{figure}

% \begin{figure}
% 	\centering
% 	% \includegraphics[width=\textwidth]{}
% 	\caption{Degenerate internal squeezing noise ellipse and signal arrow, compare to external squeezing in Fig.~\ref{fig:extSqz_ellipse_arrow} and Caves's amplifier in Fig.~\ref{fig:Cavess_amplifier}.}
% 	\label{fig:dIS_ellipse_and_arrow}
% \end{figure}

% has been explained many times before, but do it again here because it is the main reference
% how does this configuration beat the Mizuno limit (four factors in the introduction)? --> squeezing
Degenerate internal squeezing consists of a degenerate squeezer placed inside the signal-recycling cavity of an interferometer such that it squeezes the signal mode at the carrier frequency~\cite{}, as shown in Fig.~\ref{fig:dIS_config}. The internal placement means that it squeezes not just the vacuum from the readout port, like external squeezing, but also the vacuum from the intra-cavity loss ports and the gravitational-wave signal. Although in a single-pass of the squeezer the signal and noise are squeezed equally, meaning that the contributions to the variance of the light after the squeezer are scaled equally, %~\footnote{Squeezing the signal is shorthand for the de-amplification of signal resulting from squeezing the noise. \jam{(clarify)}},
the overall amount of squeezing measured at the photodetector is different for signal and noise. This will be shown shortly, but it occurs because of the different origins of the signal and noise: the signal from the arms and the noise from each cavity and the readout, which mean that the signal and noise see \jam{(colloquial?)} the signal-recycling cavity and the squeezer differently, as shown in Fig.~\ref{fig:dIS_config}. 
Therefore, degenerate internal squeezing overcomes the Mizuno Limit from Section~\ref{} by using squeezing to reduce the noise more than the signal.
%The overall effect is to de-amplify the signal but squeeze the noise more such that the sensitivity improves, as shown in Fig.~\ref{fig:dIS_ellipse_and_arrow}. Where the improvement overcomes the Mizuno Limit from Section~\ref{} through the use of squeezing. %However, this improvement only occurs around the ``sloshing'' frequency~\cite{}, the coupling frequency of the two cavities, which will be shown. 

In this section, I present a model of degenerate internal squeezing and discuss its behaviour. These results exist in the literature~\cite{}, but I demonstrate them abridged here to justify the approach I use in my work and establish what is expected from internal squeezing. In particular, I demonstrate how to consider radiation pressure, the gravitational-wave signal, and stability, which I have not yet explained.
%bridge from the OPOs in the previous chapter to my work on nondegenerate internal squeezing in the next chapter.
%The structure of the results presented here will be mirrored in the next chapter which should make those later results more convincing.  % what do I intend the reader to get out of it?
\jam{(I do not just show this model/results because I spend time with them, I need to make clear what I want the audience to get out of them.)}

	% A proposed technology to further reduce shot noise is degenerate internal squeezing where a degenerate squeezer is included inside the signal-recycling cavity as shown in the left panel of Fig.~\ref{fig:coupled_cavities}~\cite{korobkoQuantumExpanderGravitationalwave2019,adyaQuantumEnhancedKHz2020}.
	% To simplify modelling this configuration, the two coupled optical modes of concern, the differential mode from the arm cavities and the mode in the signal-recycling cavity, are approximated as coming from a pair of coupled cavities as shown in the right panel of Fig.~\ref{fig:coupled_cavities}. The quantum noise enters with the vacuum into the signal-recycling cavity, but the signal arises in the arm cavity and so sees the squeezer differently. This difference leads to no change in sensitivity at low frequencies and improved sensitivity at high frequencies around the resonance of the signal-recycling cavity as shown in Fig.~\ref{fig:sensitivity_curve}. The improvement occurs around the signal-recycling cavity resonance as this is where the most quantum noise field is present inside the cavity and so is where the internal squeezer produces the most squeezing. The resulting noise reduction is sufficient to overcome the de-amplification of the signal due to squeezing. The overall sensitivity improvement from degenerate internal squeezing is limited by optical loss in the interferometer and at the photodetector~\cite{korobkoQuantumExpanderGravitationalwave2019}. 


\subsection{Analytic model}

\jam{(is this model necessary?)}

% I have verified this model against Korobko
I construct a Hamiltonian model of degenerate internal squeezing using the formalism from Chapter~\ref{chp:background_theory}. This model is based on and verified against Ref.~\cite{Korobko2019}. %I include it here to demonstrate the Hamiltonian method that I later use in my work, particularly the radiation pressure, gravitational-wave signal, and the system's stability \jam{(repetition with section intro)}. 
Since degenerate internal squeezing is a degenerate OPO coupled to another cavity, shown in Fig.~\ref{fig:dIS_config}, I use the degenerate OPO model in Section~\ref{sec:dOPO_model} and add the extra modes associated with the arms. Let the signal-recycling cavity and output modes be labelled as in Section~\ref{sec:dOPO_model} and let the arm cavity mode at carrier frequency $\omega_0$ be $\hat a$ (resonant in the single-mode approximation) with intra-cavity loss $T_{l,a}$ into vacuum $\hat n^L_a$. Let the gravitational-wave signal $h(t)$ be coupled to the arm cavity mode by the test mass mechanical mode given by displacement $\hat x$ and momentum $\hat p$ (approximated as free-falling horizontally). 
The Hamiltonian of this system is $\hat H = \hat H_0 + \hat H_I + \hat H_\gamma + \hat H_\text{GW}$ where~\cite{}
\jam{(fill in Langevin Hamiltonian)}
\begin{align}
\hat H_0 &= \hbar \omega_0 \hat a^\dag \hat a + \hbar \omega_0 \hat b^\dag \hat b + \hbar 2\omega_0 \hat u^\dag \hat u\\
\hat H_I &= i\hbar\omega_s(\hat a\hat b^\dag-\hat a^\dag\hat b) +\hbar \frac{x}{2} (e^{i\phi} \hat u (\hat b^\dag)^2 + \text{h.c.})\\
\hat H_\gamma &= \int \ldots \\
\hat H_\text{GW} &= -\alpha (\hat{x}-L_\mathrm{arm}h(t))(\frac{\hat{a}+\hat{a}^\dag}{\sqrt{2}})+\frac{1}{2\mu}\hat{p}^2 \\
\end{align}
% \beta = \frac{\alpha L_\mathrm{arm}}{\sqrt{2}\hbar}, \beta = \sqrt{\frac{2 P_\text{circ} L_\text{arm} \omega_0}{\hbar c}}, \alpha = \sqrt{2} \hbar/L_arm= \sqrt{\frac{4 P_\text{circ} \omega_0 \hbar}{c  L_\text{arm}}}
% multiple different hamiltonians for RP
% problem with omega_s formula, only valid below one FSR of the arms
Where $\alpha=\sqrt{\frac{4 P_\text{circ} \omega_0 \hbar}{c  L_\text{arm}}}$ \jam{(check this)} is the coupling strength to the gravitational-wave signal~\cite{}, which improves with increased circulating power $P_\text{circ}$, $\mu=M/4$ is the reduced mass of the test mass with mass $M$~\cite{}, and $\omega_s\approx c\sqrt{\frac{T_\text{ITM}}{4 L_\text{arm} L_\text{SRM}}}$ is an approximation to the sloshing frequency between the coupled cavities (also known as the coupled cavity pole) which holds when it is below one FSR of the arm cavities \jam{(check this)}~\cite{}. I use $\hat H_\text{GW}$ from Ref.~\cite{Li2020,original source?} which \jam{(clarify)} couples both the mirror position and gravitational wave displacement $L_\text{arm} h(t)$ to the amplitude quadrature of the cavity mode~\footnote{Although a more natural formulation would couple the gravitational-wave strain to the mirror position and the mirror position to the cavity mode, this is equivalent~\cite{}.}. 
The Heisenberg-Langevin equations-of-motion for $\hat a, \hat b, \hat x$ and $\hat p$ can be found using the bosonic commutation relations~\cite{}, the canonical commutation relation $[\hat x,\hat p]=i\hbar$~\cite{}, and with all other commutators zero. I make the same set of approximations to the pump mode as Section~\ref{}, enter the Interaction Picture, and take fluctuating components as before. The resulting equations are
\begin{equation}\begin{cases}
\dot{\hat a}=-\omega_s \hat b- \gamma_a \hat{a} + \sqrt{2\gamma_a}\hat{n}^L_a+\frac{i \alpha}{\hbar\sqrt2}(\hat{x}-L_\mathrm{arm}h(t)) \\
\dot{\hat b}=\omega_s \hat a-i\chi e^{i\phi} \hat b^\dag - \gamma^b_\mathrm{tot} \hat{b} + \sqrt{2\gamma^b_R}\hat{B}_\mathrm{in} + \sqrt{2\gamma_b}\hat{n}^L_b\\
\dot{\hat x}=\frac{1}{\mu}\hat p\\
\dot{\hat p}=\alpha(\frac{\hat{a}+\hat{a}^\dag}{\sqrt{2}}).
\end{cases}\end{equation}
Entering the Fourier domain, I solve these equations for $\vec{\hat b}(\Omega)=[\hat b(\Omega),\hat b^\dag(-\Omega)]^\text{T}$ in terms of the vacuum sources and signal $\vec h(\Omega)=[\tilde h(\Omega),\tilde h^*(-\Omega)]^\text{T}$.
\begin{align}
\text{M}_a\vec{\hat a}(\Omega)&=-\omega_s\vec{\hat b}(\Omega) + \sqrt{2\gamma_a}\vec{\hat n}^L_a(\Omega)-i\beta\begin{bsmallmatrix}1 & 0 \\0 & -1\end{bsmallmatrix}\vec h(\Omega) \\
\text{M}_a&= (\gamma_a-i \Omega)\text{I}+\frac{i \rho}{\Omega^2 \sqrt 2}\begin{bsmallmatrix}1 & 1 \\-1 & -1\end{bsmallmatrix} \\
\therefore\text{M}_b\vec{\hat b}(\Omega)&=\sqrt{2\gamma^b_R}\vec{\hat B}_\mathrm{in}(\Omega) + \sqrt{2\gamma_b}\vec{\hat n}^L_b(\Omega)+\omega_s \text{M}_a^{-1}\left(\sqrt{2\gamma_a}\vec{\hat n}^L_a(\Omega)-i\beta\begin{bsmallmatrix}1 & 0 \\0 & -1\end{bsmallmatrix}\vec h(\Omega)\right)\\
\text{M}_b&= (\gamma^b_\mathrm{tot}-i \Omega)\text{I}+\chi \begin{bsmallmatrix}0 & i e^{i\phi}\\-i e^{-i\phi} & 0\end{bsmallmatrix}+\omega_s^2\text{M}_a^{-1}.
\end{align}
Where $\beta = \frac{\alpha L_\mathrm{arm}}{\sqrt{2}\hbar}$ and $\rho = \frac{\alpha^2}{\sqrt{2}\hbar\mu}$ are the scaled coupling constants to the gravitational-wave signal and the radiation pressure, respectively~\footnote{Where $\mu\rightarrow\infty$ turns off the radiation-pressure noise, $\rho=0$, as expected.}.
Using the same input/output relations as Section~\ref{sec:dOPO_model} and $\Gamma=\frac{1}{\sqrt2}\begin{bsmallmatrix}1 & 1 \\-i & i\end{bsmallmatrix}$, I find the quadratures at the photodetector to be
\begin{align}
\vec{\hat X}_\mathrm{PD}(\Omega)&=\text{T}\vec h(\Omega)+\text{R}_\text{in}\vec{\hat X}_\mathrm{in}(\Omega)+\text{R}^L_a\vec{\hat X}^L_a(\Omega)+\text{R}^L_b\vec{\hat X}^L_b(\Omega)+\text{R}^L_\text{PD}\vec{\hat X}^L_\text{PD}(\Omega)\\
\text{T}&=-\sqrt{1-R_\text{PD}}\omega_s(-i\beta)\Gamma \sqrt{2\gamma^b_R}\text{M}_b^{-1}\text{M}_a^{-1}\begin{bsmallmatrix}1 & 0 \\0 & -1\end{bsmallmatrix}\\
\text{R}_\text{in}&=\sqrt{1-R_\text{PD}}\Gamma\left(\text{I}-2\gamma^b_R\text{M}_b^{-1}\right)\Gamma^{-1}\\
\text{R}^L_a&=-\sqrt{1-R_\text{PD}}\omega_s\Gamma 2\sqrt{\gamma^b_R \gamma_a}\text{M}_b^{-1}\text{M}_a^{-1}\Gamma^{-1}\\
\text{R}^L_b&=-\sqrt{1-R_\text{PD}}\Gamma 2\sqrt{\gamma^b_R \gamma_b}\text{M}_b^{-1}\Gamma^{-1}\\
\text{R}^L_\text{PD}&=\sqrt{R_\text{PD}} \text{I}.
\end{align}
Where $\text{R}_\text{in},\text{R}^L_b,\text{R}^L_\text{PD}$ are the same as Eq.~\ref{eq:dOPO_PD_as_fn_of_vac} except for the different $\text{M}_b$ which accounts for the arm cavity, as expected.
The total quantum noise, which has shot noise and quantum radiation-pressure noise, is given by \begin{equation}
\text{S}_X=\text{R}_\text{in}\text{R}_\text{in}^\dag+\text{R}^L_a{\text{R}^L_a}^\dag+\text{R}^L_b{\text{R}^L_b}^\dag+\text{R}^L_\text{PD}{\text{R}^L_\text{PD}}^\dag.
\end{equation}
This matrix has similar form to Eq.~\ref{} for the degenerate OPO but with terms for the radiation pressure noise such that the variances and covariances no longer reduce to 1 and 0, respectively, when the squeezer is off. \jam{(Display the matrix in some shortened form? Maybe in an appendix?)}
The signal transfer function is with respect to $\tilde h(\Omega)$ not $\vec h(\Omega)$ but since $h(t)$ is real, $\tilde h(\Omega)=\tilde h(-\Omega)^*$ and therefore $\vec h(\Omega)=\tilde h(\Omega) \begin{bsmallmatrix}1 \\1\end{bsmallmatrix}$.
% \begin{align}
% \text{T}\begin{bsmallmatrix}1 \\1\end{bsmallmatrix}=\frac{1}{\rho \chi \cos (\phi ) (\ldots)+\Omega ^4 (\ldots)}\begin{bsmallmatrix}4 \beta ^2 \gamma_R (1-R_\text{PD}) \chi ^2 \Omega ^4 \omega_s^2 \left(\gamma_a^2+\Omega ^2\right) \cos ^2(\phi ) \\2 \beta ^2 \gamma_R (1-R_\text{PD}) \Omega ^4 \omega_s^2 \left(\left(\gamma_a^2+\Omega ^2\right) \left(2 {\gamma^b_\text{tot}}^2+\chi ^2+2 \Omega ^2\right)-4 \chi  \sin (\phi ) \left({\gamma^b_\text{tot}} \left(\gamma_a^2+\Omega ^2\right)+\gamma_a \omega_s^2\right)-\chi ^2 \left(\gamma_a^2+\Omega ^2\right) \cos (2 \phi )+4 \omega_s^2 \left(\gamma_a {\gamma^b_\text{tot}}-\Omega ^2\right)+2 \omega_s^4\right)\end{bsmallmatrix}
% \end{align}
Inspecting $\text{T}\begin{bsmallmatrix}1 \\1\end{bsmallmatrix}$ \jam{(is it worth showing this?)}, i.e.\ the vector of signal transfer functions to each quadrature, shows that there are two terms: (1) rotates between the quadratures with the pump phase and (2) stays in the second quadrature and never vanishes with the pump phase. Consider measuring the second quadrature at the photodetector since the signal is always there~\footnote{This does not mean that it is optimal to do so, since the profile of the noise with the pump phase is different to the signal, but it will suffice here~\cite{}. \jam{(what happens if I use $\phi=\pi$ and observe the first quadrature instead?)}}.
Therefore, let the sensitivity be
\begin{equation}
S_h = \frac{(\text{S}_X)_{2,2}}{\abs{(\text{T}\begin{bsmallmatrix}1 \\1\end{bsmallmatrix})_2}^2}.
\end{equation}

\subsection{Results}
% priorities: radiation pressure, gravitational-wave signal, stability
% threshold, radiation pressure, and pump phase

\jam{(Consider what order these results should come in and if they should be here at all.)}

\begin{figure}
	\centering
	% \includegraphics[width=\textwidth]{}
	\caption{Degenerate internal squeezing's quantum noise (top panel) and gravitational-wave signal (middle panel) responses and sensitivity (bottom panel). Showing the sloshing frequency $\omega_s$, threshold, and the effect of pump phase. \jam{(Which parameter set to use? Could show for Korobko and Li, matrix?)}}
	\label{fig:dIS_sensitivity}
\end{figure}

\begin{figure}
	\centering
	% \includegraphics[width=\textwidth]{}
	\caption{Degenerate internal squeezing breakdown of noise sources.}
	\label{fig:dIS_noise_budget}
\end{figure}

% comment on general performance, signal
% similarly to external squeezing, improving shot noise worsens radiation-pressure noise.
The performance of degenerate internal squeezing is shown in Fig.~\ref{fig:dIS_sensitivity}. With the squeezer off, $\chi=0$, the configuration is just the interferometer in Section~\ref{sec:coupled_cavity_approximation} and the noise, signal, and sensitivity look like Fig.~\ref{fig:simplified_sensitivity}. Turning the squeezer on appears to (1) squeeze the shot noise around the sloshing frequency $\omega_s$, (2) amplify the radiation-pressure noise, (3) de-amplify the signal around the sloshing frequency, and (4) improve the sensitivity around the sloshing frequency and worsen it at low frequencies \jam{(below ...)}. The radiation-pressure noise worsens with squeezing \jam{(does it? I think that I have misremembered the dIS response and am confusing it with nIS, check this!)} because of the trade-off with shot noise explained in Section~\ref{sec:external_squeezing}. Increasing the squeezer parameter $\chi$ further increases the effects on the noise and signal but whether this continues to improve the sensitivity depends on the losses and will be covered shortly. Compared to external squeezing, the internal squeezer indeed squeezes each of the intra-cavity losses as well, as shown in Fig.~\ref{fig:dIS_noise_budget}. 
% why improvement only around sloshing frequency
The changes to the shot noise and signal are localised to the sloshing frequency because of the resonance structure of the coupled cavity system. At the sloshing frequency, energy is strongly coupled from the arm cavity into the signal-recycling cavity and the vacuum in the signal mode is supported, these mean that the squeezer is operating efficiently and the signal and noise are strongly squeezed~\cite{}. \jam{(see Korobko, I do not understand this)} This means that degenerate internal squeezing can be added to current detector designs without affecting the sensitivity 

There is much behaviour to analyse for this configuration, which has been done in Refs.~\cite{Korobko2019,Adya2020,Korobko}, and motivates how I analyse the configuration in my work. Here, I will briefly discuss in turn: (1) threshold in the lossless case, (2) the effect of pump phase and the quantum noise in the other quadrature, (3) the optimal squeezing value, (4) the stability of the system, and (5) the system's tolerance to optical loss.

% threshold, lossless threshold but leave lossy threshold to research chapters? ``To not conflate existing knowledge with my work, I leave the lossy threshold to Section~\ref{}.'' just state result and justify later?
In the lossless case, $\gamma_a=\gamma_b=R_\text{PD}=0$, the shot noise given $\phi=\pi/2$ is 
\begin{equation}
\label{eq:dIS_lossless}
(\text{S}_X)_{2,2}=1-\frac{4 \gamma^b_R \chi \Omega ^2}{\Omega ^2 (\gamma^b_R+\chi )^2+(\Omega ^2-\omega_s^2)^2}.
\end{equation}
As shown in Fig.~\ref{fig:dIS_sensitivity}, this goes to zero at $\chi=\chi_\text{thr}=\gamma^b_R$ and $\Omega=\omega_s$ and defines threshold~\cite{}. In the anti-squeezed quadrature, $\phi=-\pi/2$ or $\chi\mapsto-\chi$ in Eq.~\ref{eq:dIS_lossless}, the quantum noise diverges at $\Omega=\omega_s$ at threshold. This behaviour is similar to the lossless degenerate OPO, except that the improvement occurs at the coupled cavity resonance instead of at DC $\Omega=0$. From the perspective of gains and losses inside the signal-recycling cavity, any light lost to the arms through $\omega_s$ must return since $\gamma_a=0$ and so the total loss remains as $\gamma^b_R$. In the lossy case, the situation is more complicated and threshold is not quoted in the literature~\cite{}. However, I will define a unified notion of threshold in my work, but to not conflate it with the existing knowledge discussed in this chapter I will leave it until Section~\ref{sec:singularity_threshold}. \jam{(explain more here, awkward end to the paragraph)}

% pump phase, foreshadow that maximum of anti-squeezing is not necessarily the minimum of squeezing
Changing the pump phase rotates between the squeezing and anti-squeezing quadratures, as shown in Fig.~\ref{fig:dIS_sensitivity}. In the anti-squeezed quadrature of the shot noise, the radiation-pressure noise is squeezed, as expected by Section~\ref{sec:external_squeezing}.
A complication when the system is lossy is that maximising the anti-squeezed quadrature is not necessarily the same as minimising the squeezed quadrature, as shown in Fig.~\ref{fig:dIS_sensitivity}, the two move apart at high arm losses. This is a result of worsening the uncertainty product in the Heisenberg Uncertainty Principle. \jam{(but explain why this is)}

\begin{figure}
	\centering
	% \includegraphics[width=\textwidth]{}
	\caption{Degenerate internal squeezing sensitivity versus quantum noise, varying the squeezer parameter up to threshold for different detection losses \jam{(show for other losses as well?)}. The optimal value of the squeezer parameter is marked. Showing squeezing and anti-squeezing.}
	\label{fig:dIS_noise_budget}
\end{figure}

% optimal squeezing curves against loss, in really high loss should antisqueeze
% difference to caves's amplifier
Although increasing the squeezer parameter continues to squeeze the quantum noise more, up to threshold, the optimal value for the sensitivity can be below threshold in the lossy case, as shown in Fig.~\ref{fig:dIS_noise_budget}. This is because increasing the squeezing might decrease the signal more than the noise past a certain point. In the high loss regime, this may occur for any positive squeezer parameter, which means that it might then be optimal to instead anti-squeeze~\cite{Korobko talk} internally. But unlike using a Caves's amplifier externally, degenerate internal anti-squeezing is only optimal in extreme loss regimes that do not resemble future detectors~\cite{}. 

\subsubsection{Stability}
% establish how stability is analysed

\begin{figure}
	\centering
	% \includegraphics[width=\textwidth]{}
	\caption{Degenerate internal squeezing sensitivity, showing imaginary part of the poles of the transfer functions against squeezer parameter, for lossless and lossy cases. Positive imaginary parts indicate instability.}
	\label{fig:dIS_stability}
\end{figure}

Configurations must be stable to be feasible. I determine the stability of configurations in this thesis by studying the poles of the transfer functions. Since the noise and signal transfer functions are rational functions that share a denominator \jam{(cite or see an appendix?)}, this is a matter of studying the zeros of the shared denominator. In the complex $\Omega$ plane, if these poles have positive imaginary part, then the system is unstable, i.e.\ all poles must be in the lower-half plane or on the real axis for the system to be stable~\cite{}~\footnote{This is often taught with respect to the Laplace Transform variable $s=i\Omega$, where positive real part of $s$ indicates instability~\cite{}.}. This is equivalent to other conditions such as \jam{(... the Nyquist criterion?)}. Degenerate internal squeezing is stable in the lossless case for all squeezer parameters below threshold, as shown in Fig.~\ref{fig:dIS_stability}. And in the lossy case, \jam{(..., is the same true?, confirm this, normalise squeezer parameter to threshold)}.


\subsection{Problems with proposal - vulnerability to optical loss}

\begin{figure}
	\centering
	% \includegraphics[width=\textwidth]{}
	\caption{Degenerate internal squeezing quantum noise and gravitational-wave signal responses (transfer functions) and sensitivity. Showing the effect of intra-cavity and detection losses. \jam{(Which parameter set to use? Could show for Korobko and Li, matrix?)}}
	\label{fig:dIS_loss_tolerance}
\end{figure}

% intracavity loss behaves differently to dOPO 
Degenerate internal squeezing has different tolerance to the different sources of optical loss, as shown in Fig.~\ref{fig:dIS_loss_tolerance}, where now the signal's tolerance must also be considered. Firstly, detection losses behave similarly to the degenerate OPO, the noise is uniformly pulled towards the vacuum and the signal is pulled towards zero. Secondly, signal-recycling intra-cavity loss, however, behaves differently to the degenerate OPO, as the noise response remains within the lossless envelope, increasing radiation-pressure noise, and worsening signal and noise around the sloshing frequency \jam{(why doesn't the response broaden like before?)}. Finally, arm intra-cavity loss, which has no analogue in the OPO, decreases the squeezing peak of the noise and moves the frequency away from the sloshing frequency, but improves the radiation-pressure noise. It also remains within the lossless signal envelope but worsens the DC response to the signal.

\begin{figure}
	\centering
	% \includegraphics[width=\textwidth]{}
	\caption{Degenerate internal squeezing shot noise response in the limit of large arm loss compared to the theoretical limiting degenerate OPO with fully reflective input test mass.}
	\label{fig:}
\end{figure}

% \subsubsection{Reduction to degenerate OPO}
% arm cavity loss gives reduction to dOPO
% The behaviour against different parameters will be similar to that seen for the degenerate OPO in Section~\ref{}.
The strange behaviour against arm intra-cavity loss can be understood as the degenerate OPO limit of degenerate internal squeezing, i.e.\ when the arm cavity is removed. Formally taking the limit $\gamma_a\rightarrow\infty$ of the shot noise shows that it reduces to a degenerate OPO with the input test mass fully-reflective, distinct from removing the end test mass. \jam{(show result or cite, also, check if equivalent to sloshing frequency to zero)} This is surprising, as one might expect the input test mass to instead become another loss port, but can be explained as the disappearance of the arm cavity mode altogether, vacuum or otherwise \jam{(check this)}. Inspecting the equation-of-motion for $\hat a$, in the limit the equation becomes $\dot{\hat a}\approx -\gamma_a \hat a$ which quickly decays, and therefore any vacuum $\hat n^L_a$ can not couple through $\hat a$ to $\hat b$ because the intermediate mode has vanished.
\jam{(However, I believe that this is a consequence of the single-mode approximation and that if a transfer matrix approach~\cite{Finesse,} was instead used where the fields are propagated inside the cavities and the cavity modes are not explicit, the limit would be a degenerate OPO with added intra-cavity loss to account for the open input test mass. This should be easy enough to test.)}
% This is confusing, I expected the initial test mass to become a loss port, but this can be understood as there no longer being an arm cavity mode, vacuum or otherwise, since it can not make a round-trip.


\begin{figure}
	\centering
	% \includegraphics[width=\textwidth]{}
	\caption{Degenerate internal squeezing sensitivity for realistic losses.}
	\label{fig:dIS_realistic_loss}
\end{figure}

Compared to the dramatic sensitivity improvement in the lossless case, if realistic losses are assumed, then the improvement significantly degrades, as shown in Fig.~\ref{fig:dIS_realistic_loss} where a high percentage of threshold is assumed. What realistic losses are for future detectors is hard to determine given the unknown progress of future technology, but it appears that, conservatively, arm intra-cavity loss below $T_{l,a}=100$~ppm (parts-per-million), signal-recycling intra-cavity loss below $T_{l,b}=1000$~ppm, and detection loss below $R_\text{PD}=10\%$ are reasonable to predict~\cite{Zhang2021,}. Therefore, detection loss will likely be the dominant source of loss in future detectors, and is responsible for most of the degradation seen in Fig.~\ref{fig:dIS_realistic_loss}. 
% \subsection{Connection to nondegenerate internal squeezing}
% at high enough losses, it becomes optimal to instead anti-squeeze internally, this means that you might as well use nondegenerate internal squeezing because then you can anti-squeeze and potentially exploit the correlations using a combined readout

% conclusions about dIS?
Although degenerate internal squeezing improves the sensitivity, its low tolerance to optical loss motivates investigating anti-squeezing methods which might improve sensitivity more, such as nondegenerate internal squeezing. \jam{(What is the optimal squeezing value given realistic losses? I need to rule out degenerate anti-squeezing.)} This does not mean that degenerate internal squeezing is not useful, it is worth further investigation, especially in low loss applications~\cite{}, but I will consider the nondegenerate case and whether it fares better.


	% Nondegenerate internal squeezing, where the internal squeezer is instead nondegenerate, has
	% been proposed as an alternative to degenerate internal squeezing~\cite{yapadyaPersonalCommunication}, although a comprehensive analysis of nondegenerate internal squeezing is yet to be done~\cite{liBroadbandSensitivityImprovement2020}. Because nondegenerate squeezing results in two entangled photons with different frequencies, these photons will not interfere with each other in the same manner as the degenerate case. Without this interference, nondegenerate internal squeezing increases the signal and the noise instead of decreasing them like the degenerate case. Therefore, nondegenerate internal squeezing is predicted to be more resistant to photodetector loss since the signal amplitude is greater. This project aims to investigate the potential benefits of nondegenerate internal squeezing over degenerate internal squeezing.



%%%%%%%%%%%%%%%%%%%%%%%%%%%%%%%%%%%%%%%%%%
\section{Existing proposal 2: stable optomechanical filtering}

\begin{figure}
	\centering
	% \includegraphics[width=\textwidth]{}
	\caption{Stable optomechanical filtering (white-light cavity) configuration.}
	\label{fig:}
\end{figure}

% how does this configuration beat the Mizuno limit (four factors in the introduction)? --> cancels arm cavity resonance

% I ommit the model of this configuration here, because in the lossless case it is exactly the model in the next chapter, see Section~\ref{}.
% but I need to talk about their results? in particular, the exceptional point of PT-symmetric, stability, and sensitivity at threshold. maybe do show a lossless, shot-noise only model -- better yet, cite the results in their paper and reference the next section for further explanation?

% Li et al, 2020

	% A recent proposal uses a stable optomechanical filter cavity to avoid this limit and increase high-frequency sensitivity without fully sacrificing low-frequency sensitivity nor increasing the power~\cite{liBroadbandSensitivityImprovement2020}. However, it requires cryogenic (around 4~K) environmental temperature and a higher mechanical quality factor than is currently possible. An all-optical alternative to this optomechanical proposal without these requirements is desirable.
	% Stable optomechanical filtering consists of an auxiliary filter cavity inside the signal-recycling cavity. One of the filter cavity's mirrors is a mechanical oscillator, such as a suspended mirror, driven by a laser whose frequency is appropriate to excite the mechanical mode~\cite{liBroadbandSensitivityImprovement2020}. This design is dynamically stable unlike previous designs for optomechanical filter cavities~\cite{miaoEnhancingBandwidthGravitationalWave2015,pageEnhancedDetectionHigh2018,miaoDesignGravitationalWaveDetectors2018}. This system has a parity-time symmetry between the differential optical mode of the interferometer and the mechanical mode; replicating this symmetry with an internal squeezer requires the squeezer to be nondegenerate to mimic the distinction between the filter cavity optical mode and the mechanical mode~\cite{liBroadbandSensitivityImprovement2020}. Stable optomechanical filtering improves high-frequency sensitivity by cancelling the effect of the resonance behaviour of the interferometer cavities. It is designed for implementation in later-generation detectors and assumes technological improvements in the near future in arm length, power, optical loss, Brownian noise, and, most stringently, in the thermal noise and quality factor of the mechanical oscillator. These assumptions are necessary to achieve the target sensitivity for astrophysical applications~\cite{miaoDesignGravitationalWaveDetectors2018}. By investigating nondegenerate internal squeezing, I aim to find more realistic requirements for a future detector.



\subsection{Problems with proposal - vulnerability to mechanical loss}
% thermal noise and mechanical quality factor

\begin{figure}
	\centering
	% \includegraphics[width=\textwidth]{}
	\caption{Stable optomechanical filtering sensitivity showing the effect of thermal noise, the dominant noise source. This data for this figure was taken from Ref.~\cite{} with permission from the authors.}
	\label{fig:}
\end{figure}


\subsubsection{Technological limit: thermal noise}


\subsection{Connection to nondegenerate internal squeezing}

\begin{figure}
	\centering
	% \includegraphics[width=\textwidth]{}
	\caption{Mode diagrams of all configurations considered in this thesis: OPOs, degenerate internal squeezing, stable optomechanical white-light cavity, and nondegenerate internal squeezing. Notice that the latter two configurations are modally the same but use different components, optomechanical and all-optical respectively, which means that their performance might be different.}
	\label{fig:}
\end{figure}

% equivalent mode structures, just the different noise sources

	% Nondegenerate internal squeezing is also motivated by a connection between it and the use of
	% a stable optomechanical filter cavity to improve high-frequency sensitivity~\cite{yapadyaPersonalCommunication,liBroadbandSensitivityImprovement2020}. The Hamiltonians of the two systems are equivalent under some mapping of the creation and annihilation operators of certain optical fields to certain mechanical fields. This connection exploits the fact that the nondegenerate squeezer interacts with three distinct frequencies to introduce a symmetry into the all-optical system that the optomechanical system has. This means that using nondegenerate internal squeezing may achieve the benefits of a stable optomechanical filter cavity without the optomechanical drawbacks of requiring cryogenic (around 4~K~\cite{miaoEnhancingBandwidthGravitationalWave2015}) environmental temperature and high mechanical quality factor. Therefore, understanding stable optomechanical filtering should lead to a better understanding of what nondegenerate internal squeezing might achieve.

%%%%%%%%%%%%%%%%%%%%%%%%%%%%%%%%%%%%%%%%%%
\section{Chapter summary}

% there are solutions in the literature to improving kilohertz sensitivity, but they have their problems


