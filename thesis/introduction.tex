\chapter{Introduction} %:gravitational waves and their detection

\section{Gravitational waves}


\subsection{Kilohertz gravitational-wave astrophysics} %i.e. sources

% motivate HF limit

\subsubsection{Astrophysical target}

% this will be the goal of my optimisation


%%%%%%%%%%%%%%%%%%%%%%%%%%%%%%%%%%%%%%%%%%
\section{Interferometric gravitational-wave detectors}

% explain what a GWD is and why there are optical cavities

% Michelson signal response function

\subsubsection{The role of optical cavities}


\subsubsection{Simplification to coupled cavities}

% mizuno limit (due to QCRB) on the integrated sensitivity, i.e. on the bandwidth-peak sensitivity product --> can be beaten by squeezing

\subsubsection{Technological limit: circulating power}
% Why can circulating power not increase?
% constraints for future detectors
% I assume that we can't increase power by a factor of 2, unlike Miao et al 2018

\subsection{Factors limiting kilohertz sensitivity}
% explain why current GWD are not sensitive at kilohertz

% cavities
% shot noise

\subsubsection{Other noise sources at kilohertz} % and why I will ignore them

%%%%%%%%%%%%%%%%%%%%%%%%%%%%%%%%%%%%%%%%%%
\section{Thesis outline}


%%%%%%%%%%%%%%%%%%%%%%%%%%%%%%%%%%%%%%%%%%
\section{Chapter summary}

