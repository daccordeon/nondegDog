\chapter{Introduction} %Background: gravitational waves and their detection
\label{chp:introduction}

\pagenumbering{arabic} 

% introduction: motivation, overview of problem, and what thesis contributes
% written to a general physics audience
% can merge with backgroup chapter/s if appropriate (background chapters are to detail the problem sufficiently and specifically)

%%%%%%%%%%%%%%%%%%%%%%%%%%%%%%%%%%%%%%%%%%
% chapter introduction

	% The detection of gravitational waves from the late inspiral of binary black-hole~\cite{GW150914} and neutron-star~\cite{GW170817} mergers is a landmark achievement of 21st-century physics and engineering that opened up a new field of astronomy dedicated to using gravitational waves to better understand the universe. % might be better in the Abstract?

In this chapter, I outline the problem that this thesis addresses. In Section~\ref{sec:gravWaves}, I explain and motivate the detection of kilohertz gravitational waves. Then, in Section~\ref{sec:intro_IFO}, I describe how gravitational-wave detectors work, explain what currently prevents the detection of kilohertz waves, and review the literature of possible solutions. Finally, in Section~\ref{sec:thesis_outline}, I outline what this thesis contributes towards detecting them in the future.


\section{Gravitational waves}
\label{sec:gravWaves}

% \begin{figure}
% 	\centering
% 	% \includegraphics[width=\textwidth]{}
% 	\caption{Gravitational wave incident on a ring of test particles, evolution over time.}
% 	\label{fig:GW_ring_of_test_particles}
% \end{figure}
\begin{figure}
	\centering
	\includegraphics[angle=-90,width=0.9\textwidth]{GW_hitting_MICH.pdf}
	\caption{The exaggerated effect of a gravitational wave incident into the page upon a Michelson interferometer. Over time, the gravitational wave complementarily stretches and squashes the perpendicular arms which changes the interference pattern at the photodetector.}
	\label{fig:GW_incident_Michelson}
\end{figure}

Gravitational waves are propagating perturbations in spacetime described by the Einstein field equations of General Relativity~\cite{cai_2017,Maggiore:2007}. %These perturbations change the proper distance and time between events in spacetime, e.g.\ the proper separation between two massive objects. Since they can move massive objects around, therefore gravitational waves carry energy~\cite{}. %\jam{(``Move'' is problematic, look at choice of frame)}
% Just as they are able to move objects, 
These waves are predicted to be emitted by the acceleration of massive objects under certain asymmetry conditions. % described by the quadrupole moment of the source~\cite{}. %, and since they carry energy the objects that emit them must lose energy by doing so. 
% tense?
Gravitational-wave signals of around 100~Hz have been detected from the late inspiral of compact binary systems, i.e.\ the last moments before the merger of, for example, two black holes or neutron stars~\cite{GWTC-1:2018}.
% -- meaning that two massive, compact objects, such as two black holes~\cite{}, two neutron stars~\cite{}, or a black hole and a neutron star~\cite{}, were closely orbiting each other. Then, as the binary system lost energy by emitting gravitational waves, the objects got closer together, the orbital frequency increased, and correspondingly the frequency of the emitted gravitational waves increased. Eventually, the objects collided and merged, leaving behind some other object, possibly a black hole or neutron star~\cite{}. The late inspiral of such systems refers to the last moments before the objects collide and merge and therefore to the highest frequency gravitational waves emitted pre-coalescence -- measured at around $100$~Hz~\cite{}. % check this?
% Despite being emitted by some of the most massive objects in the universe, the effects of gravitational waves are extremely small -- by the time they reach Earth, even the loudest gravitational waves will only move two objects separated by a kilometre by less than a thousandth the width of a proton~\cite{}. %wordchoice: loudest
% This poses a great challenge to detection.
% radiative, transverse-traceless solutions to
% Gravitational waves are described by the Einstein field equations of General Relativity~\cite{}. % will need to mention transverse-traceless, quadrupole moment?
% The initial amplitude of these waves is described by the quadrupole moment of the source -- a quantity that depends on its distribution of mass which explains why only the most massive astrophysical sources can be detected. % sources closer to the detector can be less massive
% ...
These waves alternately stretch and squash spacetime along the two axes perpendicular to the direction of propagation as shown in Fig.~\ref{fig:GW_incident_Michelson}. %That is, a gravitational wave incident normally on a ring of test particles will, over time, stretch and squash one semi-axis of the ring while conversely squashing and stretching the other semi-axis -- as shown in Fig.~\ref{fig:GW_ring_of_test_particles}\jam{(merged Figs, fix)}. %\jam{(Clarify expansion of spacetime not change in proper distances, i.e.\ nothing is moving with respect to spacetime)}
% By the time the gravitational wave reaches the ring (or detector), however, its amplitude has fallen by the inverse of the distance from the source~\cite{} which is why astrophysical sources with large quadrupole moments only produce small length changes at Earth. % 8.34 in Karl's notes
% The unitless gravitational-wave strain at the detector, $h(t)$, gives the change in length $\Delta L$ of a detector's initial length $L$ as $\Delta L(t) = L \;h(t)$. Henceforth, $h(t)$ is referred to as the gravitational-wave signal, or just: the signal.
The gravitational-wave strain $h(t)=\frac{\Delta L}{L}$ gives the fractional change in length of an axis of length $L$ over time $t$~\cite{cai_2017}. The goal of gravitational wave detection is to measure $h(t)$ accurately and precisely to extract the encoded astrophysical information about the source.

	% The orbital period of binary black-hole or neutron-star systems decays as the system radiates away energy in the form of low-amplitude waves in spacetime; these gravitational waves, correspondingly, increase in frequency over time. Current interferometric detectors are most sensitive to gravitational waves in a low-frequency window around 100~Hz that are emitted in the late inspiral of such mergers~\cite{AdvancedLIGO:2015}.

\subsection{Kilohertz gravitational-wave astrophysics} %i.e. sources
\label{sec:kilohertz_GW}

% motivate HF limit
To date, gravitational waves have only been detected around 100~Hz~\cite{GWTC-2:2020}, but there is believed to be varied and interesting astrophysics encoded in kilohertz gravitational waves.
For example, gravitational waves from $\sim1$--4~kHz are predicted to be emitted during the coalescence and from the remnant object of binary neutron-star mergers~\cite{PhysRevD.100.104029}.
% Due to the violent nature of the merger,
These signals are predicted to contain information otherwise unavailable about the exotic states of matter inside neutron stars that could better constrain the possible equations-of-state and improve our understanding of matter under extreme conditions~\cite{PhysRevD.100.104029,miaoDesignGravitationalWaveDetectors2018}. %this is a bold claim, need another reference? % due to the violent nature, really?
Other potential astrophysical science includes determining the origin of low-mass black holes by detecting binary black hole-neutron star mergers~\cite{PhysRevD.79.044030}, insights on the post-bounce dynamics of core-collapse supernovae~\cite{Ott_2009}, improving measurements of the Hubble constant independently of electromagnetic observations~\cite{PhysRevX.4.041004}, and searching the stochastic gravitational-wave background for primordial sources~\cite{miaoDesignGravitationalWaveDetectors2018}.
This possible wealth of new astrophysics motivates developing the ability to detect kilohertz gravitational waves, and I will focus on 1--4~kHz and the case example of binary neutron-star mergers.

% Non-astrophysical sources of gravitational waves?

	% However, the frequency of the emitted gravitational waves then increases beyond this window and so the merger can no longer be observed. In particular, predicted high-frequency gravitational waves around 1--4~kHz from the merger and post-merger remnant of binary neutron-star mergers are yet to be detected. These signals are predicted to contain valuable information that could be used to better constrain the possible equations-of-state of the exotic states of matter within neutron stars~\cite{miaoDesignGravitationalWaveDetectors2018}.
	% Potential astrophysical applications of high-frequency (kilohertz) gravitational-wave detection also include determining the origin of low-mass black holes by detecting binary black hole-neutron star mergers, improving measurements of the Hubble constant independently of electromagnetic observations, and searching the stochastic gravitational-wave background for exotic or primordial sources~\cite{miaoDesignGravitationalWaveDetectors2018}. 

% \subsubsection{Sensitivity target for enabling new astrophysics}
% \label{sec:GW_kilohertz_target}

% % this awkward here, should it be mentioned after Sh is explained? or in the science case?
%\jam{(is this necessary here?)}
% % this will be the goal of my optimisation
% To quantise this goal, consider the case example of the post-merger signal from a binary neutron-star merger. The estimated sensitivity required to reliably detect this signal from a typical source~\footnote{Using current understanding which is conditioned on the very equations-of-state that are to be constrained by these measurements.} is $5\times10^{-25} \mathrm{Hz}^{-1/2}$ from 1--4~kHz~\cite{miaoDesignGravitationalWaveDetectors2018}. This value is for the amplitude spectral density of the noise-to-signal ratio of a detector which will be explained in Chapter~\ref{chp:background_theory}. Other kilohertz astrophysical sources are predicted to require similar or greater sensitivity~\cite{}. For now, all that matters is that I have a target sensitivity for determining whether a given detector will be useful to kilohertz gravitational-wave detection.


%%%%%%%%%%%%%%%%%%%%%%%%%%%%%%%%%%%%%%%%%%
\section{Interferometric gravitational-wave detectors}
\label{sec:intro_IFO}

% explain what a GWD is 
Current gravitational-wave detectors are based on the \emph{Michelson interferometer} as shown in Fig.~\ref{fig:GW_incident_Michelson} where a laser beam is split down two perpendicular arms before returning and interfering at the beamsplitter to produce an interference pattern at the output~\cite{AdvancedLIGO:2015}. 
% As shown in Fig.~\ref{fig:GW_incident_Michelson},
An incident gravitational wave changes the path length difference between the kilometre-long arms by less than a thousandth the width of a proton~\cite{GW150914} which poses a great challenge for detection. %~\cite{}~\footnote{Corresponding to $\sim10^{-18}$~m which is far smaller than the wavelength of the carrier laser $\lambda_0\sim10^{-6}$~m and therefore it is not a concern that the interferometer is sensitive to changes only modulo the wavelength of the carrier.}.
% The resulting change in the interference pattern is measured as fluctuations in the arms' differential optical mode~\cite{}. %, is detected by a photodetector.
 % and constructively interferes at the input of the beamsplitter. % (this ``common'' mode then travels back towards the laser source).
% A more complete explanation would explain that\jam{(fix wording)} the gravitational wave can be considered to only move the end mirrors, whose motion phase modulates the light in each arm, which appears as amplitude modulation after the beamsplitter and can be detected, e.g.\ by a photodiode, but the simple explanation above will suffice until a formal model is introduced in Chapter~\ref{chp:proposals}.  
	%Interferometric gravitational-wave detectors are based on the Michelson interferometer as shown in the left panel of Fig.~\ref{fig:coupled_cavities}. A gravitational wave perturbs the path difference between the two arms of the interferometer by stretching and squashing spacetime and the resulting change in the interference of the light at the beamsplitter produces a signal on a photodetector. This change in the interference is known as fluctuations in the differential mode of the interferometer.
% (recall: the effect of a gravitational wave on a ring of test particles in Sect.~\ref{sec:gravWaves})
% A problem for interferometric detectors is that the perpendicular axes of deformation are determined by the polarisation of the gravitational wave~\cite{} and so, if the wave is not aligned to the arms, then the response will be less than usual or might vanish. Throughout this thesis, I will ignore this subtlety and assume the maximum detector response: from a gravitational wave exactly aligned to the arms, as this problem can be solved by a global network of detectors with different alignments~\cite{}. % -- like we currently have.
% Michelson signal response function
% unrelated but Another problem for interferometers is that they are only sensitive to phase differences between the arms modulo $2\pi$. %unrelated
% Another problem for Michelson interferometers is that the phase accumulated by the light can cancel between going from the beamsplitter to the end mirror and returning from the end mirror back to the beamsplitter.
% Therefore the signal transfer function of a Michelson interferometer, $T(\Omega)$, where the spectrum of the measurement of some quantity $\hat{X}(t)$ at the photodetector is $\tilde{\hat{X}}(\Omega) = T(\Omega) \tilde{h}(\Omega) + \text{noise terms}$ , will vanish for periods $2\pi/\Omega$ that are a multiple of the round-trip time of the light in the arms $2L/c$ where $L$ is the length of the arm and $c$ is the speed of light~\cite{}. %($\Omega$ is the angular frequency offset from the carrier frequency, explained further in Chapter~\ref{}) 
% I will ignore this complication because the corresponding frequencies $f=\Omega/(2\pi)=c/(2L)\approx38\mathrm{kHz}$ for the arm length $L=4\mathrm{km}$ of LIGO~\cite{} are far above the kilohertz frequencies of interest (e.g.\ 1--4~kHz for a neutron-star post-merger signal).

% \subsubsection{The role of optical cavities}
% why are there optical cavities

\begin{figure}
	\centering
	\includegraphics[angle=-90,width=0.9\textwidth]{DRFPMI_config.pdf}
	\caption{Dual-recycled Fabry-Perot Michelson interferometer configuration, where cavities are introduced to enhance the sensitivity of the Michelson interferometer in Fig.~\ref{fig:GW_incident_Michelson}. Technically, the recycling cavities should include their respective recycling mirror and the whole interferometer~\cite{meersRecyclingLaserinterferometricGravitationalwave1988,1995AuJPh..48..953M}. Injected, degenerate external squeezing via a Faraday isolator is shown for later reference; the orientation of the squeezing ellipse is only illustrative.}
	\label{fig:DRFPMI}
\end{figure}

\emph{Optical cavities} are introduced to improve the sensitivity of the detector~\footnote{Here, I mean the whole gravitational-wave detector not the photodetector henceforth.} as shown in Fig.~\ref{fig:DRFPMI}. % Three types of cavities are introduced: (1) arm cavities, (2) a power-recycling cavity, and (3) a signal-recycling cavity. 
Firstly, arm cavities are introduced that increase the circulating power in the arms to increase the signal response to passing gravitational waves~\cite{}.
Secondly, a power-recycling cavity is introduced between a power-recycling mirror and the Michelson to resonantly increase the power input into the interferometer~\cite{meersRecyclingLaserinterferometricGravitationalwave1988,}.
%\jam{(recycling mirror plus the locking condition of the Michelson which must include the whole interferometer including the beamsplitter in completeness -- footnote and clarify SRC in caption of 1.2)}
Finally, a signal-recycling cavity is introduced via a signal-recycling mirror that changes the overall resonance behaviour and signal response of the detector and can be tuned to achieve broadband or narrow-band enhancement~\cite{meersRecyclingLaserinterferometricGravitationalwave1988,1995AuJPh..48..953M}.
% enhances the gravitational-wave signal response by reflecting the output light back into the arms to be exposed to the gravitational wave for longer~\cite{}.  %, it will suffice until a formal model is introduced in Chapter~\ref{chp:proposals}. 
This enhanced detector is called a dual-recycled Fabry-Perot Michelson interferometer~\cite{meersRecyclingLaserinterferometricGravitationalwave1988} and will be modelled in Chapter~\ref{chp:proposals}. Without these cavities, the detection of 100~Hz gravitational waves would not be possible~\cite{}\jam{(keep this sentence in to say that the cavities must stay)}. %, and therefore their inclusion is not optional~\cite{}. 

	% Arm cavities (circled in green in Fig.~\ref{fig:coupled_cavities}) and a power-recycling cavity are introduced to increase the power within the arms since this increases the sensitivity of an interferometric detector~\cite{1995AuJPh..48..953M}. A signal-recycling cavity (circled in blue in Fig.~\ref{fig:coupled_cavities}) is introduced to further enhance the signal by changing the overall resonance behaviour of the interferometer to increase the time that the light spends coupled to the gravitational wave. Optical cavities display resonance behaviour due to the interference condition on the phase acquired by the light on each round-trip of the cavity. These cavities increase low-frequency sensitivity around 100~Hz and without them, detection would not be possible. However, their resonance behaviour significantly decreases the finite bandwidth of the interferometer and therefore reduces high-frequency (1--4~kHz) sensitivity~\footnote{Only the first resonance of the cavities is considered here because the second resonance is far above (around 37~kHz) the frequency band of astrophysical interest (here, around 1--4~kHz). Moreover, the signal response of the Michelson interferometer falls off at higher frequencies which makes the detector less sensitive at each successive resonance.}. This project considers interferometer configurations that modify the signal-recycling cavity to improve high-frequency sensitivity.


\subsection{Factors limiting kilohertz sensitivity}
\label{sec:intro_factors_limiting_kHz}
% explain why current GWD are not sensitive at kilohertz
% what currently prevents us detecting them

Gravitational-wave detectors are limited at kilohertz frequencies by several factors.

% \subsubsection{Signal response and the arm cavity resonance}

The detector's kilohertz \emph{signal response is limited by the resonance behaviour of the arm cavities}.
Optical cavities display resonance behaviour due to the different propagation phases acquired on each round-trip of the cavity by light at different frequencies~\footnote{I will not consider spatial behaviour (e.g.\ spatial modes~\cite{}) in this thesis.}~\cite{}. In the steady-state, the light entering a cavity interferes with the light that has undergone circulating round-trips which amplifies the circulating power if the cavity is on-resonance~\cite{}. %, and conversely when the cavity is off-resonance the circulating power is low. 
For an interferometer, the sensitivity below the cavity bandwidth is improved by the power amplification from the on-resonant arm cavities. However, at kilohertz, the arm cavities start going off-resonance and the signal response decreases~\cite{}. %, e.g.\ to an order less than at 100~Hz for the Advanced~Laser~Interferometric~Gravitational-Wave~Observatory~(Advanced LIGO)~\cite{}. %forAdvLIGO %-- although the arm cavities still improve the sensitivity.
Although the other cavities in the interferometer also display resonance behaviour, the long arm cavities have the shortest bandwidth, e.g.\ $\sim$~100~Hz for 4~km arms\jam{(check FWHM not FSR)}~\cite{}~\footnote{I only consider the first resonance of the arm cavities since the second resonance is above 1--4~kHz~\cite{}. Separately, I also ignore that in a Michelson interferometer the phase accumulated in the arms due to the gravitational wave can cancel between going from and coming back to the beamsplitter since this only affects frequencies above 1--4~kHz (e.g.\ 37.5~kHz for 4~km arms~\cite{}).}, and therefore limit the kilohertz signal response.
%longer arm cavities have shorter bandwidth and therefore matter at lower frequencies, such as kilohertz, as opposed to $10$~kHz etc.. 
% Due to the phase interference being modulo $2\pi$, the arm cavity resonance is periodic with frequency $\frac{\omega_\text{FSR}}{2\pi}=\frac{c}{2L_\text{arm}}$, known as one free spectral range (FSR) of the cavity (approximately $37.5$~kHz for 4~km arms), meaning that the cavities will eventually become resonant again. However, I will make a single-mode approximation to the arms that will only be valid for frequencies below the first free spectral range, and so I only discuss the interferometer's response near the first resonance -- which covers the 1--4~kHz range of interest.

% \subsubsection{Quantum shot noise} %is shot noise--dominated
% at kilohertz, going by Buikema et al, 2020: laser intensity and frequency, photodetector dark (electronic) noise, thermal (is falling off)
% It also bears mentioning the other noise sources that affect intereferometers at frequencies below kilohertz: control sensor noise, beam jitter, thermal, seismic, newtonian, gas. But I will not consider any of these in the rest of the thesis.

% \begin{figure}
% 	\centering
% 	% \includegraphics[width=\textwidth]{}
% 	\caption{Noise budget of Advanced~LIGO from Ref.~\cite{buikemaSensitivityPerformanceAdvanced2020}\jam{(Used without permission? Can I do this?)}. At kilohertz, shot noise dominates and therefore I do not consider other noise sources than quantum noise.}
% 	\label{fig:Buikemeetal2020_LIGO_noise_budget}
% \end{figure}
% \begin{figure}
% 	\centering
% 	% \includegraphics[width=\textwidth]{}
% 	\caption{Quantum noise response (top panel), signal response (middle panel), and quantum noise--limited sensitivity (bottom panel) of a gravitational-wave interferometer. It is conventional to show sensitivity as the noise-to-signal ratio (also known as the signal-normalised noise) in the gravitational-wave literature, so lower values indicate better sensitivity. At kilohertz, the sensitivity is limited by shot noise and the decreasing signal due to the arm cavity going off-resonance.}
% 	\label{fig:simplified_sensitivity}
% \end{figure}

An interferometer is subject to noise from many sources. %At kilohertz these include, in approximately decreasing size: quantum shot noise, laser intensity noise, photodetector dark noise, laser frequency noise, and thermal noise~\cite{buikemaSensitivityPerformanceAdvanced2020}. %~\footnote{Below kilohertz, there are many additional noise sources, such as noise from control systems, beam jitter, seismic noise, Newtonian noise, and gas noise, but I will not consider any of these in this thesis.}.
However, at kilohertz, the \emph{noise response is dominated by quantum shot noise}, with the sum of contributions from all other noise sources contributing less than half the total noise for the Advanced~Laser~Interferometric~Gravitational-Wave~Observatory~(Advanced LIGO)~\cite{AdvancedLIGO:2015,buikemaSensitivityPerformanceAdvanced2020}. % as shown in Fig.~\ref{fig:Buikemeetal2020_LIGO_noise_budget} taken from Ref.~\cite{buikemaSensitivityPerformanceAdvanced2020}.
Therefore, with my focus on kilohertz frequencies, this is the primary source of noise that I consider in this thesis. %~\footnote{Although I will compare my results to a configuration that is instead limited by thermal noise at kilohertz.}. %\jam{(what about thermal noise in sWLC?)}
%I will give a detailed explanation of quantum noise in Chapter~\ref{chp:background_theory}.
% Shot noise arises from the fundamental, quantum uncertainty in the phase of a state of light~\cite{}\jam{(check this, the models have noise in every quadrature)}. The vacuum fluctuations of virtual photons at a particular frequency mean that even the vacuum will have some uncertainty, a measurable variance, in its phase, distributed about zero. %phase is not a real thing, mention quadratures
% Shot noise is flat in, i.e.\ independent of, frequency, and it dominates at kilohertz because all other noise sources have been reduced below it by decades of research and innovation~\cite{}. For the quantum noise--limited sensitivity, because the signal is decreasing and the shot noise is constant, the sensitivity decreases at kilohertz.


% \subsubsection{Mizuno limit on integrated sensitivity}
\label{sec:circulating_power}
%limit on the quantum noise--limited sensitivity
% mizuno limit (due to QCRB) on the integrated sensitivity, i.e. on the bandwidth-peak sensitivity product --> can be beaten by squeezing

The \emph{Mizuno limit} states that the available circulating power limits an interferometer's integrated quantum noise--limited sensitivity, i.e.\ the product of its bandwidth and peak sensitivity~\cite{miaoFundamentalQuantumLimit2017}.
% This limit comes ultimately from the Quantum Cramer-Rao Bound where the circulating power quantifies the total energy of the detector~\cite{}\jam{(explain the physics?)}.
This means that, for a fixed circulating power, increasing bandwidth to improve kilohertz sensitivity would need to sacrifice existing low-frequency ($\sim100$~Hz) sensitivity. %, but that is not acceptable. % beyond the point of being useful for detection. %justify this?
% While this thesis is motivated by kilohertz detection, this low-frequency sensitivity should be maintained. For the case example of observing binary neutron-star mergers, maintaining low-frequency sensitivity would allow for the pre-merger inspiral to be seen, hopefully along with the merger and post-merger remnant at kilohertz, necessary to construct the full description of the merger~\cite{}.
% Therefore, this limit poses a problem to improving sensitivity at kilohertz to that required for detection.
% \subsubsection{Technological limitation on increasing circulating power}
% Why can circulating power not increase?
% constraints for future detectors
% I assume that we can't increase power by a factor of 2, unlike Miao et al 2018
The Mizuno limit could be alleviated if the circulating power could be increased, however, this is technologically challenging and is an ongoing area of research, from mitigating current effects which prevent increasing the power~\cite{Brooks_2021,PhysRevLett.114.161102} to working towards the power requirements of future detectors~\cite{}. 
% not currently technologically possible. In Advanced~LIGO, which operates at $750$~kW circulating power~\cite{}, point absorbers on the optics~\cite{Brooks_2021} and parametric instabilities~\cite{PhysRevLett.114.161102} prevent increasing the circulating power. This poses a technological challenge to future detectors which demand upwards of $3$~MW~\cite{}. % and $4.5$~MW for NEMO~\cite{}, a challenge yet to be solved but with much progress towards it~\cite{}. %\jam{(Need further information? For Evans et al, 2015, didn't they alleviate the parametric instability problem?)}
% Therefore, I will assume that $\sim3$~MW circulating power is the maximum achievable value for the next generation of detectors, assuming significant technological progress in the coming decades.
Ultimately, arbitrarily high circulating power is not possible, and, therefore, a different method is required to improve kilohertz sensitivity that can work around the Mizuno limit.

% Therefore, gravitational-wave detectors are not as sensitive at kilohertz because (1) the signal response drops off due to the arm cavity resonance, (2) the noise response is dominated by quantum shot noise, (3) the integrated sensitivity cannot improve by classical means without increasing the circulating power, and (4) the circulating power cannot increase without technological progress.

	% The amount of laser power within an interferometric gravitational-wave detector limits how much its high-frequency sensitivity can improve without its low-frequency sensitivity worsening or the power needing to increase~\cite{miaoFundamentalQuantumLimit2017}. 
	% However, gravitational waves require a wide band of sensitive frequencies to detect as they are typically not monochromatic~\cite{miaoDesignGravitationalWaveDetectors2018}, and further increasing the power is technologically problematic~\cite{Brooks_2021,PhysRevLett.114.161102}. 


%%%%%%%%%%%%%%%%%%%%%%%%%%%%%%%%%%%%%%%%%%
\subsection{Literature review: how to improve kilohertz sensitivity}
\label{sec:intro_literature_review}
% should be categorising and judging
% John Close says: don't just list papers, a powerful review identifies the goal of the thesis and puts it in context, ideally quantitatively --> what are the quantities for my thesis, sensitivity?

% The above four factors: the arm cavity resonance, shot noise, the Mizuno limit, and power limitations, can be avoided and kilohertz sensitivity can be improved, but this requires changes to the detector's configuration.
% I will review the literature of proposals to improve kilohertz sensitivity in Section~\ref{sec:literature_review}.
% For now, I will briefly mention two recent proposals: (1) degenerate internal squeezing~\cite{korobkoQuantumExpanderGravitationalwave2019,adyaQuantumEnhancedKHz2020} and (2) stable optomechanical filtering~\cite{liBroadbandSensitivityImprovement2020}. 
% The Mizuno limit can be avoided by non-classical techniques, such as quantum squeezing~\cite{}, a technique which trades off the quantum noise in a desired quantity for that in a less desired quantity. I will detail squeezing in Chapter~\ref{chp:background_theory}.
% Squeezing is already used in Advanced~LIGO externally to the coupled cavities to halve the shot noise~\cite{tseQuantumEnhancedAdvancedLIGO2019} across the band.
% Degenerate internal squeezing is a proposal that uses squeezing instead inside the signal-recycling cavity to further enhance the sensitivity by interacting with the signal and noise, which I will detail in Section~\ref{sec:dIS}. 
% The Mizuno limit can also be avoided by making the arm cavity more broadly resonant through introducing a filter cavity. This ``white-light cavity'' idea~\cite{miaoEnhancingBandwidthGravitationalWave2015,} is what stable optomechanical filtering achieves by coupling a mechanical mode to the signal-recycling cavity mode, which I will detail in Section~\ref{sec:sWLC}.
% although frontrunners in the literature~\cite{}
% However, these two configurations have their problems. Degenerate internal squeezing is susceptible to optical loss because the configuration decreases the signal and relies on correlations degradable by optical losses~\cite{korobkoQuantumExpanderGravitationalwave2019}. Stable optomechanical filtering requires low mechanical loss which demands better thermal noise and mechanical quality factor than are currently technologically possible~\cite{liBroadbandSensitivityImprovement2020,miaoEnhancingBandwidthGravitationalWave2015}. These problems could limit the feasibility of these two configurations for kilohertz detection, and therefore they motivate looking for alternative configurations. %But any alternative proposal must first avoid the limit set by the above four factors.

% \jam{(The role of this section is (1) to explain why I look at these two configurations and (2) satisfy the thesis requirement for a literature review. Check the honours marking rubric. Do I need more citations or critical analysis, perhaps? See supervisors for citations.)} %Should I also give a short review of squeezing in gravitational wave detectors and squeezing generally?

% \section{Literature review}
% \label{sec:literature_review}
% intro: beyond external changes (external squeezing and caves's amplifier) to interferometer and increasing circulating power
Over the past decade, various proposals to improve the kilohertz sensitivity of gravitational-wave detectors have been made, two of which are particularly relevant to this thesis.
% through some non-classical technique\jam{(How does sWLC avoid Mizuno limit? Why is cancelling the arm cavity resonance non-classical?)}, as outlined in Section~\ref{sec:intro_factors_limiting_kHz}. Broadband improvement is possible through external modifications to the interferometer such as external squeezing and Caves's amplification, as already discussed, but I am interested in techniques that directly address kilohertz sensitivity. In this review, I will summarise: (1) the existing gravitational-wave detectors around the world, (2) the future, third-generation of detectors\jam{(are they being constructed right now?)}, (3) the proposals that use internal squeezing, (4) the proposals that use optomechanical filtering, and (5) some of the other existing proposals. %However, I will refrain from a thorough technical explanation of these configurations, which will be given later in this chapter. 
% There is a literature of exploratory work which could be realised in a non-specific future detector, which is where my work fits into. %~\footnote{It is worth clarifying here that I am not aiming to produce a complete proposal for a future detector either, only explore the space of designs}.
% degenerate internal squeezing
One is \emph{degenerate internal squeezing} (also known as a degenerate quantum expander) % where a squeezer is placed inside the signal-recycling cavity of an interferometer, which has been considered for NEMO~\cite{}. 
which has been characterised in different operating regimes, e.g.\ to improve broadband~\cite{korobkoQuantumExpanderGravitationalwave2019} or kilohertz~\cite{adyaQuantumEnhancedKHz2020} sensitivity. 
% I will explain this configuration thoroughly later,
% This configuration has a squeezer inside the signal-recycling cavity, turning the two-cavity interferometer into an OPO coupled to the arm cavity. Unlike external squeezing, where only the readout port vacuum is squeezed, degenerate internal squeezing also squeezes the intra-cavity noise and the gravitational-wave signal. Although de-amplifying (squeezing) the signal is not ideal, the reduction in the quantum noise is significant enough to produce an improvement in sensitivity. This improvement can occur at kilohertz if the interferometer parameters are chosen correctly~\cite{}. However, degenerate internal squeezing is sensitive to intra-cavity and detection losses, since these losses reduce the signal but amplify the noise towards vacuum. 
This configuration avoids the Mizuno limit without increasing the circulating power by using a non-classical technique, quantum squeezing~\cite{miaoFundamentalQuantumLimit2017} that I will detail later. %, which I will detail in Chapter~\ref{chp:background_theory}. 
However, degenerate internal squeezing has a low tolerance to optical losses because the squeezing it uses degrades with optical loss~\cite{}. 
Although research continues today into optimising degenerate internal squeezing in the high loss regime~\cite{korobkoCompensatingQuantumDecoherenceTalk2021}, its low tolerance to loss motivates investigating configurations that are more tolerant to the realistic losses expected in a future detector. %\jam{(I need to be more critical of the literature. See goal of the literature review.)}
% for its ability to be switched into an anti-squeezing regime in the high loss limit~\footnote{This limit is beyond the expected losses in future detectors, e.g.\ above $50\%$ compared to $10\%$, but it demonstrates that internal anti-squeezing is useful under certain circumstances.}~\cite{korobkoCompensatingQuantumDecoherenceTalk2021}, which improves sensitivity similarly to a Caves's amplifier, see Section~\ref{sec:cavess_amp}, except that the amount of anti-squeezing can be different for signal and noise.
% is there any GW community interest in nondegenerate squeezing?
% But nondegenerate internal squeezing, where the internal squeezing is operated nondegenerately, is an alternative that has not yet been thoroughly considered~\cite{} -- I will identify this gap in the literature in Section~\ref{}\jam{(haven't I identified it here?)}. 

% optomechanical filtering: unstable~\cite{miaoEnhancingBandwidthGravitationalWave2015,miaoDesignGravitationalWaveDetectors2018,Page2018} and stable~\cite{liBroadbandSensitivityImprovement2020,liEnhancingInterferometerSensitivity2021}
Another existing proposal is \emph{optomechanical filtering} which improves the sensitivity without increasing the circulating power by broadening the arm cavity resonance that limits the kilohertz signal response, called a ``white-light cavity''~\cite{miaoEnhancingBandwidthGravitationalWave2015,}\jam{(but how does this avoid the Mizuno limit?)}. Optomechanical filtering in this context was first proposed in an unstable configuration which required a feedback control system to stabilise~\cite{miaoEnhancingBandwidthGravitationalWave2015}. %~\footnote{Although the white-light cavity idea was previously proposed using stable atomic media~\cite{}, this was the first proposal to instead use an optomechanical interaction.}. % previously just used stable atomic media
% This configuration couples the optical mode in the signal-recycling cavity to a mechanical mode. The optomechanical coupling can be chosen such that signal-recycling (filter) cavity imparts the opposite round-trip phase to the arm cavity at certain frequencies, which broadens the arm cavity resonance (the ``white-light cavity'' idea~\cite{}).
% When the readout occurs via the arm cavity mode, the system is unstable and needs a feedback control system,
The application to gravitational-wave detection and limitations of this unstable configuration were further investigated in Refs.~\cite{miaoDesignGravitationalWaveDetectors2018,pageEnhancedDetectionHigh2018,}. 
Then, the system was made stable without a control system by changing the readout scheme~\cite{liBroadbandSensitivityImprovement2020}\jam{(why does this change the system physically? should analyse all internal modes instead?)}. The stable configuration was recently further investigated with a more realistic model that included higher-order modes, and showed promising sensitivity improvement although it still only included mechanical loss~\cite{liEnhancingInterferometerSensitivity2021}. 
% But if the readout is changed to the signal-recycling (filter) cavity optical mode, then the system becomes stable, which has been investigated in Refs.~\cite{liBroadbandSensitivityImprovement2020,liEnhancingInterferometerSensitivity2021} and produces broadband sensitivity improvement in the lossless case.
However, to improve the sensitivity, it has a stringent requirement on the mechanical loss being low~\cite{miaoEnhancingBandwidthGravitationalWave2015,liBroadbandSensitivityImprovement2020}. Research is ongoing into how to achieve the demands of this configuration~\cite{opticalSpringRef?,catFlapRef?,}\jam{(check Miao, Page references)} which, like degenerate internal squeezing, motivates investigating more loss-tolerant configurations. 

% other configurations + configurations not based on the Michelson interferometer: speed-meters etc.
Although there are other existing proposals for kilohertz improvement~\cite{,,}, degenerate internal squeezing and stable optomechanical filtering are currently the most developed proposals\jam{(check!)}~\cite{}. These two proposals are also the most related configurations to my work in this thesis that examines \emph{nondegenerate internal squeezing}, an alternative gravitational-wave detector configuration that combines the Hamiltonian (i.e.\ the mode structure) of stable optomechanical filtering with the all-optical approach of degenerate internal squeezing. This configuration improves kilohertz sensitivity by using squeezing to avoid the Mizuno limit without increasing circulating power. %This means that it does not require increased circulating power nor, a priori, compromised low-frequency sensitivity.
Although nondegenerate squeezing has long been studied outside of gravitational-wave research~\cite{reidDemonstrationEinsteinPodolskyRosenParadox1989,schoriNarrowbandFrequencyTunable2002,,}, this particular configuration has only been mentioned once\jam{(check this!)} in the literature as its Hamiltonian is equivalent to stable optomechanical filtering under a certain mapping of optical to mechanical modes~\cite{liBroadbandSensitivityImprovement2020}.
However, in the work studying stable optomechanical filtering~\cite{liBroadbandSensitivityImprovement2020,liEnhancingInterferometerSensitivity2021}, only the dominant loss term is included in the model\jam{(check Li2021)} and the perturbations due to other realistic losses are ignored, e.g.\ it is not known how low these other losses need to be\jam{(how big are these perturbations?)}. And, although the Hamiltonian loss terms are equivalent between the two configurations~\cite{}, the realistic optical and mechanical losses, respectively, are different because of their different associated technologies which means that nondegenerate internal squeezing might not require such low loss as stable optomechanical filtering. Moreover, other aspects of nondegenerate internal squeezing are not known, such as its squeezing threshold, its behaviour in low and high loss limits, and the possible readout schemes.

% some of which are not even based on the Michelson interferometer~\cite{}. I am not aware of all that has been suggested, but some of the configurations that continue to generate interest in the literature are\jam{..., ..., and ... . (I do not know and need to read up to fill this in!)}. 
% this is just an honours project 
%, since nondegenerate internal squeezing is to degenerate internal squeezing as the nondegenerate OPO is to the degenerate OPO, and nondegenerate internal squeezing is an all-optical analogue of the stable optomechanical filtering. % I can't claim that they are frontrunners, what evidence?
% Therefore, I consider only them as a result of finite time and relevance. 
% should not be taken as a claim of their superior feasibility to the rest of the literature but as

% Therefore, I identify this as a gap in the literature, no work has fully characterised nondegenerate internal squeezing with realistic optical loss in every mode. I will fill this gap in this thesis and assess the feasibility of nondegenerate internal squeezing for kilohertz gravitational-wave detection. 
% In particular, no work has studied it in detail which I attribute to the recency of the proposals that motivate it~\ref{korobkoQuantumExpanderGravitationalwave2019,liBroadbandSensitivityImprovement2020}. The goal of this thesis is to fill that gap in the literature and assess the feasibility of nondegenerate internal squeezing for kilohertz gravitational-wave detection
% In the work studying the optomechanical analogue~\cite{liBroadbandSensitivityImprovement2020,liEnhancingInterferometerSensitivity2021}, only mechanical loss is included in the model\jam{(check Li2021)} since it is the dominant source of loss, but a full understanding of the system requires all realistic optical losses to be included\jam{(quantify difference?)}. % at the realistic levels for a future detector. % -- rather than indirectly by considering realistic mechanical losses. 

% existing detectors
% The global network of existing gravitational-wave detectors consists of Advanced~LIGO~\cite{} (both Hanford~\cite{} and Livingston~\cite{} sites), Advanced~Virgo~\cite{}, KAGRA~\cite{}, and GEO600~\cite{}\jam{(check this list)}. All of these follow the interferometer design discussed in Section~\ref{sec:intro_IFO}. % with the only major change relevant for quantum noise\jam{(check this)} being the use of external squeezing in Advanced~LIGO, discussed in Section~\ref{sec:external_squeezing}. %I will use Advanced~LIGO to represent this generation, henceforth. 
% % 3rd generation detectors: LIGO Voyager, NEMO, Einstein Telscope, Cosmic Explorer --> why I will focus on the Voyager parameter set (no increased circulating power, no change in arm lengths)
% The proposed next, third generation of ground-based detectors consists of LIGO~Voyager~\cite{}, NEMO~\cite{}, the Einstein~Telescope~\cite{}, and Cosmic~Explorer~\cite{}\jam{(check 2.5 versus 3 generation)}. They differ from the previous generation in the interferometer parameters, such as the circulating power, arm length, test mass mass, carrier wavelength, and transmissivities of the input test mass and the signal-recycling mirror~\cite{}. 
% The latter detectors, Einstein~Telescope and Cosmic~Explorer, are proposed to have significantly\jam{(quantify)} longer arm lengths, higher circulating powers, and therefore better sensitivity than LIGO~Voyager or the current generation of detectors. 
% However, I will consider the feasibility of configurations using the LIGO~Voyager parameter set~\cite{} because it will be the first of these future detectors to come online as it is a series of upgrades to the existing Advanced~LIGO detectors~\cite{}. Moreover, variations of the LIGO~Voyager parameter set are often used in the literature to assess the configurations in this chapter~\cite{liBroadbandSensitivityImprovement2020,miaoDesignGravitationalWaveDetectors2018,korobkoQuantumExpanderGravitationalwave2019,}.
% In particular, I will use\jam{(tabulate these parameters to reference later)} $3$~MW of circulating power, a $4$~km arm cavity, a $56$~m signal-recycling cavity, $200$~kg test masses, $2~\mu\text{m}$ carrier wavelength~\footnote{There is debate about $2$ versus $1.064~\mu\text{m}$ as the preferred carrier wavelength for application~\cite{}, but I will not consider this because my models are too simplified to compare the technological constraints\jam{(is this true?)}.}, $0.002$ transmission for the input test mass, and $0.046$ transmission for the signal-recycling mirror~\footnote{These parameters might be biased against one configuration over another. I have partially mitigated this by varying the coupling rates between the modes which appears to characterise the different classes of parameter sets~\cite{}.}, but I will vary the readout rate through the signal-recycling mirror by varying the length of the signal-recycling cavity~\footnote{I will also vary the input test mass's transmissivity to counteract the effect of changing the signal-recycling cavity's length on the coupling rate to the arm cavity (known as the sloshing frequency~\cite{}).}
% separate from these defined detectors is a literature of different possible configurations and proposals for these and future detectors.
% These future detectors have established science-cases\jam{(colloquial?)} and detailed proposals, but


%%%%%%%%%%%%%%%%%%%%%%%%%%%%%%%%%%%%%%%%%%
\section{Thesis outline}
\label{sec:thesis_outline}
% what the thesis contributes towards the problem: the aims from the research proposal
% quantum noise--limited sensitivity
% overview of the problem: we can't see at HF, (to be covered) and the problems with exisiting proposed solutions (the latter will be detailed more in the relevant background chapters)

\begin{table}
\centering
\begin{tabular}{@{}ll|ll@{}}
\toprule
ITM & input test mass & DC & direct current \\
ETM & end test mass & SRC & signal-recycling cavity \\
SRM & signal-recycling mirror & PT & parity-time \\
OPO & optical parametric oscillator & GW & gravitational wave \\
PD & photodetector & RP & radiation pressure \\ \bottomrule
\end{tabular}
\caption{The abbreviations used throughout this thesis in order of appearance.}
\label{tab:abbreviations}
\end{table}

In this thesis, I investigate these unexplored aspects of nondegenerate internal squeezing and present the first results on them. In particular, I characterise nondegenerate internal squeezing with realistic optical loss in every mode for different readout schemes and assess its feasibility for kilohertz gravitational-wave detection. The abbreviations used throughout this thesis are shown in Tab.~\ref{tab:abbreviations}.
\begin{itemize}
\itemsep0em 
\item In Chapter~\ref{chp:background_theory}, I will review the background physics to explain squeezing and the quantum noise response of a detector. I will demonstrate the analytic, Hamiltonian modelling that I use throughout this thesis and will discuss how squeezing is currently used to improve the sensitivity of gravitational-wave detectors. %on the simple, well-known cases of degenerate and nondegenerate optical parametric oscillators~\cite{}. 
\item In Chapter~\ref{chp:proposals}, I will discuss the benefits and limitations of degenerate internal squeezing and stable optomechanical filtering and motivate combining them into nondegenerate internal squeezing.
\item Then, in Chapter~\ref{chp:nIS_analytics}, I will derive an analytic, Hamiltonian model of nondegenerate internal squeezing. I will characterise the stability and sensitivity of the configuration, study the high and low loss limits, and derive its squeezing threshold. %limit of the amount of available squeezing.
\item Next, in Chapter~\ref{chp:science_case}, I will compare nondegenerate internal squeezing's tolerance to realistic optical loss to the existing proposals. I will consider its application to kilohertz (1--4~kHz) and broadband (0.1--4~kHz) gravitational-wave detection.
\item In Chapter~\ref{chp:idler_readout}, I will discuss the differences between alternative readout schemes. I will show that the broadband sensitivity can be further improved by combining different readout schemes. % which is an advantage of the configuration.
\item Finally, in Chapter~\ref{chp:future_work_and_conclusions}, I will consider the conclusions and limitations of my work and what avenues of future work it suggests.
\end{itemize}



% From research proposal
	% I aim to design and analytically model an all-optical interferometer configuration to improve high-frequency sensitivity beyond current detectors, while neither fully sacrificing low-frequency sensitivity nor increasing the power. I will include realistic sources of loss in the model to assess the feasibility of this configuration. Finally, I will compare this configuration to existing proposals to show the potential benefits of its design.

	% The central aim of this project is to investigate nondegenerate internal squeezing and assess how it might improve the high-frequency sensitivity of future detectors. Although nondegenerate internal squeezing has been considered before, no work has fully examined it nor assessed its feasibility by including losses in the model~\cite{liBroadbandSensitivityImprovement2020}. I will perform an analytic investigation of nondegenerate internal squeezing that will include a feasibility assessment using losses and realistic assumptions and conclude with a comparison to previously proposed configurations.

	% I will use a Hamiltonian method to analytically model nondegenerate internal squeezing and
	% calculate its quantum-noise limited sensitivity. The Hamiltonian for nondegenerate internal
	% squeezing is already known for the approximate interferometer model of a pair of coupled cavities shown in the right panel of Fig.~\ref{fig:coupled_cavities}~\cite{korobkoQuantumExpanderGravitationalwave2019,liBroadbandSensitivityImprovement2020}. %cite Schori et al?
	% This Hamiltonian includes standard harmonic oscillator terms for each of the optical fields in each cavity and at each frequency and interaction terms that represent the squeezing process and coupling between the cavities. I will solve the quantum Langevin equations of motion to find the output signal and noise fields that are measured by the photodetector and calculate the sensitivity of the overall detector. These equations combine the Heisenberg equations of motion of the field operators with input/output terms. To solve them, I will use common approximations including a semi-classical approximation of the pump field to treat it as a reservoir and a linear expansion in small fluctuations around the expectation values of certain operators. This Hamiltonian method and the approximations involved are widely used in the literature and, in particular, in the works that I will compare to and base my derivation upon~\cite{liBroadbandSensitivityImprovement2020,korobkoQuantumExpanderGravitationalwave2019}. I will repeat the analysis of stable optomechanical filtering done in Ref.~\cite{liBroadbandSensitivityImprovement2020} to practice this Hamiltonian method before attempting the novel derivation of nondegenerate internal squeezing with realistic assumptions.

	% I will assess the feasibility of nondegenerate internal squeezing by studying the effects of various noise sources on the sensitivity under realistic assumptions. The degenerate internal squeezing literature suggests that optical loss will be the dominant noise source of nondegenerate internal squeezing~\cite{korobkoQuantumExpanderGravitationalwave2019}. Therefore, I will prioritise including optical loss in the model above other noise sources. I will also include photodetector loss to test the hypothesis that nondegenerate internal squeezing is more resistant to photodetector loss than degenerate internal squeezing. Under realistic assumptions about improvements in optical loss and squeezing beyond current detectors, I will compare the sensitivity of the model to the estimated sensitivity required to detect a typical binary neutron-star post-merger signal~\cite{miaoDesignGravitationalWaveDetectors2018}. This will determine how large an improvement in these factors will need to be achieved, technologically, before the configuration is physically viable.

	% Finally, I will compare nondegenerate internal squeezing to degenerate internal squeezing and stable optomechanical filtering by comparing the sensitivity and feasibility of the three systems. Nondegenerate and degenerate internal squeezing have not been compared before. Similarly, although the connection has been previously made between the Hamiltonians of nondegenerate internal squeezing and stable optomechanical filtering~\cite{yapadyaPersonalCommunication,liBroadbandSensitivityImprovement2020}, the sensitivity and feasibility of the two systems have not been compared before. The deeper connection to stable optomechanical filtering means that I will prioritise comparing my model to it over degenerate internal squeezing.


%%%%%%%%%%%%%%%%%%%%%%%%%%%%%%%%%%%%%%%%%%
\section{Chapter summary}

% In this chapter, I motivate the detection of kilohertz gravitational waves, describe what currently prevents us detecting them, and outline what this thesis contributes to detecting them in the future.

In this chapter, I have motivated the detection of kilohertz gravitational waves for advancing our understanding of astrophysical phenomena. I have explained how detectors based on the Michelson interferometer can detect gravitational waves at around $100$~Hz but why their sensitivity cannot be simply broadened to also detect kilohertz gravitational waves. Finally, I have mentioned the two proposed configurations that motivate my work, the limitations of these configurations, and that this thesis will examine nondegenerate internal squeezing as an alternative configuration for gravitational-wave detection.


