\chapter{Introduction} %Background: gravitational waves and their detection
\label{ref:introduction}

% introduction: motivation, overview of problem, and what thesis contributes
% written to a general physics audience
% can merge with backgroup chapter/s if appropriate (background chapters are to detail the problem sufficiently and specifically)

%%%%%%%%%%%%%%%%%%%%%%%%%%%%%%%%%%%%%%%%%%
% chapter introduction

% ... (write this later)

	% The detection of gravitational waves from the late inspiral of binary black-hole~\cite{GW150914} and neutron-star~\cite{GW170817} mergers is a landmark achievement of 21st-century physics and engineering that opened up a new field of astronomy dedicated to using gravitational waves to better understand the universe. % might be better in the Abstract?

In this chapter, I motivate the detection of kilohertz gravitational waves, describe what currently prevents us from detecting them, and outline what this thesis contributes to detecting them in the future.

\section{Gravitational waves}
\label{sec:gravWaves}

Gravitational waves are propagating perturbations in spacetime~\cite{}. These perturbations change the proper distance and time between events in spacetime, e.g.\ the proper separation between two massive objects. Since they can move massive objects around, therefore gravitational waves carry energy~\cite{}. %\jam{(``Move'' is problematic, look at choice of frame)}
% Just as they are able to move objects, 
Gravitational waves are emitted by the acceleration of massive objects under certain asymmetry conditions~\cite{}, and since they carry energy the objects that emit them must lose energy by doing so. 
% tense?
Gravitational waves have been detected from the late inspiral of compact binary systems~\cite{} -- meaning that two massive, compact objects, such as two black holes~\cite{}, two neutron stars~\cite{}, or a black hole and a neutron star~\cite{}, were closely orbiting each other. Then, as the binary system lost energy by emitting gravitational waves, the objects got closer together, the orbital frequency increased, and correspondingly the frequency of the emitted gravitational waves increased. Eventually, the objects collided and merged, leaving behind some other object, possibly a black hole or neutron star~\cite{}. The late inspiral of such systems refers to the last moments before the objects collide and merge and therefore to the highest frequency gravitational waves emitted pre-coalescence -- measured at around $100$~Hz~\cite{}. % check this?
Despite being emitted by some of the most massive objects in the universe, the effects of gravitational waves are extremely small -- by the time they reach Earth, even the loudest gravitational waves will only move two objects separated by a kilometre by less than a thousandth the width of a proton~\cite{}. %wordchoice: loudest
This poses a great challenge to detection.

\begin{figure}
	\centering
	% \includegraphics[width=\textwidth]{}
	\caption{Gravitational wave incident on a ring of test particles, evolution over time.}
	\label{fig:GW_ring_of_test_particles}
\end{figure}

Gravitational waves are described by radiative, transverse-traceless solutions to the Einstein field equations of General Relativity~\cite{}. % will need to mention transverse-traceless, quadrupole moment?
The initial amplitude of these waves is described by the quadrupole moment of the source -- a quantity that depends on its distribution of mass which explains why only the most massive astrophysical sources can be detected. % sources closer to the detector can be less massive
% ...
The effect of these waves is to alternately stretch and squash spacetime along the two axes perpendicular to the direction of propagation. That is, a gravitational wave incident normally on a ring of test particles will, over time, stretch and squash one semi-axis of the ring while conversely squashing and stretching the other semi-axis -- as shown in Fig.~\ref{fig:GW_ring_of_test_particles}. 
By the time the gravitational wave reaches the ring (or detector), however, its amplitude has fallen by the inverse of the distance from the source~\cite{} which is why astrophysical sources with large quadrupole moments only produce small length changes at Earth. % 8.34 in Karl's notes
The unitless gravitational-wave strain at the detector, $h(t)$, gives the change in length $\Delta L$ of a detector's initial length $L$ as $\Delta L(t) = L \;h(t)$. The goal of gravitational wave detection is to measure $h(t)$ as accurately and precisely as possible as it encodes astrophysical information about the source~\cite{}. Henceforth, $h(t)$ is referred to as the gravitational-wave signal, or just: the signal.


	% The orbital period of binary black-hole or neutron-star systems decays as the system radiates away energy in the form of low-amplitude waves in spacetime; these gravitational waves, correspondingly, increase in frequency over time. Current interferometric detectors are most sensitive to gravitational waves in a low-frequency window around 100~Hz that are emitted in the late inspiral of such mergers~\cite{AdvancedLIGO:2015}.

\subsection{Kilohertz gravitational-wave astrophysics} %i.e. sources

% motivate HF limit
All detections of gravitational waves so far have been around $100$~Hz~\cite{}, but there is believed to be varied and interesting astrophysics encoded in higher frequency, kilohertz gravitational waves yet to be detected~\cite{}.
For example, in the merger and post-merger signals from binary neutron-star mergers, which are kilohertz gravitational waves predicted to be emitted during the coalescence and from the remnant object (possibly another neutron star) of the merger. 
Due to the violent nature of the merger, these signals are predicted to contain information otherwise unavailable about the exotic states of matter within neutron stars~\cite{}. Information that could better constrain the possible equations-of-state and improve our understanding of matter under extreme conditions~\cite{miaoDesignGravitationalWaveDetectors2018,}. %this is a bold claim, need another reference? % due to the violent nature, really?
Other astrophysical applications of kilohertz gravitational-wave detection could include determining the origin of low-mass black holes by detecting the merger and ringdown of binary black hole-neutron star mergers~\cite{}, understanding the post-bounce dynamics of core-collapse supernovae~\cite{}, improving measurements of the Hubble constant independently of electromagnetic observations, and searching the stochastic gravitational-wave background for exotic or primordial sources~\cite{miaoDesignGravitationalWaveDetectors2018}.
This possible wealth of new astrophysics motivates developing the ability to detect kilohertz gravitational waves.

% Non-astrophysical sources of gravitational waves?

	% However, the frequency of the emitted gravitational waves then increases beyond this window and so the merger can no longer be observed. In particular, predicted high-frequency gravitational waves around 1--4~kHz from the merger and post-merger remnant of binary neutron-star mergers are yet to be detected. These signals are predicted to contain valuable information that could be used to better constrain the possible equations-of-state of the exotic states of matter within neutron stars~\cite{miaoDesignGravitationalWaveDetectors2018}.
	% Potential astrophysical applications of high-frequency (kilohertz) gravitational-wave detection also include determining the origin of low-mass black holes by detecting binary black hole-neutron star mergers, improving measurements of the Hubble constant independently of electromagnetic observations, and searching the stochastic gravitational-wave background for exotic or primordial sources~\cite{miaoDesignGravitationalWaveDetectors2018}. 

\subsubsection{Sensitivity target for enabling new astrophysics}
\label{sec:GW_kilohertz_target}

% this awkward here, should it be mentioned after Sh is explained? or in the science case?
\jam{(is this necessary here?)}
% this will be the goal of my optimisation
To quantise this goal, consider the case example of the post-merger signal from a binary neutron-star merger. The estimated sensitivity required to reliably detect this signal from a typical source~\footnote{Using current understanding which is conditioned on the very equations-of-state that are to be constrained by these measurements.} is $5\times10^{-25} \mathrm{Hz}^{-1/2}$ from 1--4~kHz~\cite{miaoDesignGravitationalWaveDetectors2018}. This value is for the amplitude spectral density of the noise-to-signal ratio of a detector which will be explained in Chapter~\ref{chp:background_theory}. Other kilohertz astrophysical sources are predicted to require similar or greater sensitivity~\cite{}. For now, all that matters is that I have a target sensitivity for determining whether a given detector will be useful to kilohertz gravitational-wave detection.


%%%%%%%%%%%%%%%%%%%%%%%%%%%%%%%%%%%%%%%%%%
\section{Interferometric gravitational-wave detectors}
\label{sec:intro_IFO}

\begin{figure}
	\centering
	% \includegraphics[width=\textwidth]{}
	\caption{Gravitational wave incident on a Michelson interferometric detector. All configurations in this thesis are simplified for modelling. The gravitational wave moves the end test masses back and forth which changes the interference pattern detected.}
	\label{fig:GW_incident_Michelson}
\end{figure}

% explain what a GWD is 
Current gravitational-wave detectors are based on the Michelson interferometer as shown in Fig.~\ref{fig:GW_incident_Michelson}. An optical Michelson interferometer involves a laser beam being split down two perpendicular arms before returning and interfering at the beamsplitter to produce an interference pattern on a photodetector at the output~\cite{}. 
An incident gravitational wave, imagined as coming into the page in Fig.~\ref{fig:GW_incident_Michelson}, stretches and squashes the lengths of the arms oppositely and changes the path difference between them -- albeit by a relative length change of less than $10^{-21}$~\cite{}~\footnote{Corresponding to $\sim10^{-18}$~m which is far smaller than the wavelength of the carrier laser $\lambda_0\sim10^{-6}$~m and therefore it is not a concern that the interferometer is sensitive to changes only modulo the wavelength of the carrier.}. The resulting change in the interference pattern, known as fluctuations in the interferometer's differential mode, is detected by a photodetector. I assume that the interferometer is tuned such that the light from the arms destructively interferes at the output of the beamsplitter and constructively interferes at the input of the beamsplitter (this ``common'' mode then travels back towards the laser source).
A more complete explanation would explain that \jam{(fix wording)} the gravitational wave can be considered to only move the end mirrors, whose motion phase modulates the light in each arm, which appears as amplitude modulation after the beamsplitter and can be detected, e.g.\ by a photodiode, but the simple explanation above will suffice until a formal model is introduced in Chapter~\ref{chp:proposals}.  

	%Interferometric gravitational-wave detectors are based on the Michelson interferometer as shown in the left panel of Fig.~\ref{fig:coupled_cavities}. A gravitational wave perturbs the path difference between the two arms of the interferometer by stretching and squashing spacetime and the resulting change in the interference of the light at the beamsplitter produces a signal on a photodetector. This change in the interference is known as fluctuations in the differential mode of the interferometer.

% (recall: the effect of a gravitational wave on a ring of test particles in Sect.~\ref{sec:gravWaves})
A problem for interferometric detectors is that the perpendicular axes of deformation are determined by the polarisation of the gravitational wave~\cite{} and so, if the wave is not aligned to the arms, then the response will be less than usual or might vanish. Throughout this thesis, I will ignore this subtlety and assume the maximum detector response: from a gravitational wave exactly aligned to the arms, as this problem can be solved by a global network of detectors with different alignments~\cite{}. % -- like we currently have.

% Michelson signal response function
% unrelated but Another problem for interferometers is that they are only sensitive to phase differences between the arms modulo $2\pi$. %unrelated
Another problem for Michelson interferometers is that the phase accumulated by the light can cancel between going from the beamsplitter to the end mirror and returning from the end mirror back to the beamsplitter.
Therefore the signal transfer function of a Michelson interferometer, $T(\Omega)$, where the spectrum of the measurement of some quantity $\hat{X}(t)$ at the photodetector is $\tilde{\hat{X}}(\Omega) = T(\Omega) \tilde{h}(\Omega) + \text{noise terms}$ , will vanish for periods $2\pi/\Omega$ that are a multiple of the round-trip time of the light in the arms $2L/c$ where $L$ is the length of the arm and $c$ is the speed of light~\cite{}. %($\Omega$ is the angular frequency offset from the carrier frequency, explained further in Chapter~\ref{}) 
I will ignore this complication because the corresponding frequencies $f=\Omega/(2\pi)=c/(2L)=38\mathrm{kHz}$ for the arm length $L=4\mathrm{km}$ of LIGO~\cite{} are far above the kilohertz frequencies of interest (e.g.\ 1--4~kHz for a neutron-star post-merger signal).

\subsection{The role of optical cavities}
% why are there optical cavities

\begin{figure}
	\centering
	% \includegraphics[width=\textwidth]{}
	\caption{Dual-recycled Fabry-Perot Michelson interferometer, the various optical cavities are labelled. Abbreviations: ITM: input test mass, ETM: end test mass, PRM: power-recycling mirror, SRM: signal-recycling mirror.}
	\label{fig:DRFPMI}
\end{figure}

To the base Michelson interferometer, optical cavities are introduced to improve the sensitivity of the detector~\cite{}, principally around $100$~Hz. Three types of cavities are introduced: (1) arm cavities, (2) a power-recycling cavity, and (3) a signal-recycling cavity. 
Firstly, arm cavities (shown in Fig.~\ref{fig:DRFPMI}) are introduced via the placement of an input test mass (labelled as ITM in Fig.~\ref{fig:DRFPMI}) mirror after the beamsplitter, for each arm, forming a Fabry-Perot Michelson interferometer~\cite{}. These cavities significantly~\cite{} increase the circulating power in the arms which broadly improves the interferometer's sensitivity -- which will be shown in Chapter~\ref{chp:proposals}. 
Secondly, to further increase the power, a power-recycling cavity is introduced via a power-recycling mirror (labelled as PRM in Fig.~\ref{fig:DRFPMI}) before the beamsplitter, which reflects some of the otherwise wasted power in the common output mode back into the arms. 
Finally, a signal-recycling cavity is introduced via a signal-recycling mirror (labelled as SRM in Fig.~\ref{fig:DRFPMI}) at the output of the beamsplitter, which enhances the signal response by reflecting the output light back into the arms to be exposed to the gravitational wave for longer. Although the role of the signal-recycling cavity is more complicated than this, it changes the overall resonance behaviour of the interferometer and can be tuned to achieve a variety of outcomes from broadband to narrow-band enhancement~\cite{}, it will suffice until a formal model is introduced in Chapter~\ref{chp:proposals}. 
With these cavities the detector is called a Dual-Recycled Fabry-Perot Michelson interferometer, henceforth referred to as an interferometer. Without the improved circulating power and signal enhancement of these cavities, the detection of gravitational waves at around $100$~Hz would not have been possible~\cite{}, and therefore their inclusion is not optional~\cite{}. 

However, the resonance behaviour of these cavities limits the interferometer's kilohertz signal response. Optical cavities display resonance behaviour due to the different phases acquired by the light at different frequencies on each round-trip of the cavity~\footnote{Note that I will ignore all spatial behaviour of the light throughout this thesis.}. In the steady-state, the light entering a cavity interferes with the light that has undergone one round-trip, two round-trips, and so on. When the cavity is on-resonance for a particular frequency, this interference is constructive and the circulating power in the cavity can increase dramatically beyond the power of the incident light~\cite{}. And when the cavity is off-resonance, the converse is true and the circulating power is low. 
For an interferometer with the arm cavities on resonance at a carrier frequency, for frequency below kilohertz the sensitivity is improved by the power amplification from the arm cavities and the signal response is flat. However, at kilohertz, the arm cavities start going off-resonance and the signal response is far lower (by around an order for Advanced~LIGO~\cite{}) than at $100$~Hz -- although the arm cavities still improve the sensitivity. The other cavities in the interferometer also display resonance behaviour, but the arm cavities are the longest and therefore move off-resonance at the lowest frequency (they have the shortest bandwidth).
%longer arm cavities have shorter bandwidth and therefore matter at lower frequencies, such as kilohertz, as opposed to $10$~kHz etc.. 
Due to the phase interference being modulo $2\pi$, the arm cavity resonance is periodic with frequency $\frac{\omega_\text{FSR}}{2\pi}=\frac{c}{2L_\text{arm}}$, known as one free spectral range (FSR) of the cavity (approximately $37.5$~kHz for 4~km arms), meaning that the cavities will eventually become resonant again. However, I will make a single-mode approximation to the arms that will only be valid for frequencies below the first free spectral range, and so I only discuss the interferometer's response near the first resonance -- which covers the 1--4~kHz range of interest.

	% Arm cavities (circled in green in Fig.~\ref{fig:coupled_cavities}) and a power-recycling cavity are introduced to increase the power within the arms since this increases the sensitivity of an interferometric detector~\cite{1995AuJPh..48..953M}. A signal-recycling cavity (circled in blue in Fig.~\ref{fig:coupled_cavities}) is introduced to further enhance the signal by changing the overall resonance behaviour of the interferometer to increase the time that the light spends coupled to the gravitational wave. Optical cavities display resonance behaviour due to the interference condition on the phase acquired by the light on each round-trip of the cavity. These cavities increase low-frequency sensitivity around 100~Hz and without them, detection would not be possible. However, their resonance behaviour significantly decreases the finite bandwidth of the interferometer and therefore reduces high-frequency (1--4~kHz) sensitivity~\footnote{Only the first resonance of the cavities is considered here because the second resonance is far above (around 37~kHz) the frequency band of astrophysical interest (here, around 1--4~kHz). Moreover, the signal response of the Michelson interferometer falls off at higher frequencies which makes the detector less sensitive at each successive resonance.}. This project considers interferometer configurations that modify the signal-recycling cavity to improve high-frequency sensitivity.

\subsubsection{Simplification to pair of coupled cavities}
% does this need to be in the introduction or can it wait until Chapter 3?
% single mode approximation
\label{sec:coupled_cavity_approximation}

\begin{figure}
	\centering
	% \includegraphics[width=\textwidth]{}
	\caption{Coupled cavity approximation to the interferometer in Fig.~\ref{fig:DRFPMI}. The approximation is only valid when it is below one free spectral range of the arms.}
	\label{fig:coupled_cavity_approx}
\end{figure}


To conceptually simplify the interferometer to make it easier to understand and model, I make a single-mode approximation to the light: (1) in the differential mode of the arm cavities and (2) in the signal-recycling cavity. %Meaning what? Only the light at the carrier frequency is studied? Rather, only the light at one of the sidebands of the carrier frequency?
The motivation for this simplification is that the actual mode in each of the arms is less important than the differential mode of the interferometer which can be approximated as coming from a single ``arm'' cavity~\cite{}, coupled to the signal-recycling cavity, as shown in Fig.~\ref{fig:coupled_cavity_approx}. This is not the same as removing an arm from a Michelson interferometer, the new arm cavity mode is the differential mode of the original interferometer. 
This simplification is common in the literature~\cite{Korobko2019,Adya2020} and, although a more complete understanding requires more modes, is accurate below one free spectral range of the arm cavities~\cite{Miaoetal2015,}. In this approximation to the interferometer, the laser source and the power-recycling cavity are abstracted away into the circulating power in the arm cavity which is fixed as an input to the model.
% This simplification allows the interferometer to be understood as a pair of coupled cavities, a two-mode system, and makes clear why certain quantities, like the sloshing frequency of the two modes, should matter to the interferometer -- which is less clear when the interferometer is directly modelled.


\subsection{Factors limiting kilohertz sensitivity}
\label{sec:intro_factors_limiting_kHz}
% explain why current GWD are not sensitive at kilohertz
% what currently prevents us detecting them

Gravitational-wave detectors are not sensitive at kilohertz because: (1) the signal response drops off due to the arm cavity resonance, (2) the noise response is dominated by quantum shot noise that is flat in frequency, (3) the integrated sensitivity (the ratio of the signal response to the noise response) cannot improve by classical means without increasing the circulating power, and (4) the circulating power cannot increase without further technological progress.
The first factor limiting kilohertz sensitivity, the resonance response of the arm cavities, has been addressed above; I address the latter three factors in turn below.

\subsubsection{Noise at kilohertz is shot noise--dominated}
% at kilohertz, going by Buikema et al, 2020: laser intensity and frequency, photodetector dark (electronic) noise, thermal (is falling off)
% It also bears mentioning the other noise sources that affect intereferometers at frequencies below kilohertz: control sensor noise, beam jitter, thermal, seismic, newtonian, gas. But I will not consider any of these in the rest of the thesis.

\begin{figure}
	\centering
	% \includegraphics[width=\textwidth]{}
	\caption{Noise budget of Advanced LIGO from Ref.~\ref{Buikemeetal2020} \jam{(Used without permission? Can I do this?)}. At kilohertz, shot noise dominates and therefore I do not consider other noise sources than quantum noise.}
	\label{fig:Buikemeetal2020_LIGO_noise_budget}
\end{figure}


A real interferometer in subject to noise from many sources, at kilohertz these include, in approximate order of magnitude: quantum shot noise, laser intensity noise, photodetector dark noise, laser frequency noise, and thermal noise~\cite{Buikemeetal2020}~\footnote{Below kilohertz, there are many additional noise sources, such as noise from control systems, beam jitter, seismic noise, Newtonian noise, and gas noise, but I will not consider any of these in this thesis.}. However, at kilohertz, the noise response is dominated by shot noise, with the sum of contributions from all other noise sources making up less than half the total noise~\cite{}, as shown in Fig.~\ref{fig:Buikemeetal2020_LIGO_noise_budget} taken from Ref.~\cite{Buikemeetal2020}. As such, I consider quantum noise as the only noise source throughout this thesis. \jam{(what about thermal noise in sWLC?)}

\begin{figure}
	\centering
	% \includegraphics[width=\textwidth]{}
	\caption{Quantum noise transfer function (top panel), signal transfer function (middle panel), and quantum noise--limited sensitivity shown as noise-to-signal ratio (bottom panel) of a gravitational-wave interferometer. At kilohertz, the sensitivity is limited by shot noise and the decreasing signal due to the arm cavity going off-resonance.}
	\label{fig:simplified_sensitivity}
\end{figure}

To explain briefly, shot noise arises from the fundamental, quantum uncertainty in the phase of a state of light~\cite{} \jam{(check this, the models have noise in every quadrature)}. The vacuum fluctuations of virtual photons at a particular frequency mean that even the vacuum will have some uncertainty, a measurable variance, in its phase, distributed about zero. %phase is not a real thing, mention quadratures
Shot noise is flat in, i.e.\ independent of, frequency which leads it to currently dominate at kilohertz simply because all other noise sources have been reduced below it by decades of research and innovation~\cite{}. For sensitivity, approximated henceforth as quantum noise--limited sensitivity, the ratio of the signal response to the quantum noise response of the interferometer, because the signal is decreasing and the shot noise is constant, the sensitivity decreases at kilohertz, as shown in Fig.~\ref{fig:simplified_sensitivity}. It is conventional to show sensitivity as the noise-to-signal ratio (also known as signal-normalised noise) in the gravitational-wave literature, so lower values indicate better sensitivity in the bottom panel of Fig.~\ref{fig:simplified_sensitivity}.
A detailed explanation of quantum noise will be given in Section~\ref{sec:qnoise}.


\subsubsection{Limit on integrated sensitivity}
%limit on the quantum noise--limited sensitivity
% mizuno limit (due to QCRB) on the integrated sensitivity, i.e. on the bandwidth-peak sensitivity product --> can be beaten by squeezing

The Mizuno limit states that the circulating power within the arms limits an interferometer's integrated sensitivity, i.e.\ the product of its bandwidth and peak sensitivity~\cite{miaoFundamentalQuantumLimit2017}.
This limit comes ultimately from the Quantum Cramer-Rao Bound where the circulating power quantifies the total energy of the detector~\cite{} \jam{(explain the physics?)}.
For a fixed circulating power, this limit means that improving bandwidth to include kilohertz sensitivity would sacrifice peak sensitivity across the band, compromising existing low-frequency ($\sim100$~Hz) sensitivity. % beyond the point of being useful for detection. %justify this?
While this thesis is motivated by kilohertz detection, this low-frequency sensitivity should be maintained. For the case example of observing binary neutron-star mergers, maintaining low-frequency sensitivity would allow for the pre-merger inspiral to be seen, hopefully along with the merger and post-merger remnant at kilohertz, necessary to construct the full description of the merger~\cite{}.
Therefore, this limit poses a problem to improving sensitivity at kilohertz to that required for detection.


\subsubsection{Technological limitation: circulating power}
\label{sec:circulating_power}
% Why can circulating power not increase?
% constraints for future detectors
% I assume that we can't increase power by a factor of 2, unlike Miao et al 2018

The Mizuno limit on improving kilohertz sensitivity could be alleviated if the circulating power could be increased, however, this is not currently technologically possible. In Advanced~LIGO, which operates at $750$~kW circulating power in the arms, point absorbers on the optics~\cite{Brooks_2021} and parametric instabilities (particularly when a large number of unstable modes are available)~\cite{PhysRevLett.114.161102} prevent increasing the circulating power far beyond its current value. This poses a technological challenge to future detectors which demand upwards of $3$~MW for LIGO~Voyager~\cite{} and $4.5$~MW for NEMO~\cite{}, a challenge yet to be solved but with much progress towards it~\cite{}. %\jam{(Need further information? For Evans et al, 2015, didn't they alleviate the parametric instability problem?)}
Therefore, I will assume that $\sim3$~MW circulating power is the maximum achievable value for the next generation of detectors, assuming significant technological progress in the coming decades. Since arbitrary circulating power is not possible, a different method is required to improve kilohertz sensitivity.


	% The amount of laser power within an interferometric gravitational-wave detector limits how much its high-frequency sensitivity can improve without its low-frequency sensitivity worsening or the power needing to increase~\cite{miaoFundamentalQuantumLimit2017}. 
	% However, gravitational waves require a wide band of sensitive frequencies to detect as they are typically not monochromatic~\cite{miaoDesignGravitationalWaveDetectors2018}, and further increasing the power is technologically problematic~\cite{Brooks_2021,PhysRevLett.114.161102}. 

\subsubsection{How to improve kilohertz sensitivity}

The above four factors: the arm cavity resonance, shot noise, the Mizuno limit, and power limitations, can be avoided and kilohertz sensitivity can be improved, but this requires changes to the detector's configuration. Two configurations have been recently proposed to improve kilohertz sensitivity: (1) degenerate internal squeezing~\cite{Korobko2019,Adya2020} and (2) stable optomechanical filtering~\cite{Li2020,Miao2015}. 
The Mizuno limit, due to its origin from the Quantum Cramer-Rao Bound, can be avoided by non-classical techniques, such as squeezing~\cite{}, a technique which trades off the quantum uncertainty in a desired quantity for that in a less desired quantity, detailed in Chapter~\ref{chp:background_theory}. Squeezing is already used in Advanced~LIGO to halve the shot noise~\cite{tseQuantumEnhancedAdvancedLIGO2019} across the band. Degenerate internal squeezing is a proposal that uses additional squeezing, inside the signal-recycling cavity, to further enhance the sensitivity by interacting with the signal and noise, detailed in Section~\ref{}. 
The limit can also be avoided by instead addressing the arm cavity resonance, such as by introducing sufficient additional phase at each frequency to make the arm cavity more broadly resonant. This ``white-light cavity'' idea~\cite{Miao2015,} is what stable optomechanical filtering achieves by coupling a mechanical mode to the signal-recycling cavity mode, detailed in Section~\ref{}.

% although frontrunners in the literature~\cite{}
However, these two configurations have their problems. Degenerate internal squeezing is susceptible to optical loss because the configuration decreases the signal and relies on correlations degradable by optical losses~\cite{Korobko2019}. Stable optomechanical filtering requires a quiet mechanical mode which demands better thermal noise and mechanical quality factor than are currently technologically possible~\cite{Li2020,Miao2015}. These problems could limit the feasibility of these two configurations for kilohertz detection, and therefore they motivate looking for alternative configurations. %But any alternative proposal must first avoid the limit set by the above four factors.


%%%%%%%%%%%%%%%%%%%%%%%%%%%%%%%%%%%%%%%%%%
\section{Thesis outline}
% what the thesis contributes towards the problem: the aims from the research proposal
% quantum noise--limited sensitivity
% overview of the problem: we can't see at HF, (to be covered) and the problems with exisiting proposed solutions (the latter will be detailed more in the relevant background chapters)

In this thesis, I examine an alternative gravitational-wave detector configuration, nondegenerate internal squeezing, which combines the three-mode structure of stable optomechanical filtering with the all-optical approach of degenerate internal squeezing, in the hope that it might be more feasible for kilohertz gravitational-wave detection. This configuration avoids the four factors limiting kilohertz improvement, outlined above, by using squeezing to get around the Quantum Cramer-Rao Bound. This means that it does not require increased circulating power nor compromised low-frequency sensitivity, at least a priori. Although this configuration has been briefly mentioned in the literature~\cite{liBroadbandSensitivityImprovement2020} and the lossless Hamiltonian model is the same as stable optomechanical filtering~\cite{liBroadbandSensitivityImprovement2020}, no work has, as yet, studied it in detail nor assessed its feasibility for kilohertz detection~\cite{}. The goal of this thesis is to fill in that gap in our understanding.

\jam{(Review this after drafting the other chapters)}

I will first review the background theory of quantum noise and squeezing in Chapter~\ref{chp:background_theory}, to describe the quantum noise response of a detector. In this chapter, I will demonstrate the analytic, Hamiltonian modelling, that I will use throughout the thesis, on the simple, well-known cases of degenerate and nondegenerate optical parametric oscillators~\cite{}. Next, in Chapter~\ref{chp:proposals}, I will review the two proposals for kilohertz detection that I will later compare my work against degenerate internal squeezing and stable optomechanical filtering. This chapter will use the same Hamiltonian modelling for degenerate internal squeezing as the previous chapter but will introduce the detector's quantum noise--limited sensitivity to gravitational-wave signals. After establishing this formal model, I will be able to explain the problems with these configurations and set up the motivations for nondegenerate internal squeezing.

% quantum optics of nIS
% GW detection science case for nIS
I will then detail my Hamiltonian model of nondegenerate internal squeezing in Chapter~\ref{chp:nIS_analytics}, complete with various sources of optical loss. This derivation should reflect the modelling in the previous two chapters and therefore be reasonable and followable. Using the results of the model, I will characterise the stability and sensitivity of the configuration, demonstrate its reduction to known models, and derive a new definition of the limit of the amount of available squeezing given the assumptions of the model. Finally, in Chapter~\ref{chp:science_case}, I will examine the science case of nondegenerate internal squeezing for gravitational-wave detection, particularly kilohertz detection. Using realistic losses and the parameter space of future detections, I will argue that \jam{(... I don't know the results yet, maybe it will be that a broadband detector is feasible if idler loss can be kept low, fill this in later!)}. Comparing this feasibility to the two existing configurations, I will conclude that \jam{(..., fill this in later)}. \jam{(Idler readout chapter 6 not yet added here.)} I will finish by considering the limitations of my research and what avenues of future work might be able to support and enrich the understanding of nondegenerate internal squeezing and the possible ways to achieve kilohertz detection.

% return to the motivation
% The work in this thesis also has broader applications to cavity-based quantum metrology.
This thesis also contributes a small understanding to the broader field of cavity-based quantum metrology. While the motivation behind this work is to detect kilohertz gravitational waves, to advance astrophysics through observing new sources, the model in Chapter~\ref{chp:nIS_analytics} could be applied to other problems, such as axion detection for dark matter~\cite{}.


\jam{(Include a statement of contributions?)}

\jam{(Include a list of abbreviations? E.g.\ LIGO, KAGRA, NEMO, OPO, PT, ITM, ETM, SRM, PRM, SRC. And/or a list of symbols for the analytic models?)}


% From research proposal
	% I aim to design and analytically model an all-optical interferometer configuration to improve high-frequency sensitivity beyond current detectors, while neither fully sacrificing low-frequency sensitivity nor increasing the power. I will include realistic sources of loss in the model to assess the feasibility of this configuration. Finally, I will compare this configuration to existing proposals to show the potential benefits of its design.

	% The central aim of this project is to investigate nondegenerate internal squeezing and assess how it might improve the high-frequency sensitivity of future detectors. Although nondegenerate internal squeezing has been considered before, no work has fully examined it nor assessed its feasibility by including losses in the model~\cite{liBroadbandSensitivityImprovement2020}. I will perform an analytic investigation of nondegenerate internal squeezing that will include a feasibility assessment using losses and realistic assumptions and conclude with a comparison to previously proposed configurations.

	% I will use a Hamiltonian method to analytically model nondegenerate internal squeezing and
	% calculate its quantum-noise limited sensitivity. The Hamiltonian for nondegenerate internal
	% squeezing is already known for the approximate interferometer model of a pair of coupled cavities shown in the right panel of Fig.~\ref{fig:coupled_cavities}~\cite{korobkoQuantumExpanderGravitationalwave2019,liBroadbandSensitivityImprovement2020}. %cite Schori et al?
	% This Hamiltonian includes standard harmonic oscillator terms for each of the optical fields in each cavity and at each frequency and interaction terms that represent the squeezing process and coupling between the cavities. I will solve the quantum Langevin equations of motion to find the output signal and noise fields that are measured by the photodetector and calculate the sensitivity of the overall detector. These equations combine the Heisenberg equations of motion of the field operators with input/output terms. To solve them, I will use common approximations including a semi-classical approximation of the pump field to treat it as a reservoir and a linear expansion in small fluctuations around the expectation values of certain operators. This Hamiltonian method and the approximations involved are widely used in the literature and, in particular, in the works that I will compare to and base my derivation upon~\cite{liBroadbandSensitivityImprovement2020,korobkoQuantumExpanderGravitationalwave2019}. I will repeat the analysis of stable optomechanical filtering done in Ref.~\cite{liBroadbandSensitivityImprovement2020} to practice this Hamiltonian method before attempting the novel derivation of nondegenerate internal squeezing with realistic assumptions.

	% I will assess the feasibility of nondegenerate internal squeezing by studying the effects of various noise sources on the sensitivity under realistic assumptions. The degenerate internal squeezing literature suggests that optical loss will be the dominant noise source of nondegenerate internal squeezing~\cite{korobkoQuantumExpanderGravitationalwave2019}. Therefore, I will prioritise including optical loss in the model above other noise sources. I will also include photodetector loss to test the hypothesis that nondegenerate internal squeezing is more resistant to photodetector loss than degenerate internal squeezing. Under realistic assumptions about improvements in optical loss and squeezing beyond current detectors, I will compare the sensitivity of the model to the estimated sensitivity required to detect a typical binary neutron-star post-merger signal~\cite{miaoDesignGravitationalWaveDetectors2018}. This will determine how large an improvement in these factors will need to be achieved, technologically, before the configuration is physically viable.

	% Finally, I will compare nondegenerate internal squeezing to degenerate internal squeezing and stable optomechanical filtering by comparing the sensitivity and feasibility of the three systems. Nondegenerate and degenerate internal squeezing have not been compared before. Similarly, although the connection has been previously made between the Hamiltonians of nondegenerate internal squeezing and stable optomechanical filtering~\cite{yapadyaPersonalCommunication,liBroadbandSensitivityImprovement2020}, the sensitivity and feasibility of the two systems have not been compared before. The deeper connection to stable optomechanical filtering means that I will prioritise comparing my model to it over degenerate internal squeezing.


%%%%%%%%%%%%%%%%%%%%%%%%%%%%%%%%%%%%%%%%%%
\section{Chapter summary}

% In this chapter, I motivate the detection of kilohertz gravitational waves, describe what currently prevents us detecting them, and outline what this thesis contributes to detecting them in the future.

In this chapter, I have described gravitational waves and motivated the detection of kilohertz gravitational waves for advancing our astrophysical understanding of neutron stars, black holes, and supernovae. I have explained how detectors based on the Michelson interferometer can detect gravitational waves at around $100$~Hz, why cavities are introduced into these detectors, and why these detectors' sensitivity cannot be simply broadened to detect kilohertz gravitational waves. Finally, I have mentioned the two proposed configurations that motivate my work (degenerate internal squeezing and stable optomechanical filtering), what problems these configurations face, and that this thesis will examine nondegenerate internal squeezing as an alternative configuration for gravitational-wave detection.


