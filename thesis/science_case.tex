\chapter{Nondegenerate internal squeezing for gravitational-wave detection}
\label{chp:science_case}

% be clear about the scope of this project, I am not wanting to recommend the design of the next detector to be built, this is exploratory work into a configuration that has never been modelled this thoroughly before -- more would need to be done (e.g. add in external squeezing and thermal noise etc.) to make a proper judgement which is not my aim.
% comparison to existing proposals
% conclusions from this chapter: nIS is feasible as an alternative to sWLC and to improve the sensitivity from 100-1000 Hz, but improving 1-4 kHz sensitivity appears less likely without changing the sloshing frequency and arm bandwidth

In this chapter, I consider using nondegenerate internal squeezing in a future gravitational-wave detector. This is exploratory work and I will not recommend the optimal configuration for a future detector which would require a more detailed model. Instead, I will focus on the general feasibility of nondegenerate internal squeezing and whether it warrants further investigation.
Firstly, I will examine the tolerance of nondegenerate internal squeezing to the realistic optical losses in a future detector like LIGO~Voyager as described in Chapter~\ref{chp:proposals}. I will also discuss the optimal squeezing value for the sensitivity. Secondly, I will compare the sensitivity of nondegenerate internal squeezing to stable optomechanical filtering and argue that the low optical loss required for the all-optical configuration is as least as realistic as the low mechanical loss required for the optomechanical configuration, meaning that the nondegenerate internal squeezing is a viable, all-optical alternative. I will also compare its tolerance to optical loss to degenerate internal squeezing. Finally, I will compare the sensitivity to the target for kilohertz gravitational-wave detection from Section~\ref{sec:GW_kilohertz_target}. I will find that the kilohertz improvement is promising enough to warrant further investigation. Although this work was motivated by improving kilohertz sensitivity, I will also discuss whether the configuration is better suited to a broadband detector. %, which will be further explored in the next chapter.

\section{Tolerance to optical loss}
\label{sec:nIS_tolerance_to_losses}

% % table: parameter sets to compare
% \begin{table} %https://www.tablesgenerator.com/
% \centering
% \begin{tabular}{lllllll}
% parameter & Advanced LIGO & LIGO Voyager & NEMO & Li2020 & Korobko2019 & Miao2018 \\
% carrier wavelength &  &  &  &  &  &  \\
% arm cavity length &  &  &  &  &  &  \\
% signal-recycling cavity length &  &  &  &  &  &  \\
% circulating arm power &  &  &  &  &  &  \\
% input test mass transmissivity &  &  &  &  &  &  \\
% signal-recycling mirror transmissitivty &  &  &  &  &  &  \\
% test mass mass &  &  &  &  &  &  \\
% sloshing frequency &  &  &  &  &  &  \\
% signal readout rate &  &  &  &  &  &  \\
% optical losses &  &  &  &  &  & 
% \end{tabular}
%     \caption{\jam{(Fill in table, what detectors/parameters should be shown?)} Table of interferometer parameter sets, showing configuration and derived parameters. The parameter set from Ref.~\cite{} is based on LIGO~Voyager but directly sets the readout rate $\gamma_R$ and sloshing frequency $\omega_s$ and back-forms the corresponding physical lengths and reflectivities. \jam{(Freedoms in this process)}}
%     \label{tab:parameter_sets}
% \end{table}

% plot: tolerance to each of the four sources, matrix plot each against readout rate
% some way to mega matrix all of these, or just show sensitivity and not N and S separetely?
\begin{figure}
    \centering
    \includegraphics[width=\textwidth]{nIS_sigRO_tolerance_Rpd.pdf}
    \caption{\jam{(Purpose: show the different tolerance to different losses 1/4)} \jam{(Why does squeezer-off RP worsen with readout rate, is it because of more main vacuum?)} Nondegenerate internal squeezing tolerance to loss (1/4): detection loss. The loss uniformly scales the signal to zero and the noise to the vacuum value. The squeezed system appears generally as resilient as the non-squeezed system and more resilient around the peak, which is due to the use of anti-squeezing instead of squeezing and is an advantage over degenerate internal squeezing, see Section~\ref{sec:dIS_optical_loss}. The tolerance is independent of the readout rate but at high readout rates, e.g. 50~kHz, the squeezer does not improve the sensitivity \jam{(explain why?)}. I use the LIGO~Voyager parameters from Section~\ref{sec:literature_review} with varied signal-recycling cavity length \jam{(change this?)} and squeezer parameter normalised to the singularity threshold for each curve, henceforth.}
    \label{fig:nIS_sigRO_tolerance_Rpd}
\end{figure}
\begin{figure}
    \centering
    \includegraphics[width=\textwidth]{nIS_sigRO_tolerance_Tlb.pdf}
    % Show effect of loss on squeezer-off system, likewise for following plots. --> too many curves
    \caption{\jam{(Purpose: show the different tolerance to different losses 2/4)} \jam{(Check the effect of loss on the squeezer-off system, likewise for following plots, and mention it in the text. Check how changing the readout rate differently, i.e.\ Lsrc or Tsrm, affects the results, because only one affects the intra-cavity loss.)} Nondegenerate internal squeezing tolerance to loss (2/4): signal mode intra-cavity loss in the signal-recycling cavity. The loss decreases the sensitivity around the signal peak and at higher frequencies depending on the readout rate, but not at low frequencies except with high readout rates, e.g.\ 50~kHz. The system appears highly resilient to realistic levels of this loss compared to, e.g., idler loss, with even the unrealistic $10\%$ loss only causing a factor of two degradation in the peak sensitivity at low readout rate \jam{(check this)}. At higher readout rates, e.g.\ 50~kHz, the resilience worsens but remains high compared to other losses \jam{(quantify and explain this)}. The peak frequency of the signal and noise amplification changes because the loss changes the singularity threshold frequency \jam{(is there a physical reason?)}.}
    \label{fig:nIS_sigRO_tolerance_Tlb}
\end{figure}
\begin{figure}
    \centering
    \includegraphics[width=\textwidth]{nIS_sigRO_tolerance_Tlc.pdf}
    \caption{\jam{(Purpose: show the different tolerance to different losses 3/4)} Nondegenerate internal squeezing tolerance to loss (3/4): idler mode intra-cavity loss. For readout rates around and below 5~kHz, the loss decreases the peak sensitivity and sensitivity from 100-1000~Hz but improves the radiation-pressure noise. The system appears more vulnerable to realistic levels of idler loss, e.g.\ 1000~ppm, than the other losses, even when the readout port is closed to the idler. At increased signal readout rate, this tolerance worsens until at 50~kHz readout rate it improves \jam{(quantify)} but the squeezer does not improve the sensitivity. The peak frequency of the shot noise and signal changes because the threshold frequency changes.}
    \label{fig:nIS_sigRO_tolerance_Tlc}
\end{figure}
\begin{figure}
    \centering
    \includegraphics[width=\textwidth]{nIS_sigRO_tolerance_Tla.pdf}
    \caption{\jam{(Purpose: show the different tolerance to different losses 4/4)} Nondegenerate internal squeezing tolerance to loss (4/4): arm intra-cavity loss. At large readout rates, e.g.\ above 5~kHz \jam{(check)}, the arm intra-cavity loss appears to have a negligible effect on the signal or noise \jam{(quantify this)}, while at 500~Hz there is a sharp change around $T_{l,a}>T_{l,c}$ \jam{(check this)} above which the sensitivity degrades approximately to the no-squeezer sensitivity \jam{(quantify)} but with worse radiation-pressure noise. This is the high arm loss limit of the singularity threshold, see Section~\ref{sec:nOPO_reduction}, where the configuration is closer to the OPO limit than to the lossless PT-symmetric analogue \jam{(why doesn't this change the plots for high readout rates?)}. However, for realistic losses, $T_{l,a}<T_{l,c}$ is expected and the loss has negligible effect \jam{(quantify)} independently of readout rate.}
    \label{fig:nIS_sigRO_tolerance_Tla}
\end{figure}
\begin{figure}
    \centering
    \includegraphics[width=\textwidth]{nIS_sigRO_noise_budget.pdf}
    \caption{\jam{(Purpose: show which noise dominates.)} Nondegenerate internal squeezing separate noise transfer functions for each loss port and the total noise response. The detection loss only affects the shot noise after the interferometer and therefore is flat. The other losses are all squeezed and affected by the radiation-pressure noise because they are coupled (in)directly to the signal-recycling and arm cavity modes. Below 100~Hz, the idler loss dominates the noise, and above 100~Hz the vacuum from the readout port dominates the noise. The other losses have minimal effect, e.g.\ they are 10~dB below the total noise. This is due to the relative size of the readout $T_\text{SRM}=0.046$ and realistic loss rates and the different tolerances to the different losses.}
    \label{fig:nIS_sigRO_noise_budget}
\end{figure}

% summarise the constraints from Chapter 3, give the table of parameters
I will compare how the sensitivity of nondegenerate internal squeezing degrades with each of the optical losses in the model. I will use the parameter set in Section~\ref{sec:literature_review}, which is from Ref.~\cite{Li2020} and based on LIGO~Voyager but with a different readout rate and sloshing frequency, and the realistic losses from Section~\ref{sec:dIS_optical_loss}. I will vary the signal readout rate of the detector by changing the length of the signal-recycling cavity \jam{(clarify that the signal-recycling mirror transmissivity could also be changed, and try this. Explain why the sensitivity does not improve at 50~kHz readout rate.)}~\footnote{Since I fix the sloshing frequency $\omega_s$ and change the signal readout rate $\gamma^b_\text{tot}$, if the signal-recycling cavity length changes then so must the input test mass transmissivity.}, as the range of future detectors considered in the literature can be partially characterised by different readout rates~\cite{}. \jam{(Do I need a table of parameters here?)} E.g.\ the sloshing frequencies of Refs.~\cite{,,} lie within 2--6~kHz but their signal readout rates vary from 0.5--90~kHz. As discussed in Section~\ref{sec:literature_review}, I compromise which parameters are fixed (e.g.\ circulating power, sloshing frequency) and which are varied (e.g.\ readout rate, losses, pump power) because I cannot examine the entire parameter space, which is why I do not claim that these parameters are optimal. For this parameter set, I consider the tolerance to the four losses, detection, signal, idler, and arm, in turn.

\jam{(In the figures only show sensitivity, but say that I analyse them by examining the effect on noise and signal responses. Check that this is sufficient justification of the claims.)}

% resistant to detection losses, which is a big deal, although Caves's amplifier could alleviate this more generally
Firstly, as shown in Fig.~\ref{fig:nIS_sigRO_tolerance_Rpd}, detection loss affects nondegenerate internal squeezing similarly to the base coupled-cavity interferometer except around the anti-squeezed peak, and its tolerance does not depend on the readout rate. Detection loss uniformly scales down the signal response and pulls the noise response towards the vacuum value. At the peak and when radiation pressure noise dominates, the quantum noise is far enough above the vacuum level that the reduction in noise and signal are roughly the same and the sensitivity does not worsen. Away from the peak, the tolerance diminishes as the anti-squeezing decreases but the tolerance to detection loss is never worse than the base interferometer. This is unlike degenerate internal squeezing for the same losses in Section~\ref{sec:dIS_optical_loss} where the reliance on squeezing instead of anti-squeezing makes that system more vulnerable to detection loss. However, both configurations can improve the detection loss by the inclusion of a Caves's amplifier from Section~\ref{sec:cavess_amp} which uses the same principle of anti-squeezing.

% resistant to signal loss, sensitive to arm loss, very sensitive to idler loss (SRM must be closed to idler through e.g. dichroic)
Secondly, as shown in Fig.~\ref{fig:nIS_sigRO_tolerance_Tlb}, signal mode intra-cavity loss at the realistic level has a negligible \jam{(quantify)} effect on nondegenerate internal squeezing. This is because the signal mode is dominated by loss through the readout port at $T_\text{SRM}=0.046=46000$~ppm compared to $T_{l,b}=1000~\text{ppm}$, where to change the readout rate I fix the transmissivity of the signal-recycling mirror and change the cavity length \jam{(why don't I just change Tsrm? do this to compare.)}. However, in the high signal loss limit, e.g.\ $T_{l,b}=0.1$, the loss dominates the readout rate, the peak frequency changes because the singularity threshold frequency is affected, and the tolerance worsens with the readout rate. \jam{(why?)} % because the bandwidth of the peak increases or the shorter cavity length at high readout rates also increases the intra-cavity loss rate?
But this is not of concern to future detectors which are in the low signal loss regime.

% changing the readout rate changes the cavity length and therefore increases the idler loss
Thirdly, as shown in Fig.~\ref{fig:nIS_sigRO_tolerance_Tlc}, idler intra-cavity loss at the realistic level with zero idler readout rate already significantly degrades the sensitivity \jam{(quantify)}. And opening the idler readout port is equivalent to further increasing the idler loss for signal readout~\footnote{E.g.\ a readout port symmetric between signal and idler increases the idler loss to a transmissivity of $0.046$ in Fig.~\ref{fig:nIS_sigRO_tolerance_Tlc} \jam{(update this if Tsrm changes)}.}. Therefore, the idler readout port should be closed for signal readout. Increasing the idler loss changes the peak frequency because of the singularity threshold frequency, but unlike signal loss, it also decreases the radiation-pressure noise \jam{(why?)}. With zero idler readout rate, the decrease in sensitivity from 100-1000~Hz by increasing $T_{l,c}$ from 100 to 1000~ppm is similar to introducing $50\%$ detection loss \jam{(check this, does this idler loss dominate 0.1 detection loss?)}. This effect is increased at higher signal readout rates because as the length of the cavity decreases all of the signal and idler loss rates increase \jam{(fix this?)}. 
% idler loss modal equivalence to mechanical loss dominating sWLC?
That idler loss is the dominant~\footnote{As measured in sensitivity change due to optical loss. The vacuum from the signal readout port is still the main vacuum source.} optical loss with these realistic losses \jam{(check this)} agrees with the optomechanical analogue being dominated by mechanical loss in the mechanical idler mode, see Section~\ref{sec:sWLC_loss}. Unlike detection loss or the vacuum from the readout port, idler intra-cavity loss cannot be lowered by the use of external squeezers from Section~\ref{sec:external_squeezing}, making it a greater problem for using this configuration in future detectors.

Finally, as shown in Fig.~\ref{fig:nIS_sigRO_tolerance_Tla}, arm intra-cavity loss at the realistic level below 100~ppm has a negligible \jam{(quantify)} effect on the sensitivity. Although in the high arm loss limit in Section~\ref{sec:nOPO_reduction}, the noise response changes to become closer to a nondegenerate OPO than the lossless system, future detectors lie outside this regime since the realistic arm loss is small. 
% Namely, future detectors lie in the regime explained in Section~\ref{sec:singularity_threshold} where the singularities have not merged, given by $$\gamma^c_\text{tot}\omega_s^2\geq\gamma_a\left(\omega_s^2+\gamma_a(\gamma^b_\mathrm{tot}+\gamma^c_\text{tot})\right).$$
But this limit explains the change in behaviour for $T_{l,a}=0.01$ \jam{(why not when it's equal to idler loss?)} in Fig.~\ref{fig:nIS_sigRO_tolerance_Tla} when the signal readout rate is large \jam{(why?)}.

% summarise tolerances, noise budget in Fig.~\ref{fig:nIS_sigRO_noise_budget}
\jam{(explain noise budget more)}
Therefore, as shown in Fig.~\ref{fig:nIS_sigRO_noise_budget}, for realistic losses, the noise of nondegenerate internal squeezing is dominated by idler loss whether or not the idler readout port is open, the detection loss has a smaller effect \jam{(quantify)}, and signal and arm intra-cavity losses have negligible effects \jam{(check)}. This agrees with the optomechanical analogue being limited by mechanical idler loss, see Section~\ref{sec:sWLC_loss}. 
\jam{(have I wrung all the information out of these plots?)}


\subsection{Optimal squeezing}
\label{sec:nIS_optimal_squeezing}

% plot optimal squeezing curve (blue-green swoosh)
\begin{figure}
	\centering
	\includegraphics[width=0.9\textwidth]{nIS_optimum_squeezing.pdf}
	\caption{\jam{(Purpose: show that optimum is below threshold. Is this plot necessary?)} \jam{(Could cut the signal vs noise subplot.)} Nondegenerate internal squeezing's sensitivity versus noise (top panel) and signal versus noise (bottom panel) at a given (or ``probe'') frequency of $2.5$~kHz which is above the singularity threshold frequency \jam{(which is what value?)}, varying the squeezing parameter from $0$ to $0.95$ for realistic losses. The sensitivity, noise, and signal are measured relative to their respective reference values without squeezing, which for the noise is the vacuum value because the chosen frequency is shot noise--dominated. Increasing the squeezing parameter increases the signal more than the noise up to a point. This shows that the optimal squeezer parameter for maximum sensitivity, shown with a red dot, is below threshold. Although the optimal squeezer parameter also maximises the signal here, this is not necessarily always the case. Compare to degenerate internal squeezing in Fig.~\ref{fig:dIS_optimal_squeezing}.}
	\label{fig:nIS_optimum_squeezing}
\end{figure}

\jam{(Is this subsection necessary? Could just move the explanation to singularity threshold section to explain that maximising sensitivity at a given frequency cannot find threshold. Could explain in comparison to dIS section whether the peak sensitivity frequency moves.)}

% emphasise that threshold is not optimal squeezing for sensitivity
% Without detection loss, the optimal squeezing parameter for sensitivity is below threshold because the probe frequency is higher than the on-threshold peak frequency and with increased squeezer parameter the peak moves to lower frequencies. After the peak passes the probe frequency \jam{(show this?)} the signal and the noise start decreasing \jam{(check this)}. Increasing detection loss $R_\text{PD}$ decreases the signal $T$ as $\sqrt{1-R_\text{PD}}T$.
The optimal amount of squeezing for the maximum sensitivity is not necessarily on threshold. As shown in Fig.~\ref{fig:nIS_optimum_squeezing}, the sensitivity at a given non-threshold frequency, here 2.5~kHz, peaks at a point before threshold beyond which the amount that the signal is amplified more than the noise decreases.
This is because the peak frequency of the signal and noise changes with the squeezer parameter, and the optimal sensitivity is when it is aligned with the given frequency. This is unlike degenerate internal squeezing in Fig.~\ref{fig:dIS_optimal_squeezing} where it remains at the sloshing frequency for realistically low loss. \jam{(The lossless case is not shown here because then the noise has no anti-squeezed peak. )}
% mention that this can be done for probe frequency, peak frequency, integral etc.
Conversely, this also demonstrates that using the sensitivity at a particular frequency cannot reliably find threshold, which is also true for other metrics such as peak and integrated sensitivity \jam{(is it? check this using optimisation?)}. % and instead some other metric, e.g.\ the peak or integrated sensitivity, should be used.
% Choosing the metric to judge a configuration by is a difficult task because of the innate trade-off of peak sensitivity and bandwidth. I will return to this problem later when I compare the kilohertz and broadband performances of nondegenerate internal squeezing.


\section{Comparison to existing proposals}

% The problem with making a judgement, emphasise that this result is not a clean decision, the best I can say is that they are comparable, nIS is feasible as an alternative to sWLC 
I will now compare the sensitivity of nondegenerate internal squeezing to the two existing proposals in Chapter~\ref{chp:proposals} to answer whether it is a viable, all-optical alternative to stable optomechanical filtering and whether it is more resistant to optical loss than degenerate internal squeezing. Neither of these comparisons will provide a definitive judgement on the best configuration for future detectors, but I will demonstrate that, at the very least, nondegenerate internal squeezing warrants equal consideration to these two configurations.

\subsection{All-optical versus optomechanical analogues}
\label{sec:nIS_vs_sWLC}
% ultimately: is the all-optical approach a viable alternative? yes! but sWLC is not ruled out

% plot: with data from Fig. 5 in Li --> extract the plot and don't mess with the .nb further!
\begin{figure}
	\centering
	\includegraphics[width=\textwidth]{nIS_vs_sWLC.pdf}
	\caption{\jam{(Purpose: show that nIS is comparable to sWLC)} Nondegenerate internal squeezing compared to stable optomechanical filtering where I use the data from Fig.~5 in Ref.~\cite{Li2020} with permission from the authors~\cite{personalCorrespondence}. The all-optical system's quantum noise--limited sensitivity with realistic optical loss is worse than the optomechanical system's quantum and thermal noise--limited sensitivity with low mechanical loss but is comparable to it if more ideal optical losses of 75~ppm in the arms and 100~ppm in the idler are used. I use the same ratio to singularity threshold of $98.6\%$ as the lossless case \jam{(check what sWLC does, is $\chi_m$ or the ratio fixed?)}. I validate this comparison using the lossless models and the model for a single-cavity detector (with no signal-recycling cavity), which agree with Ref.~\ref{Li2020}.}
	\label{fig:nIS_vs_sWLC}
\end{figure}

As discussed in Section~\ref{sec:modal_equivalence}, the only difference in the models of nondegenerate internal squeezing and stable optomechanical filtering is that the idler mode is optical and mechanical, respectively. This means that the idler loss is modelled by the same Langevin terms in the Hamiltonian but with different realistic loss rates between optical and mechanical loss because of the different associated technologies. Therefore, to determine whether nondegenerate internal squeezing is a viable, all-optical alternative to the optomechanical analogue, I find the optical loss required to achieve the same sensitivity as the results in Ref.~\ref{Li2020} that assume low mechanical loss and determine whether this optical loss is as realistic as that low mechanical loss. 
% The key result is that I've found losses for nIS that replicate the results for sWLC: arm loss 75 ppm, idler loss 100 ppm. 
    %  Are these loss values more realistic than the thermal and mechanical constraints of sWLC? I think so, but I need back-up on this, forecasting future technological progress is hard to do rigourously or scientifically.
This is shown in Fig.~\ref{fig:nIS_vs_sWLC}, where I compare the sensitivity to Fig.~5 of Ref.~\cite{Li2020}. To validate this comparison, I confirm that the lossless sensitivity and the sensitivity of a single-cavity detector agree with that reference. I use the same ratio to singularity threshold of $98.6\%$ to compare the configurations \jam{(check if $\chi_m$ changes)} and show that reducing it by $3\%$ decreases the peak but increases the bandwidth for the all-optical configuration and therefore I expect the same behaviour for the optomechanical analogue \jam{(check this?)}. In Fig.~\ref{fig:nIS_vs_sWLC}, I show the sensitivity for the realistic losses from Section~\ref{sec:dIS_optical_loss} and for the lower losses of 75~ppm arm loss and 100~ppm idler loss with no idler readout rate. For these ``ideal'' losses, the peak sensitivity and bandwidth are better than the optomechanical configuration with low mechanical loss determined by $T_\text{env}=4$~K and $Q_m=8\times10^9$, but for the realistic losses they are worse \jam{(check this)}. Although predicting future technological progress is not rigorous, and so what is ideal and what is realistic is unclear, these ideal losses might be technologically possible~\cite{Zhang2020,,} and are at least as realistic \jam{(are they? what can I say here?)} as the mechanical loss required for the optomechanical analogue~\cite{,}. And because of the equivalence in Section~\ref{sec:modal_equivalence}, I predict that the tolerance to other factors, e.g.\ pump power, would be the same between the two configurations for equivalent losses. Therefore, nondegenerate internal squeezing is a viable, all-optical alternative to stable optomechanical filtering. \jam{(have I justified this conclusion? what are the limitations of this approach?)}


\subsection{Degenerate versus nondegenerate internal squeezing}
% do I need to include this? it would be nice --> above comparison is the focus though,  keep this brief, might not be a subsection worth to say
% how to quantify difference in tolerance? is nIS more resistant than dIS?

As discussed in Section~\ref{sec:nIS_tolerance_to_losses}, the tolerance to optical loss is different between nondegenerate and degenerate internal squeezing. Comparing the two configurations for the same interferometer parameters and realistic losses shown in Figs.~\ref{fig:dIS_realistic_loss}~\ref{fig:nIS_general_sens} \jam{(check that these figures compare, maybe use a different fig for nIS, and show lossless case in each)} shows that the nondegenerate case is more resilient to loss \jam{(quantify this, give total change in integrated sensitivity?)}. 
This is because of the difference between squeezing and anti-squeezing the signal and noise, discussed for detection loss in Section~\ref{sec:cavess_amp}, where loss always reduces the signal but can increase or decrease the noise depending on whether it is below or above the vacuum value, respectively. And although degenerate internal squeezing can also perform internal anti-squeezing, it is only optimal to do so when the losses are high and beyond the regime of future detectors, as discussed in Section~\ref{sec:dIS_results}. 
Therefore, as predicted in Section~\ref{sec:dIS_optical_loss}, nondegenerate internal squeezing is more resilient to optical loss than degenerate internal squeezing \jam{(can I claim this generally? what can I say that is substantive?)}. However, the sensitivity curves are not directly comparable and therefore for different parameter sets and metrics the degenerate case might be better, e.g.\ to not affect the radiation-pressure noise \jam{(why does nondegenerate affect RPN? do I also need to show the results for different parameters, e.g.\ Korobko2019?)}. %And Other considerations, such as the different tolerances to pump power, are necessary to decide between the two. 


Compared to the two configurations from Chapter~\ref{chp:proposals}, nondegenerate internal squeezing is neither definitively better nor worse, meaning that it warrants equal consideration as them \jam{(this evasive but what else can I say?)}, particularly for its relative resilience to loss. % particularly when the optical loss is high but the mechanical loss higher \jam{(I cannot directly compare these, what do I mean to say?)}.


\section{Feasibility for gravitational-wave detection}
\label{sec:nIS_sigRO_feasibility}

I now return to the motivating problem of improving kilohertz gravitational-wave sensitivity to detect new astrophysical sources. I will explore the feasibility of nondegenerate internal squeezing for both kilohertz (e.g. 1--4~kHz) and broadband (e.g. 100--4000~Hz) detection. I will reinforce that nondegenerate internal squeezing warrants further investigation by showing that its sensitivity is promising for detection even without external squeezing or increased circulating power. Again, I do not aim to find the best configuration for future detectors but instead will look at nondegenerate internal squeezing's performance for the LIGO~Voyager parameter set with varied readout rates which partially characterises the possible configurations.
% mention how optimisation could be done towards either goal against a variety of metrics, list some mutrics, but leave to future work
% Also, determining the best configuration would depend on the goal, e.g.\ kilohertz versus broadband sensitivity, and the metric used, e.g.\ sensitivity peak or integral against a kernel that biases certain frequencies~\cite{}, which I leave to future work with a more detailed model. 

% stress that the goal of this thesis is not to provide a recommendation to the designers of future gravitational wave detectors of what parameters to use, that task is much harder and would require the modelling of many other effects. This section (and thesis) is only exploratory in nature, and my conclusion is that nIS warrants more study because it appears to be able to get close enough to the targets (without increasing power) and is comparable to optomechanical alternative, and more resistant to detection losses than degenerate internal squeezing (operated in the squeezing not anti-squeezing pump phase).
% directly address the vagueness with ``is this configuration useful to GW detection'', talking about a particular parameter set or the best param set possible?

\subsection{Application to kilohertz detection}

% plot: curve optimised for kilohertz
\begin{figure}
    \centering
    \includegraphics[width=0.9\textwidth]{nIS_ideal_losses.pdf}
    \caption{\jam{(Purpose: talk about the sensitivity target and GW-detection feasibility)} \jam{(Confirm that $\alpha$ GW-coupling is correct.)} \jam{(Check external squeezing 10dB definition)} Nondegenerate internal squeezing's sensitivity compared to the kilohertz sensitivity target of $5\times10^{-25}\text{Hz}^{-1/2}$ from 1--4~kHz~\cite{Miao2018}. Under ideal conditions, i.e.\ $95\%$ threshold, arm and idler losses at 75 and 100~ppm, respectively, and signal readout rate 500~Hz, the target can be achieved at the peak frequency. For more realistic losses, decreased squeezer parameter, and/or higher readout rate the target is not achieved for these parameters \jam{(quantify how far off?)}. Alternatively, realistic losses with 10~dB frequency-dependent external squeezing~\cite{}, as discussed in Section~\ref{sec:external_squeezing}, can achieve the target at the peak frequency, where 10~dB external squeezing means that in the lossless case the noise would be reduced by 10~dB uniformly but with realistic losses the noise is only reduced by $7.2$~dB \jam{(check if this is the definition)}. Among the losses, squeezer parameter, and readout rate, the changes in readout rate have the greatest effect \jam{(quantify change and effect, how to compare changes?)}. % and the squeezer parameter affects the low readout rate cases more \jam{(explain why)}.
    Separate from the kilohertz target, the integrated sensitivity from 100--4000~kHz also improves \jam{(quantify)}. These results have not been optimised and are without external squeezing, unless stated otherwise, or increased circulating power.}
    \label{fig:nIS_sens_target}
\end{figure}

% achieving target across the band looks very unlikely, but target might be achievable at a particular kilohertz frequency
I consider using nondegenerate internal squeezing to detect kilohertz gravitational waves, represented by the astrophysical target sensitivity of $\sqrt{S_h}=5\times10^{-25}\text{Hz}^{-1/2}$ at 1--4~kHz from Section~\ref{sec:GW_kilohertz_target} required to detect the post-merger signal from a typical binary neutron-star merger. In Fig.~\ref{fig:nIS_sens_target}, I compare this target to the sensitivity of nondegenerate internal squeezing with the realistic and ideal loss cases from Section~\ref{sec:nIS_vs_sWLC}, the idler readout port closed, and different squeezer parameters and readout rates. Although the configuration does not achieve the target from 1--4~kHz, the results are still promising.
From varying the readout rate, e.g.\ in Fig.~\ref{fig:nIS_sigRO_tolerance_Rpd}, I found that a low readout rate of $\sim 500$~Hz, such that the bandwidth of the peak is short compared to the arm cavity free-spectral range of $37.5$~kHz, achieves the best sensitivity from 1--4~kHz \jam{(why?)}. For this readout rate at $95\%$ threshold, the target is achieved at the peak frequency \jam{(give exact values)} with ideal losses but is not achieved for realistic losses \jam{(quantify how far off)} and/or decreased ratio to threshold, e.g.\ at $90\%$ threshold \jam{(check this)}. I have experimented with what would be necessary to achieve the target across 1--4~kHz and I found that even at $99\%$ threshold with ideal losses, increased sloshing frequency \jam{(to what?)}, and readout rate to move the peak to the middle of the band \jam{(give value)}, the target could only be met from \jam{($\sim$~1--3~kHz, for the old value of alpha, check what is now possible)} where the arm cavity bandwidth limits further improvement \jam{(can I say this without proof?)}. 
% note the limitations with the target (conditions on a particular EoS of the neutron star etc.)
However, this target sensitivity and frequency range are not definitive as they depend on the equation-of-state of the neutron stars which is not well understood, indeed, understanding it better is one of the goals of kilohertz detection \jam{(can I criticise this approach more?)}. And so, kilohertz improvement close to the target might be sufficient~\cite{}.
Therefore, nondegenerate internal squeezing does not meet the target sensitivity from 1--4~kHz but improves kilohertz sensitivity enough by itself that together with other improvements, such as external squeezing (injected and Caves's amplification) \jam{(quantify how much)}, it would be feasible with realistic losses, 90--95$\%$ squeezer parameter, and low readout rate \jam{(is there a problem with low readout rate?)} to achieve it for part of the band, as shown for 10~dB external squeezing and realistic losses in Fig.~\ref{fig:nIS_sens_target}. This agrees with Ref.~\cite{Miao2018} which achieved the target from 1--4~kHz using unstable~\footnote{There is not an exact correspondence between nondegenerate internal squeezing and the unstable case, see Section~\ref{sec:modal_equivalence}, but I reference it as a similar configuration that cannot achieve the target by itself but can with other improvements.} optomechanical filtering with 10~dB injected squeezing and twice the circulating power, but I avoid increasing the circulating power as explained in Section~\ref{sec:circulating_power}. The problem of improving kilohertz sensitivity is not resolved but this all-optical configuration shows some progress towards kilohertz detection.

    % Sensitivity target of 5e-25 from 1--4 kHz.
    %   With high squeezer ratio, i.e. 95\%, and ideal losses, then target can be hit for less than 1kHz of the band. With even higher squeezer ratio, i.e 99\%, and increased omegaS to move the peak into kHz (and corresponding increased gammaR), then the target can be hit for 1--3kHz.
    %       Bottom line: nIS can achieve the sensitivity target partially across the band, with more and more ideal conditions/optimisation of omegaS and gammaR necessary to widen the range that it achieves it at.
    %       Recommending a detector design isn't the goal of my project, but I do want to say something re: the sensitivity target. Right now, it sounds like nIS can achieve it at some peak frequency (under ideal conditions) but not from 1--4kHz for a Voyager-like detector.

\subsection{Application to broadband detection}

% plot: curve optimised for broadband --> not necessary? just point to Fig.~\ref{fig:nIS_sens_target} and mention optimisation as future work (and that preliminary results show ..., if that)
% \begin{figure}
%   \centering
%   % \includegraphics[width=\textwidth]{}
%   \caption{Nondegenerate internal squeezing sensitivity optimised for broadband detection (i.e.\ with integrated sensitivity from 0 to $\infty$ optimised).}
%   \label{fig:}
% \end{figure}

% kilohertz detection was the aim of the thesis but it looks like nIS might be more useful for a different purpose, set up idler readout chapter?
Although kilohertz detection motivated this work, the broadband improvement from 100--4000 Hz offered by nondegenerate internal squeezing, shown in Fig.~\ref{fig:nIS_sens_target}, could be used to detect more of the sources that detectors like Advanced~LIGO currently see~\cite{} but over a broader range of frequencies. For example, to observe a binary neutron-star merger from the inspiral to post-merger remnant~\cite{}, where only the late inspiral is currently seen~\cite{}. \jam{(are there other science applications of broadband detection?)} The next generation of detectors like LIGO~Voyager aim to improve the sensitivity beyond the current Advanced LIGO sensitivity of $\sqrt{S_h}=8\times10^{-24}\text{Hz}^{-1/2}$ around 100~Hz~\cite{Martynov2018} \jam{(if I quote this sensitivity, then it should be shown in the plot)}, but Fig.~\ref{fig:nIS_sens_target} shows that nondegenerate internal squeezing further improves the sensitivity from 100--1000~Hz \jam{(quantify)}, along with the kilohertz improvement above, for realistic losses, $95\%$ threshold, and readout rates below 5 Hz. Where this broadband sensitivity improvement is less than the kilohertz improvement above \jam{(quantify)} because of the trade-off between peak sensitivity and bandwidth. This improvement is feasible and promising as long as the worsened radiation-pressure noise below 50~Hz is not an issue, e.g.\ it is not for binary neutron-star mergers~\cite{} \jam{(what is it an issue for? and give a 100-1000 Hz astrophysical target?)}. Therefore, nondegenerate internal squeezing should not just be considered for kilohertz sensitivity.


To improve both kilohertz and broadband detection, nondegenerate internal squeezing is a feasible configuration for future gravitational-wave detectors and should be considered for future work. To determine the optimal parameter set for a future detector, a more detailed model with effects such as pump depletion and higher-order modes \jam{(what would these effects change?)} could be optimised against some astrophysical metric~\cite{}. For example, for a broadband detector, the sensitivity integrated against a kernel that biases achieving a target sensitivity but not going beyond it might be the appropriate metric. \jam{(should I give an equation? I want to show that I have considered this, but how much detail is sufficient?)}


%%%%%%%%%%%%%%%%%%%%%%%%%%%%%%%%%%%%%%%%%%
\section{Chapter summary}

In this chapter, I have applied my model of nondegenerate internal squeezing to gravitational-wave detection and found that it warrants consideration for future detectors.
Firstly, I calculated the tolerance to the realistic optical loss expected in future detectors and found that the signal readout is limited by idler loss, affected by detection loss, and that signal and arm intra-cavity losses are negligible. In particular, signal readout requires the idler readout port to be closed to reduce the idler loss. I also demonstrated that when optimising the configuration's sensitivity the metric chosen might mean that the optimal squeezer parameter is below threshold \jam{(is this worth mentioning again?)}.
Secondly, I compared nondegenerate internal squeezing to the existing and related proposals of stable optomechanical filtering and degenerate internal squeezing. I showed that the optical loss required to match the performance of the optomechanical analogue was beyond the realistic level of optical loss but would require a comparable amount of technological progress to the low mechanical loss assumed for the optomechanical analogue \jam{(is this clear? is ``realistic'' a poor choice of word?)}. Therefore, I concluded that nondegenerate internal squeezing is a viable, all-optical alternative to that configuration. I also showed that nondegenerate internal squeezing is more resilient to optical loss than the degenerate case but that their sensitivity curves are not directly comparable which makes the cases useful for different purposes, e.g.\ the nondegenerate case affects the radiation-pressure noise \jam{(have I explained why?)} but the degenerate case does not.
Finally, I considered the motivating problem of improving kilohertz sensitivity and showed that, while nondegenerate internal squeezing does not achieve the particular astrophysical target for the entire band, it improves the sensitivity enough that it feasibly would enable and increase the detections of kilohertz gravitational waves from 1--4~kHz. I also considered whether it is suited to broadband detection from 100--4000~kHz and found that it would improve the sensitivity in that band at the cost of radiation-pressure noise, making it a possible consideration for broadband detectors as well.

\jam{(Have I justified my conclusions and shown deep interpretation of the results to find the physics? I do not know what to add.)}


