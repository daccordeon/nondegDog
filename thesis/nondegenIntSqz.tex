\chapter{Analytic model of nondegenerate internal squeezing} %Research chapter: ...

% results: interpret deep and far, and be very critical of assessing your approach

%%%%%%%%%%%%%%%%%%%%%%%%%%%%%%%%%%%%%%%%%%
% chapter introduction




\section{Literature review}
\jam{(justify novelty of proposal)}


% EPR experiments and ``EPR'' squeezing

% \section{Analytic model}

\section{Hamiltonian model}

This derivation is based on the lossless model in Ref.~\cite{liBroadbandSensitivityImprovement2020} with the differential mode $\hat{a}$, the signal mode in the signal-recycling cavity $\hat{b}$, and the idler mode in the signal-recycling cavity $\hat{c}$. \jam{Approximations have already been made: single mode and semi-classical pump}. In that model, the only source of vacuum is into the $\hat{b}$ mode from the readout -- i.e. behind the signal-recycling mirror.
To that model, I add in optical loss: (1) intra-cavity to each of the three modes $\hat{a}, \hat{b}, \hat{c}$ through loss ports \jam{(explain loss port, give a picture)} with transmissivity $T_{l,a}, T_{l,b}, T_{l,c}$, respectively; and (2) at the photodetector, through a beamsplitter with reflectivity $R_{PD}$ \jam{shown in fig}.
To study the low-frequency quantum noise, I introduce radiation pressure effects through coupling the gravitational-wave signal $h(t)$ to the end test-mass mirror motion $\hat{x}$ with associated momentum $\hat{p}$.
Following the advice \jam{(word choice)} in Ref.~\cite{liBroadbandSensitivityImprovement2020}, I also couple the $\hat{c}$ mode to a back-action evading mechanical mode $\hat{y}$ with momentum $\hat{q}$ and negative effective mass $-\mu$. This is done to retain the PT-symmetry of the system with the P symmetry now exchanging $\hat{a}\leftrightarrow\hat{c}$ and $\hat{x}\leftrightarrow\hat{y}$.


The full Hamiltonian of the system is given by: $$\hat{H}=\hat{H}_0+\hat{H}_\mathrm{I}+\hat{H}_{\mathrm{GW}}+\hat{H}_{\mathrm{BAE}}+\hat{H}_{\gamma_R}+\hat{H}_{\gamma_a}+\hat{H}_{\gamma_b}+\hat{H}_{\gamma_c}.$$ Where each term describes:
\begin{itemize}
\item $\hat{H}_0=\hbar\omega_0(\hat{a}^\dagger\hat{a}+\frac{1}{2})+\hbar\omega_0(\hat{b}^\dagger\hat{b}+\frac{1}{2})+\hbar\omega_\mathrm{idler}(\hat{c}^\dagger\hat{c}+\frac{1}{2})$, the uncoupled behaviour of the optical modes, in the Interaction Picture, the model can ignore this behaviour (moving it from the operators onto the states) \jam{(is RWA also used?)} % and uses a rotating frame at the carrier frequency in order to ignore the evolution of the optical modes from $\hat{H}_0$ \jam{(be careful, are we? -- also, is RWA used?)}
\item $\hat{H}_\mathrm{I}=i\hbar\omega_s(\hat{a}\hat{b}^\dag-\hat{a}^\dag\hat{b})+i\hbar\chi(\hat{b}^\dag\hat{c}^\dag-\hat{b}\hat{c})$, the interaction of the three optical modes $\hat{a}, \hat{b}, \hat{c}$, where a semi-classical approximation has been taken to the pump field \jam{(spell out what this means for $\chi$)} and a single-mode approximation to each of the cavity modes which are assumed to be on resonance \jam{(this needs more discussion, do more in deg. int. sqz. section)}
\item $\hat{H}_\mathrm{GW}=-\alpha (\hat{x}-L_\mathrm{arm}h)(\frac{\hat{a}+\hat{a}^\dag}{\sqrt{2}})+\frac{1}{2\mu}\hat{p}^2$, \jam{($\alpha$ is $\alpha_\mathrm{GW}$, not the alpha in Li)} the coupling to the gravitational-wave signal $h(t)$
\item $\hat{H}_{\mathrm{BAE}}=-\alpha \hat{y}(\frac{\hat{c}+\hat{c}^\dag}{\sqrt{2}})-\frac{1}{2\mu}\hat{q}^2$, the PT-symmetry enabling, back-action evasion \jam{(clarify, what is BAE?)} mode
\item $\hat{H}_{\gamma_R}$ \jam{(formula, state but just use Langevin terms)}, the readout of the $\hat{b}$ mode
\item $\hat{H}_{\gamma_i}$ for $i=a,b,c$, the intra-cavity loss ports, these and $\hat{H}_{\gamma_R}$ give the standard \jam{(word choice)} Langevin terms in the Heisenberg-Langevin equations of motion \jam{(cite Gardiner and Collete?)}.
\end{itemize}

From this Hamiltonian $\hat{H}$, I find the Heisenberg-Langevin equations of motion in the Interaction Picture (separating away the evolution with respect to $\hat{H}_0$):
$$\begin{cases}
\dot{\hat{a}}=-\omega_s\hat{b} - \gamma_a \hat{a} + \sqrt{2\gamma_a}\hat{n}^L_a+\frac{i}{\hbar}\alpha(\hat{x}-L_\mathrm{arm}h)\frac{1}{\sqrt{2}}\\
\dot{\hat{b}}=\omega_s\hat{a} + \chi\hat{c}^\dagger - \gamma^b_\mathrm{tot} \hat{b} + \sqrt{2\gamma_R}\hat{B}_\mathrm{in} + \sqrt{2\gamma_b}\hat{n}^L_b\\
\dot{\hat{c}}=\chi\hat{b}^\dagger - \gamma_c \hat{c} + \sqrt{2\gamma_c}\hat{n}^L_c + \frac{i}{\hbar}\alpha \hat{y}\frac{1}{\sqrt{2}}\\
\dot{\hat{x}}=\frac{1}{\mu}\hat{p},\quad \dot{\hat{p}}=\alpha(\frac{\hat{a}+\hat{a}^\dag}{\sqrt{2}})\\
\dot{\hat{y}}=-\frac{1}{\mu}\hat{q},\quad \dot{\hat{q}}=\alpha(\frac{\hat{c}+\hat{c}^\dag}{\sqrt{2}})
\end{cases}$$
I separate out the fluctuating part of each mode $\hat{a}(t)=\langle\hat{a}\rangle+\delta\hat{a}(t)$ from its time average (or large classical motion) $\langle\hat{a}\rangle$. This does not change the form of the equations. \jam{(Why? This is not clear, the LHS does not change but the time-averages remain)} 
% solving the equations
In Fourier space, these equations become (with the simplified notation $\tilde{\delta\hat{Q}}(\Omega)\mapsto\hat{Q}(\Omega)$ and sign convention $\partial_t\mapsto-i\Omega$):
\begin{equation}
\begin{cases}
\label{eq:nIS-2}
-i\Omega\hat{a}(\Omega)=-\omega_s\hat{b}(\Omega) - \gamma_a \hat{a}(\Omega) + \sqrt{2\gamma_a}\hat{n}^L_a(\Omega)+i(\frac{1}{-\Omega^2}\rho_\mathrm{RP}(\frac{\hat{a}(\Omega)+\hat{a}^\dag(-\Omega)}{\sqrt{2}})-\beta\tilde{h}(\Omega))\\
-i\Omega\hat{b}(\Omega)=\omega_s\hat{a}(\Omega) + \chi\hat{c}^\dagger(-\Omega) - \gamma^b_\mathrm{tot} \hat{b}(\Omega) + \sqrt{2\gamma_R}\hat{B}_\mathrm{in}(\Omega) + \sqrt{2\gamma_b}\hat{n}^L_b(\Omega)\\
-i\Omega\hat{c}(\Omega)=\chi\hat{b}^\dagger(-\Omega) - \gamma_c \hat{c}(\Omega) + \sqrt{2\gamma_c}\hat{n}^L_c(\Omega) + \frac{-i}{-\Omega^2}\rho_\mathrm{BAE}(\frac{\hat{c}(\Omega)+\hat{c}^\dag(-\Omega)}{\sqrt{2}}).
\end{cases}
\end{equation}

Where $\beta=\frac{\alpha L_\mathrm{arm}}{\sqrt{2}\hbar}$ and $\rho_\mathrm{RP}=\rho_\mathrm{BAE}=\frac{\alpha^2}{\sqrt{2}\hbar\mu}$ where the radiation pressure (RP) and back-action evasion (BAE) effects have been separated even though the coupling constant is the same, for PT-symmetry.

Solving these Equations~\ref{eq:nIS-2} for simultaneous solutions of the $\hat{Q}(\Omega)$ and $\hat{Q}^\dag(-\Omega)$ fields is easier when expressed in matrix form. Let $\vec{\hat{Q}}(\Omega)=[\hat{Q}(\Omega),\hat{Q}^\dag(-\Omega)]^T$, then the above equations can be re-written as:
$$\begin{cases}
\label{eq:nIS-3}
((\gamma_a-i\Omega)\mathrm{I}+\frac{i\rho_\mathrm{RP}}{\Omega^2 \sqrt{2}}\begin{bsmallmatrix}
1 & 1 \\ 
-1 & -1
\end{bsmallmatrix})\vec{\hat{a}}(\Omega)=-\omega_s\vec{\hat{b}}(\Omega) + \sqrt{2\gamma_a}\vec{\hat{n}}^L_a(\Omega)-i\beta\begin{bsmallmatrix}
1 & 0 \\ 
0 & -1
\end{bsmallmatrix}\vec{\tilde{h}}(\Omega)\\
(\gamma^b_\mathrm{tot}-i\Omega)\vec{\hat{b}}(\Omega)=\omega_s\vec{\hat{a}}(\Omega) + \chi\begin{bsmallmatrix}
0 & 1 \\ 
1 & 0
\end{bsmallmatrix}\vec{\hat{c}}(\Omega) + \sqrt{2\gamma_R}\vec{\hat{B}}_\mathrm{in}(\Omega) + \sqrt{2\gamma_b}\vec{\hat{n}}^L_b(\Omega)\\
((\gamma_c-i\Omega)\mathrm{I}-\frac{i\rho_\mathrm{BAE}}{\Omega^2\sqrt{2}}\begin{bsmallmatrix}
1 & 1 \\ 
-1 & -1
\end{bsmallmatrix})\vec{\hat{c}}(\Omega)=\chi\begin{bsmallmatrix}
0 & 1 \\ 
1 & 0
\end{bsmallmatrix}\vec{\hat{b}}(\Omega) + \sqrt{2\gamma_c}\vec{\hat{n}}^L_c(\Omega).
\end{cases}$$

Where $\mathrm{I}$ is the $2\times2$ identity matrix. Solving for $\vec{\hat{b}}(\Omega)$, these give
\begin{align}
\vec{\hat{b}}(\Omega)=\mathrm{M}_b^{-1}( &\omega_s\mathrm{M}_a^{-1}(\sqrt{2\gamma_a}\vec{\hat{n}}^L_a(\Omega)-i\beta\begin{bsmallmatrix}
1 & 0 \\ 
0 & -1
\end{bsmallmatrix}\vec{\tilde{h}}(\Omega)) + \chi\begin{bsmallmatrix}
0 & 1 \\ 
1 & 0
\end{bsmallmatrix}\mathrm{M}_c^{-1}\sqrt{2\gamma_c}\vec{\hat{n}}^L_c(\Omega)\\&+ \sqrt{2\gamma_R}\vec{\hat{B}}_\mathrm{in}(\Omega) + \sqrt{2\gamma_b}\vec{\hat{n}}^L_b(\Omega)).
\end{align}
Where the matrices $\mathrm{M}_i,\; i=a,b,c$ are given by:
\begin{equation}
\begin{cases}
\mathrm{M}_a = (\gamma_a-i\Omega)\mathrm{I}+\frac{i\rho_\mathrm{RP}}{\Omega^2 \sqrt{2}}\begin{bsmallmatrix}
1 & 1 \\ 
-1 & -1
\end{bsmallmatrix}\\
\mathrm{M}_b = (\gamma^b_\mathrm{tot}-i\Omega)\mathrm{I} + \omega_s^2 \mathrm{M}_a^{-1} - \chi^2\begin{bsmallmatrix}
0 & 1 \\ 
1 & 0
\end{bsmallmatrix} \mathrm{M}_c^{-1} \begin{bsmallmatrix}
0 & 1 \\ 
1 & 0
\end{bsmallmatrix}\\
\mathrm{M}_c = (\gamma_c-i\Omega)\mathrm{I}-\frac{i\rho_\mathrm{BAE}}{\Omega^2\sqrt{2}}\begin{bsmallmatrix}
1 & 1 \\ 
-1 & -1
\end{bsmallmatrix}.
\end{cases}
\end{equation}

Having found the light inside the cavities, I want to find the light at the photodetector. The light right outside and travelling away from the signal-recycling mirror is given by the Input/Output (I/O) relation: $\hat{B}_\mathrm{out}=\hat{B}_\mathrm{in}-\sqrt{2\gamma_R}\hat{b}$. Using the beamsplitter model of detection loss at the photodiode, for a beamsplitter of reflectivity $R_\mathrm{PD}$, the quadratures of light at the photodetector are given by: $\vec{\hat{X}}_\mathrm{PD}=\sqrt{R_\mathrm{PD}} \vec{\hat{X}}_\mathrm{PD}^L + \sqrt{1-R_\mathrm{PD}} \vec{\hat{X}}_{B_\mathrm{out}}$. Where $\vec{\hat{X}}_Q=\Gamma \vec{\hat{Q}}=[\hat{X}_{1,Q},\hat{X}_{2,Q}]^T$ for $\hat{X}_{i,Q}$ the ith quadrature of $Q$ and $\Gamma=\frac{1}{\sqrt{2}}\begin{bsmallmatrix}
1 & 1 \\ 
-i & i\end{bsmallmatrix}$ the quadrature matrix.
Putting everything together, I find the quadratures of the light at the photodetector to be given by:
\begin{align}
\vec{\hat{X}}_\mathrm{PD}(\Omega)&=
% \sqrt{R_\mathrm{PD}} \vec{\hat{X}}_\mathrm{PD}^L(\Omega) + \sqrt{1-R_\mathrm{PD}} \Gamma(\hat{B}_\mathrm{in}(\Omega)-\sqrt{2\gamma_R}\hat{b}(\Omega))\\
% &=  \sqrt{R_\mathrm{PD}} \vec{\hat{X}}_\mathrm{PD}^L(\Omega) + \sqrt{1-R_\mathrm{PD}} \Gamma(\mathrm{I}-2\gamma_R\mathrm{M}_b^{-1})\Gamma^{-1}\vec{\hat{X}}_{B_\mathrm{in}}(\Omega)\\
% &-\sqrt{1-R_\mathrm{PD}} \Gamma\sqrt{2\gamma_R}\mathrm{M}_b^{-1}
% (\omega_s\mathrm{M}_a^{-1}\sqrt{2\gamma_a}\vec{\hat{n}}^L_a(\Omega)
% -\omega_s\mathrm{M}_a^{-1}i\beta\begin{bsmallmatrix}
% 1 & 0 \\ 
% 0 & -1
% \end{bsmallmatrix}\vec{\tilde{h}}(\Omega) \\& \hspace{5cm}+ \chi\begin{bsmallmatrix}
% 0 & 1 \\ 
% 1 & 0
% \end{bsmallmatrix}\mathrm{M}_c^{-1}\sqrt{2\gamma_c}\vec{\hat{n}}^L_c(\Omega) + \sqrt{2\gamma_b}\vec{\hat{n}}^L_b(\Omega))\\&=
\mathrm{R_{in}}\vec{\hat{X}}_{B_\mathrm{in}}(\Omega)
+ \mathrm{R}^L_a\vec{\hat{X}}^L_a(\Omega)
+ \mathrm{R}^L_b\vec{\hat{X}}^L_b(\Omega)
+ \mathrm{R}^L_c\vec{\hat{X}}^L_c(\Omega)
+ \mathrm{R}^L_\mathrm{PD}\vec{\hat{X}}_\mathrm{PD}^L(\Omega)
+ \mathrm{T}\vec{\tilde{h}}(\Omega).
\end{align}

Where $\vec{\hat{X}}_{B_\mathrm{in}}(\Omega)$ is the quadrature vector of the vacuum from the main vacuum port behind the signal-recycling mirror, $\vec{\hat{X}}^L_a(\Omega)$ is the vacuum from the $\hat{a}$ intra-cavity loss, etc.. And where the transfer function matrices $\mathrm{R}_i, \mathrm{T}$ for the noises and signal, respectively, are given by the following (where the frequency dependence is inside each $\mathrm{M}_i^{-1}$):
\begin{equation}
\begin{cases}
\mathrm{R_{in}}=\sqrt{1-R_\mathrm{PD}} \Gamma(\mathrm{I}-2\gamma_R\mathrm{M}_b^{-1})\Gamma^{-1}\\
\mathrm{R}^L_a=-\sqrt{1-R_\mathrm{PD}} 2\sqrt{\gamma_R\gamma_c}\omega_s\Gamma\mathrm{M}_b^{-1}\mathrm{M}_a^{-1}\Gamma^{-1}\\
\mathrm{R}^L_b=-\sqrt{1-R_\mathrm{PD}} 2\sqrt{\gamma_R\gamma_b}\Gamma\mathrm{M}_b^{-1}\Gamma^{-1}\\
\mathrm{R}^L_c=-\sqrt{1-R_\mathrm{PD}} 2\sqrt{\gamma_R\gamma_c}\chi\Gamma\mathrm{M}_b^{-1}\begin{bsmallmatrix}
0 & 1 \\ 
1 & 0
\end{bsmallmatrix}\mathrm{M}_c^{-1}\Gamma^{-1}\\
\mathrm{R}^L_\mathrm{PD}=\sqrt{R_\mathrm{PD}}\\
\mathrm{T}=\sqrt{1-R_\mathrm{PD}} \sqrt{2\gamma_R}i\omega_s\beta \Gamma\mathrm{M}_b^{-1}\mathrm{M}_a^{-1}\begin{bsmallmatrix}
1 & 0 \\ 
0 & -1
\end{bsmallmatrix}.
\end{cases}
\end{equation} 


\jam{(set up the combined noise, spectral densities matrix Sx)}

\begin{equation}
\mathrm{S}_X(\Omega)2\pi\delta(\Omega-\Omega')=\ev{\vec{\hat{X}}_\mathrm{PD}(\Omega)\cdot\vec{\hat{X}}_\mathrm{PD}(\Omega')^\dag}
\end{equation}

\jam{(explain uncorrelated vacuum assumption -- i and j are not indices of vectors)}

\begin{equation}
\ev{\vec{\hat{X}}_{(i)}(\Omega)\cdot\vec{\hat{X}}_{(j)}(\Omega')^\dag}=\delta_{i,j}\mathrm{I}\,2\pi\delta(\Omega-\Omega'),\quad \mathrm{S_{vac}}=\mathrm{I}=\begin{bsmallmatrix}
1 & 0 \\ 
0 & 1
\end{bsmallmatrix}
\end{equation}

\jam{(versus single-sided power spectral density of single function, not a vector: $A\circ B=\dfrac{1}{2}(A\cdot B+B\cdot A)$, check factors of two again --> single sided, vac=1, quadratures normalised as 1/rt2, quadrature is Hermitian and therefore commutes with its dagger)}


\begin{equation}
\mathcal{S}_X(\Omega)2\pi\delta(\Omega-\Omega')=(\text{S}_X)_{2,2}(\Omega)2\pi\delta(\Omega-\Omega')=\ev{\hat{X}_{\mathrm{PD},2}(\Omega)\circ\hat{X}_{\mathrm{PD},2}(\Omega')^\dag},\quad \mathcal{S}_\mathrm{vac}=1
\end{equation}

\begin{equation}
\mathcal{T}(\Omega)=(\mathrm{T}.\begin{bsmallmatrix}
1\\ 
1
\end{bsmallmatrix})_2
\end{equation}

I define the sensitivity of the detector to be the noise-to-signal ratio, i.e. the ratio of the noise and signal transfer functions (or, the signal-normalised noise).
Assuming that we measure the second quadrature, e.g. via balanced homodyne readout \jam{(address optimisation of quadrature angle, might not be 2 just because signal is only there. Also, need to describe homodyne readout somewhere)}, this gives:
$$S_h(\Omega)=\frac{(\mathrm{S}_X(\Omega))_{2,2}}{\abs{\mathcal{T}(\Omega)}^2}.$$




\jam{(continue writing up derivation...)}

\jam{(back up and include pump phase in model to show its non-effect)}



% \begin{align}
% \gamma^b_\mathrm{tot}&=\gamma_R+\gamma_b\\
% \hat{B}_\mathrm{out} &= \hat{B}_\mathrm{in} - \sqrt{2\gamma_R}\hat{b}\\
% &\implies R^{(i)}(\Omega), T(\Omega)\\ S_h(\Omega)&= \frac{\sum_i \left| R^{(i)}(\Omega) \right|^2+\frac{R_{\mathrm{PD}}}{1-R_\mathrm{PD}}}{\left| T(\Omega) \right|^2}
% \end{align}

\section{Reduction to known systems}

\subsection{Lossless, PT-symmetric limit}

% symmetry broken by RP, repair by BAE but too complicated for here

\subsection{Nondegenerate OPO limit}


\section{Stability}


\section{Singularity threshold}

I have devised a definition of threshold: ``threshold is the (real) value of the squeezing parameter such that the anti-squeezed quadrature of the quantum noise has a singularity at some (real) frequency; if multiple singularities exist then the threshold is the smallest squeezer value''. Here, I am only looking for singularities that diverge to infinity (along all angles of approach). In order to reduce confusion, I do not call these ``poles'' since the quantum noise is not defined on $\mathbb{C}$. \jam{(Check this: looking at the imaginary part of the poles for stability and looking for real poles (i.e. poles on the real axis) for threshold are the same solutions, surely?)}
% NB: I am calling the points "singularities" for now to avoid confusion, but they are just the poles of the transfer functions (against complex \[CapitalOmega]) at real \[Chi] such that they are on the real axis (and therefore real \[CapitalOmega]).

For this unified, anti-squeezed singularity definition of threshold, I find the real singularities of each system to be at:
\begin{align}
\label{eq:unified-threshold}
\mathrm{dOPO}&: \Omega=0, \chi=\gamma^b_\mathrm{tot}\\
\mathrm{nOPO}&: \Omega=0, \chi=\sqrt{\gamma^b_\mathrm{tot}\gamma_c}\\
\mathrm{dIS}&:\begin{cases}
\Omega=0, \chi=\gamma^b_\mathrm{tot}+\dfrac{\omega_s^2}{\gamma_a};&\gamma_a\neq0\\
\Omega=\sqrt{\omega_s^2-\gamma_a^2}, \chi=\gamma^b_\mathrm{tot}+\gamma_a;&\omega_s\geq\gamma_a\geq0
\end{cases}\\
\mathrm{nIS}&:\begin{cases}
\Omega=0, \chi=\sqrt{\gamma_c(\gamma^b_\mathrm{tot}+\dfrac{\omega_s^2}{\gamma_a})};&\gamma_c\neq0,\gamma_a\neq0\\
\Omega=\sqrt{\dfrac{\gamma_c\omega_s^2+\gamma_a(-\omega_s^2-\gamma_a(\gamma^b_\mathrm{tot}+\gamma_c))}{\gamma^b_\mathrm{tot}+\gamma_c}}, \chi=\sqrt{(\gamma_a+\gamma^b_\mathrm{tot})(\gamma_a+\gamma_c+\dfrac{\omega_s^2}{\gamma^b_\mathrm{tot}+\gamma_c})};&\gamma_c\neq0,(*)
\end{cases}\\
(*)&:\gamma_c\omega_s^2\geq\gamma_a\omega_s^2+\gamma_a^2(\gamma^b_\mathrm{tot}+\gamma_c)
\end{align}

For dIS, by Eq.~\ref{eq:unified-threshold}, the definition reduces to dOPO threshold $(\Omega,\chi)\xrightarrow[\gamma_a\rightarrow\infty]{}(0,\gamma^b_\mathrm{tot})$ and to the lossless threshold $(\Omega,\chi)\xrightarrow[\gamma_a\rightarrow0]{}(\omega_s,\gamma^b_\mathrm{tot})$.
For nIS, this definition reduces to nOPO threshold $(\Omega,\chi)\xrightarrow[\gamma_a\rightarrow\infty]{}(0,\sqrt{\gamma^b_\mathrm{tot}\gamma_c})$ and to the lossless threshold in the formal limit $(\Omega,\chi)\xrightarrow[\gamma_a=0,\gamma_c\rightarrow0]{}(0,\omega_s)$. This last limit makes sense from the stability \jam{(make sure stability already studied)} analysis, since the lossless system is marginally stable at threshold and this anti-squeezed singularity definition looks for when the poles are on the real axis.

Radiation pressure and back-action evasion do not change threshold because the quantum noise transfer function denominator has the same positive zeros as the shot noise transfer function without radiation pressure (and BAE).

\jam{(Threshold plots)}

\subsubsection{Degenerate internal squeezing - different notions of threshold}
%shouldn't this be in the dIS sub-chapter?

\jam{(One explanation for the difference between position of the singularity of the anti-squeezed quadrature and the minimum of the squeezed quadrature is that the expression for the sloshing frequency in Korobko et al, 2019 is not valid at large arm losses. If this is so, then what does this fixed--sloshing frequency Hamiltonian correspond to? And for that system, why does the supposedly Gaussian squeezing not maximise antisqz at the frequency that it minimises squeezing? --> Look at pump phase and/or covariance matrix)}


% \subsection{PT-symmetry}
% % mention removing back-action evasion

%%%%%%%%%%%%%%%%%%%%%%%%%%%%%%%%%%%%%%%%%%
\section{Chapter summary}


%%%%%%%%%%%%%%%%%%%%%%%%%%%%%%%%%%%%%%%%%%
\chapter{Gravitational-wave detection science case for nondegenerate internal squeezing}

% comparison to existing proposals

\section{Experimental constraints} %on parameter space

% call back to optical loss


% \section{Sensitivity optimisation}

% % chosen metrics (mutrics), optimisation to hit astrophysical target
% % Realistic experimental constraints

% \subsection{Kilohertz detector}

% \subsection{Broadband detector}

\section{Feasibility for gravitational-wave detection}

% miao et al, 2018 astro target and mutrics

% motivation was kilohertz detection but works better as broadband detector?
% cover two detector designs

\subsection{Optimisation for kilohertz detection}

\subsection{Optimisation for broadband detection}


\section{Comparison to existing proposals}


%%%%%%%%%%%%%%%%%%%%%%%%%%%%%%%%%%%%%%%%%%
\section{Chapter summary}


