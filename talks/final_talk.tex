% honours final talk
\documentclass[12pt,xcolor=dvipsnames]{beamer}
\mode<presentation> {
\usetheme{Madrid}
\setbeamertemplate{navigation symbols}{}
}

\usepackage[utf8]{inputenc}
\usepackage{amsmath, amssymb, amsthm}
\usepackage{amsfonts}
\usepackage{graphicx}
\usepackage{float}
\usepackage[center]{caption}
\usepackage{hyperref}
\usepackage{mathtools}

\usepackage{subfig}
\usepackage{comment}

\usepackage{multimedia}
\usepackage{anyfontsize}


\graphicspath{{./}}

\setbeamertemplate{enumerate items}[default]


\title[]{Improving future gravitational-wave detectors using nondegenerate internal~squeezing}

\author[James Gardner]{\texorpdfstring{\large{James~Gardner}\\\small{Min~Jet~Yap, Vaishali~Adya, Sheon~Chua, and David~McClelland}}{James~Gardner}}
\institute[]{\small The Centre for Gravitational Astrophysics, ANU}
\date{November 10, 2021}

\begin{document}

%%%%%%%%%%%%%%%%%%%%%%%%%%%%%%%%%%%%%%%%%%

{
  \usebackgroundtemplate{\includegraphics[width=\paperwidth]{title_slide_with_logos.pdf}}
  \begin{frame}[label=titleframe,noframenumbering]
  \titlepage
  \end{frame}
}

\begin{frame}{Motivation: kilohertz gravitational waves}

\vskip0pt plus 1filll
\centering
{\tiny Potential astrophysics from [M. Breschi, S. Bernuzzi, F. Zappa, et al. 2019. \emph{Phys. Rev. D}, 100:104029]}\\
{\vspace{-0.2cm}\tiny and [H.~Miao, H.~Yang, and D.~Martynov., 2018, \emph{Phys. Rev. D}, 98(4):044044]}
\end{frame}

\begin{frame}{Gravitational-wave detectors}
% simple michelson
\centering
\includegraphics[width=0.75\textwidth]{ligo_sites_Christopher_Berry.png}
\includegraphics[angle=-90,width=\textwidth]{GW_hitting_MICH.pdf}
\vskip0pt plus 1filll
{\tiny (top) image credit: [Christopher Berry, 2015], (bottom) reference: [J. Aasi et al., 2015. \emph{Class. Quantum Grav.}, 32:074001]}
\end{frame}

\begin{frame}{Goal: kilohertz sensitivity}
% + overview of how nIS improves SNR
% no sensitivity graph yet
\centering
\includegraphics[angle=-90,width=0.8\textwidth]{talk_SNR_goal.pdf}
\end{frame}

\begin{frame}{Quantum noise and squeezing}

\end{frame}

\begin{frame}{External squeezing and cavities}
%+ cavities affect signal response
\centering
\includegraphics[angle=-90,width=\textwidth]{talk_dES_config.pdf}
\end{frame}

\begin{frame}{Internal squeezing}


\vskip0pt plus 1filll
\centering
{\tiny Degenerate internal squeezing from [M. Korobko, Y. Ma, Y. Chen, et al., 2019, \emph{Light Sci. Appl.}, 8(1):118]}\\
{\vspace{-0.2cm}\tiny and [V. B. Adya, M. J. Yap, D. Töyrä, et al., 2020, \emph{Class. Quantum Grav.}, 37(7):07LT02]}
\end{frame}

\begin{frame}{Optical loss}
\centering
{
\includegraphics[angle=-90,width=0.85\textwidth]{talk_ball_and_stick_loss.pdf}
\hspace*{1cm}
}
\end{frame}

\begin{frame}{Method} 
% clearly say what I did
% analytic model
% approximations (no pump depl.)
% validation
\end{frame}

\begin{frame}{Results: nondegenerate internal squeezing}
\centering
\includegraphics[width=\textwidth]{talk_nIS_N_S_NSR_annotated.pdf}
% N_S_NSR, explain sens!
% explain parameter set
\end{frame}

\begin{frame}{Threshold}
\centering
\includegraphics[angle=-90,width=1\textwidth]{gain_and_loss.pdf}
\\\vspace*{1cm}
\large{threshold: gain$=$loss}
\\\vspace*{0.5cm}threshold + no pump depletion$\implies$borderline unstable
% blowup
\end{frame}

\begin{frame}{Threshold via stability}
% this is my technique
\centering
% threshold + no pump depletion$\implies$borderline unstable
% traj
% \pause
\includegraphics[width=0.7\textwidth]{talk_nIS_threshold.pdf}
\end{frame}

\begin{frame}{Tolerance to detection optical loss}
% confirm comparison to dIS but no fig for dIS
\begin{columns}
\column{0.55\textwidth}
\centering
\includegraphics[height=0.77\textheight]{talk_nIS_tolerance.pdf}
\column{0.45\textwidth}
\centering
\hspace*{-.3cm}
\includegraphics[height=0.742\textheight]{talk_dIS_tolerance.pdf} % uses a different parameter set, readout rate = 5kHz
\end{columns}
\vskip0pt plus 1filll
\centering
{\tiny Degenerate internal squeezing model from [M. Korobko, Y. Ma, Y. Chen, et al., 2019, \emph{Light Sci. Appl.}, 8(1):118]}\\
\end{frame}

\begin{frame}{Kilohertz sensitivity}
\centering
\includegraphics[width=0.95\textwidth]{talk_nIS_ideal_losses.pdf}
{\tiny\vspace{-0.2cm}Astrophysical target from [H.~Miao, H.~Yang, and D.~Martynov., 2018, \emph{Phys. Rev. D}, 98(4):044044]}
\end{frame}

\begin{frame}{Highlights of other results}
\vspace{2cm}
	\begin{enumerate}
	\item Optimal squeezing
	\item Comparison to optomechanical analogue
	\item Broadband sensitivity using a different readout scheme %(measurement) 
	\end{enumerate}
\vskip0pt plus 1filll
\centering
{\tiny\vspace{-0.15cm}Optomechanical analogue in [X. Li, M. Goryachev, Y. Ma, et al., 2020, \emph{arXiv:2012.00836 [quant-ph]}\ ]}
\end{frame}

\begin{frame}{Future work}
	\begin{enumerate}
	\item \textbf{Combined readout scheme} %Optimal sensitivity
	\item Extended analytic model %$\implies$ validate threshold
	\item Numerical validation % validation already done with limits, sufficient physics extracted with analytic model to absorb all of research time
	\item Parity-time symmetry
	\item Experimental table-top demonstration 
	\end{enumerate}
\end{frame}

\begin{frame}{Conclusions}
% return to motivation
\begin{block}{Nondegenerate internal squeezing}
\begin{enumerate}
\item Detection loss--resistant, all-optical configuration
\item Well-characterised by analytic model
\item Can improve kilohertz (1--4~kHz) or broadband (0.1--4~Hz) sensitivity to gravitational waves 
\end{enumerate}
\end{block}
\centering
\vspace{0.5cm}
{\large gravitational-wave detection$\implies$ \underline{new physics}!}
\end{frame}



%%%%%%%% repete primeiro slide? %%%%%%%%
% {
% \usebackgroundtemplate{\includegraphics[width=\paperwidth]{title_slide_with_logos.pdf}}
% 	\againframe<3>[noframenumbering]{titleframe}
% }

\setbeamercolor{frametitle}{fg=Blue,bg=Blue!20}

\begin{frame}[noframenumbering]{Stable optomechanical filtering}
\centering
\includegraphics[angle=-90,width=0.8\textwidth]{sWLC_config.pdf}
\end{frame}

\begin{frame}[noframenumbering]{Abstract mode structure}
\centering
\includegraphics[angle=-90,width=\textwidth]{all_mode_structures.pdf}
\end{frame}

\begin{frame}[noframenumbering]{Stability of nondegenerate internal squeezing}
\centering
\includegraphics[width=\textwidth]{nIS_stability.pdf}
\end{frame}

\begin{frame}[noframenumbering]{Threshold of degenerate internal squeezing}
\centering
\includegraphics[width=0.8\textwidth]{dIS_threshold_traj_compressed.pdf}
\end{frame}

\begin{frame}[noframenumbering]{Alternative readout schemes}
\begin{columns}
\column{0.5\textwidth}
\includegraphics[width=\textwidth]{idler_RO_config.pdf}
\centering
\column{0.5\textwidth}
\centering
\includegraphics[width=\textwidth]{nIS_signal_vs_idler_ROs.pdf}
\end{columns}
\end{frame}



\end{document}
