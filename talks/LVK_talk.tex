% LVK March 2022 meeting presentation
\documentclass[12pt,xcolor=dvipsnames,aspectratio=169]{beamer}
\mode<presentation> {
\usetheme{Madrid}
\setbeamertemplate{navigation symbols}{}
}

\usepackage[utf8]{inputenc}
\usepackage{amsmath, amssymb, amsthm}
\usepackage{amsfonts}
\usepackage{graphicx}
\usepackage{float}
\usepackage[center]{caption}
\usepackage{hyperref}
\usepackage{mathtools}
\usepackage{booktabs}

\usepackage{subfig}
\usepackage{comment}

\usepackage{multimedia}
% \usepackage{media9}
\usepackage{anyfontsize}


\graphicspath{{./}{/home/james/Code/honours/nondegDog/talks/ozgrav_template/}{/home/james/Code/honours/nondegDog/paper/figures_for_publication/}}

\setbeamertemplate{enumerate items}[default]

\newcommand{\vframefill}{\vskip0pt plus 1filll}


\title[\url{https://dcc.ligo.org/LIGO-P2200052}]{Nondegenerate internal squeezing}
\author[James~W.~Gardner et al.]{\texorpdfstring{\large{James~W.~Gardner}\\\small{Min~Jet~Yap, Vaishali~Adya, Sheon~Chua, Bram~J.~J.~Slagmolen, and David~McClelland}}{James~Gardner}}
\institute[]{\small The Centre for Gravitational Astrophysics, ANU}
\date{LVK Meeting March 2022}

\begin{document}

%%%%%%%%%%%%%%%%%%%%%%%%%%%%%%%%%%%%%%%%%%

{\usebackgroundtemplate{\includegraphics[width=\paperwidth]{ozgrav_template_title_slide.pdf}}
\begin{frame}[label=titleframe,noframenumbering]
\end{frame}}

{\usebackgroundtemplate{\includegraphics[width=\paperwidth]{ozgrav_template_AoC_slide.pdf}}
\begin{frame}[noframenumbering]
\end{frame}}

% body templates are all too busy and/or ugly, just use normal Madrid beamer
% {\usebackgroundtemplate{\includegraphics[width=\paperwidth]{ozgrav_template_body_slide.pdf}}
\begin{frame}{Motivation: kilohertz gravitational waves}
\centering
\includegraphics[height=4.5cm,width=6cm]{merger_first_slide.png}
% \vframefill
\vframefill
\begin{enumerate}
\item Neutron-star equation-of-state % and exotic states of matter
\item Origin of low-mass black holes
\item Post-bounce dynamics of core-collapse supernovae
\item Primordial sources %\item Primordial sources in the gravitational-wave background
\end{enumerate}
\vframefill
\centering
{\tiny video credit: [NASA/Goddard Space Flight Center, 2010]}\\
{\vspace{-0.2cm}\tiny Potential astrophysical science from [K.~Ackley, V.~B.~Adya, and P.~Agrawal et al., 2020, \emph{Publ. Astron. Soc. Aust.}, 37]} %[H.~Miao, H.~Yang, and D.~Martynov., 2018, \emph{Phys. Rev. D}, 98(4):044044]
\end{frame}

\begin{frame}{Current gravitational-wave detectors}
% simple michelson
\centering
\includegraphics[width=0.75\textwidth]{ligo_sites_Christopher_Berry.png}
\includegraphics[angle=-90,width=\textwidth]{GW_hitting_MICH.pdf}
\vframefill
{\tiny (top) image credit: [Christopher Berry, 2015], (bottom) [J. Aasi et al., 2015. \emph{Class. Quantum Grav.}, 32:074001]}
\end{frame}

\begin{frame}{Goal: kilohertz sensitivity}
% + overview of how nIS improves SNR
% no sensitivity graph yet
\centering
\includegraphics[angle=-90,width=0.8\textwidth]{talk_SNR_goal.pdf}
\end{frame}

\begin{frame}{Quantum noise and squeezing} % start with HUP ``like the position and momentum of a massive particle''
% maximum squeezing set by ``threshold'', explained later
\begin{picture}(320,250)%(width,height)(xoffset,yoffset)
\put(50,121.3){\includegraphics<1>[width=0.75\textwidth]{talk_squeezing_intro_1.pdf}}
\put(50,121.3){\includegraphics<2>[width=0.75\textwidth]{talk_squeezing_intro_2.pdf}}
\put(50,35){\includegraphics<3>[width=0.75\textwidth]{talk_squeezing_intro_3.pdf}}
\put(5,25){\only{\vspace*{-0.1cm}\fontsize{5.5pt}{10pt}\selectfont Review of squeezing for gravitational-wave detection in [S. L. Danilishin and F. Y. Khalili. 2012. \emph{Living Rev. Relativ.}, 15(1):5.]}<3>}
\end{picture}
% \centering
% \includegraphics<1>[width=0.77\textwidth]{talk_squeezing_intro_half.pdf}
% \includegraphics<2>[width=0.77\textwidth]{talk_squeezing_intro_full.pdf}
% \vframefill
\end{frame}

\begin{frame}{Cavities and external squeezing} % explain cavities first
%+ cavities affect signal response
\centering
\vspace*{0.5cm}%\hspace*{1.2cm}
\includegraphics[width=0.9\textwidth]{talk_dES_config.pdf}
\vframefill
{\tiny External squeezing in LIGO from [M. Tse, H. Yu, N. Kijbunchoo, et al. 2019. \emph{Phys. Rev. Lett.}, 123(23):231107.]}
\end{frame}

\begin{frame}{Degenerate internal squeezing}
\centering
\vspace*{0.5cm}
\includegraphics[width=0.9\textwidth]{talk_dIS_config.pdf}
\vframefill\centering
{\tiny Degenerate internal squeezing from [M. Korobko, Y. Ma, Y. Chen, et al., 2019, \emph{Light Sci. Appl.}, 8(1):118]}
% \\{\vspace{-0.2cm}\tiny and [V. B. Adya, M. J. Yap, D. Töyrä, et al., 2020, \emph{Class. Quantum Grav.}, 37(7):07LT02]}
\end{frame}

\begin{frame}{Nondegenerate internal squeezing}
\centering
\vspace*{0.5cm}
\includegraphics[width=0.9\textwidth]{talk_nIS_config.pdf}
\vframefill
\end{frame}

\begin{frame}{Methods}
\begin{flushleft}
Analytic model of nondegenerate internal squeezing:
\end{flushleft}
\centering
% \vspace*{0.5cm}
\includegraphics[width=0.7\textwidth]{talk_model_flowchart.pdf}
\vframefill
{\fontsize{5.9}{10}\selectfont\vspace{-0.1cm}Lossless Hamiltonian from [X. Li, M. Goryachev, Y. Ma, et al., 2020, \emph{arXiv:2012.00836 [quant-ph]}\ ]}
\end{frame}

\begin{frame}{Results}
% \centering
\vspace*{1.1cm}
\begin{enumerate}
\item Validation %: reduces to correct~limits
\item Dynamical stability and squeezing threshold -- a new method %via poles
% \only{\vspace*{3cm}}<2>
% \only{
\item \textbf{Characterisation of sensitivity}
\item \textbf{Tolerance to detection optical loss} and other losses
\item Comparison to optomechanical analogue
\item \textbf{Comparison to astrophysical kilohertz target}
\item \textbf{Idler readout scheme} %}<3>
\end{enumerate}
\vframefill\centering
{\fontsize{5.9}{10}\selectfont\vspace{-0.1cm}Optomechanical analogue from [X. Li, M. Goryachev, Y. Ma, et al., 2020, \emph{arXiv:2012.00836 [quant-ph]}\ ]} 
\end{frame}

\begin{frame}{Characterisation of sensitivity} % internal squeezing only
\centering
\vspace*{1cm}
\includegraphics[width=\textwidth]{talk_nIS_N_S_NSR_annotated.pdf} %0.986, 0.95, 0.85, 0.75 % threshold
% N_S_NSR, explain sens!
% explain parameter set
\vframefill\centering
{\tiny LIGO Voyager parameter set from [R. X. Adhikari, K. Arai, A. F. Brooks, et al. 2020. \emph{Class. Quantum Grav.}, 37(16):165003.]}
\end{frame}

\begin{frame}{Tolerance to detection optical loss} % compared ``apples-to-apples''
% confirm comparison to dIS but no fig for dIS
% \begin{columns}
% \column{0.55\textwidth}
% \centering
% \includegraphics[height=0.77\textheight]{talk_nIS_tolerance.pdf}
% \column{0.45\textwidth}
% \centering
% \hspace*{-.3cm}
% \includegraphics[height=0.742\textheight]{talk_dIS_tolerance.pdf} % uses a different parameter set, readout rate = 5kHz
% \end{columns}
% \vframefill\centering
% {\tiny Degenerate internal squeezing model from [M. Korobko, Y. Ma, Y. Chen, et al., 2019, \emph{Light Sci. Appl.}, 8(1):118]}\\
\begin{picture}(320,250)%(width,height)(xoffset,yoffset)
\put(0,50){\includegraphics<1>[height=0.7\textheight]{talk_nIS_tolerance_1.pdf}}
\put(180,50){\includegraphics<1>[height=0.7\textheight]{talk_dIS_tolerance_1.pdf}}
\put(0,50){\includegraphics<2>[height=0.7\textheight]{talk_nIS_tolerance_2.pdf}}
\put(180,50){\includegraphics<2>[height=0.7\textheight]{talk_dIS_tolerance_2.pdf}}
\put(20,25){\vspace*{-0.1cm}\tiny Degenerate internal squeezing model from [M. Korobko, Y. Ma, Y. Chen, et al., 2019, \emph{Light Sci. Appl.}, 8(1):118]}
\end{picture}
\end{frame}

\begin{frame}{Comparison to astrophysical kilohertz target} % explain target, no increase in circulating power + 4 km arms
\centering
\includegraphics[width=0.95\textwidth]{talk_nIS_ideal_losses_w_NS.pdf}
% low frequencies are not improved by external squeezing because they are dominated by the idler loss, not the readout port vacuum
{\tiny\vspace{-0.2cm}Astrophysical target from [H.~Miao, H.~Yang, and D.~Martynov., 2018, \emph{Phys. Rev. D}, 98(4):044044]}
\end{frame}

\begin{frame}{Idler readout scheme} % this is my technique
\begin{columns}
\column{0.5\textwidth}
\includegraphics[width=\textwidth]{talk_idler_RO.pdf}
\centering
\column{0.5\textwidth}
\centering
% \hspace*{-0.1cm}
\includegraphics[width=\textwidth]{talk_sigVsIdler_ROs} % variational readout of idler
\end{columns}
\end{frame}

\begin{frame}{Future work}
	\begin{enumerate}
	\item \textbf{Coherently combined readout scheme} %Optimal sensitivity
	\item Extended model %$\implies$ validate threshold
		\begin{enumerate}
		\item Analytic additions, e.g.\ pump depletion
		\item Numerical validation % validation already done with limits, sufficient physics extracted with analytic model to absorb all of research time
		\item Parity-time symmetry -- future collaboration
		\end{enumerate}
	\item Experimental table-top demonstration 
	\end{enumerate}
\end{frame}

\begin{frame}{Conclusions}
% return to motivation
\begin{block}{Nondegenerate internal squeezing}
\begin{enumerate}
\item Detection loss--resistant, all-optical configuration
\item Well-characterised by analytic model
\item Can improve kilohertz (1--4~kHz) or broadband (0.1--4~kHz) sensitivity to gravitational waves 
\end{enumerate}
\end{block}
\vspace{0.5cm}
{
	\setbeamercolor{block body}{bg=red!20}
  % \setbeamercolor{block title}{#3}	
	\begin{block}{}
	\centering
	{\large gravitational-wave detection$\implies$\underline{new physics}!}
	\end{block}
}
\end{frame}

{\usebackgroundtemplate{\includegraphics[width=\paperwidth]{ozgrav_template_thanks_slide.pdf}}
\begin{frame}[noframenumbering]
\end{frame}}

\setbeamercolor{frametitle}{fg=Blue,bg=Blue!20}

\begin{frame}[noframenumbering]{Stability and singularity threshold}
\centering
% \includegraphics[width=\textwidth]{}
\end{frame}

\begin{frame}[noframenumbering]{Table of parameters}
\centering
\vspace*{2cm}
% \resizebox{\textwidth}{!}{%
% 	% \begin{tabular}{@{}ll|ll@{}}
% 	% \toprule
% 	% carrier wavelength, $\lambda_0$ & 2 $\mu\text{m}$ &  signal mode transmissivity, $T_{\text{SRM},b}$ & 0.046  \\
% 	% arm cavity length, $L_\text{arm}$ & 4 km & signal readout rate, $\gamma^b_R$ & 500 Hz \\
% 	% signal-recycling cavity length, $L_\text{SRC}$ & 1.124 km & idler mode transmissivity, $T_{\text{SRM},c}$ & 0 \\
% 	% circulating arm power, $P_\text{circ}$ & 3 MW & idler readout rate, $\gamma^c_R$ & 0 \\
% 	% test mass mass, $M$ & 200 kg & arm intra-cavity loss, $T_{l,a}$ & 100 ppm \\
% 	% input test mass transmissivity, $T_\text{ITM}$ & 0.197 & signal mode intra-cavity loss, $T_{l,b}$ & 1000 ppm \\
% 	% sloshing frequency, $\omega_s$ & 5 kHz & idler mode intra-cavity loss, $T_{l,c}$ & 1000 ppm \\
%  %  &  & detection loss, $R_\text{PD}$ & $10\%$ \\ \bottomrule
% 	% \end{tabular}
% }
\vframefill\centering
{\tiny LIGO Voyager parameter set from [R. X. Adhikari, K. Arai, A. F. Brooks, et al. 2020. \emph{Class. Quantum Grav.}, 37(16):165003.]}
\end{frame}


\end{document}
