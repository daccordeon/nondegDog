% LVK March 2022 meeting presentation
\documentclass[12pt,xcolor=dvipsnames,aspectratio=169]{beamer}
\mode<presentation> {
\usetheme{Madrid}
\setbeamertemplate{navigation symbols}{}
}

\usepackage[utf8]{inputenc}
\usepackage{amsmath, amssymb, amsthm}
\usepackage{amsfonts}
\usepackage{graphicx}
\usepackage{float}
\usepackage[center]{caption}
\usepackage{hyperref}
\usepackage{mathtools}
\usepackage{booktabs}

\usepackage{subfig}
\usepackage{comment}

\usepackage{multimedia}
% \usepackage{media9}
\usepackage{anyfontsize}


\graphicspath{{./}{/home/james/Code/honours/nondegDog/talks/ozgrav_template/}{/home/james/Code/honours/nondegDog/paper/figures_for_publication/}{/home/james/Code/honours/nondegDog/source/plots/LVK_talk/}}

\setbeamertemplate{enumerate items}[default]

\newcommand{\vframefill}{\vskip0pt plus 1filll}


\title[\url{https://dcc.ligo.org/LIGO-P2200052}]{Nondegenerate internal squeezing}
\author[James~W.~Gardner et al.]{\texorpdfstring{\large{James~W.~Gardner}\\\small{Min~Jet~Yap, Vaishali~Adya, Sheon~Chua, Bram~J.~J.~Slagmolen, and David~McClelland}}{James~Gardner}}
\institute[]{\small The Centre for Gravitational Astrophysics, ANU}
\date{LVK Meeting March 2022}

\begin{document}

%%%%%%%%%%%%%%%%%%%%%%%%%%%%%%%%%%%%%%%%%%

{\usebackgroundtemplate{\includegraphics[width=\paperwidth]{ozgrav_template_title_slide.pdf}}
\begin{frame}[label=titleframe,noframenumbering]
\end{frame}}

{\usebackgroundtemplate{\includegraphics[width=\paperwidth]{ozgrav_template_AoC_slide.pdf}}
\begin{frame}[noframenumbering]
\end{frame}}

% body templates are all too busy and/or ugly, just use normal Madrid beamer
% {\usebackgroundtemplate{\includegraphics[width=\paperwidth]{ozgrav_template_body_slide.pdf}}
\begin{frame}{Motivation: kilohertz gravitational waves}
\begin{columns}
\column{0.48\textwidth}
\centering
% \includegraphics[height=4.5cm,width=6cm]{merger_first_slide.png}
\movie[width=6cm,height=4.5cm,showcontrols,autostart,poster]{\includegraphics[height=4.5cm,width=6cm]{merger_first_slide.png}}{bns_merger_20s.mp4}
\\
{\tiny Media credit: [NASA/Goddard Space Flight Center, 2010]}
\column{0.52\textwidth}
\begin{enumerate}
\item Neutron-star equation-of-state
\item Origin of low-mass black holes
\item Post-bounce dynamics of core-collapse supernovae
\item Primordial sources
\end{enumerate}
$\implies$ want to improve \\ quantum~noise--limited sensitivity 
\\
\vspace{0.5cm}
\centering
{\tiny [R. X. Adhikari et al. 2020. Class. Quantum Grav., 37(16):165003.],
\\[0pt] [K.~Ackley et al., 2020, Publ. Astron. Soc. Aust., 37], 
\\[-7pt] [H. Miao et al. 2018. Phys. Rev. D, 98(4):044044.]}
\end{columns}
\end{frame}

% mention: alternative names, sWLC connection to nIS, limitation of low tolerance to loss
\begin{frame}{Existing proposals} % related to my work (not leading!)
\begin{columns}
\column{0.5\textwidth}
\centering
Degenerate internal squeezing\\\vspace{0.1cm}
% (also known as: degenerate quantum expansion)
\includegraphics[width=0.9\textwidth]{LVK_talk_dIS_config.pdf}
	\\{\tiny [M. Korobko et al. 2019. Light Sci. Appl., 8(1):118.] \\[-7pt]  [V. B. Adya et al. 2020. Class. Quantum Grav., 37(7):07LT02.]}
\column{0.5\textwidth}
\centering
Stable optomechanical filtering\\\vspace{0.2cm}
% (also known as: a stable white-light cavity)
\includegraphics[width=0.8\textwidth]{talk_sWLC_config.pdf}
\\{\tiny [X. Li et al. 2020. arXiv:2012.00836 [quant-ph]] \\[-7pt] [X. Li et al. 2021. Phys. Rev. D, 103:122001.]}
\end{columns}
\end{frame}

\begin{frame}{Nondegenerate internal squeezing}
\centering
\includegraphics[width=\textwidth]{talk_fig1_paper_nIS_config.pdf}
% \\{\tiny{Review of relevant quantum optics in [S. L. Danilishin et al. 2012. Living Rev. Relativ., 15(1):5.]}}
\end{frame}

\begin{frame}{Application in a modified LIGO Voyager}
\begin{columns}
\column{0.45\textwidth}
\includegraphics[width=\textwidth]{talk_fig1_paper_nIS_config_(a).pdf}
\column{0.55\textwidth}
\centering
Parameters of LIGO Voyager$^1$ except\\\vspace*{0.2cm}
\begin{tabular}{@{}ll@{}}
\toprule signal-recycling cavity length & 1.124 km \\
input test mass transmissivity & 0.197 \\[5pt]
arm intra-cavity loss & 100 ppm \\
SRC intra-cavity loss & 1000 ppm \\
detection loss & $10\%$ \\\bottomrule
\end{tabular}
\\\vspace*{0.5cm}
{\tiny ${}^1$[R. X. Adhikari et al. 2020. Class. Quantum Grav., 37(16):165003.]}
\end{columns}
\end{frame}

\begin{frame}{Characterisation of sensitivity} % for signal readout, explain signal response, also discuss astrophysical target
\centering
\includegraphics[height=0.8\textheight]{talk_fig3_nIS_N_S_NSR_vertical_annotated.pdf}
\end{frame}

\begin{frame}{Tolerance to detection optical loss} % don’t mention the paper when briefly talking about threshold
\begin{columns}
\column{0.5\textwidth}
\centering
\includegraphics[height=0.78\textheight]{talk_fig4_nIS_tolerance_to_detection_loss_sens_reduction_combined_vertical_1.pdf}
\column{0.5\textwidth}
\centering
\includegraphics[width=\textwidth]{talk_fig4_nIS_tolerance_to_detection_loss_sens_reduction_combined_vertical_2.pdf}
\\\vspace*{0.2cm}
\begin{tabular}{@{}ll@{}}
\toprule squeezer parameter, $\chi$ & 95$\%$ threshold \\
arm intra-cavity loss & 100 ppm \\
SRC intra-cavity loss & 1000 ppm \\\bottomrule
\end{tabular}
\end{columns}
\end{frame}

\begin{frame}{Idler readout scheme} % make dES clear, explain variational readout
\begin{columns}
\column{0.5\textwidth}
\centering\vspace*{1.5cm}
\includegraphics[height=0.8\textheight]{LVK_talk_idler_RO.pdf}
\column{0.5\textwidth}
% use picture environment to exactly specify positions to make clean animations
\begin{picture}(150,250)%(width,height)(xoffset,yoffset)
\put(0,60){\includegraphics<1>[width=0.9\textwidth]{talk_fig5_idlerRO_fixed_vs_variational_4curves.pdf}}
\put(0,60){\includegraphics<2>[width=0.9\textwidth]{fig5_idlerRO_fixed_vs_variational.pdf}}
\put(30,50){\vspace*{-0.1cm}\tiny All curves are without injection of external squeezing.}
\end{picture}
\end{columns}
\end{frame}

\begin{frame}{Combined readout scheme}
\begin{columns}
\column{0.5\textwidth}
\centering
\includegraphics[height=0.8\textheight]{talk_combined_RO_config.pdf}
\column{0.5\textwidth}
\centering
\includegraphics[width=0.9\textwidth]{talk_fig6_optimal_filter_vs_variational_readouts_vs_optNoVar_vertical.pdf}
\\{\tiny All curves are without injection of external squeezing.}
\end{columns}
\end{frame}

\begin{frame}{Summary}
\begin{enumerate}
\item Single-mode analytic Hamiltonian model
\item Dynamical stability and squeezing threshold % -- a new method
\item \textbf{Characterisation of sensitivity}
% \item \textbf{Comparison to astrophysical kilohertz target}
\item \textbf{Tolerance to detection optical loss} and other optical losses
\item Comparison to existing configurations
\item \textbf{Alternative readout schemes}
\end{enumerate}
% \vframefill\centering
{\tiny Content shown in boldface is covered in this presentation} 
\end{frame}

\begin{frame}{Conclusions} % return to motivation, talk about arm cavity FSR in future work (for CE/ET seems promising)
Nondegenerate internal squeezing
\begin{enumerate}
\item Tolerant to detection optical loss
\item Broadband sensitivity improvement using optimal filtering
\item Dynamically stable
\end{enumerate}
$\implies$ an all-optical, loss-resistant alternative %to existing proposals

\vspace*{0.5cm}
Potential future work
\begin{enumerate}
\item Multi-mode analysis
\item Experimental demonstration
\end{enumerate}
\end{frame}

{\usebackgroundtemplate{\includegraphics[width=\paperwidth]{ozgrav_template_thanks_slide.pdf}}
\begin{frame}[noframenumbering]
\end{frame}}

\setbeamercolor{frametitle}{fg=Blue,bg=Blue!20}

\begin{frame}[noframenumbering]{Table of parameters}
\centering
\vspace*{0.6cm}
\resizebox{1.2\textheight}{!}{%
    \begin{tabular}{@{}ll@{}}
    \toprule
    carrier wavelength, $\lambda_0=2\pi c/\omega_0$ & 2 $\mu\text{m}$ \\
    arm cavity length, $L_\text{arm}$ & 4 km \\
    \textbf{signal-recycling cavity length, $L_\text{SRC}$} & \textbf{1.124 km} \\
    circulating arm power, $P_\text{circ}$ & 3 MW \\
    test mass mass, $M$ & 200 kg \\
    \textbf{input test mass transmissivity, $T_\text{ITM}$} & \textbf{0.197} \\
    signal-recycling mirror transmissivity, $T_{\text{SRM},b}$ ($T_{\text{SRM},c}$) & 0.046 (0) \\[0.2cm]
    \textbf{sloshing frequency, $\omega_s$} & \textbf{5 kHz} \\
    \textbf{readout rate, $\gamma^b_R$ ($\gamma^c_R$)} & \textbf{0.5 (0) kHz} \\
    arm cavity free-spectral range & 37.5~kHz \\[0.2cm] 
    arm intra-cavity loss, $T_{l,a}$ & 100 ppm \\
    signal-recycling intra-cavity loss, $T_{l,b}$ ($T_{l,c}$) & 1000 (1000) ppm \\
    detection loss, $R_\text{PD}$ & $10\%$ \\ \bottomrule
    \end{tabular}}
\vframefill
{\tiny [R. X. Adhikari et al. 2020. Class. Quantum Grav., 37(16):165003.]}
\end{frame}

\begin{frame}[noframenumbering]{Stability and singularity threshold}
\centering
\includegraphics[height=0.8\textheight]{talk_fig2_nIS_stability_plane_w_colorbar_vertical.pdf}
\end{frame}

\end{document}
